http://medicina.med.up.pt/im/trabalhos_10_11/Sites/Turma4/article.pdf


ABSTRACT
Our main aim is to assess if the Bland Altman‟s method has been correctly applied in articles and
proceeding papers. From the 18360 documents indexed on ISI which quote the method, we excluded
1071 because we were interested in documents that could have applied the method, which means
that only articles and proceeding papers were considered, in a total of 17289 documents. From those,
we randomly selected a sample of 70 articles (to guarantee that all documents could be read and
analyzed and, simultaneously, to have an amount big enough to enable us to draw valid conclusions)
using SPSS program and assigned 35 to each class, 3 for each student. The data were collected
using a checklist and each document was analyzed twice by two different students to verify the interobserver
reproducibility of the checklist. After that we built a data base on SPSS with the collected
data and used a chi-square statistical test to data analysis. In the results we estimated, with a
confidence interval of 95%, the proportion of articles that applied the method and the proportion of
articles that applied each of the checklist‟s conditions, sorted by impact factor of the journals where
they were published and publication year. We found that 91% of our sample indeed applied Bland
Altman‟s method, but only 14% have correctly applied the method, fulfilling all of the checklist‟s
conditions. On the other hand, we can only take conclusion about one of the checklist‟s items (the
second assumption) because only in that case we obtained a p-value less than 0.05, being able to
say that the more recent articles verify the second assumption more often than the older ones, and
that was in line with what we have expected.
