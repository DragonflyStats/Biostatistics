\documentclass[12pt, a4paper]{article}
\usepackage{natbib}
\usepackage{vmargin}
\usepackage{graphicx}
\usepackage{epsfig}
\usepackage{subfigure}
%\usepackage{amscd}
\usepackage{amssymb}
\usepackage{subfiles}
\usepackage{subfigure}
\usepackage{framed}
\usepackage{subfiles}
\usepackage{amsbsy}
\usepackage{amsthm, amsmath}
%\usepackage[dvips]{graphicx}
\bibliographystyle{chicago}
\renewcommand{\baselinestretch}{1.1}

% left top textwidth textheight headheight % headsep footheight footskip
\setmargins{3.0cm}{2.5cm}{15.5 cm}{23.5cm}{0.25cm}{0cm}{0.5cm}{0.5cm}

\pagenumbering{arabic}
%-------------------------------------------------------------------Simplifying GLS by KH -%


\begin{document}

%\subfiles{IntroGLS}
%\subfile{AugmentedGLMs}


%---------------------------------------------------------------------------%
%1.1 Introduction to Influence Analysis
%1.2 Extension of techniques to LME Models
%1.3 Residual Diagnostics
%1.4 Standardized and studentized residuals
%1.5 Covariance Parameters
%1.6 Case Deletion Diagnostics
%1.7 Influence Analysis
%1.8 Terminology for Case Deletion
%1.9 Cook's Distance (Classical Case)
%1.10 Cook's Distance (LME Case)
%1.11 Likelihood Distance
%1.12 Other Measures
%1.13 CPJ Paper
%1.14 Matrix Notation of Case Deletion
%1.15 CPJ's Three Propositions
%1.16 Other measures of Influence
\tableofcontents
	
	\begin{framed} 
		\begin{itemize}
			\item \texttt{R} command and \texttt{R} object - Typewriter Font
			\item \texttt{R} Package name - Italics
			\item Selected Acronyms and Proper Nouns - Italics
		\end{itemize}
	\end{framed}
\medskip
	
\begin{itemize}	
\item This chapter is broken into two parts. The first part is a review of diagnostics methods for linear models, intended to acquaint the
	reader with the subject, and also to provide a basis for material covered in the second part. Particular attention is drawn to graphical methods.
	
\item The second part of the chapter looks at diagnostics techniques for LME models, firsly covering the theory, then proceeding to a discussion on 
	implementing these using \texttt{R} code.
\item While a substantial body of work has been developed in this area, ther are still area worth exploring. 
	In particular the development of graphical techniques pertinent to LME models should be looked at.
\end{itemize}



\section{Introduction}


\chapter{Model Diagnostics}

\subsection*{Abstract}
This chapter is broken into two parts. The first part is a review of diagnostics methods for linear models, intended to acquaint the reader with the subject, and also to provide a basis for material covered in the second part. Particular attention is drawn to graphical methods.



\section{Analysis of  Influence}


%--------------------------------------%

\subsection{Further Assumptions of Linear Models}

As with fitted models, the assumption of normality of residuals and homogeneity of variance is applicable to LMEs also. 

%--------------------------------------%


Homoscedascity is the technical term to describe the variance of the residuals being constant across the range of predicted values.
Heteroscedascity is the converse scenario : the variance differs along the range of values.

%--Marginal and Conditional Residuals

% \subfile{ResidualsLMEs.tex}
% \subfile{iterativemethods.tex}



In recent years, mixed models have become invaluable tools in the analysis of experimental and observational
data. In these models, more than one term can be subject to random variation. Mixed model technology enables you to analyze complex experimental data with hierarchical random processes, temporal,
longitudinal, and spatial data, to name just a few important applications. 


%====================================================================================================================%
\subsection{Summary of Schabenberger's Paper}

On occasion, quantification is not possible. Assume, for example, that a data point is removed
and the new estimate of the G matrix is not positive definite. This may occur if a variance component
estimate now falls on the boundary of the parameter space. Thus, it may not be possible to compute certain
influence statistics comparing the full-data and reduced-data parameter estimates. However, knowing that
a new singularity was encountered is important qualitative information about the data point’s influence on
the analysis.

The basic procedure for quantifying influence is simple:

\begin{enumerate}
	\item Fit the model to the data and obtain estimates of all parameters.
	\item Remove one or more data points from the analysis and compute updated estimates of model parameters.
	\item Based on full- and reduced-data estimates, contrast quantities of interest to determine how the absence
	of the observations changes the analysis.
\end{enumerate}
We use the subscript (U) to denote quantities obtained without the observations in the set U. For example,
%βb
(U) denotes the fixed-effects “\textit{\textbf{leave-U-out}}” estimates. Note that the set U can contain multiple observations.


%===================================================================================
If the global measure suggests that the points in U are influential, you should next determine the nature of
that influence. In particular, the points can affect
\begin{itemize}
	\item the estimates of fixed effects
	\item the estimates of the precision of the fixed effects
	\item the estimates of the covariance parameters
	\item the estimates of the precision of the covariance parameters
	\item fitted and predicted values
\end{itemize}

It is important to further decompose the initial finding to determine whether data points are actually troublesome.
Simply because they are influential “somehow”, should not trigger their removal from the analysis or
a change in the model. For example, if points primarily affect the precision of the covariance parameters
without exerting much influence on the fixed effects, then their presence in the data may not distort hypothesis
tests or confidence intervals about $\beta$.
%They will only do so if your inference depends on an estimate of the
%precision of the covariance parameter estimates, as is the case for the Satterthwaite and Kenward-Roger
%degrees of freedom methods and the standard error adjustment associated with the DDFM=KR option.

%------------------------------------------------------------%
\subsection{Summary of Paper}

Standard residual and influence diagnostics for linear models can be extended to LME models.
The dependence of the fixed effects solutions on the covariance parameters has important ramifications on the perturbation analysis.	
Calculating the studentized residuals-And influence statistics whereas each software procedure can calculate both conditional and marginal raw residuals, only SAs Proc Mixed is currently the only program that provide studentized residuals Which ave preferred for model diagnostics. The conditional Raw residuals ave not well suited to detecting outliers as are the studentized conditional residuals. (schabenbege r)


LME are flexible tools for the analysis of clustered and repeated measurement data. LME extend the capabilities of standard linear models by allowing unbalanced and missing data, as long as the missing data are MAR. Structured covariance matrices for both the random effects G and the residuals R. missing at Random.

A conditional residual is the difference between the observed valve and the predicted valve of a dependent variable- Influence diagnostics are formal techniques that allow the identification observation that heavily influence estimates of parameters.
To alleviate the problems with the interpretation of conditional residuals that may have unequal variances, we consider sealing.
Residuals obtained in this manner ave called studentized residuals.

\begin{itemize}
	\item Standard residual and influence diagnostics for linear models can be extended to linear mixed models. The dependence of fixed-effects solutions on the covariance parameter estimates has important ramifications in perturbation analysis. 
	\item To gauge the full impact of a set of observations on the analysis, covariance parameters need to be updated, which requires refitting of the model. 
	%	\item The experimental INFLUENCE option of the MODEL statement in the MIXED procedure (SAS 9.1) enables you to perform iterative and noniterative influence analysis for individual observations and sets of observations.
	
	\item The conditional (subject-specific) and marginal (population-averaged) formulations in the linear mixed model enable you to consider conditional residuals that use the estimated BLUPs of the random effects, and marginal residuals which are deviations from the overall mean. 
	\item Residuals using the BLUPs are useful to diagnose whether the random effects components in the model are specified correctly, marginal residuals are useful to diagnose the fixed-effects components. 
	\item Both types of residuals are available in SAS 9.1 as an experimental option of the MODEL statement in the MIXED procedure.
	
	\item It is important to note that influence analyses are performed under the assumption that the chosen model is correct. Changing the model structure can alter the conclusions. Many other variance models have been fit to the data presented in the repeated measures example. You need to see the conclusions about which model component is affected in light of the model being fit.
	%	\item  For example, modeling these data with a random intercept and random slope for each child or an unstructured covariance matrix will affect your conclusions about which children are influential on the analysis and how this influence manifests itself.
\end{itemize}



	
	\section{Case Deletion Diagnostics for LME models}

	%%%\citet{christensen}
	
	Christensen (19XX)  describes three propositions that are required
	for efficient case-deletion in LME models. The first proposition
	decribes how to efficiently update $V$ when the $i$th element is
	deleted.
	\begin{equation}
	V_{[i]}^{-1} = \Lambda_{[i]} - \frac{\lambda
		\lambda\prime}{\nu^{}ii}
	\end{equation}
	
	
	The second of Christensen's propostions is the following set of
	equations, which are variants of the Sherman Wood bury updating
	formula.
	\begin{eqnarray}
	X'_{[i]}V_{[i]}^{-1}X_{[i]} &=& X' V^{-1}X -
	\frac{\hat{x}_{i}\hat{x}'_{i}}{s_{i}}\\
	(X'_{[i]}V_{[i]}^{-1}X_{[i]})^{-1} &=& (X' V^{-1}X)^{-1} +
	\frac{(X' V^{-1}X)^{-1}\hat{x}_{i}\hat{x}' _{i}
		(X' V^{-1}X)^{-1}}{s_{i}- \bar{h}_{i}}\\
	X'_{[i]}V_{[i]}^{-1}Y_{[i]} &=& X\prime V^{-1}Y -
	\frac{\hat{x}_{i}\hat{y}' _{i}}{s_{i}}
	\end{eqnarray}
	
	
	
	
	
	
	
	
	In LME models, fitted by either ML or REML, an important overall
	influence measure is the likelihood distance \citep{cook82}. The
	procedure requires the calculation of the full data estimates
	$\hat{\psi}$ and estimates based on the reduced data set
	$\hat{\psi}_{(U)}$. The likelihood distance is given by
	determining
	
	
	\begin{eqnarray}
	LD_{(U)} &=& 2\{l(\hat{\psi}) - l( \hat{\psi}_{(U)}) \}\\
	RLD_{(U)} &=& 2\{l_{R}(\hat{\psi}) - l_{R}(\hat{\psi}_{(U)})\}
	\end{eqnarray}
	




\subsection{Residual Diagnostics in LME models}
\begin{itemize}
	\item A \textbf{residual} is the difference between the observed quantity and the predicted value. In LME models a distinction is made between marginal residuals and conditional residuals.
	
	\item A \textbf{Marginal residual} is the difference between the observed data and the estimated marginal mean (Schabenberger  pg3)
	The computation of case deletion diagnostics in the classical model is made simple by the fact that important estimates can be computed without refitting the model. 
	
	\item Such update formulae are available in the mixed model only if you assume that the covariance parameters are not affect by the removal of the observation in question. Schabenberger remarks that this is not a reasonable assumption.
	
\end{itemize}


Basic procedure for quantifying influence is simple

\begin{enumerate}
	\item  	Fit the model to the data
	\item   	Remove one or more data points from the analysis and compute updated estimates of model parameters
	\item  	Based on the full and reduced data estimates, contrast quantities of interest to determine how the absence of the observations changed the analysis.
\end{enumerate}
The likelihood distance is a global summary measure expressing the joint influence of the observations in the set U on all parameters in $\Psi$ that were subject to updating.


\section{Case Deletion Diagnostics for LME models}

\citet{HaslettDillane} remark that linear mixed effects models
didn't experience a corresponding growth in the use of deletion
diagnostics, adding that \citet{McCullSearle} makes no mention of
diagnostics whatsoever.

\citet{Christensen} describes three propositions that are required
for efficient case-deletion in LME models. The first proposition
decribes how to efficiently update $V$ when the $i$th element is
deleted.
\begin{equation}
V_{[i]}^{-1} = \Lambda_{[i]} - \frac{\lambda
	\lambda\prime}{\nu^{}ii}
\end{equation}


The second of christensen's propostions is the following set of
equations, which are variants of the Sherman Wood bury updating
formula.
\begin{eqnarray}
X'_{[i]}V_{[i]}^{-1}X_{[i]} &=& X' V^{-1}X -
\frac{\hat{x}_{i}\hat{x}'_{i}}{s_{i}}\\
(X'_{[i]}V_{[i]}^{-1}X_{[i]})^{-1} &=& (X' V^{-1}X)^{-1} +
\frac{(X' V^{-1}X)^{-1}\hat{x}_{i}\hat{x}' _{i}
	(X' V^{-1}X)^{-1}}{s_{i}- \bar{h}_{i}}\\
X'_{[i]}V_{[i]}^{-1}Y_{[i]} &=& X\prime V^{-1}Y -
\frac{\hat{x}_{i}\hat{y}' _{i}}{s_{i}}
\end{eqnarray}








In LME models, fitted by either ML or REML, an important overall
influence measure is the likelihood distance \citep{cook82}. The
procedure requires the calculation of the full data estimates
$\hat{\psi}$ and estimates based on the reduced data set
$\hat{\psi}_{(U)}$. The likelihood distance is given by
determining


\begin{eqnarray}
LD_{(U)} &=& 2\{l(\hat{\psi}) - l( \hat{\psi}_{(U)}) \}\\
RLD_{(U)} &=& 2\{l_{R}(\hat{\psi}) - l_{R}(\hat{\psi}_{(U)})\}
\end{eqnarray}

\section{Carstensen}
Let $y_{mir} $ be the $r$th replicate measurement on the $i$th item by the $m$th method, where $m=1,2,$ $i=1,\ldots,N,$ and $r = 1,\ldots,n_i.$ When the design is balanced and there is no ambiguity we can set $n_i=n.$ The LME model underpinning Roy's approach can be written
\begin{equation}
y_{mir} = \beta_{0} + \beta_{m} + b_{mi} + \epsilon_{mir}.
\end{equation}
Here $\beta_0$ and $\beta_m$ are fixed-effect terms representing, respectively, a model intercept and an overall effect for method $m.$
The $\beta$ terms can be gathered together into (fixed effect) intercept terms $\alpha_m=\beta_0+\beta_m.$ The $b_{1i}$ and $b_{2i}$ terms are correlated random effect parameters having $\mathrm{E}(b_{mi})=0$ with $\mathrm{Var}(b_{mi})=g^2_m$ and $\mathrm{Cov}(b_{mi}, b_{m^\prime i})=g_{12}.$ The random error term for each response is denoted $\epsilon_{mir}$ having $\mathrm{E}(\epsilon_{mir})=0$, $\mathrm{Var}(\epsilon_{mir})=\sigma^2_m$, $\mathrm{Cov}(b_{mir}, b_{m^\prime ir})=\sigma_{12}$, $\mathrm{Cov}(\epsilon_{mir}, \epsilon_{mir^\prime})= 0$ and $\mathrm{Cov}(\epsilon_{mir}, \epsilon_{m^\prime ir^\prime})= 0.$ Two methods of measurement are in complete agreement if the null hypotheses $\mathrm{H}_1\colon \beta_1 = \beta_2$ and $\mathrm{H}_2\colon \sigma^2_1 = \sigma^2_2 $ and $\mathrm{H}_3\colon g^2_1= g^2_2$ hold simultaneously. \citet{roy} proposes a Bonferroni correction to control the familywise error rate for tests of $\{\mathrm{H}_1, \mathrm{H}_2, \mathrm{H}_3\}$ and account for difficulties arising due to multiple testing. Let $\omega^2_m = \sigma^2_m + g^2_m$ represent the overall variability of method $m.$  Roy also integrates $\mathrm{H}_2$ and $\mathrm{H}_3$ into a single testable hypothesis $\mathrm{H}_4\colon \omega^2_1=\omega^2_2.$ CONCERNS?

\bigskip

% Complete paragraph by specifying variances and covariances for epsilons.
% I thing that these are your sigmas?
% Also, state equality of the parameters in this model when each of the three hypotheses above are true.
\citet{Roy} demonstrates how to implement a method comparison study further to model (1) using the SAS proc mixed package.
%------------------------------------------------------------------------------------------------%
\citet{BXC2008} demonstrates how to construct limits of agreement using SAS, STATA and R. In the case of SAS, the PROC MIXED procedure is used.
Implementation in R is performed using the nlme package \citep{pb2000}.

\citet{BXC2008} remarks that the implementation using R is quite ``arcane".

As R is freely available, this paper demonstrates an implementation of Roy's model using R.

The R statistical software package is freely available.

%------------------------------------------------------------------------------------------------%
The LME model is very easy to implement using PROC MIXED of SAS and the results are also easy to interpret.
The SAS proc mixed procedure has very simple syntax.

As the required code to fit the models is complex, R code necessary to fit the models is provided. 

A demonstration is provided on how to use the output to perform the tests, and to compute limits of agreement.



We assume the data are formatted as a dataset with four columns named:

meth, method of measurement, the number of methods being M,
item, items (persons, samples) measured by each method, of which there are I,
repl, replicate indicating repeated measurement of the same item by the same method, and
y, the measurement.







%=======================================================================================%

\subsection{Remarks on the Multivariate Normal Distribution}

Diligence is required when considering the models. Carstensen specifies his models in terms of the univariate normal distribution. Roy's model is specified using the bivariate normal distribution.
This gives rises to a key difference between the two model, in that a bivariate model accounts for covariance between the variables of interest.
The multivariate normal distribution of a $k$-dimensional random vector $X = [X_1, X_2, \ldots, X_k]$
can be written in the following notation:
\[
X\ \sim\ \mathcal{N}(\mu,\, \Sigma),
\]
or to make it explicitly known that $X$ is $k$-dimensional,
\[
X\ \sim\ \mathcal{N}_k(\mu,\, \Sigma).
\]
with $k$-dimensional mean vector
\[ \mu = [ \operatorname{E}[X_1], \operatorname{E}[X_2], \ldots, \operatorname{E}[X_k]] \]
and $k \times k$ covariance matrix
\[ \Sigma = [\operatorname{Cov}[X_i, X_j]], \; i=1,2,\ldots,k; \; j=1,2,\ldots,k \]

\bigskip

\begin{enumerate}
	\item Univariate Normal Distribution
	
	\[
	X\ \sim\ \mathcal{N}(\mu,\, \sigma^2),
	\]
	
	\item Bivariate Normal Distribution
	
	\begin{itemize}
		\item[(a)] \[  X\ \sim\ \mathcal{N}_2(\mu,\, \Sigma), \vspace{1cm}\]
		\item[(b)] \[    \mu = \begin{pmatrix} \mu_x \\ \mu_y \end{pmatrix}, \quad
		\Sigma = \begin{pmatrix} \sigma_x^2 & \rho \sigma_x \sigma_y \\
		\rho \sigma_x \sigma_y  & \sigma_y^2 \end{pmatrix}.\]
	\end{itemize}
\end{enumerate}


%\chapter{Limits of Agreement}

\section{Modelling Agreement with LME Models}

% Carstensen pages 22-23


Roys uses and LME model approach to provide a set of formal tests for method comparison studies.\\

Four candidates models are fitted to the data.\\

These models are similar to one another, but for the imposition of equality constraints.\\

These tests are the pairwise comparison of candidate models, one formulated without constraints, the other with a constraint.\\


Roy's model uses fixed effects $\beta_0 + \beta_1$ and $\beta_0 + \beta_1$ to specify the mean of all observationsby \\ methods 1 and 2 respectuively.





Roy adheres to Random Effect ideas in ANOVA

Roy treats items as a sample from a population.\\

Allocation of fixed effects and random effects are very different in each model\\

Carstensen's interest lies in the difference between the population from which they were drawn.\\

Carstensen's model is a mixed effects ANOVA.\\

\[
Y_{mir}  =  \alpha_m + \mu_i + c_{mi} + e_{mir}, \qquad c_{mi} \sim \mathcal{\tau^2_m}, \qquad e_{mir} \sim \mathcal{\sigma^2_m},
\]

This model includes a method by item iteration term.\\

Carstensen presents two models. One for the case where the replicates, and a second for when they are linked.\\

Carstensen's model does not take into account either between-item or within-item covariance between methods.\\


In the presented example, it is shown that Roy's LoAs are lower than those of Carstensen.
Carstensen makes some interesting remarks in this regard.

\begin{quote}
	The only slightly non-standard (meaning "not often used") feature is the differing residual variances between methods.
\end{quote}





It is also desirable to measure the influence of the case deletions on the covariance matrix of $\hat{\beta}$.



%===================================================================%

\begin{itemize}
	\item \textit{
		The previous Section (Section 4) is a literary review of residual diagnostics and influence procedures
		for Linear Mixed Effects Models, drawing heavily on Schabenberger and Zewotir.}
	
	\item \textit{	Section 4 begins with an introduction to key topics in residual diagnostics, such as influence, leverage, outliers
		and Cook's distance. Other concepts such as DFFITS and DFBETAs will be introduced briefly, mostly to explain why the are not particularly useful for
		the Method Comparison context, and therefore are not elaborated upon.}
	
	\item \textit{	In brief, Variable Selection is not applicable to Method Comparison Studies, in the 
		commonly used used context. 
		Testing a rather simplisticy specificied model against one with more random effects terms is tractable, but this research question is of secondary importance.}
\end{itemize}



\subsection{Matrix Notation for Case Deletion} %1.14

%\subsection{Case deletion notation} %1.14.1

For notational simplicity, $\boldsymbol{A}(i)$ denotes an $n \times m$ matrix $\boldsymbol{A}$ with the $i$-th row
removed, $a_i$ denotes the $i$-th row of $\boldsymbol{A}$, and $a_{ij}$ denotes the $(i, j)-$th element of $\boldsymbol{A}$.
%
%\subsection{Partitioning Matrices} %1.14.2
%Without loss of generality, matrices can be partitioned as if the $i-$th omitted observation is the first row; i.e. $i=1$.












\section{Extension of Diagnostic Methods to LME models}

When similar notions of statistical influence are applied to mixed models,
things are more complicated. Removing data points affects fixed effects and covariance parameter estimates.
Update formulas for “\textit{leave-one-out}” estimates typically fail to account for changes in covariance
parameters. 
%
%
%In LME models, there are two types of residuals, marginal residuals and conditional residuals. A
%marginal residual is the difference between the observed data and the estimated marginal mean. A conditional residual is the
%difference between the observed data and the predicted value of the observation. In a model without random effects, both sets of residuals coincide \citep{schab}.

\citet{Christiansen} noted the case deletion diagnostics techniques have not been applied to linear mixed effects models and seeks to develop methodologies in that respect. \citet{Christiansen} develops these techniques in the context of REML.

\citet{CPJ} noted the case deletion diagnostics techniques had not been applied to linear mixed effects models and seeks to develop methodologies in that respect. \citet{CPJ} develops these techniques in the context of REML.

%\citet{CPJ} develops \index{case deletion diagnostics} case deletion diagnostics, in particular the equivalent of \index{Cook's distance} Cook's distance, a well-known metric, for diagnosing influential observations when estimating the fixed effect parameters and variance components. Deletion diagnostics provide a means of assessing the influence of an observation (or groups of observations) on inference on the estimated parameters of LME models. We shall provide a fuller discussion of Cook's distance in due course.


\citet{Demi} extends several regression diagnostic techniques commonly used in linear regression, such as leverage, infinitesimal influence, case deletion diagnostics, Cook's distance, and local influence to the linear mixed-effects model. In each case, the proposed new measure has a direct interpretation in terms of the effects on a parameter of interest, and reduces to the familiar linear regression measure when there are no random effects. 

The new measures that are proposed by \citet{Demi} are explicitly defined functions and do not require re-estimation of the model, especially for cluster deletion diagnostics. The basis for both the cluster deletion diagnostics and Cook's distance is a generalization of Miller's simple update formula for case deletion for linear models. Furthermore \citet{Demi} shows how Pregibon's infinitesimal case deletion diagnostics is adapted to the linear mixed-effects model. 
%A simple compact matrix formula is derived to assess the local influence of the fixed-effects regression coefficients. 


%
%
%\section{Case Deletion Diagnostics for LME models} %1.6
%
%Data from single individuals, or a small group of subjects may influence non-linear mixed effects model selection. Diagnostics routinely applied in model building may identify such individuals, but these methods are not specifically designed for that purpose and are, therefore, not optimal. 

\citet{Demi} proposes two likelihood-based diagnostics for identifying individuals that can influence the choice between two competing models.




\section{Cook's Distance for LMEs} %1.10
Diagnostic methods for fixed effects are generally analogues of methods used in classical linear models.
Diagnostic methods for variance components are based on `one-step' methods. \citet{cook86} gives a completely general method for assessing the influence of local departures from assumptions in statistical models.

For fixed effects parameter estimates in LME models, the \index{Cook's distance} Cook's distance can be extended to measure influence on these fixed effects.

\[
\mbox{CD}_{i}(\beta) = \frac{(c_{ii} - r_{ii}) \times t^2_{i}}{r_{ii} \times p}
\]

For random effect estimates, the \index{Cook's distance} Cook's distance is

\[
\mbox{CD}_{i}(b) = g{\prime}_{(i)} (I_{r} + \mbox{var}(\hat{b})D)^{-2}\mbox{var}(\hat{b})g_{(i)}.
\]
Large values for Cook's distance indicate observations for special attention.










\subsubsection{Random Effects}

A large value for $CD(u)_i$ indicates that the $i-$th observation is influential in predicting random effects.

\subsubsection{linear functions}

$CD(\psi)_i$ does not have to be calculated unless $CD(\beta)_i$ is large.

%---------------------------------------------------------------------------%
%

For LME models, two formulations exist; a \index{Cook's distance}Cook's distance that examines the change in fixed fixed parameter estimates, and another that examines the change in random effects parameter estimates. The outcome of either Cook's distance is a scaled change in either $\beta$ or $\theta$.

%If $V$ is known, Cook's D can be calibrated according to a chi-square distribution with degrees of freedom equal to the rank of $\boldsymbol{X}$ \citep{cpj92}.

%
%%---------------------------------------------------------------------------%
%\newpage
%\section{Cook's Distance for LMEs} %1.10
%Diagnostic methods for fixed effects are generally analogues of methods used in classical linear models.
%Diagnostic methods for variance components are based on `one-step' methods. \citet{cook86} gives a completely general method for assessing the influence of local departures from assumptions in statistical models.
%
%For fixed effects parameter estimates in LME models, the \index{Cook's distance} Cook's distance can be extended to measure influence on these fixed effects.
%
%\[
%\mbox{CD}_{i}(\beta) = \frac{(c_{ii} - r_{ii}) \times t^2_{i}}{r_{ii} \times p}
%\]
%
%For random effect estimates, the \index{Cook's distance} Cook's distance is
%
%\[
%\mbox{CD}_{i}(b) = g{\prime}_{(i)} (I_{r} + \mbox{var}(\hat{b})D)^{-2}\mbox{var}(\hat{b})g_{(i)}.
%\]
%Large values for Cook's distance indicate observations for special attention.
%



\subsubsection{linear functions}

$CD(\psi)_i$ does not have to be calculated unless $CD(\beta)_i$ is large.

%---------------------------------------------------------------------------%
%

For LME models, two formulations exist; a \index{Cook's distance}Cook's distance that examines the change in fixed fixed parameter estimates, and another that examines the change in random effects parameter estimates. The outcome of either Cook's distance is a scaled change in either $\beta$ or $\theta$.

%If $V$ is known, Cook's D can be calibrated according to a chi-square distribution with degrees of freedom equal to the rank of $\boldsymbol{X}$ \citep{cpj92}.

%
%%---------------------------------------------------------------------------%
%\newpage
%\section{Cook's Distance for LMEs} %1.10
%Diagnostic methods for fixed effects are generally analogues of methods used in classical linear models.
%Diagnostic methods for variance components are based on `one-step' methods. \citet{cook86} gives a completely general method for assessing the influence of local departures from assumptions in statistical models.
%
%For fixed effects parameter estimates in LME models, the \index{Cook's distance} Cook's distance can be extended to measure influence on these fixed effects.
%
%\[
%\mbox{CD}_{i}(\beta) = \frac{(c_{ii} - r_{ii}) \times t^2_{i}}{r_{ii} \times p}
%\]
%
%For random effect estimates, the \index{Cook's distance} Cook's distance is
%
%\[
%\mbox{CD}_{i}(b) = g{\prime}_{(i)} (I_{r} + \mbox{var}(\hat{b})D)^{-2}\mbox{var}(\hat{b})g_{(i)}.
%\]
%Large values for Cook's distance indicate observations for special attention.
%




\subsection{Change in the precision of estimates}

The effect on the precision of estimates is separate from the effect on the point estimates. Data points that
have a small \index{Cook's distance} Cook's distance, for example, can still greatly affect hypothesis tests and confidence intervals, if their  influence on the precision of the estimates is large.








\section{Simplifying GLS}

It is straightforward to show that $\hat{\beta} = (X^tV^{-l}X)^{-l}X^tV^{-l}Y = BY$ and at the minimum the sum of squares is $Y^{t} (V^{-l}  - V^{-l}(X^tV^{-l}X)^{-l}X^tV^{-l})Y = Y^{t}QY$.\\
\bigskip

\emph{The purpose of this note is to give emphasis to one derivation, based on Lagrange multipliers, which leads to a system of equations that is very intuitive and lends itself readily to specialization. This approach is in fact standard in the geostatistical treatment of \textbf{kriging} (see Matheron 1962; Journel and Huijbregts 1981; Ripley 1981; Cressie 1993). In the genetics literature it is associated with the name of Henderson (1983); or in the classical statistical literature Hocking (1996, p. 73) is a suitable reference.}

\emph{The approach based on Lagrange multipliers deemphasizes the explicit determination of $\hat{\beta}$ and leads to a clearer understanding of the complementary (but for some confusing) tasks known as best linear unbiased estimation (BLUE) and best linear unbiased prediction (BLUP). Regrettably, Robinson-despite offering four derivations, and having as his main concern the interplay of BLUP and BLUE-gives it little prominence.}

It has recently been discussed by Searle (1997, p; 278) who said that it makes another approach (Searle, Casella, and McCulloch 1992, p. 271) seem "obtuse and unnecessarily complicated." By contrast, our treatment emphasizes the fact that it leads to a single set of equations whose solution sheds simplifying light on very many issues in general least squares.

The American Statistician's Teacher's Corner (e.g., McLean, Sanders, and Stroup 1991; Puntanen and Styan 1989) has already played host to previous attempts to simplify the explanation of such topics. Various authors (CPJ, Haslett Hayes ,Martin ) have visited the more specialized area of diagnostics and have developed \textbf{\emph{down-dating}} (leave-$k$-out) formulas.

The conventional approach here is via tricky identities based on the inverses of partitioned matrices. Here again the Lagrange system of equations leads to a much simplified and-we claim-much more intuitive derivation of these more technical results.


\emph{
	The essence of the approach is to seek that linear combination of the available data Y which is best for the
	estimation of Z among those linear estimators which are constrained to be unbiased. We adopt therefore a constrained minimization approach, using Lagrange multipliers. By best we mean that combination $\hat{Z}(Y) = \lambda_{z}^{t}Y$ which has least mean square error $E( Z- \lambda_{z}^{t}Y)^2$, and by unbiased we mean $E( Z- \lambda_{z}^{t}Y)) = 0$. }
Here $Z$ denotes that scalar which is to be the objective of the estimation. This estimator is written as $\hat{Z}(Y)$ to make its dependence on $Y$ explicit. Note that the term "best" is applied in the context of minimizing the prediction variance $var(Z - Z(Y))$. We shall see that Z may be used to denote either a random variable or an unknown parameter, and that it will be sufficient to specify Z via $E[Z]$ and $cov(Z, Y)$. If $Z$ is not a random variable then of course the latter is zero and $E[Z] = Z$. We establish-very simply, as below-a general solution in terms of A and $\operatorname{cov}(Z, Y)$ and achieve particular tasks by identification of these. Our presentation is for a scalar Z, but the notation facilitates generalization to vector Z.


%--------------------------------------------------------------------%
\newpage




\section{Repeated measurements in LME models}

In many statistical analyzes, the need to determine parameter estimates where multiple measurements are available on each of a set of variables often arises. Further to \citet{lam}, \citet{hamlett} performs an analysis of the correlation of replicate measurements, for two variables of interest, using LME models.

Let $y_{Aij}$ and $y_{Bij}$ be the $j$th repeated observations of the variables of interest $A$ and $B$ taken on the $i$th subject. The number of repeated measurements for each variable may differ for each individual.
Both variables are measured on each time points. Let $n_{i}$ be the number of observations for each variable, hence $2\times n_{i}$ observations in total.

It is assumed that the pair $y_{Aij}$ and $y_{Bij}$ follow a bivariate normal distribution.
\begin{eqnarray*}
	\left(
	\begin{array}{c}
		y_{Aij} \\
		y_{Bij} \\
	\end{array}
	\right) \sim \mathcal{N}(
	\boldsymbol{\mu}, \boldsymbol{\Sigma})\mbox{   where } \boldsymbol{\mu} = \left(
	\begin{array}{c}
		\mu_{A} \\
		\mu_{B} \\
	\end{array}
	\right)
\end{eqnarray*}

The matrix $\Sigma$ represents the variance component matrix between response variables at a given time point $j$.

\[
\boldsymbol{\Sigma} = \left( \begin{array}{cc}
\sigma^2_{A} & \sigma_{AB} \\
\sigma_{AB} & \sigma^2_{B}\\
\end{array}   \right)
\]

$\sigma^2_{A}$ is the variance of variable $A$, $\sigma^2_{B}$ is the variance of variable $B$ and $\sigma_{AB}$ is the covariance of the two variable. It is assumed that $\boldsymbol{\Sigma}$ does not depend on a particular time point, and is the same over all time points.

\bibliographystyle{chicago}
\bibliography{DB-txfrbib}
\end{document}


