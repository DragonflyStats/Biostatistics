
\documentclass[12pt, a4paper]{article}
\usepackage{natbib}
\usepackage{vmargin}
\usepackage{graphicx}
\usepackage{epsfig}
\usepackage{subfigure}
%\usepackage{amscd}
\usepackage{amssymb}
\usepackage{subfigure}
\usepackage{amsbsy}
\usepackage{amsthm, amsmath}
%\usepackage[dvips]{graphicx}
\bibliographystyle{chicago}
\renewcommand{\baselinestretch}{1.4}

% left top textwidth textheight headheight % headsep footheight footskip
\setmargins{3.0cm}{2.5cm}{15.5 cm}{23.5cm}{0.5cm}{0cm}{1cm}{1cm}

\pagenumbering{arabic}


\begin{document}
\chapter{Zewotir's Paper}
	

%-------------------------------------------------------------------------------------------------------------------------------------%



\section{Demidenko's I Influence} %2.6
The concept of I Influence is generalized  to the non linea regression model.


\subsection{Extension of techniques to LME Models} %1.2

Model diagnostic techniques, well established for classical models, have since been adapted for use with linear mixed effects models. Diagnostic techniques for LME models are inevitably more difficult to implement, due to the increased complexity.

Beckman, Nachtsheim and Cook (1987) \citet{Beckman} applied the \index{local influence}local influence method of Cook (1986) to the analysis of the linear mixed model.

While the concept of influence analysis is straightforward, implementation in mixed models is more complex. Update formulae for fixed effects models are available only when the covariance parameters are assumed to be known.

If the global measure suggests that the points in $U$ are influential, the nature of that influence should be determined. In particular, the points in $U$ can affect the following

\begin{itemize}
	\item the estimates of fixed effects,
	\item the estimates of the precision of the fixed effects,
	\item the estimates of the covariance parameters,
	\item the estimates of the precision of the covariance parameters,
	\item fitted and predicted values.
\end{itemize}








%--------------------------------------------------%
\subsection{Residual Analysis for Linear Models, LME models and GLMs}

\textbf{Keywords:}

\begin{itemize}
	\item Residuals (\emph{Beginners}), 
	\item Testing the Assumption of Normality (\emph{Beginners})
	\item Diagnostic Plots with the \texttt{plot} function
	\item Cook's Distance
	\item DFFits and DFBeta
	\item Standardized and Studentized Residuals
	\item Influence Leverage and Outlierness
\end{itemize}


\subsection{Diagnostics for Random Effects}
Empirical best linear unbiased predictors EBLUPS provide the a useful way of diagnosing random effects.

EBLUPs are also known as ``shrinkage estimators" because they tend to be smaller than the estimated effects would be if they were computed by treating a random factor as if it was fixed (West et al )







\subsection{Case Deletion Diagnostics for Mixed Models}

\citet{Christiansen} notes the case deletion diagnostics techniques have not been applied to linear mixed effects models and seeks to develop methodologies in that respect.

\citet{Christiansen} develops these techniques in the context of REML

\subsection{Methods and Measures}
The key to making deletion diagnostics useable is the development of efficient computational formulas, allowing one to obtain the \index{case deletion diagnostics} case deletion diagnostics by making use of basic building blocks, computed only once for the full model.


\citet{Zewotir} lists several established methods of analyzing influence in LME models. These methods include \begin{itemize}
	\item Cook's distance for LME models,
	\item \index{likelihood distance} likelihood distance,
	\item the variance (information) ration,
	\item the \index{Cook-Weisberg statistic} Cook-Weisberg statistic,
	\item the \index{Andrews-Prebigon statistic} Andrews-Prebigon statistic.
\end{itemize}

\section{Zewotir Measures of Influence in LME Models}%2.2
%Zewotir page 161
\citet{Zewotir} describes a number of approaches to model diagnostics, investigating each of the following;
\begin{itemize}
	\item Variance components
	\item Fixed effects parameters
	\item Prediction of the response variable and of random effects
	\item likelihood function
\end{itemize}

\citet{Zewotir} lists several established methods of analyzing influence in LME models. These methods include \begin{itemize}
	\item Cook's distance for LME models,
	\item \index{likelihood distance} likelihood distance,
	\item the variance (information) ration,
	\item the \index{Cook-Weisberg statistic} Cook-Weisberg statistic,
	\item the \index{Andrews-Prebigon statistic} Andrews-Prebigon statistic.
\end{itemize}




%---------------------------------------------------------------------------%
\newpage

\section{Zewotir Notepad}

\begin{quote}
	Abstract: Linear mixed models are extremely sensitive to outlying responses and extreme points in the fixed and random effect design spaces. Few diagnostics are available in standard computing packages. We provide routine diagnostic tools, which are computationally inexpensive. The diagnostics
	are functions of basic building blocks: studentized residuals, error contrast matrix, and the inverse of the response variable covariance matrix. The basic building blocks are computed only once from the complete data analysis and provide information on the influence of the data on different aspects
	of the model fit. Numerical examples provide analysts with the complete pictures of the diagnostics.
\end{quote}
Key words: Case deletion, influential observations, randomeffects, statistical
diagnostics, variance components ratios.

%-------------------------------------------------------------------------------------------------------%

Description: The influence of observations on statistical inference is of importance in statistical data analysis. 
A practical and well-established approach to influence analysis is based on case deletion. 
We provide computationally inexpensive diagnostic tools for linear mixed models. 
The diagnostics are a function of basic building blocks, computed only once from the complete data analysis, 
and provide information on the influence of the data on different aspects of the model fit.


%-------------------------------------------------------------------------------------------------------%
\bigskip 

Residual standard deviation.

Roy Subject effects, replicate in subject,

Cardiac data PEFR data from BLand


Royal Melbourne Hospital.


Roy demonstrates that correlation can be described under the model formulation.

\[Y_i = x.\beta + Z.u + \epsilon]\]


Laird Ware form (litte et al)



%-------------------------------------------------------------------------------------------------------%

For the purpose of comparison of both approaches, we compute the limits of agreement for two methods described in 
well known data sets.
%Examples Done On Other Notepad

%-------------------------------------------------------------------------------------------------------%

% Zewotir
$\epsilon$ is an $n \times 1$ vector of error terms
Zewotir provides routine diagnostics tools that are computationally inexpensive.
$\boldsymbol{u}_i$ is a $q_i \times 1$  vector of random variables from $\mathcal{N}(0 \sigma^2.I)$


% 2. Model Definiton and Estimation 

%-------------------------------------------------------------------------------------------------------%  
% 3. Background, Notation and Update Formulae




\section{Zewotir Measures of Influence in LME Models}%2.2
%Zewotir page 161
\citet{Zewotir} describes a number of approaches to model diagnostics, investigating each of the following;
\begin{itemize}
	\item Variance components
	\item Fixed effects parameters
	\item Prediction of the response variable and of random effects
	\item likelihood function
\end{itemize}

\subsection{Cook's Distance}
\begin{itemize}
	\item For variance components $\gamma$: $CD(\gamma)_i$,
	\item For fixed effect parameters $\beta$: $CD(\beta)_i$,
	\item For random effect parameters $\boldsymbol{u}$: $CD(u)_i$,
	\item For linear functions of $\hat{beta}$: $CD(\psi)_i$
\end{itemize}


\subsubsection{Random Effects}

A large value for $CD(u)_i$ indicates that the $i-$th observation is influential in predicting random effects.

\subsubsection{linear functions}

$CD(\psi)_i$ does not have to be calculated unless $CD(\beta)_i$ is large.





\section{Efficient Updating Theorem} %2.1
\citet{Zewotir} describes the basic theorem of efficient updating.
\begin{itemize}
	\item \[ m_i = {1 \over c_{ii}}\]
	%\item
	%item
	%\item
\end{itemize}





\subsubsection{Random Effects}


A large value for $CD(u)_i$ indicates that the $i-$th observation is influential in predicting random effects.


\subsubsection{linear functions}


$CD(\psi)_i$ does not have to be calculated unless $CD(\beta)_i$ is large.




\subsection{Information Ratio}




%--------------------------------------------------------------%
\newpage

\section{Computation and Notation } %2.3
with $\boldsymbol{V}$ unknown, a standard practice for estimating $\boldsymbol{X \beta}$ is the estime the variance components $\sigma^2_j$,
compute an estimate for $\boldsymbol{V}$ and then compute the projector matrix $A$, $\boldsymbol{X \hat{\beta}}  = \boldsymbol{AY}$.


\citet{zewotir} remarks that $\boldsymbol{D}$ is a block diagonal with the $i-$th block being $u \boldsymbol{I}$
%--------------------------------------------------------------%





\section{Zewotir: Computation and Notation } %2.3
with $\boldsymbol{V}$ unknown, a standard practice for estimating $\boldsymbol{X \beta}$ is the estime the variance components $\sigma^2_j$,
compute an estimate for $\boldsymbol{V}$ and then compute the projector matrix $A$, $\boldsymbol{X \hat{\beta}}  = \boldsymbol{AY}$.


Zewotir remarks that $\boldsymbol{D}$ is a block diagonal with the $i-$th block being $u \boldsymbol{I}$
%===============================================================================================================%


\section{Section 3}

\[  \boldsymbol{X} = \left[  \begin{array}{c} x^{\prime} \\ \boldsymbol{X} \end{array} \right]   \]

\[  \boldsymbol{Z} = \left[  \begin{array}{c} z^{\prime} \\ \boldsymbol{Z} \end{array} \right]   \] 

$\boldsymbol{A}_{(i)}$ denote an $n\times m$ matrix $\boldsymbol{A}$ with the $i-$th row removed.

%% \[ X= \left[ \begin{array} x_i \\ X(i) \end{array} \right] \]
%-------------------------------------------------------------------------------------------------------%
%Page 158 Top Half

CPJ used certain statistics as the basic building blocks of case deletion diagnostics.


%-------------------------------------------------------------------------------------------------------%
%Page 158 Lower Half

\textbf{Theorem 2 :} Basic Theorem of efficient updating (Zewotir)

%-------------------------------------------------------------------------------------------------------%
%Zewotir section 4
\section{Measures of Influence}
% 4.1 influence on variance component rations
%     4.1.1 Analogue of Cook’s Distance
%     4.1.2 Analogue of Information Ratio

%Page  161
Cook's Distance
\[  CD_{i}(\gamma) = \boldsymbol{g}^{\prime}_{(i)} ( \boldsymbol{I} + var(\hat{\gamma}) \boldsymbol{G} + \boldsymbol{g}\]

Large values of $CD$ highlights observations for special attention.

Information Ratio
\[ IR(\gamma) = det( \boldsymbol{I}_r + var(\hat{\gamma})\boldsymbol{G} \]

\begin{itemize}
\item $det(A)$ denotes the determinant of the square matrix $\boldsymbol{A}$.
\item
\end{itemize}
%-------------------------------------------------------------------------------------------------------%

4.2. Influence on fixed effects parameter estimates
\subsubsection*{4.2.1 Analogue of Cook’s Distance}
The Cook’s distance can be extended to measure influence on the fixed effects in the mixed models.
Large values of $CD_i(\beta$) indicates points for further consideration
%------------------------------------%
4.2.2. Analogue of the variance ratio

The variance ratio measures the change of the determinant of the variance of the fixed effects parameter estimates when the i-th case is deleted.

%------------------------------------%
%Page 163
\subsubsection*{4.2.3 Analogue of the Cook-Weisberg statistic}

This statistic is used to measure the change of the confidence ellipsoid value of $\beta$.
\[ \boldsymbol{y} \sim N ( boldsymbol{X}\beta , \sigma^2_epsilon boldsymbol{H})\]

The $100(1-\alpha)\%$ confidence ellipsoid for $\beta$ is....

Cook and Weisberg proposed the logarithm of the ratio $E_{(i)}$ to E as a measure of influence.


%-------------------------------------------------------------------------------------------------------%


\subsubsection*{4.2.4 Analogue of the Andrews Pregibons statistic}
This is another measure based on the volume of the confidence ellipsoid. AP$_i$
%-------------------------------------------------------------------------------------------------------%

\subsubsection*{4.3 Influence on random effects prediction.}
Analogue of Cook’s Distance
A large $CD_i(u)$ indicates that the i-th observation is influential in predicting random effects.
%-------------------------------------------------------------------------------------------------------%

\subsubsection*{4.4 Influence on the likelihood function}
Likelihood Distance ($LD_i$)
%-------------------------------------------------------------------------------------------------------%

4.5 influence on the linear functions of the fixed effect parameters.
All the diagnostics are a function of the following basic building blocks
1)  	Studentized residuals
2)  	Error contrast matrix
3)  	The inverse of the response variable covariance matrix.
The basic building blocks are computed once from the complete data set.
Zewotir assumes that D is block diagonal with the i-th block being $\gamma. I$.
%-------------------------------------------------------------------------------------------------------% 
Applications in other notepad
Studentized Residuals
\[ e^s_i = \frac{e_i}{s^2_{(i)}(1-h_i)} \]
where
$e^s_i $ - studentized residual
$s^2_{(i)}$ - standard deviation where $i$th obs is deleted
$h_i$ - leverage statistic
Belsley et al (1980) recommend the use of studentized Residuals to determine whether there is an outlier.
%-------------------------------------------------------------------------------%
Data set 1: Beckmans' aerosol data ( Zewotir)
high efficiency particulate air filter.
toxic dust, radionuclides, and mists/
Both Beckman and CPJ rank the 14th observation as the most influential.
%-------------------------------------------------------------------------------%
Data set 2: Metal oxide analysis data ( Zewotir)
Process and measurement variation on the properties of lots of metal oxides.
type 1 and 2
chemist 1 and 2
Sample 1 and 2
Maximum likelihood variance component ratios.
Full data and with cases removed.
%-------------------------------------------------------------------------------%
http://www.tandfonline.com/doi/abs/10.1080/03610910600716795
http://www.ingentaconnect.com/content/routledg/cjas/2011/00000038/00000005/art00016
http://www.sciencedirect.com/science/article/pii/S0047259X09001213



Margina Means ( population averaged)
$E[Y_{ij}]$

Conditional means ( Cluster specific)
$E[Y_{ij}|\gamma_{i}]$





\bibliography{DB-txfrbib}
\end{document}
