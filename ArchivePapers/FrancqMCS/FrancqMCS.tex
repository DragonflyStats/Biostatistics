\textbf{Measurement methods comparison with errors-in-variables regressions. From horizontal to vertical OLS regression, review and new perspectives}

%%- http://dial.uclouvain.be/pr/boreal/en/object/boreal%3A144249

\begin{itemize}
	\item 

This paper summarizes and confronts the relationships between six well-known regressions applied in the context of measurement methods comparison with or without replicated data. When two measurement methods are equivalent, it can be expected that there is no analytical bias between them and they must provide the same results on average notwithstanding the measurement errors. Usually, there is no golden-standard (like a calibration problem) and each measurement method is compared to another one and vice-versawith a linear regression analysis.
\item  If there is no bias, the results should not be far from the identity line Y=X. Unfortunately, the measurement errors in each axis must be taken into account in the regression analysis (and ideally the switching of axes) by applying a linear errors-in-variables regression. Therefore, the OLS (Ordinary Least Square) regression lines provide biased estimated lines while the errors-in-variables regressions lines lie between the two extreme OLS lines. 

\item DR (Deming Regression) and BLS (Bivariate Least Square) regressions are the most general regressions and provide (asymptotically) unbiased estimated lines which are confounded under homoscedasticity. The biases and coverage probabilities of the six regressions are compared with simulations where the results are displayed in charts instead of large data sets. As the precision ratio $\lambda$ of two measurement methods is a very important parameter in errors-in-variables regressions, it will be a leitmotiv in this paper and all the charts will be linked to it. 
\item An unbiased estimator of $\lambda$ is also given. The approximate CI (Confidence Intervals) provided by BLS are very easy to compute and provide coverage probabilities close to the nominal level. On the other hand, the CI provided by DR are less easy to compute but provide better coverage probabilities whatever $\lambda$ for large sample sizes ($N > 20$). 
\item Finally, the regression techniques are applied with real data and this paper proposes to plot all the possible values of the estimated parameters and their CI in a chartwith respect to $\lambda$. Then, the equivalence hypothesis can sometimes be rejected (or not rejected) whatever $\lambda$. This is very usefulwhen $\lambda$ is unknown and inestimable because of no replicated data.
\end{itemize}



%-----------------------------------------------------------------------------%
\newpage

How to regress and predict in a Bland-Altman plot? Review and contribution based on tolerance intervals and correlated-errors-in-variables models
%%- http://dial.uclouvain.be/pr/boreal/object/boreal:172848

\begin{itemize}
	\item Two main methodologies for assessing equivalence in method-comparison studies are presented separately in the literature. The first one is the well-known and widely applied Bland–Altman approach with its agreement intervals, where two methods are considered interchangeable if their differences are not clinically significant. 
	\item The second approach is based on \textbf{\textit{errors-in-variables regression}} in a classical (X,Y) plot and focuses on confidence intervals, whereby two methods are considered equivalent when providing similar measures notwithstanding the random measurement errors. 
	\item This paper reconciles these two methodologies and shows their similarities and differences using both real data and simulations. A new consistent correlated-errors-in-variables regression is introduced as the errors are shown to be correlated in the Bland–Altman plot. 
	\item Indeed, the coverage probabilities collapse and the biases soar when this correlation is ignored. Novel tolerance intervals are compared with agreement intervals with or without replicated data, and novel predictive intervals are introduced to predict a single measure in an (X,Y) plot or in a Bland–Atman plot with excellent coverage probabilities.
	\item We conclude that the (correlated)-errors-in-variables regressions should not be avoided in method comparison studies, although the Bland–Altman approach is usually applied to avert their complexity. We argue that tolerance or predictive intervals are better alternatives than agreement intervals, and we provide guidelines for practitioners regarding method comparison studies.
	
\end{itemize}
%-----------------------------------------------------------------------------%
\newpage
% - 2016 
% - http://dial.uclouvain.be/pr/boreal/en/object/boreal%3A144248
Hyperbolic confidence bands of errors-in-variables regression lines applied to method comparison studies

\begin{itemize}
	\item This paper focuses on the confidence bands of errors-in-variables regression lines applied to method comparison studies. When comparing two measurement methods, the goal is to ’proof’ that they are equivalent. Without analytical bias, they must provide the same results on average notwithstanding the errors of measurement. 
	
\item The results should not be far from the identity line Y = X (slope ($\beta$) equal to 1 and intercept ($\alpha$) equal to 0). A joint-CI is ideally used to test this joint hypothesis and the measurement errors in both axes must be taken into account. \textbf{DR (Deming Regression)} and \textbf{BLS (Bivariate Least Square regression)} regressions provide consistent estimated lines (confounded under homoscedasticity). 

\item Their joint-CI with a shape of ellipse in a ($\beta$;$\alpha$) plane already exist in the literature. However, this paper proposes to transform these joint-CI into hyperbolic confidence bands for the line in the (X;Y) plane which are easier to display and interpret. 

\item Four methodologies are proposed based on previous papers and their properties and advantages are discussed. The proposed confidence bands are mathematically identical to the ellipses but a detailed comparison is provided with simulations and real data.
	
\item When the error variances are known, the coverage probabilities are very close to each other but the joint-CI computed with the maximum likelihood (ML) or the method of moments provide slightly better coverage probabilities. Under unknown and heteroscedastic error variances, the ML coverage probabilities drop drastically while the BLS provide better coverage probabilities.
	
\end{itemize}

\newpage

\textbf{How to regress and predict in a Bland and Altman plot? Review and contribution based on tolerance intervals and correlated errors in variables models}
%%- http://dial.uclouvain.be/pr/boreal/en/object/boreal%3A165158

\begin{itemize}
	\item To assess equivalence in method comparison studies, two main methodologies are presented separately in the literature. First, the well-known and widely applied Bland and Altman approach with its agreement intervals where two devices are considered interchangeable if their differences are not meaningful in practice. 
	\item The second approach is based on errors-in-variables regressions in a classical (X,Y) plot and focuses on confidence intervals. Two devices are considered equivalent when providing similar measures notwithstanding the random measurement errors. \item This paper reconciles these two methodologies, shows their similarities and complementarity with real data and simulations. 
	\item A new consistent \textbf{\textit{Correlated-Errors-in-Variables (CEIV)}} regression is introduced to compare these two approaches. Indeed, the errors are shown to be correlated in a Bland and Altman plot.
	\item  When ignoring this correlation, the coverage probabilites collapse drastically and the biases soar considerably. Novel tolerance intervals are compared to agreement intervals with or without replicated data as well as novel predictive intervals to predict single measure in a (X,Y) plot or in a Bland and Atman plot with a robust estimator of the measurement differences. \item It will be then concluded that (C)EIV regressions can unfortunately not be avoided in method comparison studies, although the Bland and Altman approach is, usually, applied to avert the complexity of this statistical method. Tolerance or predictive intervals are presented in this paper as better alternatives than agreement intervals. Tips for the user are provided.
\end{itemize}

%-----------------------------------------------------------%
\newpage

\textbf{Bootstrap in Errors-in-Variables Regressions Applied to Methods Comparison Studies}
%%-http://dial.uclouvain.be/pr/boreal/en/object/boreal%3A163036

\begin{itemize}
	\item In method comparison studies, the measurements taken by two methods are compared to assess whether they are equivalent. If there is no analytical bias between the methods, they should provide the same results on average notwithstanding the measurement errors.
	\item This equivalence can be assessed with regression techniques by taking into account the measurement errors. Among them, the paper focuses on Deming Regression (DR) and bivariate least-squares regression. The confidence intervals (CI's) of the regression parameters are useful to assess the presence or absence of bias. 
	\item These CI's computed by errors-in-variables regressions are approximate (except the one for slope estimated by DR), which leads to coverage probabilities lower than the nominal value. 
	\item Six bootstrap approaches and the jackknife are assessed in the paper as means to improve the coverage probabilities of the CI's.
\end{itemize}


\newpage
%---------------------------------------------------------------------------%
\textbf{Errors-in-variables regressions to assess equivalence in method comparison studies}
%% - http://dial.uclouvain.be/pr/boreal/en/object/boreal%3A135862
\begin{itemize}
\item Analytical laboratories continuously assess the uncertainties and reliability of their measurement systems so that their clients can take the right decision. This leads to the development of new measurement methods which should be more precise, less expensive but chiefly equivalent. To assess equivalence in method comparison studies, two main methodologies are presented separately in the literature. 
\item First, the approach based on errors-in-variables regressions that focuses on confidence intervals. Two devices are considered equivalent when they provide similar measures notwithstanding the random measurement errors. In this thesis, several regressions are reviewed and compared; several new results are proposed as well as charts and tips for the user. The second methodology is the well-known Bland and Altman approach with its agreement intervals. 
\item Two devices are considered interchangeable when the differences in their measurements are not meaningful in practice. Tolerance intervals are presented as better alternatives and two new consistent regressions are proposed to compute predictive intervals. Finally, this thesis reconciles the two methodologies, shows their similarities and how they complement each other.
\end{itemize}
