\documentclass[12pt, a4paper]{article}
\usepackage{natbib}
\usepackage{vmargin}
\usepackage{graphicx}
\usepackage{epsfig}
\usepackage{subfigure}
%\usepackage{amscd}
\usepackage{amssymb}
\usepackage{subfiles}
\usepackage{subfigure}
\usepackage{framed}
\usepackage{subfiles}
\usepackage{amsbsy}
\usepackage{amsthm, amsmath}
%\usepackage[dvips]{graphicx}
\bibliographystyle{chicago}
\renewcommand{\baselinestretch}{1.1}

% left top textwidth textheight headheight % headsep footheight footskip
\setmargins{3.0cm}{2.5cm}{15.5 cm}{23.5cm}{0.25cm}{0cm}{0.5cm}{0.5cm}

\pagenumbering{arabic}
%-------------------------------------------------------------------Simplifying GLS by KH -%


\begin{document}

%\subfiles{IntroGLS}
%\subfile{AugmentedGLMs}


	
	\begin{framed} 
		\begin{itemize}
			\item \texttt{R} command and \texttt{R} object - Typewriter Font
			\item \texttt{R} Package name - Italics
			\item Selected Acronyms and Proper Nouns - Italics
		\end{itemize}
	\end{framed}
	\newpage
	
	
	This chapter is broken into two parts. The first part is a review of diagnostics methods for linear models, intended to acquaint the
	reader with the subject, and also to provide a basis for material covered in the second part. Particular attention is drawn to graphical methods.
	
	The second part of the chapter looks at diagnostics techniques for LME models, firsly covering the theory, then proceeding to a discussion on 
	implementing these using \texttt{R} code.
	While a substantial body of work has been developed in this area, ther are still area worth exploring. 
	In particular the development of graphical techniques pertinent to LME models should be looked at.
	\newpage


\subsection{Introduction}
\subsubsection{Robinson's (1991) review}
\emph{ Robinson's (1991) review of best linear unbiased prediction (BLUP), together with the subsequent discussion, has emphasized the very considerable range of models that may be addressed via the general least squares (GLS) solution to the general linear model $Y = X\beta + \varepsilon$, where $E(\varepsilon) = 0$ and $var(\varepsilon) = V$. These include linear mixed models, geostatistics, time series and multivariate regression.}


\emph{ The texts by Christensen (1996, 1991) and the connections to modern topics of image analysis, quality analysis, Bayesian methods, and splines (all in Robinson and discussion) make it an eminently suitable topic for teaching in any course concerning statistical linear models. }


\emph{Nevertheless some of the matrix algebra that results from solving the normal equations for individual specifications of the general linear model will be daunting, and far from intuitive for many students, even those who are at home in linear space. The conventional approach to prediction and estimation from data $Y$ associated with covariates X via the general linear model $Y = X\beta + \varepsilon$ is essentially a two-stage process.}

The first stage is to determine the best,in the GLS sense, estimator $\hat{\beta}$ of $\beta$ and subsequently to determine everything else from this.

The estimator is said to be best if it minimizes the generalization of the sum of squares $\hat{e}^{t}V^{-1}\hat{e}$, where $\hat{e} = Y- X\hat{\beta}$

%---------------------------------------------------%
%Simplifying GLS
\newpage

It is straightforward to show that $\hat{\beta} = (X^tV^{-l}X)^{-l}X^tV^{-l}Y = BY$ and at the minimum the sum of squares is $Y^{t} (V^{-l}  - V^{-l}(X^tV^{-l}X)^{-l}X^tV^{-l})Y = Y^{t}QY$.\\
\bigskip

\emph{The purpose of this note is to give emphasis to one derivation, based on Lagrange multipliers, which leads to a system of equations that is very intuitive and lends itself readily to specialization. This approach is in fact standard in the geostatistical treatment of \textbf{kriging} (see Matheron 1962; Journel and Huijbregts 1981; Ripley 1981; Cressie 1993). In the genetics literature it is associated with the name of Henderson (1983); or in the classical statistical literature Hocking (1996, p. 73) is a suitable reference.}

\emph{The approach based on Lagrange multipliers deemphasizes the explicit determination of $\hat{\beta}$ and leads to a clearer understanding of the complementary (but for some confusing) tasks known as best linear unbiased estimation (BLUE) and best linear unbiased prediction (BLUP). Regrettably, Robinson-despite offering four derivations, and having as his main concern the interplay of BLUP and BLUE-gives it little prominence.}

It has recently been discussed by Searle (1997, p; 278) who said that it makes another approach (Searle, Casella, and McCulloch 1992, p. 271) seem "obtuse and unnecessarily complicated." By contrast, our treatment emphasizes the fact that it leads to a single set of equations whose solution sheds simplifying light on very many issues in general least squares.

The American Statistician's Teacher's Corner (e.g., McLean, Sanders, and Stroup 1991; Puntanen and Styan 1989) has already played host to previous attempts to simplify the explanation of such topics. Various authors (CPJ, Haslett Hayes ,Martin ) have visited the more specialized area of diagnostics and have developed \textbf{\emph{down-dating}} (leave-$k$-out) formulas.

The conventional approach here is via tricky identities based on the inverses of partitioned matrices. Here again the Lagrange system of equations leads to a much simplified and-we claim-much more intuitive derivation of these more technical results.


\emph{
The essence of the approach is to seek that linear combination of the available data Y which is best for the
estimation of Z among those linear estimators which are constrained to be unbiased. We adopt therefore a constrained minimization approach, using Lagrange multipliers. By best we mean that combination $\hat{Z}(Y) = \lambda_{z}^{t}Y$ which has least mean square error $E( Z- \lambda_{z}^{t}Y)^2$, and by unbiased we mean $E( Z- \lambda_{z}^{t}Y)) = 0$. }
Here $Z$ denotes that scalar which is to be the objective of the estimation. This estimator is written as $\hat{Z}(Y)$ to make its dependence on $Y$ explicit. Note that the term "best" is applied in the context of minimizing the prediction variance $var(Z - Z(Y))$. We shall see that Z may be used to denote either a random variable or an unknown parameter, and that it will be sufficient to specify Z via $E[Z]$ and $cov(Z, Y)$. If $Z$ is not a random variable then of course the latter is zero and $E[Z] = Z$. We establish-very simply, as below-a general solution in terms of A and cov(Z, Y) and achieve particular tasks by identification of these. Our presentation is for a scalar Z, but the notation facilitates generalization to vector Z.


%--------------------------------------------------------------------%

\subsection{Predictors and Estimators}

\emph{We note that Robinson (1991) stated "A convention has somehow developed that estimators of random effects are called predictors while estimators of fixed effects are called estimators." We agree that this distinction is confusing and indeed unnecessary.} \\ \bigskip



We seek $\hat{Z}(Y) = \lambda_{z}^{t}Y$, where $ \lambda_{z}^{t}$, is an $n \times 1$ vector of estimation coefficients. It is convenient to specify $E[Z]=A\beta$ for known $A$. In this context $A$ denotes a row vector, but we generalize this in the following. The constraint requiring $\hat{Z}(Y)$ to be unbiased now reduces to $(A -  \lambda_{z}^{t}X) = 0$. A solution is found by minimizing $var(Z -  \lambda_{z}^{t}Y) + \gamma^t_z (X^t\lambda_{z} - A^t)$, where $\gamma_z$ is a $p \times 1$ vector of Lagrange multipliers, where $p$ is the length of the parameter vector $\beta$. Setting to zero the derivatives with respect to $\lambda_{z}$ and $\gamma_z $ yields the system.




\begin{equation}
\left(
  \begin{array}{cc}
    V & X \\
    X^t & 0 \\
  \end{array}
\right)\left(
  \begin{array}{c}
    \lambda_{z}\\
   \gamma_z \\
  \end{array}
\right)=\left(
  \begin{array}{c}
    \mbox{cov}(Y,Z)\\
   A^{t} \\
  \end{array}
\right)
\end{equation}


If the inverse exists we have that
\begin{equation}
\left(
  \begin{array}{c}
    \lambda_{z}\\
   \gamma_z \\
  \end{array}
\right)=\left(
  \begin{array}{cc}
    V & X \\
    X^t & 0 \\
  \end{array}
\right) ^{-1}\left(
  \begin{array}{c}
    \mbox{cov}(Y,Z)\\
   A^{t} \\
  \end{array}
\right)
\end{equation}



so that
\[ \hat{Z}(Y) =
\left(
  \begin{array}{cc}
    \lambda_{z}^{t}&
   \gamma_z^{t} \\
  \end{array}
\right)=\left(
  \begin{array}{c}
    Y \\
    0 \\
  \end{array}
\right) \]

In terms of the estimation problem being considered the square matrix on the left-hand side of (1) concerns "what we have," namely, the data plus constraints.

The matrix does not depend on Z and consequently need only be constructed once before application to a range of problems. The right- hand side contains the term $cov(Z,Y)$ and can be specified for whatever Z is being considered.

It is this feature of system (1) that makes a generic approach to estimation possible.

\chapter{Model Diagnostics}
%---------------------------------------------------------------------------%
%1.1 Introduction to Influence Analysis
%1.2 Extension of techniques to LME Models
%1.3 Residual Diagnostics
%1.4 Standardized and studentized residuals
%1.5 Covariance Parameters
%1.6 Case Deletion Diagnostics
%1.7 Influence Analysis
%1.8 Terminology for Case Deletion
%1.9 Cook's Distance (Classical Case)
%1.10 Cook's Distance (LME Case)
%1.11 Likelihood Distance
%1.12 Other Measures
%1.13 CPJ Paper
%1.14 Matrix Notation of Case Deletion
%1.15 CPJ's Three Propositions
%1.16 Other measures of Influence
\tableofcontents
%===========================================================================%
\newpage

\subsection*{Abstract}
This chapter is broken into two parts. The first part is a review of diagnostics methods for linear models, intended to acquaint the reader with the subject, and also to provide a basis for material covered in the second part. Particular attention is drawn to graphical methods.


\section{Framework for Model Validation using Residual Diagnostics}
In statistical modelling, the process of model validation is a critical step, but also a step that is too often overlooked. A very simple procedure is to examine commonly encountered
metrics, such as the $R^2$ value. However, using a small handful of simple measures and methods is insufficient to properly assess the quality of a fitted model. To do so properly, a full and comprehensive
analysis that tests of all of the assumptions, as far as possible, must be carried out. A statistical model, whether of the fixed-effects or mixed-effects variety, represents how you think your data
were generated. Following model specification and estimation, it is of interest to explore the model-data
agreement by raising questions such as
\begin{itemize}
	\item Does the model-data agreement support the model assumptions?
	\item Should model components be refined, and if so, which components? For example, should regressors
	be added or removed, and is the covariation of the observations modeled properly?
	\item Are the results sensitive to model and/or data? Are individual data points or groups of cases particularly
	influential on the analysis?
\end{itemize}


\subsection{Residual Analysis}
A residual is the difference between an observed quantity and its
estimated or predicted value. 
Residual analysis is a widely used model validation technique. A residual is simply the difference between an observed value and the corresponding fitted value, as predicted by the model. The rationale is that, if the model is properly fitted to the model, then the residuals would approximate the random errors that one should expect.
that is to say, if the residuals behave randomly, with no discernible trend, the model has fitted the data well. If some sort of non-random trend is evident in the model, then the model can be considered to be poorly fitted.
Statistical software environments, such as the \texttt{R} Programming language, provides a suite of tests and graphical procedure sfor appraising a fitted linear model, with several 
of these procedures analysing the model residuals.

In classical linear models, an examination of model-data agreement has traditionally revolved around

The second part of the chapter looks at diagnostics techniques for LME models, firsly covering the theory, then proceeding to a discussion on 
implementing these using \texttt{R} code.

While a substantial body of work has been developed in this area, there is still areas worth exploring. 
In particular the development of graphical techniques pertinent to LME models should be looked at.




%\section{Introduction (Page 1)}
%
%Linear models for uncorrelated data have well established measures to gauge the influence of one or more
%observations on the analysis. For such models, closed-form update expressions allow efficient computations
%without refitting the model. 
%
%
%When similar notions of statistical influence are applied to mixed models,
%things are more complicated. Removing data points affects fixed effects and covariance parameter estimates.
%Update formulas for “\textit{leave-one-out}” estimates typically fail to account for changes in covariance
%parameters. 
%
%Moreover, in repeated measures or longitudinal studies, one is often interested in multivariate
%influence, rather than the impact of isolated points. 

% This paper examines extensions of influence measures
% in linear mixed models and their implementation in the MIXED procedure.









\newpage
%=========================================================================%
\section{Model Validation Framework}
%\section{Model Validation using Residual Diagnostics}
In statistical modelling, the process of model validation is a critical step of model fitting process, but also a step that is too often overlooked. A very simple procedure is to examine commonly-used
metrics, such as the $R^2$ value. However, using a small handful of simple measures and methods is insufficient to properly assess the quality of a fitted model. To do so properly, a full and comprehensive
analysis that tests of all of the assumptions, as far as possible, must be carried out.

%=========================================================================%
%\subsection{Model Validation Framework}
%In classical linear models, this examination of model-data agreement has traditionally revolved around
\citet{schab} describes the model validatin framework as comprised of the following tasks
>>>>>>> origin/master
\begin{itemize}
	\item  overall measures of goodness-of-fit
	\item the informal, graphical examination of estimates of model errors to assess the quality of distributional
	assumptions: residual analysis
	
	
	\item the quantitative assessment of the inter-relationship of model components; for example, collinearity 	diagnostics
	\item the qualitative and quantitative assessment of influence of cases on the analysis, i.e. influence analysis.
\end{itemize}
<<<<<<< HEAD
The sensitivity of a model is studied through measures that express its stability under perturbations. You
are not interested in a model that is either overly stable or overly sensitive. Changes in the data or model
components should produce commensurate changes in the model output. The difficulty is to determine
when the changes are substantive enough to warrant further investigation, possibly leading to a reformulation
of the model or changes in the data (such as dropping outliers). This paper is primarily concerned
with stability of linear mixed models to perturbations of the data; that is, with influence analysis. 
=======
%The sensitivity of a model is studied through measures that express its stability under perturbations. You
%are not interested in a model that is either overly stable or overly sensitive. Changes in the data or model
%components should produce commensurate changes in the model output. The difficulty is to determine
%when the changes are substantive enough to warrant further investigation, possibly leading to a reformulation
%of the model or changes in the data (such as dropping outliers).

% This paper is primarily concerned with stability of linear mixed models to perturbations of the data; that is, with influence analysis.
%========================================================================================================= %
%\subsection{Residual}



%========================================================================================================= %
%\subsection{Residual Analysis}

Residual analysis is a widely used model validation technique. A residual is simply the difference between an observed value and the corresponding fitted value, as predicted by the model. The rationale is that, if the model is properly fitted to the model, then the residuals would approximate the random errors that one should expect.
that is to say, if the residuals behave randomly, with no discernible trend, the model has fitted the data well. If some sort of non-random trend is evident in the model, then the model can be considered to be poorly fitted.

%========================================================================================================= %
%\subsection{Introduction}
%A statistical model, whether of the fixed-effects or mixed-effects variety, represents how you think your data were generated. 
%Following model specification and estimation, it is of interest to explore the model-data
%agreement by raising questions such as

Statistical software environments, such as the \texttt{R} Programming language, provides a suite of tests and graphical procedure sfor appraising a fitted linear model, with several 
of these procedures analysing the model residuals.





%========================================================================================================= %
\subsection{Outliers and Leverage}



The question of whether or not a point should be considered an outlier must also be addressed. An outlier is an observation whose true value is unusual given its value on the predictor variables. The leverage of an observation is a further consideration. Leverage describes an observation with an extreme value on a predictor variable is a point with high leverage. High leverage points can have a great amount of effect on the estimate of regression coefficients.
% - Leverage is a measure of how far an independent variable deviates from its mean.

Influence can be thought of as the product of leverage and outlierness. An observation is said to be influential if removing the observation substantially changes the estimate of the regression coefficients. The \texttt{R} programming language has a variety of methods used to study each of the aspects for a linear model. While linear models and GLMS can be studied with a wide range of well-established diagnostic technqiues, the choice of methodology is much more restricted for the case of LMEs.

%---------------------------------------------------------------------------%
%\newpage
%\section{Residual diagnostics} %1.3
For classical linear models, residual diagnostics are typically conducted using a plot of the observed residuals and the predicted values. A visual inspection for the presence of trends inform the analyst on the validity of distributional assumptions, and to detect outliers and influential observations.

%\section{Case Deletion Diagnostics}
%
%
%Linear models for uncorrelated data have well established measures to gauge the influence of one or more
%observations on the analysis. For such models, closed-form update expressions allow efficient computations
%without refitting the model. 
%
%
%Since the pioneering work of Cook in 1977, deletion measures have been applied to many statistical models for identifying influential observations. Case-deletion diagnostics provide a useful tool for identifying influential observations and outliers.
%
%The key to making deletion diagnostics useable is the development of efficient computational formulas, allowing one to obtain the \index{case deletion diagnostics} case deletion diagnostics by making use of basic building blocks, computed only once for the full model.
%
%The computation of case deletion diagnostics in the classical model is made simple by the fact that estimates of $\beta$ and $\sigma^2$, which exclude the $i-$th observation, can be computed without re-fitting the model. %\subsection{Terminology for Case Deletion diagnostics} %1.8
%
%\citet{preisser} describes two type of diagnostics. When the set consists of only one observation, the type is called
%`\textit{observation-diagnostics}'. For multiple observations, Preisser describes the diagnostics as `\textit{cluster-deletion}' diagnostics. When applied to LME models, such update formulas are available only if one assumes that the covariance parameters are not affected by the removal of the observation in question. However, this is rarely a reasonable assumption.
%
%
%
%
%%---------------------------------------------------------------------------%
\subsection{Matrix Notation for Case Deletion} %1.14

%\subsection{Case deletion notation} %1.14.1

For notational simplicity, $\boldsymbol{A}(i)$ denotes an $n \times m$ matrix $\boldsymbol{A}$ with the $i$-th row
removed, $a_i$ denotes the $i$-th row of $\boldsymbol{A}$, and $a_{ij}$ denotes the $(i, j)-$th element of $\boldsymbol{A}$.
%
%\subsection{Partitioning Matrices} %1.14.2
%Without loss of generality, matrices can be partitioned as if the $i-$th omitted observation is the first row; i.e. $i=1$.



%-------------------------------------------------------------------------------------------------------------------------------------%
%--------------------------------------%
\subsection{Extension of Diagnostic Methods to LME models}

<<<<<<< HEAD

When similar notions of statistical influence are applied to mixed models,
things are more complicated. Removing data points affects fixed effects and covariance parameter estimates.
Update formulas for “\textit{leave-one-out}” estimates typically fail to account for changes in covariance
parameters. 
%
%
%In LME models, there are two types of residuals, marginal residuals and conditional residuals. A
%marginal residual is the difference between the observed data and the estimated marginal mean. A conditional residual is the
%difference between the observed data and the predicted value of the observation. In a model without random effects, both sets of residuals coincide \citep{schab}.

\citet{Christiansen} noted the case deletion diagnostics techniques have not been applied to linear mixed effects models and seeks to develop methodologies in that respect. \citet{Christiansen} develops these techniques in the context of REML.
=======
\citet{CPJ} noted the case deletion diagnostics techniques had not been applied to linear mixed effects models and seeks to develop methodologies in that respect. \citet{CPJ} develops these techniques in the context of REML.
>>>>>>> origin/master

%\citet{CPJ} develops \index{case deletion diagnostics} case deletion diagnostics, in particular the equivalent of \index{Cook's distance} Cook's distance, a well-known metric, for diagnosing influential observations when estimating the fixed effect parameters and variance components. Deletion diagnostics provide a means of assessing the influence of an observation (or groups of observations) on inference on the estimated parameters of LME models. We shall provide a fuller discussion of Cook's distance in due course.


\citet{Demi} extends several regression diagnostic techniques commonly used in linear regression, such as leverage, infinitesimal influence, case deletion diagnostics, Cook's distance, and local influence to the linear mixed-effects model. In each case, the proposed new measure has a direct interpretation in terms of the effects on a parameter of interest, and reduces to the familiar linear regression measure when there are no random effects. 

The new measures that are proposed by \citet{Demi} are explicitly defined functions and do not require re-estimation of the model, especially for cluster deletion diagnostics. The basis for both the cluster deletion diagnostics and Cook's distance is a generalization of Miller's simple update formula for case deletion for linear models. Furthermore \citet{Demi} shows how Pregibon's infinitesimal case deletion diagnostics is adapted to the linear mixed-effects model. 
%A simple compact matrix formula is derived to assess the local influence of the fixed-effects regression coefficients. 


%
%
%\section{Case Deletion Diagnostics for LME models} %1.6
%
%Data from single individuals, or a small group of subjects may influence non-linear mixed effects model selection. Diagnostics routinely applied in model building may identify such individuals, but these methods are not specifically designed for that purpose and are, therefore, not optimal. 

\citet{Demi} proposes two likelihood-based diagnostics for identifying individuals that can influence the choice between two competing models.


\newpage




\section{Analysis of  Influence}


>>>>>>> origin/master

%%\subfile{InfluenceforLMEs.tex}
% \subfile{ApplicationsToMCS.tex}  - Not Ready
% \subfile{SideNotes.tex} - Pregibon etc
% \subfile{HaslettHayes.tex} - Build this up 
%%\subfile{LikelihoodDistances.tex}
%%<<<<<<< HEAD


%
%=======
%\subfile{Influence}
%\subfile{ResidualsLMEs.tex}
%\subfile{iterativemethods.tex}




%-------------------------------------------------------------------------------------------------Chapter 3------------------------%



% --- \subsection{Importance-Weighted Least-Squares (IWLS)}  %3.3
% ---   \subsection{H-Likelihood}

%\chapter{Model Diagnostics}
%---------------------------------------------------------------------------%
%1.1 Introduction to Influence Analysis
%1.2 Extension of techniques to LME Models
%1.3 Residual Diagnostics
%1.4 Standardized and studentized residuals
%1.5 Covariance Parameters
%1.6 Case Deletion Diagnostics
%1.7 Influence Analysis
%1.8 Terminology for Case Deletion
%1.9 Cook's Distance (Classical Case)
%1.10 Cook's Distance (LME Case)
%1.11 Likelihood Distance
%1.12 Other Measures
%1.13 CPJ Paper
%1.14 Matrix Notation of Case Deletion
%1.15 CPJ's Three Propositions
%1.16 Other measures of Influence

%--------------------------------------%

\subsection{Further Assumptions of Linear Models}

As with fitted models, the assumption of normality of residuals and homogeneity of variance is applicable to LMEs also. 

%--------------------------------------%


Homoscedascity is the technical term to describe the variance of the
residuals being constant across the range of predicted values.
Heteroscedascity is the converse scenario : the variance differs along
the range of values.

%--Marginal and Conditional Residuals

% \subfile{ResidualsLMEs.tex}
% \subfile{iterativemethods.tex}



In recent years, mixed models have become invaluable tools in the analysis of experimental and observational
data. In these models, more than one term can be subject to random variation. Mixed model
technology enables you to analyze complex experimental data with hierarchical random processes, temporal,
longitudinal, and spatial data, to name just a few important applications. 

\subsection{Stating the LME Model}
The general linear mixed
model is
\[
Y = X\beta + Zu + \varepsilon\]
where Y is a $(n\times1)$ vector of observed data, X is an $(n\times p)$ fixed-effects design or regressor matrix of rank
k, Z is a $(n \times g)$ random-effects design or regressor matrix, $u$ is a $(g \times 1)$ vector of random effects, and $\varepsilon$ is
an $(n\times1)$ vector of model errors (also random effects). The distributional assumptions made by the MIXED
procedure are as follows: γ is normal with mean 0 and variance G; $\varepsilon$ is normal with mean 0 and variance
R; the random components $u$ and $\varepsilon$ are independent. Parameters of this model are the fixed-effects β and
all unknowns in the variance matrices G and R. The unknown variance elements are referred to as the
covariance parameters and collected in the vector $theta$.
%===========================================================================%

The concept of critiquing the model-data agreement applies in mixed models in the same way as in linear
fixed-effects models. In fact, because of the more complex model structure, you can argue that model and
data diagnostics are even more important. For example, you are not only concerned with capturing the
important variables in the model. You are also concerned with “distributing” them correctly between the
fixed and random components of the model. The mixed model structure presents unique and interesting
challenges that prompt us to reexamine the traditional ideas of influence and residual analysis.
%==========================================================================%
This paper presents the extension of traditional tools and statistical measures for influence and residual
analysis to the linear mixed model and demonstrates their implementation in the MIXED procedure (experimental
features in SAS 9.1). The remainder of this paper is organized as follows. The “Background” section
briefly discusses some mixed model estimation theory and the challenges to model diagnosis that result
from it.

%	 The diagnostics implemented in the MIXED procedure are discussed in the “Residual Diagnostics
%	in the MIXED Procedure” section (page 3) and the “Influence Diagnostics in the MIXED Procedure” section
%	(page 5). The syntax options and suboptions you use to request the various diagnostics are briefly sketched
%	in the “Syntax” section (page 9). The presentation concludes with an example.
%	
%	
%====================================================================================================================%
\subsection{Summary of Schabenberger's Paper}
=======
\newpage
%\subsection{INFLUENCE DIAGNOSTICS IN THE MIXED PROCEDURE}
%Key to the implementations of influence diagnostics in the MIXED procedure is the attempt to quantify
%influence, where possible, by drawing on the basic definitions of the various statistics in the classical linear
%model. 

On occasion, quantification is not possible. Assume, for example, that a data point is removed
and the new estimate of the G matrix is not positive definite. This may occur if a variance component
estimate now falls on the boundary of the parameter space. Thus, it may not be possible to compute certain
influence statistics comparing the full-data and reduced-data parameter estimates. However, knowing that
a new singularity was encountered is important qualitative information about the data point’s influence on
the analysis.

The basic procedure for quantifying influence is simple:

\begin{enumerate}
	\item Fit the model to the data and obtain estimates of all parameters.
	\item Remove one or more data points from the analysis and compute updated estimates of model parameters.
	\item Based on full- and reduced-data estimates, contrast quantities of interest to determine how the absence
	of the observations changes the analysis.
\end{enumerate}
We use the subscript (U) to denote quantities obtained without the observations in the set U. For example,
%βb
(U) denotes the fixed-effects “\textit{\textbf{leave-U-out}}” estimates. Note that the set U can contain multiple observations.


%===================================================================================
If the global measure suggests that the points in U are influential, you should next determine the nature of
that influence. In particular, the points can affect
\begin{itemize}
	\item the estimates of fixed effects
	\item the estimates of the precision of the fixed effects
	\item the estimates of the covariance parameters
	\item the estimates of the precision of the covariance parameters
	\item fitted and predicted values
\end{itemize}

It is important to further decompose the initial finding to determine whether data points are actually troublesome.
Simply because they are influential “somehow”, should not trigger their removal from the analysis or
a change in the model. For example, if points primarily affect the precision of the covariance parameters
without exerting much influence on the fixed effects, then their presence in the data may not distort hypothesis
tests or confidence intervals about $\beta$.
%They will only do so if your inference depends on an estimate of the
%precision of the covariance parameter estimates, as is the case for the Satterthwaite and Kenward-Roger
%degrees of freedom methods and the standard error adjustment associated with the DDFM=KR option.

%------------------------------------------------------------%
\subsection{Summary of Paper}
>>>>>>> origin/master
%Summary of Schabenberger
Standard residual and influence diagnostics for linear models can be extended to LME models.
The dependence of the fixed effects solutions on the covariance parameters has important ramifications on the perturbation analysis.	
Calculating the studentized residuals-And influence statistics whereas each software procedure can calculate both conditional and marginal raw residuals, only SAs Proc Mixed is currently the only program that provide studentized residuals Which ave preferred for model diagnostics. The conditional Raw residuals ave not well suited to detecting outliers as are the studentized conditional residuals. (schabenbege r)


LME are flexible tools for the analysis of clustered and repeated measurement data. LME extend the capabilities of standard linear models by allowing unbalanced and missing data, as long as the missing data are MAR. Structured covariance matrices for both the random effects G and the residuals R. missing at Random.

A conditional residual is the difference between the observed valve and the predicted valve of a dependent variable- Influence diagnostics are formal techniques that allow the identification observation that heavily influence estimates of parameters.
To alleviate the problems with the interpretation of conditional residuals that may have unequal variances, we consider sealing.
Residuals obtained in this manner ave called studentized residuals.

\begin{itemize}
	\item Standard residual and influence diagnostics for linear models can be extended to linear mixed models. The dependence of fixed-effects solutions on the covariance parameter estimates has important ramifications in perturbation analysis. 
	\item To gauge the full impact of a set of observations on the analysis, covariance parameters need to be updated, which requires refitting of the model. 
	%	\item The experimental INFLUENCE option of the MODEL statement in the MIXED procedure (SAS 9.1) enables you to perform iterative and noniterative influence analysis for individual observations and sets of observations.
	
	\item The conditional (subject-specific) and marginal (population-averaged) formulations in the linear mixed model enable you to consider conditional residuals that use the estimated BLUPs of the random effects, and marginal residuals which are deviations from the overall mean. 
	\item Residuals using the BLUPs are useful to diagnose whether the random effects components in the model are specified correctly, marginal residuals are useful to diagnose the fixed-effects components. 
	\item Both types of residuals are available in SAS 9.1 as an experimental option of the MODEL statement in the MIXED procedure.
	
	\item It is important to note that influence analyses are performed under the assumption that the chosen model is correct. Changing the model structure can alter the conclusions. Many other variance models have been fit to the data presented in the repeated measures example. You need to see the conclusions about which model component is affected in light of the model being fit.
	%	\item  For example, modeling these data with a random intercept and random slope for each child or an unstructured covariance matrix will affect your conclusions about which children are influential on the analysis and how this influence manifests itself.
\end{itemize}



	
	\section{Case Deletion Diagnostics for LME models}
	
	%%% Haslett \& Dillane (19XX) }
	
	Haslett \& Dillane (19XX) remark that linear mixed effects models
	didn't experience a corresponding growth in the use of deletion
	diagnostics, adding that \citet{McCullSearle} makes no mention of
	diagnostics whatsoever.
	
	%%%\citet{christensen}
	
	Christensen (19XX)  describes three propositions that are required
	for efficient case-deletion in LME models. The first proposition
	decribes how to efficiently update $V$ when the $i$th element is
	deleted.
	\begin{equation}
	V_{[i]}^{-1} = \Lambda_{[i]} - \frac{\lambda
		\lambda\prime}{\nu^{}ii}
	\end{equation}
	
	
	The second of Christensen's propostions is the following set of
	equations, which are variants of the Sherman Wood bury updating
	formula.
	\begin{eqnarray}
	X'_{[i]}V_{[i]}^{-1}X_{[i]} &=& X' V^{-1}X -
	\frac{\hat{x}_{i}\hat{x}'_{i}}{s_{i}}\\
	(X'_{[i]}V_{[i]}^{-1}X_{[i]})^{-1} &=& (X' V^{-1}X)^{-1} +
	\frac{(X' V^{-1}X)^{-1}\hat{x}_{i}\hat{x}' _{i}
		(X' V^{-1}X)^{-1}}{s_{i}- \bar{h}_{i}}\\
	X'_{[i]}V_{[i]}^{-1}Y_{[i]} &=& X\prime V^{-1}Y -
	\frac{\hat{x}_{i}\hat{y}' _{i}}{s_{i}}
	\end{eqnarray}
	
	
	
	
	
	
	
	
	In LME models, fitted by either ML or REML, an important overall
	influence measure is the likelihood distance \citep{cook82}. The
	procedure requires the calculation of the full data estimates
	$\hat{\psi}$ and estimates based on the reduced data set
	$\hat{\psi}_{(U)}$. The likelihood distance is given by
	determining
	
	
	\begin{eqnarray}
	LD_{(U)} &=& 2\{l(\hat{\psi}) - l( \hat{\psi}_{(U)}) \}\\
	RLD_{(U)} &=& 2\{l_{R}(\hat{\psi}) - l_{R}(\hat{\psi}_{(U)})\}
	\end{eqnarray}
	
	
	% Haslett Dillane
	%==================================================================%
	Haslett \& Dillane (199X) offers an
	procedure to assess the influences for the variance components
	within the linear model, complementing the existing methods for
	the fixed components. 
	
	
	The essential problem is that there is no
	useful updating procedures for $\hat{V}$, or for $\hat{V}^{-1}$.
	Haslett \& Dillane (199X) propose an alternative , and
	computationally inexpensive approach, making use of the
	`\texttt{delete=replace}' identity.
	
	\citet{Haslett99} considers the effect of `leave k out'
	calculations on the parameters $\beta$ and $\sigma^{2}$, using
	several key results from \citet{HaslettHayes} on partioned
	matrices.
	
	% - Haslett \& Dillane (199X)  - Haslett \& Dillane (19XX) }
	% - Haslett (1999) - \citet{Haslett99}
	
	
	%---------------------------------------------------------------------------%
	\newpage
	\section{Haslett's Analysis} %2.5
	For fixed effect linear models with correlated error structure Haslett (1999) showed that the effects on
	the fixed effects estimate of deleting each observation in turn could be cheaply computed from the fixed effects model predicted residuals.

\newpage

\newpage
\section{Haslett's Analysis} %2.5
For fixed effect linear models with correlated error structure Haslett (1999) showed that the effects on
the fixed effects estimate of deleting each observation in turn could be cheaply computed from the fixed effects model predicted residuals.


A general theory is presented for residuals from the general linear model with correlated errors.
It is demonstrated that there are two fundamental types of residual associated with this model,
referred to here as the marginal and the conditional residual.


These measure respectively the distance to the global aspects of the model as represented by the expected value
and the local aspects as represented by the conditional expected value.


These residuals may be multivariate.


\citet{HaslettHayes} developes some important dualities which have simple implications for diagnostics.


%The results are illustrated by reference to model diagnostics in time series and in classical multivariate analysis with independent cases.
%============================================================================================== %
\newpage

\section{Haslett's Analysis} %2.5
For fixed effect linear models with correlated error structure Haslett (1999) showed that the effects on
the fixed effects estimate of deleting each observation in turn could be cheaply computed from the fixed effects model predicted residuals.


A general theory is presented for residuals from the general linear model with correlated errors. 
It is demonstrated that there are two fundamental types of residual associated with this model, 
referred to here as the marginal and the conditional residual. 

These measure respectively the distance to the global aspects of the model as represented by the expected value 
and the local aspects as represented by the conditional expected value. 

These residuals may be multivariate. 

\citet{HaslettHayes} developes some important dualities which have simple implications for diagnostics. 

%The results are illustrated by reference to model diagnostics in time series and in classical multivariate analysis with independent cases.
%------------------------------------------------------------%
\section*{Haslett and Hayes - Residuals}
Haslett and Hayes (1998) and Haslett (1999) considered the case of an LME model with correlated covariance structure.

\subsection{Residual Diagnostics in LME models}
\begin{itemize}
	\item A \textbf{residual} is the difference between the observed quantity and the predicted value. In LME models a distinction is made between marginal residuals and conditional residuals.
	
	\item A \textbf{Marginal residual} is the difference between the observed data and the estimated marginal mean (Schabenberger  pg3)
	The computation of case deletion diagnostics in the classical model is made simple by the fact that important estimates can be computed without refitting the model. 
	
	\item Such update formulae are available in the mixed model only if you assume that the covariance parameters are not affect by the removal of the observation in question. Schabenberger remarks that this is not a reasonable assumption.
	
\end{itemize}


Basic procedure for quantifying influence is simple

\begin{enumerate}
	\item  	Fit the model to the data
	\item   	Remove one or more data points from the analysis and compute updated estimates of model parameters
	\item  	Based on the full and reduced data estimates, contrast quantities of interest to determine how the absence of the observations changed the analysis.
\end{enumerate}
The likelihood distance is a global summary measure expressing the joint influence of the observations in the set U on all parameters in $\Psi$ that were subject to updating.


\section{Case Deletion Diagnostics for LME models}

\citet{HaslettDillane} remark that linear mixed effects models
didn't experience a corresponding growth in the use of deletion
diagnostics, adding that \citet{McCullSearle} makes no mention of
diagnostics whatsoever.

\citet{Christensen} describes three propositions that are required
for efficient case-deletion in LME models. The first proposition
decribes how to efficiently update $V$ when the $i$th element is
deleted.
\begin{equation}
V_{[i]}^{-1} = \Lambda_{[i]} - \frac{\lambda
	\lambda\prime}{\nu^{}ii}
\end{equation}


The second of christensen's propostions is the following set of
equations, which are variants of the Sherman Wood bury updating
formula.
\begin{eqnarray}
X'_{[i]}V_{[i]}^{-1}X_{[i]} &=& X' V^{-1}X -
\frac{\hat{x}_{i}\hat{x}'_{i}}{s_{i}}\\
(X'_{[i]}V_{[i]}^{-1}X_{[i]})^{-1} &=& (X' V^{-1}X)^{-1} +
\frac{(X' V^{-1}X)^{-1}\hat{x}_{i}\hat{x}' _{i}
	(X' V^{-1}X)^{-1}}{s_{i}- \bar{h}_{i}}\\
X'_{[i]}V_{[i]}^{-1}Y_{[i]} &=& X\prime V^{-1}Y -
\frac{\hat{x}_{i}\hat{y}' _{i}}{s_{i}}
\end{eqnarray}








In LME models, fitted by either ML or REML, an important overall
influence measure is the likelihood distance \citep{cook82}. The
procedure requires the calculation of the full data estimates
$\hat{\psi}$ and estimates based on the reduced data set
$\hat{\psi}_{(U)}$. The likelihood distance is given by
determining


\begin{eqnarray}
LD_{(U)} &=& 2\{l(\hat{\psi}) - l( \hat{\psi}_{(U)}) \}\\
RLD_{(U)} &=& 2\{l_{R}(\hat{\psi}) - l_{R}(\hat{\psi}_{(U)})\}
\end{eqnarray}
%
%
%\addcontentsline{toc}{section}{Bibliography}
%
%\bibliography{transferbib}
%\end{document}
Let $y_{mir} $ be the $r$th replicate measurement on the $i$th item by the $m$th method, where $m=1,2,$ $i=1,\ldots,N,$ and $r = 1,\ldots,n_i.$ When the design is balanced and there is no ambiguity we can set $n_i=n.$ The LME model underpinning Roy's approach can be written
\begin{equation}
y_{mir} = \beta_{0} + \beta_{m} + b_{mi} + \epsilon_{mir}.
\end{equation}
Here $\beta_0$ and $\beta_m$ are fixed-effect terms representing, respectively, a model intercept and an overall effect for method $m.$
The $\beta$ terms can be gathered together into (fixed effect) intercept terms $\alpha_m=\beta_0+\beta_m.$ The $b_{1i}$ and $b_{2i}$ terms are correlated random effect parameters having $\mathrm{E}(b_{mi})=0$ with $\mathrm{Var}(b_{mi})=g^2_m$ and $\mathrm{Cov}(b_{mi}, b_{m^\prime i})=g_{12}.$ The random error term for each response is denoted $\epsilon_{mir}$ having $\mathrm{E}(\epsilon_{mir})=0$, $\mathrm{Var}(\epsilon_{mir})=\sigma^2_m$, $\mathrm{Cov}(b_{mir}, b_{m^\prime ir})=\sigma_{12}$, $\mathrm{Cov}(\epsilon_{mir}, \epsilon_{mir^\prime})= 0$ and $\mathrm{Cov}(\epsilon_{mir}, \epsilon_{m^\prime ir^\prime})= 0.$ Two methods of measurement are in complete agreement if the null hypotheses $\mathrm{H}_1\colon \beta_1 = \beta_2$ and $\mathrm{H}_2\colon \sigma^2_1 = \sigma^2_2 $ and $\mathrm{H}_3\colon g^2_1= g^2_2$ hold simultaneously. \citet{roy} proposes a Bonferroni correction to control the familywise error rate for tests of $\{\mathrm{H}_1, \mathrm{H}_2, \mathrm{H}_3\}$ and account for difficulties arising due to multiple testing. Let $\omega^2_m = \sigma^2_m + g^2_m$ represent the overall variability of method $m.$  Roy also integrates $\mathrm{H}_2$ and $\mathrm{H}_3$ into a single testable hypothesis $\mathrm{H}_4\colon \omega^2_1=\omega^2_2.$ CONCERNS?

\bigskip

% Complete paragraph by specifying variances and covariances for epsilons.
% I thing that these are your sigmas?
% Also, state equality of the parameters in this model when each of the three hypotheses above are true.
\citet{Roy} demonstrates how to implement a method comparison study further to model (1) using the SAS proc mixed package.
%------------------------------------------------------------------------------------------------%
\citet{BXC2008} demonstrates how to construct limits of agreement using SAS, STATA and R. In the case of SAS, the PROC MIXED procedure is used.
Implementation in R is performed using the nlme package \citep{pb2000}.

\citet{BXC2008} remarks that the implementation using R is quite ``arcane".

As R is freely available, this paper demonstrates an implementation of Roy's model using R.

The R statistical software package is freely available.

%------------------------------------------------------------------------------------------------%
The LME model is very easy to implement using PROC MIXED of SAS and the results are also easy to interpret.
The SAS proc mixed procedure has very simple syntax.

As the required code to fit the models is complex, R code necessary to fit the models is provided. 

A demonstration is provided on how to use the output to perform the tests, and to compute limits of agreement.



We assume the data are formatted as a dataset with four columns named:

meth, method of measurement, the number of methods being M,
item, items (persons, samples) measured by each method, of which there are I,
repl, replicate indicating repeated measurement of the same item by the same method, and
y, the measurement.





\newpage
\section{Regression Of Differences On Averages}
Further to Carstensen, we can formulate the two measurements
$y_{1}$ and $y_{2}$ as follows:
\\
$y_{1} = \alpha + \beta\mu + \epsilon_{1}$
\\
$y_{2} = \alpha + \beta\mu + \epsilon_{2}$









\newpage
\subsection{Remarks on the Multivariate Normal Distribution}

Diligence is required when considering the models. Carstensen specifies his models in terms of the univariate normal distribution. Roy's model is specified using the bivariate normal distribution.
This gives rises to a key difference between the two model, in that a bivariate model accounts for covariance between the variables of interest.
The multivariate normal distribution of a $k$-dimensional random vector $X = [X_1, X_2, \ldots, X_k]$
can be written in the following notation:
\[
X\ \sim\ \mathcal{N}(\mu,\, \Sigma),
\]
or to make it explicitly known that $X$ is $k$-dimensional,
\[
X\ \sim\ \mathcal{N}_k(\mu,\, \Sigma).
\]
with $k$-dimensional mean vector
\[ \mu = [ \operatorname{E}[X_1], \operatorname{E}[X_2], \ldots, \operatorname{E}[X_k]] \]
and $k \times k$ covariance matrix
\[ \Sigma = [\operatorname{Cov}[X_i, X_j]], \; i=1,2,\ldots,k; \; j=1,2,\ldots,k \]

\bigskip

\begin{enumerate}
	\item Univariate Normal Distribution
	
	\[
	X\ \sim\ \mathcal{N}(\mu,\, \sigma^2),
	\]
	
	\item Bivariate Normal Distribution
	
	\begin{itemize}
		\item[(a)] \[  X\ \sim\ \mathcal{N}_2(\mu,\, \Sigma), \vspace{1cm}\]
		\item[(b)] \[    \mu = \begin{pmatrix} \mu_x \\ \mu_y \end{pmatrix}, \quad
		\Sigma = \begin{pmatrix} \sigma_x^2 & \rho \sigma_x \sigma_y \\
		\rho \sigma_x \sigma_y  & \sigma_y^2 \end{pmatrix}.\]
	\end{itemize}
\end{enumerate}
\newpage

\subsection{Note 1: Coefficient of Repeatability}
The coefficient of repeatability is a measure of how well a
measurement method agrees with itself over replicate measurements
\citep{BA99}. Once the within-item variability is known, the
computation of the coefficients of repeatability for both methods
is straightforward.

\subsection{Note 2: Carstensen model in the single measurement case}
\citet{BXC2004} presents a model to describe the relationship between a value of measurement and its real value.
The non-replicate case is considered first, as it is the context of the Bland-Altman plots.
This model assumes that inter-method bias is the only difference between the two methods.


\begin{equation}
y_{mi}  = \alpha_{m} + \mu_{i} + e_{mi} \qquad  e_{mi} \sim \mathcal{N}(0,\sigma^{2}_{m})
\end{equation}

The differences are expressed as $d_{i} = y_{1i} - y_{2i}$.

For the replicate case, an interaction term $c$ is added to the model, with an associated variance component.




\subsection{Note 3: Model terms}
It is important to note the following characteristics of this model.
\begin{itemize}
	\item Let the number of replicate measurements on each item $i$ for both methods be $n_i$, hence $2 \times n_i$ responses. However, it is assumed that there may be a different number of replicates made for different items. Let the maximum number of replicates be $p$. An item will have up to $2p$ measurements, i.e. $\max(n_{i}) = 2p$.
	
	% \item $\boldsymbol{y}_i$ is the $2n_i \times 1$ response vector for measurements on the $i-$th item.
	% \item $\boldsymbol{X}_i$ is the $2n_i \times  3$ model matrix for the fixed effects for observations on item $i$.
	% \item $\boldsymbol{\beta}$ is the $3 \times  1$ vector of fixed-effect coefficients, one for the true value for item $i$, and one effect each for both methods.
	
	\item Later on $\boldsymbol{X}_i$ will be reduced to a $2 \times 1$ matrix, to allow estimation of terms. This is due to a shortage of rank. The fixed effects vector can be modified accordingly.
	\item $\boldsymbol{Z}_i$ is the $2n_i \times  2$ model matrix for the random effects for measurement methods on item $i$.
	\item $\boldsymbol{b}_i$ is the $2 \times  1$ vector of random-effect coefficients on item $i$, one for each method.
	\item $\boldsymbol{\epsilon}$  is the $2n_i \times  1$ vector of residuals for measurements on item $i$.
	\item $\boldsymbol{G}$ is the $2 \times  2$ covariance matrix for the random effects.
	\item $\boldsymbol{R}_i$ is the $2n_i \times  2n_i$ covariance matrix for the residuals on item $i$.
	\item The expected value is given as $\mbox{E}(\boldsymbol{y}_i) = \boldsymbol{X}_i\boldsymbol{\beta}.$ \citep{hamlett}
	\item The variance of the response vector is given by $\mbox{Var}(\boldsymbol{y}_i)  = \boldsymbol{Z}_i \boldsymbol{G} \boldsymbol{Z}_i^{\prime} + \boldsymbol{R}_i$ \citep{hamlett}.
\end{itemize}
\newpage

%\chapter{Limits of Agreement}

%\section{Modelling Agreement with LME Models}

% Carstensen pages 22-23


Roys uses and LME model approach to provide a set of formal tests for method comparison studies.\\

Four candidates models are fitted to the data.\\

These models are similar to one another, but for the imposition of equality constraints.\\

These tests are the pairwise comparison of candidate models, one formulated without constraints, the other with a constraint.\\


Roy's model uses fixed effects $\beta_0 + \beta_1$ and $\beta_0 + \beta_1$ to specify the mean of all observationsby \\ methods 1 and 2 respectuively.





Roy adheres to Random Effect ideas in ANOVA

Roy treats items as a sample from a population.\\

Allocation of fixed effects and random effects are very different in each model\\

Carstensen's interest lies in the difference between the population from which they were drawn.\\

Carstensen's model is a mixed effects ANOVA.\\

\[
Y_{mir}  =  \alpha_m + \mu_i + c_{mi} + e_{mir}, \qquad c_{mi} \sim \mathcal{\tau^2_m}, \qquad e_{mir} \sim \mathcal{\sigma^2_m},
\]

This model includes a method by item iteration term.\\

Carstensen presents two models. One for the case where the replicates, and a second for when they are linked.\\

Carstensen's model does not take into account either between-item or within-item covariance between methods.\\


In the presented example, it is shown that Roy's LoAs are lower than those of Carstensen.
Carstensen makes some interesting remarks in this regard.

\begin{quote}
	The only slightly non-standard (meaning "not often used") feature is the differing residual variances between methods.
\end{quote}

\section{Cook's Distance} %1.9
%
%\citet{cook77} greatly expanded the study of residuals and influence measures. Cook's key observation was the effects of deleting each observation in turn could be computed without undue additional computational expense. Consequently deletion diagnostics have become an integral part of assessing linear models.
%---------------------------------------------------------------------------%
\citet{cook77} greatly expanded the study of residuals and influence measures. \index{Cook's distance}Cook's Distance , denoted as$D_{(i)}$, is a well known diagnostic technique used in classical linear models, used as an overall measure of the combined impact of the $i$th case of all estimated regression coefficients. Cook's key observation was the effects of deleting each observation in turn could be calculated with little additional computation. That is to say, $D_{(i)}$ can be calculated without fitting a new regression coefficient each time an observation is deleted.  Consequently deletion diagnostics have become an integral part of assessing linear models. 

The focus of this analysis is related to the estimation of point estimates (i.e. regression coefficients). It must be pointed out that the effect on the precision of estimates is separate from the effect on the point estimates. Data points that
have a small \index{Cook's distance}Cook's distance, for example, can still greatly affect hypothesis tests and confidence intervals, if their  influence on the precision of the estimates is large.

As well as individual observations, Cook's distance can be used to analyse the influence of observations in subset $U$ on a vector of parameter estimates \citep{cook77}.
%\section{Effects on fitted and predicted values}
\begin{eqnarray}
\hat{e_{i}}_{(U)} = y_{i} - x\hat{\beta}_{(U)}\\
\delta_{(U)} = \hat{\beta} - \hat{\beta}_{(U)}
\end{eqnarray}
%It uses the same structure for measuring the combined impact of the differences in the estimated regression coefficients when the $k$th case is deleted. 

%======================================================= %
%===================================================================%
\newpage
\begin{itemize}
	\item \textit{
		The previous Section (Section 4) is a literary review of residual diagnostics and influence procedures
		for Linear Mixed Effects Models, drawing heavily on Schabenberger and Zewotir.}
	
	\item \textit{	Section 4 begins with an introduction to key topics in residual diagnostics, such as influence, leverage, outliers
		and Cook's distance. Other concepts such as DFFITS and DFBETAs will be introduced briefly, mostly to explain why the are not particularly useful for
		the Method Comparison context, and therefore are not elaborated upon.}
	
	\item \textit{	In brief, Variable Selection is not applicable to Method Comparison Studies, in the 
		commonly used used context. 
		Testing a rather simplisticy specificied model against one with more random effects terms is tractable, but this research question is of secondary importance.}
\end{itemize}

%=============================================== %
\newpage
\subsection*{Appendix to Section 4}

As an appendix to section 4, an appraisal of the current state of development (or lack thereof) for current implemenations for LME models, particularly for
\texttt{nlme} and \texttt{lme4} fitted models.

Crucially, a review of internet resources indicates that almost all of the progress in this regard has been done for \texttt{lme4} fitted models, specifically the \textit{Influence.ME} \texttt{R} package. (Nieuwenhuis et 2012)

Conversely there is very little for \texttt{nlme} models. To delve into this mor, one would immediately investigate the current development workflow for both packages.

%======================%
% Douglas Bates

As an aside, Douglas Bates was arguably the most prominent \texttt{R} developer working in the LME area. 
However Bates has now prioritised the development of LME models in another computing environment , i.e Julia. 
% The current version of this is XXXX

%======================%
% nlme
\subsubsection*{The \texttt{nlme} package}

With regards to \texttt{nlme}, the torch has been passed to Galecki Galecki \& Burzykowski (UMich. and Hasselt respecitely).  Galecki \& Burzykowski published \textit{Linear Mixed Effects Models using \texttt{R}}. 
Also, the accompanying \texttt{R} package, nlmeU package is under current development, with a version being released XXXX.


%======================%
% lme4 and influence.ME
\subsubsection*{The \texttt{lme4} package}

The \texttt{lme4} package is also under active development, under the leadership of Ben Bolker (McMaster University). According to CRAN, the LME4 package, fits linear and generalized linear mixed-effects models

\begin{quote}
	The models and their components are represented using S4 classes and methods. The core computational algorithms are implemented using the Eigen C++ library for numerical linear algebra and RcppEigen "glue".
	(CRAN)
\end{quote}

%=====================%
% Important Consideration for MCS

The key issue is that \texttt{nlme} allows for the particular specification of Roy's Model, speciifically direct spefiication of the VC matrices for within subject and between subject residuals.
The \texttt{lme4} package does not allow for this.
To advance the ideas that eminate from Roys' paper, one is required to use the \texttt{nlme} context. However, to take advantage of the infrastructure already provided for \texttt{lme4} models, one may change the research question away from that of Roy's paper. 
To this end, an exploration of what textit{influence.ME} can accomplished is merited.
As a complement to this, one can also consider how to properly employ the $R^2$ measure, in the context of Methoc Comparison Studies, further to the work by Edwards et al, namely ``An $R^2$ statistic for fixed effects in the linear mixed model".
%================================================= %
\newpage
\begin{framed}
	
	\begin{quote}
		\textbf{Abstract for ``An $R^2$ statistic for fixed effects in the linear mixed model"}
		Statisticians most often use the linear mixed model to analyze Gaussian longitudinal data. 
		
		The value and familiarity of the R2 statistic in the linear univariate model naturally creates great interest in extending it to the linear mixed model. We define and describe how to compute a model R2 statistic for the linear mixed model by using only a single model. 
		
		The proposed R2 statistic measures multivariate association between the repeated outcomes and the fixed effects in the linear mixed model. The R2 statistic arises as a 1–1 function of an appropriate F statistic for testing all fixed effects (except typically the intercept) in a full model. 
		
		The statistic compares the full model with a null model with all fixed effects deleted (except typically the intercept) while retaining exactly the same covariance structure. 
		
		Furthermore, the R2 statistic leads immediately to a natural definition of a partial R2 statistic. A mixed model in which ethnicity gives a very small p-value as a longitudinal predictor of blood pressure (BP) compellingly illustrates the value of the statistic. 
		
		In sharp contrast to the extreme p-value, a very small $R^2$ , a measure of statistical and scientific importance, indicates that ethnicity has an almost negligible association with the repeated BP outcomes for the study.
	\end{quote}
\end{framed}
%====================%
% Diagnostics with nlmeU
\newpage
\subsection*{Leave-One-Out Diagnostics with \texttt{lmeU}}
Galecki et al discuss the matter of LME influence diagnostics in their book, although not into great detail.


The command \texttt{lmeU} fits a model with a particular subject removed. The identifier of the subject to be removed is passed as the only argument

A plot ofthe per-observation diagnostics individual subject log-likelihood contributions can be rendered.

\subsubsection*{Likelihood Displacement}
%% Page 503 Galecki




%====================================================================%
\newpage

\subsection*{Missing Data in Method Comparison Studies}

The matter of missing data has not been commonly encountered in either Method Comparison Studies or Linear Mixed Effects Modelling. However Roy (2009) deals with the relevant assumptions regrading missing data.

Galecki \& Burzykowski (2013) tackles the subject of missing data in LME Modelling.

Furthermore the nlmeU package includes the \texttt{patMiss} function, which ``allows to compactly present pattern of missing data in a given vector/matrix/data
frame or combination of thereof".

\newpage
\section{Exention of Cook's Distance methodology to LME models}
\index{Cook's distance} Cook's Distance is extended to LME models.  For LME models, two formulations exist; a \index{Cook's distance}Cook's distance that examines the change in fixed fixed parameter estimates, and another that examines the change in random effects parameter estimates. The outcome of either Cook's distance is a scaled change in either $\beta$ or $\theta$.

Diagnostic methods for variance components are based on `one-step' methods. \citet{cook86} gives a completely general method for assessing the influence of local departures from assumptions in statistical models. For fixed effects parameter estimates in LME models, the \index{Cook's distance} Cook's distance can be extended to measure influence on these fixed effects.

\[
\mbox{CD}_{i}(\beta) = \frac{(c_{ii} - r_{ii}) \times t^2_{i}}{r_{ii} \times p}
\]

For random effect estimates, the \index{Cook's distance} Cook's distance is

\[
\mbox{CD}_{i}(b) = g{\prime}_{(i)} (I_{r} + \mbox{var}(\hat{b})D)^{-2}\mbox{var}(\hat{b})g_{(i)}.
\]
Large values for Cook's distance indicate observations for special attention.

\index{Cook's distance}Cook's Distance was extended from classical linear models to LME models.  For linear mixed effects models, Cook's distance can be extended to model influence diagnostics by definining.

\[ CD_{\beta i} = {(\hat{\beta} - \hat{\beta}_{[i]})^{T}(\boldsymbol{X}^{\prime}\boldsymbol{V}^{-1}\boldsymbol{X}) (\hat{\beta} - \hat{\beta}_{[i]}) \over p}\]

It is also desirable to measure the influence of the case deletions on the covariance matrix of $\hat{\beta}$.

%================================================================== %



\subsection{Cook's Distance}
\begin{itemize}
	\item For variance components $\gamma$: $CD(\gamma)_i$,
	\item For fixed effect parameters $\beta$: $CD(\beta)_i$,
	\item For random effect parameters $\boldsymbol{u}$: $CD(u)_i$,
	\item For linear functions of $\hat{beta}$: $CD(\psi)_i$
\end{itemize}



It is also desirable to measure the influence of the case deletions on the covariance matrix of $\hat{\beta}$.



\section{Cook's Distance for LMEs} %1.10
Diagnostic methods for fixed effects are generally analogues of methods used in classical linear models.
Diagnostic methods for variance components are based on `one-step' methods. \citet{cook86} gives a completely general method for assessing the influence of local departures from assumptions in statistical models.

For fixed effects parameter estimates in LME models, the \index{Cook's distance} Cook's distance can be extended to measure influence on these fixed effects.

\[
\mbox{CD}_{i}(\beta) = \frac{(c_{ii} - r_{ii}) \times t^2_{i}}{r_{ii} \times p}
\]

For random effect estimates, the \index{Cook's distance} Cook's distance is

\[
\mbox{CD}_{i}(b) = g{\prime}_{(i)} (I_{r} + \mbox{var}(\hat{b})D)^{-2}\mbox{var}(\hat{b})g_{(i)}.
\]
Large values for Cook's distance indicate observations for special attention.

\newpage
\subsubsection{Random Effects}

A large value for $CD(u)_i$ indicates that the $i-$th observation is influential in predicting random effects.

\subsubsection{linear functions}

$CD(\psi)_i$ does not have to be calculated unless $CD(\beta)_i$ is large.

%---------------------------------------------------------------------------%
%

For LME models, two formulations exist; a \index{Cook's distance}Cook's distance that examines the change in fixed fixed parameter estimates, and another that examines the change in random effects parameter estimates. The outcome of either Cook's distance is a scaled change in either $\beta$ or $\theta$.

%If $V$ is known, Cook's D can be calibrated according to a chi-square distribution with degrees of freedom equal to the rank of $\boldsymbol{X}$ \citep{cpj92}.

%
%%---------------------------------------------------------------------------%
%\newpage
%\section{Cook's Distance for LMEs} %1.10
%Diagnostic methods for fixed effects are generally analogues of methods used in classical linear models.
%Diagnostic methods for variance components are based on `one-step' methods. \citet{cook86} gives a completely general method for assessing the influence of local departures from assumptions in statistical models.
%
%For fixed effects parameter estimates in LME models, the \index{Cook's distance} Cook's distance can be extended to measure influence on these fixed effects.
%
%\[
%\mbox{CD}_{i}(\beta) = \frac{(c_{ii} - r_{ii}) \times t^2_{i}}{r_{ii} \times p}
%\]
%
%For random effect estimates, the \index{Cook's distance} Cook's distance is
%
%\[
%\mbox{CD}_{i}(b) = g{\prime}_{(i)} (I_{r} + \mbox{var}(\hat{b})D)^{-2}\mbox{var}(\hat{b})g_{(i)}.
%\]
%Large values for Cook's distance indicate observations for special attention.
%



\subsection{Change in the precision of estimates}

The effect on the precision of estimates is separate from the effect on the point estimates. Data points that
have a small \index{Cook's distance} Cook's distance, for example, can still greatly affect hypothesis tests and confidence intervals, if their  influence on the precision of the estimates is large.

%------------------------------------------------------------%



\subsection{Introduction}
\subsubsection{Robinson's (1991) review}
\emph{ Robinson's (1991) review of best linear unbiased prediction (BLUP), together with the subsequent discussion, has emphasized the very considerable range of models that may be addressed via the general least squares (GLS) solution to the general linear model $Y = X\beta + \varepsilon$, where $E(\varepsilon) = 0$ and $var(\varepsilon) = V$. These include linear mixed models, geostatistics, time series and multivariate regression.}


\emph{ The texts by Christensen (1996, 1991) and the connections to modern topics of image analysis, quality analysis, Bayesian methods, and splines (all in Robinson and discussion) make it an eminently suitable topic for teaching in any course concerning statistical linear models. }


\emph{Nevertheless some of the matrix algebra that results from solving the normal equations for individual specifications of the general linear model will be daunting, and far from intuitive for many students, even those who are at home in linear space. The conventional approach to prediction and estimation from data $Y$ associated with covariates X via the general linear model $Y = X\beta + \varepsilon$ is essentially a two-stage process.}

The first stage is to determine the best,in the GLS sense, estimator $\hat{\beta}$ of $\beta$ and subsequently to determine everything else from this.

The estimator is said to be best if it minimizes the generalization of the sum of squares $\hat{e}^{t}V^{-1}\hat{e}$, where $\hat{e} = Y- X\hat{\beta}$

%---------------------------------------------------%
%Simplifying GLS
\newpage

It is straightforward to show that $\hat{\beta} = (X^tV^{-l}X)^{-l}X^tV^{-l}Y = BY$ and at the minimum the sum of squares is $Y^{t} (V^{-l}  - V^{-l}(X^tV^{-l}X)^{-l}X^tV^{-l})Y = Y^{t}QY$.\\
\bigskip

\emph{The purpose of this note is to give emphasis to one derivation, based on Lagrange multipliers, which leads to a system of equations that is very intuitive and lends itself readily to specialization. This approach is in fact standard in the geostatistical treatment of \textbf{kriging} (see Matheron 1962; Journel and Huijbregts 1981; Ripley 1981; Cressie 1993). In the genetics literature it is associated with the name of Henderson (1983); or in the classical statistical literature Hocking (1996, p. 73) is a suitable reference.}

\emph{The approach based on Lagrange multipliers deemphasizes the explicit determination of $\hat{\beta}$ and leads to a clearer understanding of the complementary (but for some confusing) tasks known as best linear unbiased estimation (BLUE) and best linear unbiased prediction (BLUP). Regrettably, Robinson-despite offering four derivations, and having as his main concern the interplay of BLUP and BLUE-gives it little prominence.}

It has recently been discussed by Searle (1997, p; 278) who said that it makes another approach (Searle, Casella, and McCulloch 1992, p. 271) seem "obtuse and unnecessarily complicated." By contrast, our treatment emphasizes the fact that it leads to a single set of equations whose solution sheds simplifying light on very many issues in general least squares.

The American Statistician's Teacher's Corner (e.g., McLean, Sanders, and Stroup 1991; Puntanen and Styan 1989) has already played host to previous attempts to simplify the explanation of such topics. Various authors (CPJ, Haslett Hayes ,Martin ) have visited the more specialized area of diagnostics and have developed \textbf{\emph{down-dating}} (leave-$k$-out) formulas.

The conventional approach here is via tricky identities based on the inverses of partitioned matrices. Here again the Lagrange system of equations leads to a much simplified and-we claim-much more intuitive derivation of these more technical results.


\emph{
	The essence of the approach is to seek that linear combination of the available data Y which is best for the
	estimation of Z among those linear estimators which are constrained to be unbiased. We adopt therefore a constrained minimization approach, using Lagrange multipliers. By best we mean that combination $\hat{Z}(Y) = \lambda_{z}^{t}Y$ which has least mean square error $E( Z- \lambda_{z}^{t}Y)^2$, and by unbiased we mean $E( Z- \lambda_{z}^{t}Y)) = 0$. }
Here $Z$ denotes that scalar which is to be the objective of the estimation. This estimator is written as $\hat{Z}(Y)$ to make its dependence on $Y$ explicit. Note that the term "best" is applied in the context of minimizing the prediction variance $var(Z - Z(Y))$. We shall see that Z may be used to denote either a random variable or an unknown parameter, and that it will be sufficient to specify Z via $E[Z]$ and $cov(Z, Y)$. If $Z$ is not a random variable then of course the latter is zero and $E[Z] = Z$. We establish-very simply, as below-a general solution in terms of A and cov(Z, Y) and achieve particular tasks by identification of these. Our presentation is for a scalar Z, but the notation facilitates generalization to vector Z.


%--------------------------------------------------------------------%

\subsection{Predictors and Estimators}

\emph{We note that Robinson (1991) stated "A convention has somehow developed that estimators of random effects are called predictors while estimators of fixed effects are called estimators." We agree that this distinction is confusing and indeed unnecessary.} \\ \bigskip



We seek $\hat{Z}(Y) = \lambda_{z}^{t}Y$, where $ \lambda_{z}^{t}$, is an $n \times 1$ vector of estimation coefficients. It is convenient to specify $E[Z]=A\beta$ for known $A$. In this context $A$ denotes a row vector, but we generalize this in the following. The constraint requiring $\hat{Z}(Y)$ to be unbiased now reduces to $(A -  \lambda_{z}^{t}X) = 0$. A solution is found by minimizing $var(Z -  \lambda_{z}^{t}Y) + \gamma^t_z (X^t\lambda_{z} - A^t)$, where $\gamma_z$ is a $p \times 1$ vector of Lagrange multipliers, where $p$ is the length of the parameter vector $\beta$. Setting to zero the derivatives with respect to $\lambda_{z}$ and $\gamma_z $ yields the system.




\begin{equation}
\left(
\begin{array}{cc}
V & X \\
X^t & 0 \\
\end{array}
\right)\left(
\begin{array}{c}
\lambda_{z}\\
\gamma_z \\
\end{array}
\right)=\left(
\begin{array}{c}
\mbox{cov}(Y,Z)\\
A^{t} \\
\end{array}
\right)
\end{equation}


If the inverse exists we have that
\begin{equation}
\left(
\begin{array}{c}
\lambda_{z}\\
\gamma_z \\
\end{array}
\right)=\left(
\begin{array}{cc}
V & X \\
X^t & 0 \\
\end{array}
\right) ^{-1}\left(
\begin{array}{c}
\mbox{cov}(Y,Z)\\
A^{t} \\
\end{array}
\right)
\end{equation}



so that
\[ \hat{Z}(Y) =
\left(
\begin{array}{cc}
\lambda_{z}^{t}&
\gamma_z^{t} \\
\end{array}
\right)=\left(
\begin{array}{c}
Y \\
0 \\
\end{array}
\right) \]

In terms of the estimation problem being considered the square matrix on the left-hand side of (1) concerns "what we have," namely, the data plus constraints.

The matrix does not depend on Z and consequently need only be constructed once before application to a range of problems. The right- hand side contains the term $cov(Z,Y)$ and can be specified for whatever Z is being considered.

It is this feature of system (1) that makes a generic approach to estimation possible.





\bibliographystyle{chicago}
\bibliography{DB-txfrbib}
\end{document}


