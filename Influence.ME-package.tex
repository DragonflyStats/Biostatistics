To apply the same logic to mixed effects models one has to measure the influence of a particular higher
level unit on the estimates of a higher level predictor. 

This means that the mixed effect model has to
be adjusted to neutralize the unit’s influence on that estimate, while at the same time allowing the unit’s
lower-level cases to help estimate the effects of the lower-level predictors in the model. 

This procedure is
based on a modification of the intercept and the addition of a dummy variable for the cases that might be
influential. 

Influence.ME provides several measures of influential cases, and is specifically designed for use
with mixed effects regression models using the afore mentioned modified intercept and dummy approach.

Using both ‘real’ and simulated data from Social Science applications of mixed effects models, five tools to
detect influential cases which are available in the package will be discussed:
\begin{itemize}
\item Cook’s Distance
\item DFBETAS
\item Percent change of the estimated parameter magnitude
\item Changes in statistical significance of parameter estimates
\item Changes in the sign of parameter estimates
\end{itemize}
In contrast with other algorithms for detecting influential cases, influence.ME is capable to uncover
groups of cases that are influential. Since this rapidly becomes computationally highly intensive, additional
script functions are provided that assist in manually dividing the computation into multiple sessions, or to
possibly to share the computations between different computers.
%=========================================================================================================%


influence.ME is an R package for detecting influential data in multilevel regression models (or, mixed effects models as they are referred to in the R community). The application of multilevel models has become common practice, but the development of diagnostic tools has lagged behind. Hence, we developed influence.ME, which calculates standardized measures of influential data for the point estimates of generalized multilevel models, such as DFBETAS, Cook’s distance, as well as percentile change and a test for changing levels of significance. influence.ME calculates these measures of influence while accounting for the nesting structure of the data. A paper detailing this package was published in the R Journal (available from the R Journal (.PDF) and my researchgate.net profile).

influence.ME depends on lme4. As the authors of lme4 have completely revised the inner workings of lme4 and are currently releasing version 1.0, 

%==============================================================================%
\section{influence.ME: Tools for detecting influential data in mixed effects models}
\begin{itemize}
\item influence.ME provides a collection of tools for detecting influential cases in generalized mixed effects models. 
\item It analyses models that were estimated using lme4. 
\item The basic rationale behind identifying influential data is that when iteratively single units are omitted from the data, 
models based on these data should not produce substantially different estimates. 
\item To standardize the assessment of how influential a (single group of) observation(s) is, several measures of influence 
are common practice, such as DFBETAS and Cook's Distance. In addition, we provide a measure of percentage change of the 
fixed point estimates and a simple procedure to detect changing levels of significance.


%--------------------------------------------------------------------------------------%

%  - http://www.r-bloggers.com/influence-me-tools-for-detecting-influential-data-in-multilevel-regression-models/
\item Despite the increasing popularity of multilevel regression models, the development of diagnostic tools lagged behind. Typically, in the social sciences multilevel regression models are used to account for the nesting structure of the data, such as students in classes, migrants from origin-countries, and individuals in countries. The strength of multilevel models lies in analyzing data on a large number of groups with only a couple of observations within each group, such as for instance students in classes.

\item Nevertheless, in the social sciences multilevel models are often used to analyze data on a limited number of groups with per group a large number of observations. A typical example would be the analysis of data on individuals nested within countries. By nature, only a limited number of countries exists. In practice, typical country-comparative analyses are based on about 25 countries. With such a small number of groups (e.g. countries), observations on a single group can easily be overly influential to the outcomes. This means that the conclusions based on the multilevel regression model could no longer hold when a single group is removed from the data.

\item In our recent publication in the R Journal, we introduce influence.ME, software that provides tools for detecting influential data in multilevel regression models (or: in mixed effects models, as these are commonly referred to in statistics). influence.ME is a publically available R package that evaluates multilevel regression models that were estimated with the lme4.0 package. It calculates standardized measures of influential data for the point estimates of generalized mixed effects models, such as DFBETAS, Cook’s distance, as well as percentile change and a test for changing levels of significance. influence.ME calculates these measures of influence while accounting for the nesting structure of the data. The package and measures of influential data are introduced, a practical example is given, and strategies for dealing with influential data are suggested.

\item With this publication, and of course with the software that was available for quite some time, we hope to contribute to a better usage of multilevel regression models. The provided example and guidelines were geared towards applications in the social sciences, but are applicable in all disciplines.
\end{itemize}
%--------------------------------------------------------------------------------------%

\end{document}
