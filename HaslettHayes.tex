\documentclass[00-MASTER.tex]{subfiles}
\begin{document}

\newpage
\section{Haslett's Analysis} %2.5
For fixed effect linear models with correlated error structure Haslett (1999) showed that the effects on
the fixed effects estimate of deleting each observation in turn could be cheaply computed from the fixed effects model predicted residuals.


A general theory is presented for residuals from the general linear model with correlated errors.
It is demonstrated that there are two fundamental types of residual associated with this model,
referred to here as the marginal and the conditional residual.


These measure respectively the distance to the global aspects of the model as represented by the expected value
and the local aspects as represented by the conditional expected value.


These residuals may be multivariate.


\citet{HaslettHayes} developes some important dualities which have simple implications for diagnostics.


%The results are illustrated by reference to model diagnostics in time series and in classical multivariate analysis with independent cases.
%============================================================================================== %
\newpage

\section{Haslett's Analysis} %2.5
For fixed effect linear models with correlated error structure Haslett (1999) showed that the effects on
the fixed effects estimate of deleting each observation in turn could be cheaply computed from the fixed effects model predicted residuals.


A general theory is presented for residuals from the general linear model with correlated errors. 
It is demonstrated that there are two fundamental types of residual associated with this model, 
referred to here as the marginal and the conditional residual. 

These measure respectively the distance to the global aspects of the model as represented by the expected value 
and the local aspects as represented by the conditional expected value. 

These residuals may be multivariate. 

\citet{HaslettHayes} developes some important dualities which have simple implications for diagnostics. 

%The results are illustrated by reference to model diagnostics in time series and in classical multivariate analysis with independent cases.
\end{document}