\newpage
\section*{Deming Regression}

Performance of Deming regression analysis in case of misspecified analytical error ratio in method comparison studies

%-----------------------------------------------------------------------------------------------%
Application of Deming regression analysis to interpret method comparison data presupposes specification of the 
squared analytical error ratio ($\lambda$, but in cases involving only single measurements by each method, this 
ratio may be unknown and is often assigned a default value of one. 

On the basis of simulations, this practice was evaluated in situations with real error ratios deviating from one. 
Comparisons of two electrolyte methods and two glucose methods were simulated. 

In the first case, misspecification of $\lambda$ produced a bias that amounted to two-thirds of the maximum bias of the 
ordinary least-squares regression method. Standard errors and the results of hypothesis-testing also became misleading. 
In the second situation, a misspecified error ratio resulted only in a negligible bias. 

Thus, given a short range of values in relation to the measurement errors, it is important that $\lambda$ is correctly 
estimated either from duplicate sets of measurements or, in the case of single measurement sets, specified from 
quality-control data. However, even with a misspecified error ratio, Deming regression analysis is likely to perform 
better than least-squares regression analysis.
