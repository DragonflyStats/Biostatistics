\documentclass[12pt, a4paper]{article}
\usepackage{natbib}
\usepackage{vmargin}
\usepackage{graphicx}
\usepackage{epsfig}
\usepackage{subfigure}
%\usepackage{amscd}
\usepackage{amssymb}
\usepackage{subfigure}
\usepackage{amsbsy}
\usepackage{amsthm, amsmath}
%\usepackage[dvips]{graphicx}
\bibliographystyle{chicago}
\renewcommand{\baselinestretch}{1.4}

% left top textwidth textheight headheight % headsep footheight footskip
\setmargins{3.0cm}{2.5cm}{15.5 cm}{23.5cm}{0.5cm}{0cm}{1cm}{1cm}

\pagenumbering{arabic}

%---------------------------------------------------------------------------%
\newpage


\section{Likelihood Distance} %1.11
The \index{likelihood distance} likelihood distance is a global summary measure that expresses the joint influence of the subsets of observations, $U$, on all parameters in $\phi$ that were subject to updating. \citet{schab} points out the likelihood distance gives the amount by which the log-likelihood of the model fitted from the full data changes if one were
to estimate the model from a reduced-data estimates. Importantly $LD(\psi_{(U)})$ is not the log-likelihood obtained by fitting the model to the reduced data set. It is obtained by evaluating the likelihood function based on the full data set (containing all $n$ observations) at the reduced-data estimates.


%---------------------------------------------------------- %
%Likelihood Displacement.
\[  LD(\boldsymbol{(U)})= 2[l\boldsymbol{\hat{(\phi)}} - l\boldsymbol{\hat{\phi}_\omega} ] \]
\[  RLD(\boldsymbol{(U)})= 2[ l_R\boldsymbol{\hat{(\phi)}} - l_R\boldsymbol{\hat{(\phi)}_\omega} ] \]
%	Large values indicate that $\boldsymbol{\hat{\theta}}$ and $\boldsymbol{\hat{\theta}_\omega}$ differ considerably.




%---------------------------------------------------------------------------%
\newpage
\section{Likelihood Distance} %1.11
The likelihood distance gives the amount by which the log-likelihood of the full data changes if one were
to evaluate it at the reduced-data estimates. The important point is that $l(\psi_{(U)})$ is not the log-likelihood
obtained by fitting the model to the reduced data set.

It is obtained by evaluating the likelihood function based on the full data set (containing all n observations) at the reduced-data estimates.

The likelihood distance is a global, summary measure, expressing the joint influence of the observations in
the set $U$ on all parameters in $\psi$  that were subject to updating.
%------------%

\subsection{Likelihood Distance}

The \index{likelihood distance} likelihood distance is a global, summary measure, expressing the joint influence of the observations in the set $U$ on all parameters in $\phi$  that were subject to updating.

\end{document}
