\documentclass[]{article}

%opening
\title{Introduction to Method Comparison Studies}

\begin{document}


%\title{Spring 2011}
\tableofcontents
\addcontentsline{toc}{section}{Bibliography}

%----------------------------------------------------------------------------------------%
%----------------------------------------------------------------------------------------%
\newpage
\chapter{Limits of Agreement}





\newpage
\chapter{LME models for MCS}






%-----------------------------------------------------------------------------------------------------%
\newpage






\citet{pkcng} generalize this approach to account for situations
where the distributions are not identical, which is commonly the
case. The TDI is not consistent and may not preserve its
asymptotic nominal level, and that the coverage probability
approach of \citet{lin2002} is overly conservative for moderate
sample sizes. This methodology proposed by \citet{pkcng} is a
regression based approach that models the mean and the variance of
differences as functions of observed values of the average of the
paired measurements.
%%%%%%%%%%%%%%%%%%%%%%%%%%%%%%%%%%%%%%%%%%%%%%%%%%%%%%%%%%%%%%%%%%%%%%%%%%%%%%%%%%%%%%%%%%%%%%%%%%%%%%%%%5

Maximum likelihood estimation is used to estimate the parameters.
The REML estimation is not considered since it does not lead to a
joint distribution of the estimates of fixed effects and random
effects parameters, upon which the assessment of agreement is
based.

\section{Random Effects and MCS}
The methodology comprises two calculations. The second calculation
is for the standard deviation of means Before the modified Bland
and Altman method can be applied for repeated measurement data, a
check of the assumption that the variance of the repeated
measurements for each subject by each method is independent of the
mean of the repeated measures. This can be done by plotting the
within-subject standard deviation against the mean of each subject
by each method. Mean Square deviation measures the total deviation
of a


\subsection{Random coefficient growth curve model} (Chincilli
1996) Random coefficient growth curve model, a special type of
mixed model have been proposed a single measure of agreement for
repeated measurements.
\begin{equation}
\textbf{d}= \textbf{Xb} + \textbf{Zu} + \textbf{e}
\end{equation}
The distributional asummptions also require \textbf{d} to
\textbf{N}

%------------------------------------------------------------------
\newpage
\section{Other Approaches}

\subsection{Random coefficient growth curve model} (Chincilli
1996) Random coefficient growth curve model, a special type of
mixed model have been proposed  a single measure of agreement for
repeated measurements.
\subsection{Marginal Modelling}
(Diggle 2002) proposes the use of marginal models as an
alternative to mixed models.m Marginal models are appropriate when
interences about the mean response are of specific interest.
\section{KP}
Most residual covariance structures are design for one
within-subject factor. However two or more may be present. For
such cases,an approppriate approach would be the residual
covariance structure using Kronecker product of the underlying
within-subject factor specific covariances structure.




%-----------------------------------------------------------------------------------------------------%
\newpage
\bibliography{DB-txfrbib}
\end{document}