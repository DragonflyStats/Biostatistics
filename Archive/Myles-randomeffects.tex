\documentclass[MAIN.tex]{subfiles}

% Load any packages needed for this document
\begin{document}
	
	
Using the Bland–Altman method to measure agreement with repeated measures
http://bja.oxfordjournals.org/content/99/3/309.full

P. S. Myles
Department of Anaesthesia and Perioperative Medicine, Alfred Hospital, Commercial Road, Melbourne, Victoria 3004, Australia
Department of Epidemiology and Preventive Medicine, Monash University, Melbourne, Australia

\section{Random effects Model} 

\citet{Myles} proposes the use of
Random effects models to address the issue of repeated
measurement. 

Myles proposes a formulation of the Bland–Altman
plot, using the within-subject variance estimated by the random
effects model, with the time of the measurement taken as a random
effect. He states that \emph{random effects models account for the
	dependent nature of the data, and additional explanatory
	variables, to provide reliable estimates of agreement in this
	setting.}
\\
Agreement between methods is reflected by the between-subject
variation.The Random Effects Model takes this into account before
calculating the within-subject standard deviation.

\subsection{Myers Random Effects Model} The presentation of the
95\% limits of agreement is for visual judgement of how well two
methods of measurement agree. The smaller the range between the
two, the better the agreement is The question of small is small is
a question of clinical judgement


Repeated measurements for each subjects are often used in clinical
research.



\subsection{Random Effects Modelling}
Random effects models are used to examine the within-subject
variation after adjusting for known and unknown variables, in
which each subject has a different intercept and slope over a time
period period.


\citet{Myles} remarks that the random effects model is an
extension of the analysis of variance method, accounting for more
covariates.

A random effect (in Myles's case, time of measurement) is chosen
to reflect the different intercept and slope for each subject with
respect to their change of measurements over the time period.

In Myles's methodology, the standard deviation of difference
between the means of the repeated measurements can be calculated
based on the within-subject standard deviation estimates.

A random effects model (also variance components model)is a type
of hierarchical linear model. Hierarchical linear modelling (HLM)
is a more advanced form of simple linear regression and multiple
linear regression. HLM is appropriate for use with nested
data.\\Faraway comments that the random effects approach is
\emph{more ambitious than the LME model in that it attempts to say
	something about the wider population beyond the particular
	sample}.



\section{Random Effects and MCS}
The methodology comprises two calculations. The second calculation
is for the standard deviation of means Before the modified Bland
and Altman method can be applied for repeated measurement data, a
check of the assumption that the variance of the repeated
measurements for each subject by each method is independent of the
mean of the repeated measures. This can be done by plotting the
within-subject standard deviation against the mean of each subject
by each method. Mean Square deviation measures the total deviation
of a


\subsection{Random coefficient growth curve model} (Chincilli
1996) Random coefficient growth curve model, a special type of
mixed model have been proposed a single measure of agreement for
repeated measurements.
\begin{equation}
\textbf{d}= \textbf{Xb} + \textbf{Zu} + \textbf{e}
\end{equation}
The distributional asummptions also require \textbf{d} to
\textbf{N}



\section{Repeated Measurements}
The original Bland–Altman paper (1983) has been cited on more than
$11,500$ occasions— indicative of its importance in medical
research.
\\
\\
Myers also remarks up on this : \emph{The Bland–Altman method can
even include estimation of confidence intervals for the bias and
limits of agreement, but these are often omitted in research
papers.}
\\
\\
When repeated measures data are available, it is desirable to use
all the data to compare the two methods. However, the original
Bland–Altman method was developed for two sets of measurements
done on one occasion (i.e. independent data), and so this approach
is not suitable for repeated measures data. However, as a naý¨ve
analysis, it may be used to explore the data because of the
simplicity of the method.
\\
\\
Myers comments upon the misuse of the Bland Altman methods is
literature, citing several research papers.
\\
\\
Myers et al propose using random effects models.
\\
\\

% P.S. Myles
% Using the Bland Altman Method to measure agreement with repeated measures
% British Journal of Anaesthesia
% Volume 99, number 3, September 2007 pages 309-311
%-------------------------------------------------------------------------------%

%PREVALENCE OF BLAND ALTMAN

\begin{quote}
The original Bland Altman Publication has been cited on more than 11,500 occasions, compelling evidence of its 
importance in medical research.
\end{quote}

The Bland Altman method can even include estimation of confidence intervals for the bias and limits of agreement, 
but these are often omitted in research papers.

The presentation of the 95% limits of agreement is for visual judgement of how well two methods of measurement agree.
The smaller the range between the two, the better the agreement is 
The question of small is small is a question of clinical judgement

Repeated measurements for each subjects are often used in clinical research.


The original Bland Altman Method was developed for two sets of measurements done on one occasion (i.e. independent data), and so this approach is not suitable for repeated measures data. However, as a naïve analysis, it may be used to explore the data because of the simplicity of the method. Myles states that such misuse of the standards Bland Altman method is widespread in Anaesthetic and critical care literature.

Bland and Altman have provided a modification for analysing repeated measures under stable or chaninging conditions, where repearted data is collected over a period of time. 
Myers proposes an alternative Random effects model for this purpose.

%--------------------------------------------------------------------------%
\section{Random Effects Model  (Myers)}
\begin{itemize}
\item	With repeated measures data, we can calculate the mean of the repeated measurements by each method on each individuals
\item	The pairs of means can then be used to compare the two methods based on the 95% limits of agreement for the difference of means. The bias between the two methods will not be affected by averaging the repeated measurements.
\item	However the variation of the differences will be underestimated by this practice because the measurement error is, to some extent, removed. Some advanced statistical calculations are needed to take into account these measurement errors.
\item	Random effects models can be used to estimate the within-subject variation after accounting for other observed and unobserved variations, in which each subject has a different intercept and slope over the observation period
\item	On the basis of the within-subject variance estimated by the random effects model, we can then create an appropriate Bland Altman Plot.
\item	The sequence or the time of the measurement over the observation period can be taken as a random effect.
\end{itemize}
% {note – need to explain this better}
\end{document}
The main difference between  Myers proposed method and the Bland Altman is that the random effects model is used to estimate the within-subject variance after adjusting for known and unknown variables. The Bland Altman approach uses one way analysis of variance to estimate the within subject variance.
In general, the random effects model is an extension of the analysis of the ANOVA method and it can adjust for many more covariates than the ANOVA method


\bibliographystyle{chicago}
\bibliography{2017bib}




\end{document} 



