Survival analytical techniques were used to assess agreement of a quantitative variable

*J Llorca, M Delgado-Rodríguez - Journal of clinical epidemiology*

### Overview
1. **Background and Objective:** Survival–agreement plots have been suggested as a new graphical approach to assess agreement in
quantitative variables. We propose that survival analytical techniques can complement this method, providing a new analytical insight
for agreement.
2. **Methods:** Two survival–agreement plots are used to detect the bias between to measurements of the same variable. The presence of
bias is tested with log-rank test, and its magnitude with Cox regression.
3. **Results:** An example on C-reactive protein determinations shows how survival analytical methods would be interpreted in the context
of assessing agreement.
4. **Conclusion:** Log-rank test, Cox regression, or other analytical methods could be used to assess agreement in quantitative variables;
correct interpretations require good clinical sense

<hr>
- In clinical or epidemiologic research, the measurement of variables always implies some degree of error.
Because it is impossible to control the various sources of variation, the assessment of the reliability of a
measurement is essential. 
- Otherwise, concordance analysis must take into account the "clinical" interpretation 
of the measurement under study, because its practical usefulness is of central importance. In this article, 
we propose a new approach to assess the reliability of a quantitative measurement. We use a graphical approach 
familiar to statisticians and data analysts of the biomedical area, associating to it the useful feature of 
interpretation based on the proportion of concordant cases. 
- We believe that the proposed graphical approach 
can serve as a complement, or as a alternative, to the Altman-Bland method for agreement analysis. It allows
a simple interpretation of agreement that takes into account the "clinical" importance of the differences 
between observers or methods. 
-In addition, it allows the analysis of reliability or agreement, by means of survival analysis techniques.

<hr>

### Discussion
The proposed approach presents advantages and disadvantages compared to the Altman-Bland proposal or to
mountain plot. These two approaches, depending on the research interests, can complement or serve as alternative
one to another. For small data sets, the Altman-Bland approach should be used, and the incorporation of any index
would be of little value to the analysis.In general, the use of measures of agreement—particularly
in the case of a quantitative variable—is difficult, because these measures are calculated, and interpreted, almost
strictly, from a statistical standpoint. The proposed approach has the advantage of expressing, by means of a graphic, an
index that is easily interpreted, and depends upon the degree of relevance of the agreement, as judged by the researcher.

Moreover, the form of the resulting step function can also yield much information: very high steps indicating that a
Fig. 2. Proportion of discordance between systolic blood pressure by two observrs (J and R) and a semiautomatic device (S) until “tolerance” limits.
better agreement will be reached more rapidly, that is, for smaller differences. However, in choosing the module of the difference,
for example, we loose sight of some characteristics of the differences, very much evident in the Altman-Bland
approach or in the mountain plot. These characteristics can have a great impact in the study of agreement [21]. 

The average of the differences can serve as an estimate of the bias among the methods. Besides the bias, a tendency in the
differences as a function of the magnitude of the measurement or an increase of the differences indicating a greater
error due to the measurement are not considered in the proposed approach, because it does not take into account
the magnitude of the measurement. Nevertheless, if the magnitude of the bias does not have “clinical” relevance, it
might be that the absence of this information may not impair the analysis.

On the other hand, if considering the magnitude of the measurement is of importance, the proposed graphic could
take it into account through the calculation of relative differences.

For instance, in the X-axis of the graphic, instead of showing the modules of the differences, we would present the
differences relative to the magnitude of the measurement,
as follows:
|AB| / [(AB)/2]
Therefore, the graphic would be completely adimensional, what could be an advantage, for the sake of comparison.
Although our proposal is descriptive in nature, it allows the use of inference resources, by means of tests associated to
the Kaplan-Meier analysis. It is possible, for example, to use the log-rank test to evaluate whether the difference between
two curves of agreement, for a certain categorical covariate is statistically significant. The sample size must
be always taken into account, in particular, when the sample is subdivided.

\newpage



Survival analytical techniques were used to assess agreement of a quantitative variable

*J Llorca, M Delgado-Rodríguez - Journal of clinical epidemiology*

### Overview
1. **Background and Objective:** Survival–agreement plots have been suggested as a new graphical approach to assess agreement in
quantitative variables. We propose that survival analytical techniques can complement this method, providing a new analytical insight
for agreement.
2. **Methods:** Two survival–agreement plots are used to detect the bias between to measurements of the same variable. The presence of
bias is tested with log-rank test, and its magnitude with Cox regression.
3. **Results:** An example on C-reactive protein determinations shows how survival analytical methods would be interpreted in the context
of assessing agreement.
4. **Conclusion:** Log-rank test, Cox regression, or other analytical methods could be used to assess agreement in quantitative variables;
correct interpretations require good clinical sense

<hr>
- In clinical or epidemiologic research, the measurement of variables always implies some degree of error.
Because it is impossible to control the various sources of variation, the assessment of the reliability of a
measurement is essential. 
- Otherwise, concordance analysis must take into account the "clinical" interpretation 
of the measurement under study, because its practical usefulness is of central importance. In this article, 
we propose a new approach to assess the reliability of a quantitative measurement. We use a graphical approach 
familiar to statisticians and data analysts of the biomedical area, associating to it the useful feature of 
interpretation based on the proportion of concordant cases. 
- We believe that the proposed graphical approach 
can serve as a complement, or as a alternative, to the Altman-Bland method for agreement analysis. It allows
a simple interpretation of agreement that takes into account the "clinical" importance of the differences 
between observers or methods. 
-In addition, it allows the analysis of reliability or agreement, by means of survival analysis techniques.

<hr>

### Discussion
The proposed approach presents advantages and disadvantages compared to the Altman-Bland proposal or to
mountain plot. These two approaches, depending on the research interests, can complement or serve as alternative
one to another. For small data sets, the Altman-Bland approach should be used, and the incorporation of any index
would be of little value to the analysis.In general, the use of measures of agreement—particularly
in the case of a quantitative variable—is difficult, because these measures are calculated, and interpreted, almost
strictly, from a statistical standpoint. The proposed approach has the advantage of expressing, by means of a graphic, an
index that is easily interpreted, and depends upon the degree of relevance of the agreement, as judged by the researcher.

Moreover, the form of the resulting step function can also yield much information: very high steps indicating that a
Fig. 2. Proportion of discordance between systolic blood pressure by two observrs (J and R) and a semiautomatic device (S) until “tolerance” limits.
better agreement will be reached more rapidly, that is, for smaller differences. However, in choosing the module of the difference,
for example, we loose sight of some characteristics of the differences, very much evident in the Altman-Bland
approach or in the mountain plot. These characteristics can have a great impact in the study of agreement [21]. 

The average of the differences can serve as an estimate of the bias among the methods. Besides the bias, a tendency in the
differences as a function of the magnitude of the measurement or an increase of the differences indicating a greater
error due to the measurement are not considered in the proposed approach, because it does not take into account
the magnitude of the measurement. Nevertheless, if the magnitude of the bias does not have “clinical” relevance, it
might be that the absence of this information may not impair the analysis.

On the other hand, if considering the magnitude of the measurement is of importance, the proposed graphic could
take it into account through the calculation of relative differences.

For instance, in the X-axis of the graphic, instead of showing the modules of the differences, we would present the
differences relative to the magnitude of the measurement,
as follows:
|AB| / [(AB)/2]
Therefore, the graphic would be completely adimensional, what could be an advantage, for the sake of comparison.
Although our proposal is descriptive in nature, it allows the use of inference resources, by means of tests associated to
the Kaplan-Meier analysis. It is possible, for example, to use the log-rank test to evaluate whether the difference between
two curves of agreement, for a certain categorical covariate is statistically significant. The sample size must
be always taken into account, in particular, when the sample is subdivided.
