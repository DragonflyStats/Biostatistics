Cook's Distance
Lilelihood Distance

%===========================================================================%

technology acceptance model is coommonly encountered in the context of software systems or technology, but many of the key lessons can be re-interpreted for other 
fields. With specific reference to the Bland-Altman methodology, we will explore some of them here.

\begin{itemize}
\item Perceived ease of use.
\item Perceived Functionality.
\end{itemize}

% The Technology Acceptance Model (TAM) is an information systems theory that models how users come to accept and use a technology. The model suggests that when users are presented with a new technology, a number of factors influence their decision about how and when they will use it, notably:
% Perceived usefulness (PU) - This was defined by Fred Davis as "the degree to which a person believes that using a particular system would enhance his or her job performance".
% Perceived ease-of-use (PEOU) -C Davis defined this as "the degree to which a person believes that using a particular system would be free from effort" (Davis 1989).

In summary, if one is to propose an advancement of method comparison methods, one must consider how these methods are to likely to be adopted by widespread use. 

% http://en.wikipedia.org/wiki/Unified_theory_of_acceptance_and_use_of_technology


%=================================================================================%


% - http://www.nss.gov.au/nss/home.nsf/pages/Confidentiality+-+Managing+the+risk+of+disclosure+in+the+release+of+microdata
% - Types of disclosure risk in microdata


\begin{itemize}
\item R command and R object - Typewriter Font
\item R Package name - Italics
\item Selected Accroymns - Italics
\end{itemize}

%===================================================================%
\textit{
The previous Section (Section 4) is a literary review of residual diagnostics and influence procedures
for Linear Mixed Effects Models, drawing heavily on Schabenberger and Zewotir.

Section 4 begins with an introduction to key topics in residual diagnostics, such as influence, leverage, outliers
and Cook's distance. Other concepts such as \textit{DFFITS} and \textit{DFBETA}s will be introduced briefly, mostly to explain why the are not particularly useful for
the Method Comparison context, and therefore are not elaborated upon.

In brief, Variable Selection is not applicable to Method Comparison Studies, in the 
commonly used used context. 
Testing a rather simplisticy specificied model against one with more random effects terms is tractable, but this research question is of secondary importance.

}

\newpage

%% Page 503 Galecki


\newpage
%====================================================================%



R PACKAGES

nlmeU
nlmeU package contains datasets and utility functions enhancing functionality of nlme package. Datasets, functions and scripts are described in Galecki, Burzykowski (2013) book. Package is under development.

nlmeUpdK Download: linux(.tar.gz) | windows(.zip) | macosx(.tgz)
nlmeUpdK is a companion package to nlmeU and contains scripts and functions used in Galecki and Burzykowski (2013) book pertaining to a new pdMat class (Sections 17.6-17.8 and 20.2)


%-http://www.r-bloggers.com/influence-me-tools-for-detecting-influential-data-in-multilevel-regression-models/

\begin{quote}

In our recent publication in the R Journal, we introduce influence.ME, software that provides tools for detecting influential data in multilevel regression models (or: in mixed effects models, as these are commonly referred to in statistics). influence.ME is a publically available R package that evaluates multilevel regression models that were estimated with the lme4.0 package. It calculates standardized measures of influential data for the point estimates of generalized mixed effects models, such as DFBETAS, Cook’s distance, as well as percentile change and a test for changing levels of significance. influence.ME calculates these measures of influence while accounting for the nesting structure of the data. The package and measures of influential data are introduced, a practical example is given, and strategies for dealing with influential data are suggested.

With this publication, and of course with the software that was available for quite some time, we hope to contribute to a better usage of multilevel regression models. The provided example and guidelines were geared towards applications in the social sciences, but are applicable in all disciplines.

On a final note, the editorial of the R Journal describes how this journal is quickly ranking up in the degree of (academic) recognition it receives:

Thomson Reuters has informed us that The R Journal has been accepted for listing in the Science Citation Index-Expanded (SCIE), including the Web of Science, and the ISI Alerting Service, starting with volume 1, issue 1 (May 2009). This complements the current listings by EBSCO and the Directory of Open Access Journals (DOAJ), and completes a process started by Peter Dalgaard in 2010.

\end{quote}


%=========================================================%
\end{document}
