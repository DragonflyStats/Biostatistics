Question 1
In a random sample of 100 patients at a clinic, you would like to test whether the mean RDI is x or more using a one sided 5% type 1 error rate. The sample mean RDI had a mean of 12 and a standard deviation of 4. What value of x (testing H0:μ=x versus Ha:μ>x) would you reject for?
Your Answer		Score	Explanation
Any value of about 11.3 or lower			
Any value of about 11.26 or lower			
Any value of about 11.26 or higher			
Any value of about 11.3 or higher	Inorrect	0.00	
Total		0.00 / 1.00	
Question Explanation

We will be more likely to reject for smaller values of x. Thus, the question is, what's the smallest value of x for which we would reject? The boundary occurs at 100−−−√(12−x)/4=1.645 or 12 - 1.645 * 4 / 10 = 11.34
%------------------------------------------------------%
Question 2
A pharmaceutical company is interested in testing a potential blood pressure lowering medication. Their first examination considers only subjects that received the medication at baseline then two weeks later. The data are as follows (SBP in mmHg)

Baseline	Week 2
140	138
138	136
150	148
148	146
135	133
Test the hypothesis that there was a mean reduction in blood pressure. Compare the difference between a paired and unpaired test for a two sided 5% level test.

Your Answer		Score	Explanation
Reject for the paired; fail to reject for the unpaired	Correct	1.00	
Reject for the paired; reject for the unpaired			
Fail to reject for the paired; reject for the unpaired			
Fail to reject for the paired; fail to reject for the unpaired			
Total		1.00 / 1.00	
Question Explanation

bl <- c(140, 138, 150, 148, 135)
fu <- c(138, 136, 148, 146, 133)
bl - fu
[1] 2 2 2 2 2
t.test(fu, bl, alternative = "two.sided", paired = FALSE)

    Welch Two Sample t-test

data:  fu and bl 
t = -0.4868, df = 8, p-value = 0.6395
alternative hypothesis: true difference in means is not equal to 0 
95 percent confidence interval:
 -11.474   7.474 
sample estimates:
mean of x mean of y 
    140.2     142.2 
Notice that the Pvalue is 0 for the paired test

Question 3
Brain volumes for 9 men yielded a 90 % confidence interval of 1,077 cc to 1,123 cc. Would you reject in a two sided 5% hypothesis test of H0:μ=1,078?
Your Answer		Score	Explanation
You would fail to reject the null hypothesis.			
It can not be ascertained from the information given.			
You would reject the null hypothesis.	Inorrect	0.00	
Total		0.00 / 1.00	
Question Explanation

No, you would fail to reject. The 95% interval would be wider than the 90% interval. Since 1,078 is in the narrower 90% interval, it would also be in the wider 95% interval. Thus, in either case it's in the interval and so you would fail to reject.
%------------------------------------------------------%
Question 4
In an effort to improve efficiency, hospital administrators are evaluating a new triage system for their emergency room. In an validation study of the system, 5 patients were tracked in a mock (simulated) ER under the new and old triage system. Their waiting times in natural log of hours were:

Subject	1	2	3	4	5
New	0.929	-1.745	1.677	0.701	0.128
Old	2.233	-2.513	1.204	1.938	2.533
Give a Pvalue for the test of the hypothesis that the new system resulted in lower waiting times for a one sided t test.

Your Answer		Score	Explanation
0.1405			
0.2597			
0.5194	Inorrect	0.00	
0.281			
Total		0.00 / 1.00	
Question Explanation

We'll use a paired t test

new <- c(0.929, -1.745, 1.677, 0.701, 0.128)
old <- c(2.233, -2.513, 1.204, 1.938, 2.533)
t.test(new - old, alternative = "less")

    One Sample t-test

data:  new - old 
t = -1.245, df = 4, p-value = 0.1405
alternative hypothesis: true mean is less than 0 
95 percent confidence interval:
   -Inf 0.5277 
sample estimates:
mean of x 
   -0.741 
Here's some of the incorrect answers

t.test(new, old, alternative = "two.sided", paired = TRUE)

    Paired t-test

data:  new and old 
t = -1.245, df = 4, p-value = 0.281
alternative hypothesis: true difference in means is not equal to 0 
95 percent confidence interval:
 -2.3932  0.9112 
sample estimates:
mean of the differences 
                 -0.741 
t.test(new, old, alternative = "less", paired = FALSE)

    Welch Two Sample t-test

data:  new and old 
t = -0.6799, df = 6.704, p-value = 0.2597
alternative hypothesis: true difference in means is less than 0 
95 percent confidence interval:
  -Inf 1.338 
sample estimates:
mean of x mean of y 
    0.338     1.079 
t.test(new, old, alternative = "two.sided", paired = FALSE)

    Welch Two Sample t-test

data:  new and old 
t = -0.6799, df = 6.704, p-value = 0.5194
alternative hypothesis: true difference in means is not equal to 0 
95 percent confidence interval:
 -3.341  1.859 
sample estimates:
mean of x mean of y 
    0.338     1.079 
%------------------------------------------------------%
Question 5
Refer to the previous question. Give a 95% T confidence interval for the ratio of the waiting times (recall that the measurements were natural logged).

Here's the data and setting again. 

In an effort to improve efficiency, hospital administrators are evaluating a new triage system for their emergency room. In an validation study of the system, 5 patients were tracked in a mock (simulated) ER under the new and old triage system. Their waiting times in natural log of hours were:

Subject	1	2	3	4	5
New	0.929	-1.745	1.677	0.701	0.128
Old	2.233	-2.513	1.204	1.938	2.533
Your Answer		Score	Explanation
-2.01 to 0.528			
-2.39 to 0.91			
0.134 to1.69			
.09 to 2.49	Correct	1.00	
Total		1.00 / 1.00	
Question Explanation

Refer to the previous solution
exp(t.test(new - old)$conf.int)
[1] 0.09133 2.48743
attr(,"conf.level")
[1] 0.95
%------------------------------------------------------%
Question 6
Suppose that 18 obese subjects were randomized, 9 each, to a new diet pill and a placebo. Subjects’ body mass indices (BMIs) were measured at a baseline and again after having received the treatment or placebo for four weeks. The average difference from follow-up to the baseline (followup - baseline) was −3 kg/m2 for the treated group and 1 kg/m2 for the placebo group. The corresponding standard deviations of the differences was 1.5 kg/m2 for the treatment group and 1.8 kg/m2 for the placebo group. Does the change in BMI over the two year period appear to differ between the treated and placebo groups? Assuming normality of the underlying data and a common population variance, give a pvalue for a two sided t test.
Your Answer		Score	Explanation
Around 0.00005	Inorrect	0.00	
Around 0.01			
Around 0.1			
Around 0.001			
Around 0.0001			
Total		0.00 / 1.00	
Question Explanation

n1 <- n2 <- 9
x1 <- -3  ##treated
x2 <- 1  ##placebo
s1 <- 1.5  ##treated
s2 <- 1.8  ##placebo
s <- sqrt(((n1 - 1) * s1^2 + (n2 - 1) * s2^2)/(n1 + n2 - 2))
ts <- (x1 - x2)/(s * sqrt(1/n1 + 1/n2))
2 * pt(ts, n1 + n2 - 2)
[1] 0.0001025
%------------------------------------------------------%
Question 7
Consider a one sided α level single group Z test of H0:μ=μ0 versus Ha:μ>μ0 with the data X¯ for the sample mean and s for the sample standard deviation with n measurements. What are the collection of points for which you would fail to reject the hypothesis?
Your Answer		Score	Explanation
[X+Z1−αs/n√,∞)	Inorrect	0.00	
[X−Z1−αs/n√,∞)			
(−∞,X−Z1−αs/n√]			
The confidence interval X¯±Z1−α/2s/n√			
(−∞,X+Z1−αs/n√]			
Total		0.00 / 1.00	
Question Explanation

We will fail to reject if n√(X¯−μ0)/s≤Z1−α. Thus if μ0≥X¯−Z1−αs/n√ or [X−Z1−αs/n√,∞).

Total		0.00 / 1.00	
Question Explanation

Confusing, but conceptually easy problem. The test statistic is n√(X¯−μ0)/σ If X¯ is large, μ0 and σ is small, the test statistic will be large and we'll be most inclined to reject.
