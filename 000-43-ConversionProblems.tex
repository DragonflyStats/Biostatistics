\documentclass[MAIN.tex]{subfiles}

% Load any packages needed for this document
\begin{document}
	\section{The Conversion Problem}
	
	In this section, we will reconsider the conversion problem, where by the methods of measurements are denominated in different units.
	Conversion problems arise when the comparison is between two 
	approximate methods of measurement each of which measures the quantity in different units.
	
	This situation can arise when the methods in question proceed by measuring different proxies for the underlying 
	quantity of interest. (lewis 1991)
	
	For the single measurement case, the author can not foresee any scope for insights that are not already offered by using a structural relation model, as proposed by lewis et 1991, or error-in-variables regression. 
	In the case of orthonormal regression, it is not reasonable to assume that both methods have equal measurement variance, when they are denominated in different units.
	The analyst may attempt to mitigate the problem by scaling the variance of one method, but even still problems remain.
	Similarly for Deming regression, no further insights on how to properly estimate the variance ratio can be offered.
	
	For the case of conversion problem with replicate measurements, a framework that incorporates the ideas offered by Roy (2009) can be proposed. Estimates for between-subject and within-subject variances may be sought.
	However Roy's tests on variability are no longer applicable, as one would not expect the method to have similar estimates. An estimate for the scaling factor $\beta$ may be sought, where $Y_i \approx \beta X$.
	
	
	\[ X_i = \tau_i + \delta_i \]
	\[ Y_i = \alpha + \beta X \tau_i + \epsilon_i\]
	
	
	We will simulate a data set based in lewis conversion problems, provide three replicates values for both measurements. To acheive this we add ``jitter noise" to three copies of each original measurement.
	
	\section{Other Types of Studies}
	\citet{lewis} categorize method comparison studies into three
	different types.  The key difference between the first two is
	whether or not a `gold standard' method is used. In situations
	where one instrument or method is known to be `accurate and
	precise', it is considered as the`gold standard' \citep{lewis}. A
	method that is not considered to be a gold standard is referred to
	as an `approximate method'. In calibration studies they are
	referred to a criterion methods and test methods respectively.
	
	
	\textbf{1. Calibration problems}. The purpose is to establish a
	relationship between methods, one of which is an approximate
	method, the other a gold standard. The results of the approximate
	method can be mapped to a known probability distribution of the
	results of the gold standard \citep{lewis}. (In such studies, the
	gold standard method and corresponding approximate method are
	generally referred to a criterion method and test method
	respectively.) \citet*{BA83} make clear that their methodology is
	not intended for calibration problems.
	
	\bigskip \textbf{2. Comparison problems}. When two approximate
	methods, that use the same units of measurement, are to be
	compared. This is the case which the Bland-Altman methodology is
	specfically intended for, and therefore it is the most relevant of
	the three.
	
	\bigskip \textbf{3. Conversion problems}. When two approximate
	methods, that use different units of measurement, are to be
	compared. This situation would arise when the measurement methods
	use 'different proxies', i.e different mechanisms of measurement.
	\citet{lewis} deals specifically with this issue. In the context
	of this study, it is the least relevant of the three.
	
	\citet[p.47]{DunnSEME} cautions that`gold standards' should not be
	assumed to be error free. `It is of necessity a subjective
	decision when we come to decide that a particular method or
	instrument can be treated as if it was a gold standard'. The
	clinician gold standard , the sphygmomanometer, is used as an
	example thereof.  The sphygmomanometer `leaves considerable room
	for improvement' \citep{DunnSEME}. \citet{pizzi} similarly
	addresses the issue of glod standards, `well-established gold
	standard may itself be imprecise or even unreliable'.
	
	
	The NIST F1 Caesium fountain atomic clock is considered to be the
	gold standard when measuring time, and is the primary time and
	frequency standard for the United States. The NIST F1 is accurate
	to within one second per 60 million years \citep{NIST}.
	
	Measurements of the interior of the human body are, by definition,
	invasive medical procedures. The design of method must balance the
	need for accuracy of measurement with the well-being of the
	patient. This will inevitably lead to the measurement error as
	described by \citet{DunnSEME}. The magnetic resonance angiogram,
	used to measure internal anatomy,  is considered to the gold
	standard for measuring aortic dissection. Medical test based upon
	the angiogram is reported to have a false positive reporting rate
	of 5\% and a false negative reporting rate of 8\%. This is
	reported as sensitivity of 95\% and a specificity of 92\%
	\citep{ACR}.
	
	In literature they are, perhaps more accurately, referred to as
	`fuzzy gold standards' \citep{phelps}. Consequently when one of the methods is
	essentially a fuzzy gold standard, as opposed to a `true' gold
	standard, the comparison of the criterion and test methods should
	be consider in the context of a comparison study, as well as of a
	calibration study.

\newpage

	\section{Lewis Conversion} 
	While regarding a comparison of two pump meters under operational conditions
	
	‘..It is suspected that the various assumptions made by each method are weak under operational conditions’
	Lewis listed several sources of variation that relate to the practical aspects of each measurement method.
	
	‘There is little reasons to believe that the laboratory conditions of the devise provide a suitable basis for the conversion of data gathered under operational conditions.
	
	
	%-------------------------------------------------------------%
	Latent variables are variables that are not measured (i.e. not observed) but whose values is observed from other observed variables. One advantage of using latent variables is that it reduces the dimensionality of data. A large number of observable variables can be aggregated in a model to represent an underlying concept, making it easier for humans to understand the data.	[wikipedia]

\bibliographystyle{chicago}
\bibliography{DB-txfrbib}
\end{document}