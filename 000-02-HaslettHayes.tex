\documentclass[12pt, a4paper]{article}
\usepackage{natbib}
\usepackage{vmargin}
\usepackage{graphicx}
\usepackage{epsfig}
\usepackage{subfigure}
%\usepackage{amscd}
\usepackage{amssymb}
\usepackage{subfiles}
\usepackage{subfigure}
\usepackage{framed}
\usepackage{subfiles}
\usepackage{amsbsy}
\usepackage{amsthm, amsmath}
%\usepackage[dvips]{graphicx}
\bibliographystyle{chicago}
\renewcommand{\baselinestretch}{1.1}

% left top textwidth textheight headheight % headsep footheight footskip
\setmargins{3.0cm}{2.5cm}{15.5 cm}{23.5cm}{0.25cm}{0cm}{0.5cm}{0.5cm}

\pagenumbering{arabic}
%-------------------------------------------------------------------Simplifying GLS by KH -%


\begin{document}




\section*{Haslett and Hayes - Residuals}
Haslett and Hayes (1998) and Haslett (1999) considered the case of an LME model with correlated covariance structure.

\section{Haslett's Analysis} %2.5
For fixed effect linear models with correlated error structure Haslett (1999) showed that the effects on
the fixed effects estimate of deleting each observation in turn could be cheaply computed from the fixed effects model predicted residuals.



A general theory is presented for residuals from the general linear model with correlated errors. 
It is demonstrated that there are two fundamental types of residual associated with this model, 
referred to here as the marginal and the conditional residual. 

These measure respectively the distance to the global aspects of the model as represented by the expected value 
and the local aspects as represented by the conditional expected value. 

These residuals may be multivariate. 

\citet{HaslettHayes} developes some important dualities which have simple implications for diagnostics. 

%The results are illustrated by reference to model diagnostics in time series and in classical multivariate analysis with independent cases.
%------------------------------------------------------------%

	
	% Haslett Dillane
	%==================================================================%
	Haslett \& Dillane (199X) offers an
	procedure to assess the influences for the variance components
	within the linear model, complementing the existing methods for
	the fixed components. 
	
	
	The essential problem is that there is no
	useful updating procedures for $\hat{V}$, or for $\hat{V}^{-1}$.
	Haslett \& Dillane (199X) propose an alternative , and
	computationally inexpensive approach, making use of the
	`\texttt{delete=replace}' identity.
	
	\citet{Haslett99} considers the effect of `leave k out'
	calculations on the parameters $\beta$ and $\sigma^{2}$, using
	several key results from \citet{HaslettHayes} on partioned
	matrices.
	


	
	%%% Haslett \& Dillane (19XX) }
	
	Haslett \& Dillane (19XX) remark that linear mixed effects models
	didn't experience a corresponding growth in the use of deletion
	diagnostics, adding that \citet{McCullSearle} makes no mention of
	diagnostics whatsoever.















\bibliographystyle{chicago}
\bibliography{DB-txfrbib}
\end{document}
