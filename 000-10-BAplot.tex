\chapter{The Bland Altman Plot}
\section{Bland Altman Plots}
The issue of whether two measurement methods are comparable to the
extent that they can be used interchangeably with sufficient
accuracy is encountered frequently in scientific research.
Historically comparison of two methods of measurement was carried
out by use of matched pairs correlation coefficients or simple
linear regression. Bland and Altman recognized the inadequacies of
these analyses and articulated quite thoroughly the basis on which
of which they are unsuitable for comparing two methods of
measurement \citep*{BA83}.

As an alternative they proposed a simple statistical methodology
specifically appropriate for method comparison studies. They
acknowledge that there are other valid methodologies, but argue
that a simple approach is preferable to complex approaches,
\emph{"especially when the results must be explained to
	non-statisticians"} \citep*{BA83}.

The first step recommended which the authors argue should be
mandatory is construction of a simple scatter plot of the data.
The line of equality ($X=Y$) should also be shown, as it is
necessary to give the correct interpretation of how both methods
compare. A scatter plot of the Grubbs data is shown in figure 2.1.
A visual inspection thereof confirms the previous conclusion that
there is an inter method bias present, i.e. Fotobalk device has a
tendency to record a lower velocity.



In light of shortcomings associated with scatterplots,
\citet*{BA83} recommend a further analysis of the data. Firstly
differences of measurements of two methods on the same subject
should  be calculated, and then the average of those measurements
(Table 1.1). The averages of the two measurements is considered by
Bland and Altman to the best estimate for the unknown true value.
Importantly both methods must measure with the same units. These
results are then plotted, with differences on the ordinate and
averages on the abscissa (figure 1.2). \citet*{BA83}express the
motivation for this plot thusly:
\begin{quote}
	"From this type of plot it is much easier to assess the magnitude
	of disagreement (both error and bias), spot outliers, and see
	whether there is any trend, for example an increase in
	(difference) for high values. This way of plotting the data is a
	very powerful way of displaying the results of a method comparison
	study."
\end{quote}
\newpage
% latex table generated in R 2.6.0 by xtable 1.5-5 package
% Thu Aug 27 16:31:52 2009
\begin{table}[tbh]
	\begin{center}
		
		\begin{tabular}{|c|c|c|c|c|}
			\hline
			Round & Fotobalk [F] & Counter [C] & Differences [F-C] & Averages [(F+C)/2] \\
			\hline
			1 & 793.80 & 794.60 & -0.80 & 794.20 \\
			2 & 793.10 & 793.90 & -0.80 & 793.50 \\
			3 & 792.40 & 793.20 & -0.80 & 792.80 \\
			4 & 794.00 & 794.00 & 0.00 & 794.00 \\
			5 & 791.40 & 792.20 & -0.80 & 791.80 \\
			6 & 792.40 & 793.10 & -0.70 & 792.80 \\
			7 & 791.70 & 792.40 & -0.70 & 792.00 \\
			8 & 792.30 & 792.80 & -0.50 & 792.50 \\
			9 & 789.60 & 790.20 & -0.60 & 789.90 \\
			10 & 794.40 & 795.00 & -0.60 & 794.70 \\
			11 & 790.90 & 791.60 & -0.70 & 791.20 \\
			12 & 793.50 & 793.80 & -0.30 & 793.60 \\
			\hline
		\end{tabular}
		\caption{Fotobalk and Counter Methods: Differences and Averages}
	\end{center}
\end{table}




\subsection{Repeated Measurements }
In cases where there are repeated measurements by each of the two
methods on the same subjects , Bland Altman suggest calculating
the mean for each method on each subject and use these pairs of
means to compare the two methods.
\\
The estimate of bias will be unaffected using this approach, but
the estimate of the standard deviation of the differences will be
too small, because of the reduction of the effect of repeated
measurement error. Bland Altman propose a correction for this.
\\
Carstensen attends to this issue also, adding that another
approach would be to treat each repeated measurement separately.







\subsection{Introduction}
\begin{itemize}
	\item Comparing two methods of measurement is normally done by computing limits of agreement (LoA), i.e. prediction limits for
	a future difference between measurements with the two methods. When the difference is not constant it is not clear what
	this means, since the difference between the methods depends on the average; hence, unlike the case where the difference is
	constant, LoA cannot directly be translated into a prediction interval for a measurement by one method given that of another.
	\item The main point in the paper by Bland and Altman [1] is however different from the outlook in this paper; Bland and Altman
	mainly discuss whether two methods of measurement can be used interchangeably and how to assess this with the help of
	proper statistical methods to derive LoA, i.e. prediction limits for differences between two methods.
	This paper takes as starting point that the classical LoA can be converted to a prediction interval for one method given a
	measurement by the other (details in the next section). This sort of relationship can be shown in a plot as a line with slope 1
	and prediction limits as lines also with slope 1; applicable for the prediction both from method 1 to method 2 and vice versa. In
	the case of non-constant difference it would be desirable to be able to produce a similar plot, usable both ways. Thus, the aim
	of this paper is to produce a conversion from one method to another that also applies in the case where the difference between
	methods is not constant.
	\item In this paper, I set up a proper model for data for method comparison studies which in the case of constant difference between
	methods leads to the classical LoA, and in the case of linear bias gives a simple formula for the prediction. The paper only
	addresses the situation where only one measurement by each method is available, although replicate measurements by each
	method are desirable whenever possible [2]. Moreover, the situation with non-constant variance over the range of measurements
	is not covered either.
\end{itemize}
\newpage
\subsection{Discussion}
%----------------------------------------------------------------------------------------------------------------%
I have here proposed a simple twist to the results from regression of the differences on the sums in the case of a linear relationship
between two methods of measurement. It is consistent with the obvious underlying model, and exploits the fact that although
the parameters of the model cannot be estimated, those functions of the parameters that are needed for creating predictions
can be estimated.
%----------------------------------------------------------------------------------------------------------------%
The prediction limits provided have the attractive property that if the prediction line with limits is drawn in a coordinate
system, the chart will apply in both ways; hence, both the line and the limits are symmetric. Precisely as the prediction intervals
derived from the classical LoA are in the case where the difference between methods is constant.
%----------------------------------------------------------------------------------------------------------------%
The drawback is that the regression of the differences on the means ignores that the averages are correlated with the residuals
(i.e. the error terms), and therefore gives biased estimates if the slope linking the two methods is far from 1 or the residual
variances are very different. However, both of these are rather uncommon in method comparison studies, so the method proposed
here is widely applicable.
%----------------------------------------------------------------------------------------------------------------%
When considering LoA, the only feasible transformation is the log-transform, which gives LoA for the ratio of measurements,
which is immediately understandable. If, for example, the measurements are fractions where some are close to either 0 or 1 a
logit transform may be adequate. 

LoA would then be for (log) odds-ratios, not very easily understood. For other more arbitrarily
chosen transformation the situation may be even worse. But if a plot with conversion lines and limits are constructed, then the
plot is readily back-transformed to the original scale for practical use.
%---------------------------------------------------------------------------------

\subsection{Distribution of Maxima} It is possible to use Order
Statistics theory to assess conditional probabilities. With two
random variables $T_{0}$ and $T_{1}$, we define two variables $Z$
and $W$ such that they take the maximum and minimum values of the
pair of $T$ values.\subsection{Plot of the Maxima against the
	Minima}


In Figure 1,  The Maximas are plotted against their corresponding
minima. The Critical values of the Maxima and Minima are displayed
in the dotted lines.The Line of Equality depicts the obvious
logical constraint of the each Maximum value being greater than
its corresponding minimum value.



The scientific question at hand is the correct approach to
assessing whether two methods can be used interchangeably.
\citet{BA99} expresses this as follows:
\begin{quote}We want to
	know by how much (one) method is likely to differ from the
	(other), so that if it not enough to cause problems in the
	mathematical interpretation we can ... use the two
	interchangeably.
\end{quote}



Consequently, of the categories of method comparison study,
comparison studies, the second category, is of particular
importance, and the following discussion shall concentrate upon
it. Less emphasis shall be place on the other three categories.

\bigskip Further to \citet{BA86}, 'equivalence' of two methods expresses
that both can be used interchangeably.
\citet[p.49]{DunnSEME} remarks that this is a very restrictive
interpretation of equivalence, and that while agreement indicated
equivalence, equivalence does not necessarily reflect agreement.

The main difference between Myers proposed method and the Bland
Altman is that the random effects model is used to estimate the
within-subject variance after adjusting for known and unknown
variables. The Bland Altman approach uses one way analysis of
variance to estimate the within subject variance. In general, the
random effects model is an extension of the analysis of the ANOVA
method and it can adjust for many more covariates than the ANOVA
method






\newpage
\section{Conclusions about Existing Methodologies}

Scatterplots are recommended by \citet{BA83} for an initial
examination of the data, facilitating an initial judgement and
helping to identify potential outliers. They are not useful for a
thorough examination of the data. \citet{BritHypSoc} notes that
data points will tend to cluster around the line of equality,
obscuring interpretation.


The Bland Altman methodology is well noted for its ease of use,
and can be easily implemented with most software packages. Also it
doesn't require the practitioner to have more than basic
statistical training. The plot is quite informative about the
variability of the differences over the range of measurements. For
example, an inspection of the plot will indicate the 'fan effect'.
They also can be used to detect the presence of an outlier.

\citet{ludbrook97,ludbrook02}criticizes these plots on the
basis that they presents no information on effect of constant bias
or proportional bias. These plots are only practicable when both
methods measure in the same units. Hence they are totally
unsuitable for conversion problems. The limits of agreement are
somewhat arbitrarily constructed. They may or may not be suitable
for the data in question. It has been found that the limits given
are too wide to be acceptable. There is no guidance on how to deal
with outliers. Bland and Altman recognize effect they would have
on the limits of agreeement, but offer no guidance on how to
correct for those effects.

There is no formal testing procedure provided. Rather, it is upon
the practitioner opinion to judge the outcome of the methodology.






%%%%%%%%%%%%%%%%%%%%%%%%%%%%%%%%%%%%%%%%%%%%%%%%%%%%%%%%%%%%%%%%%%%%%%%%%
%9 Appendix                  %%%%%%%%%%%%%%%%%%%%%%%%%%%%%%%%%%%%%%%%%%%%%
%%%%%%%%%%%%%%%%%%%%%%%%%%%%%%%%%%%%%%%%%%%%%%%%%%%%%%%%%%%%%%%%%%%%%%%%%
\section{The Bland Altman Plot}
In 1986 Bland and Altman published a paper in the Lancet proposing
the difference plot for use for method comparison purposes. It has
proved highly popular ever since. This is a simple, and widely
used , plot of the differences of each data pair, and the
corresponding average value. An important requirement is that the
two measurement methods use the same scale of measurement.
\\
Variations of the Bland Altman plot is the use of ratios, in the
place of differences.
\begin{equation}
D_{i} = X_{i} - Y_{i}   \label{BA01}
\end{equation}
Altman and Bland suggest plotting the within subject differences $
D = X_{1} - X_{2} $ on the ordinate versus the average of $x_{1}$
and  $x_{2}$ on the abscissa.




\subsection{Treatment of Outliers}
Bland and Altman attend to the issue of outliers in their 1986
paper, wherein they present a data set with an extreme outlier

\section{Bland Altman Plots In Literature}
\citet{mantha} contains a study the use of Bland Altman plots of
44 articles in several named journals over a two year period. 42
articles used Bland Altman's limits of agreement, wit the other
two used correlation and regression analyses. \citet{mantha}
remarks that 3 papers, from 42 mention predefined maximum width
for limits of agreement which would not impair medical care.

The conclusion of \citet{mantha} is that there are several
inadequacies and inconsistencies in the reporting of results ,and
that more standardization in the use of Bland Altman plots is
required. The authors recommend the prior determination of limits
of agreement before the study is carried out. This contention is
endorsed by \citet{lin}, which makes a similar recommendation for
the sample size, noting that\emph{sample sizes required either was
	not mentioned or no rationale for its choice was given}.

\begin{quote}
	In order to avoid the appearance of "data dredging", both the
	sample size and the (limits of agreement) should be specified and
	justified before the actual conduct of the trial. \citep{lin}
\end{quote}

\citet{Dewitte} remarks that the limits of agreement should be
compared to a clinically acceptable difference in measurements.
%%%%%%%%%%%%%%%%%%%%%%%%%%%%%%%%%%%%%%%%%%%%%%%%%%%%%%%%%%%%%%%%%%%%%%%%%
%4 Inappropriate assessment of Agreement       %%%%%%%%%%%%%%%%%%%%%%%%%%
%%%%%%%%%%%%%%%%%%%%%%%%%%%%%%%%%%%%%%%%%%%%%%%%%%%%%%%%%%%%%%%%%%%%%%%%%


\subsection{Gold Standard} This is considered to be the most
accurate measurement of a particular parameter.


