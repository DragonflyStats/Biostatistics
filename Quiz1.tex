Question 1
In a random sample of 100 patients at a clinic, you would like to test whether the mean RDI is x or more using a one sided 5% type 1 error rate. The sample mean RDI had a mean of 12 and a standard deviation of 4. What value of x (testing H0:μ=x versus Ha:μ>x) would you reject for?
Your Answer		Score	Explanation
Any value of about 11.3 or lower			
Any value of about 11.26 or lower			
Any value of about 11.26 or higher			
Any value of about 11.3 or higher	Inorrect	0.00	
Total		0.00 / 1.00	
Question Explanation

We will be more likely to reject for smaller values of x. Thus, the question is, what's the smallest value of x for which we would reject? The boundary occurs at 100−−−√(12−x)/4=1.645 or 12 - 1.645 * 4 / 10 = 11.34
Question 2
A pharmaceutical company is interested in testing a potential blood pressure lowering medication. Their first examination considers only subjects that received the medication at baseline then two weeks later. The data are as follows (SBP in mmHg)

Baseline	Week 2
140	138
138	136
150	148
148	146
135	133
Test the hypothesis that there was a mean reduction in blood pressure. Compare the difference between a paired and unpaired test for a two sided 5% level test.

Your Answer		Score	Explanation
Reject for the paired; fail to reject for the unpaired	Correct	1.00	
Reject for the paired; reject for the unpaired			
Fail to reject for the paired; reject for the unpaired			
Fail to reject for the paired; fail to reject for the unpaired			
Total		1.00 / 1.00	
Question Explanation

bl <- c(140, 138, 150, 148, 135)
fu <- c(138, 136, 148, 146, 133)
bl - fu
[1] 2 2 2 2 2
t.test(fu, bl, alternative = "two.sided", paired = FALSE)

    Welch Two Sample t-test

data:  fu and bl 
t = -0.4868, df = 8, p-value = 0.6395
alternative hypothesis: true difference in means is not equal to 0 
95 percent confidence interval:
 -11.474   7.474 
sample estimates:
mean of x mean of y 
    140.2     142.2 
Notice that the Pvalue is 0 for the paired test

Question 3
Brain volumes for 9 men yielded a 90 % confidence interval of 1,077 cc to 1,123 cc. Would you reject in a two sided 5% hypothesis test of H0:μ=1,078?
Your Answer		Score	Explanation
You would fail to reject the null hypothesis.			
It can not be ascertained from the information given.			
You would reject the null hypothesis.	Inorrect	0.00	
Total		0.00 / 1.00	
Question Explanation

No, you would fail to reject. The 95% interval would be wider than the 90% interval. Since 1,078 is in the narrower 90% interval, it would also be in the wider 95% interval. Thus, in either case it's in the interval and so you would fail to reject.
Question 4
In an effort to improve efficiency, hospital administrators are evaluating a new triage system for their emergency room. In an validation study of the system, 5 patients were tracked in a mock (simulated) ER under the new and old triage system. Their waiting times in natural log of hours were:

Subject	1	2	3	4	5
New	0.929	-1.745	1.677	0.701	0.128
Old	2.233	-2.513	1.204	1.938	2.533
Give a Pvalue for the test of the hypothesis that the new system resulted in lower waiting times for a one sided t test.

Your Answer		Score	Explanation
0.1405			
0.2597			
0.5194	Inorrect	0.00	
0.281			
Total		0.00 / 1.00	
Question Explanation

We'll use a paired t test

new <- c(0.929, -1.745, 1.677, 0.701, 0.128)
old <- c(2.233, -2.513, 1.204, 1.938, 2.533)
t.test(new - old, alternative = "less")

    One Sample t-test

data:  new - old 
t = -1.245, df = 4, p-value = 0.1405
alternative hypothesis: true mean is less than 0 
95 percent confidence interval:
   -Inf 0.5277 
sample estimates:
mean of x 
   -0.741 
Here's some of the incorrect answers

t.test(new, old, alternative = "two.sided", paired = TRUE)

    Paired t-test

data:  new and old 
t = -1.245, df = 4, p-value = 0.281
alternative hypothesis: true difference in means is not equal to 0 
95 percent confidence interval:
 -2.3932  0.9112 
sample estimates:
mean of the differences 
                 -0.741 
t.test(new, old, alternative = "less", paired = FALSE)

    Welch Two Sample t-test

data:  new and old 
t = -0.6799, df = 6.704, p-value = 0.2597
alternative hypothesis: true difference in means is less than 0 
95 percent confidence interval:
  -Inf 1.338 
sample estimates:
mean of x mean of y 
    0.338     1.079 
t.test(new, old, alternative = "two.sided", paired = FALSE)

    Welch Two Sample t-test

data:  new and old 
t = -0.6799, df = 6.704, p-value = 0.5194
alternative hypothesis: true difference in means is not equal to 0 
95 percent confidence interval:
 -3.341  1.859 
sample estimates:
mean of x mean of y 
    0.338     1.079 
Question 5
Refer to the previous question. Give a 95% T confidence interval for the ratio of the waiting times (recall that the measurements were natural logged).

Here's the data and setting again. 

In an effort to improve efficiency, hospital administrators are evaluating a new triage system for their emergency room. In an validation study of the system, 5 patients were tracked in a mock (simulated) ER under the new and old triage system. Their waiting times in natural log of hours were:

Subject	1	2	3	4	5
New	0.929	-1.745	1.677	0.701	0.128
Old	2.233	-2.513	1.204	1.938	2.533
Your Answer		Score	Explanation
-2.01 to 0.528			
-2.39 to 0.91			
0.134 to1.69			
.09 to 2.49	Correct	1.00	
Total		1.00 / 1.00	
Question Explanation

Refer to the previous solution
exp(t.test(new - old)$conf.int)
[1] 0.09133 2.48743
attr(,"conf.level")
[1] 0.95
Question 6
Suppose that 18 obese subjects were randomized, 9 each, to a new diet pill and a placebo. Subjects’ body mass indices (BMIs) were measured at a baseline and again after having received the treatment or placebo for four weeks. The average difference from follow-up to the baseline (followup - baseline) was −3 kg/m2 for the treated group and 1 kg/m2 for the placebo group. The corresponding standard deviations of the differences was 1.5 kg/m2 for the treatment group and 1.8 kg/m2 for the placebo group. Does the change in BMI over the two year period appear to differ between the treated and placebo groups? Assuming normality of the underlying data and a common population variance, give a pvalue for a two sided t test.
Your Answer		Score	Explanation
Around 0.00005	Inorrect	0.00	
Around 0.01			
Around 0.1			
Around 0.001			
Around 0.0001			
Total		0.00 / 1.00	
Question Explanation

n1 <- n2 <- 9
x1 <- -3  ##treated
x2 <- 1  ##placebo
s1 <- 1.5  ##treated
s2 <- 1.8  ##placebo
s <- sqrt(((n1 - 1) * s1^2 + (n2 - 1) * s2^2)/(n1 + n2 - 2))
ts <- (x1 - x2)/(s * sqrt(1/n1 + 1/n2))
2 * pt(ts, n1 + n2 - 2)
[1] 0.0001025
Question 7
Consider a one sided α level single group Z test of H0:μ=μ0 versus Ha:μ>μ0 with the data X¯ for the sample mean and s for the sample standard deviation with n measurements. What are the collection of points for which you would fail to reject the hypothesis?
Your Answer		Score	Explanation
[X+Z1−αs/n√,∞)	Inorrect	0.00	
[X−Z1−αs/n√,∞)			
(−∞,X−Z1−αs/n√]			
The confidence interval X¯±Z1−α/2s/n√			
(−∞,X+Z1−αs/n√]			
Total		0.00 / 1.00	
Question Explanation

We will fail to reject if n√(X¯−μ0)/s≤Z1−α. Thus if μ0≥X¯−Z1−αs/n√ or [X−Z1−αs/n√,∞).

Question 8
Researchers would like to conduct a study of n healthy adults to detect a four year mean brain volume loss of .01 mm3. Assume that the standard deviation of four year volume loss in this population is .04 mm3. What would be the value of n needded for 90% power of type one error rate of 5% one sided test versus a null hypothesis of no volume loss?

Your Answer		Score	Explanation
Around 140			
Around 20			
Around 50	Inorrect	0.00	
Around 100			
Total		0.00 / 1.00	
Question Explanation

The hypothesis is H0:μΔ=0 versus Ha:μΔ>0 where μΔ is volume loss (change defined as Baseline - Four Weeks). The test statistics is X¯Δ.04/n√ which is rejected if it is larger than Z.95=1.645.
We want to calculate
P(X¯ΔσΔ/n√>1.645 | μΔ=.01)=P(X¯Δ−.01.04/n√>1.645−.01.04/n√ | μΔ=.01)=P(Z>1.645−n√/4)=.90
So we need 1.645−n√/4=Z.10=−1.282 and thus n=(4∗(1.645+1.282))2.

ceiling((4 * (qnorm(0.95) - qnorm(0.1)))^2)
[1] 138
Question 9
The Daily Planet ran a recent story about Kryptonite poisoning in the water supply after a recent event in Metropolis. Their usual field reporter, Clark Kent, called in sick and so Lois Lane reported the stories. Researchers plan to sample 288 individuals from Metropolis and control city Gotham and will compare mean blood Kryptonite levels (in Lex Luthors per milliliter, LL/ml). The expect to find a mean difference in LL/ml of around 2. Assoming a two sided Z test of the relevant hypothesis at 5%, what would be the power. Assume that the standard deviation is 12 for both groups.

Your Answer		Score	Explanation
Around 70%			
Around 40%	Inorrect	0.00	
Around 90%			
Around 20%			
Around 10%			
Around 30%			
Around 80%			
Around 60%			
Around 50%			
Total		0.00 / 1.00	
Question Explanation

H0:μMetropolis=μGotham versus Ha:μMetropolis≠μGotham.

Let
ts=X¯Metropolis−X¯Gothamσ(1nMetropolis+1nGotham)½
We will reject if abs(ts)>1.96. Thus we want
P(|ts|>1.96)=P(|X¯Metropolis−X¯Gotham|>1.96∗σ(1nMetropolis+1nGotham)½)
where the probability is calculated under the alternative. Note σ(1nMetropolis+1nGotham)½=1. Under Ha X¯Metropolis−X¯Gotham is N(2,1).

pnorm(-1.96, mean = 2, sd = 1) + pnorm(1.96, mean = 2, sd = 1)
[1] 0.4841
Note the first probability is effectively 0, so people usually don't even bother calculating it.

Question 10
As you increase the type one error rate, α, what happens to power?
Your Answer		Score	Explanation
It is impossible to say			
The power goes down			
The power goes up			
Power stays the same			
Total		0.00 / 1.00	
Question Explanation

As you require less evidence to reject, i.e. your α rate goes up, you will have larger power.
Question 11
Consider a setting with iid data from a N(μ,σ2) distribution testing H0:μ=0 verus Ha:μ>0. The null hypothesis is rejected if n√X¯/σ>1.645. What happens to the type I error rate as n goes to infinity?

Your Answer		Score	Explanation
It is 5% regardless of the size of n	Correct	1.00	
It limits to 5%, but can be a different value for small sample sizes			
It is 10% regardless of the size of n			
It limits to 10%, but can be a different value for small sample sizes			
Total		1.00 / 1.00	
Question Explanation

The type 1 error rate is set to be .05 by choosing the value 1.645. Since the data are assumed Gaussian with a known variance, the type one error rate is not approximate by the CLT. Thus it is .05 regardless of the size of n.

Question 12
Suppose that you have three independent samples from a N(μ1,σ2), N(μ2,σ2) and N(μ3,σ2) respectively of size n1, n2 and n3. Let S21, S22 and S23 be the associated sample variances. Define the pooled variance as
S2p=(n1−1)S21+(n2−1)S22+(n3−1)S23n1+n2+n3−3
Consider testing H0:aμ1+bμ2+cμ3=0. Let
TS=aX¯1+bX¯2+cX¯3Sp(a2n1+b2n2+c2n3)1/2
What distribution do you think TS has under H0?

Your Answer		Score	Explanation
A T distribution with n1+n2+n3−3 degrees of freedom.	Correct	1.00	
Standard normal.			
Approximately T distributed			
Chi squared with n1+n2+n3−3 df			
Total		1.00 / 1.00	
Question Explanation

Note that (n1+n2+n3)S2p/σ2 is Chi Squared with n1+n2+n3 df, as it is the sum of three independent Chi squared rvs. Also aX¯1+bX¯2+cX¯3 is N(0,σ2(a2n1+b2n2+c2n3)) under H0. Thus putting these two together we see that TS is standard normal divided by a Chi squared divided by its degrees of freedom. (With some hand waiving about independence.)

Question 13
Consider a one sample Z test of $H_0 : \mu = \mu_0$ versus $H_a : \mu > \mu_0$. All else equal, which scenarios will be closer to rejecting the null hypothesis?
Your Answer		Score	Explanation
X¯ is small, μ0 is large, σ is small			
X¯ is large, μ0 is large, σ is small			
X¯ is small, μ0 is small, σ is small			
X¯ is large, μ0 is small, σ is large	Inorrect	0.00	
X¯ is large, μ0 is large, σ is large			
X¯ is small, μ0 is large, σ is large			
X¯ is large, μ0 is small, σ is small			
X¯ is small, μ0 is small, σ is large			
Total		0.00 / 1.00	
Question Explanation

Confusing, but conceptually easy problem. The test statistic is n√(X¯−μ0)/σ If X¯ is large, μ0 and σ is small, the test statistic will be large and we'll be most inclined to reject.
