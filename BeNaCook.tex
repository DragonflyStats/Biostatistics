% Beckman Nachtsheim Cook
% Diagnostics for Mixed-Model Analysis of Variance

% Abstract:
% We describe a new method for assessment of model inadequacy in maximum-likelihood mixed-model analysis of variance. 
% In particular, we discuss its use in diagnosing perturbations from the usual assumption of constant error variance and 
% from the assumption that each realization of a given random factor has been drawn from the same normal population. 
% Computer implementation of the procedure is described, and an example is presented, involving the analysis of filter 
% cartridges used with commercial respirators.
% --------------------------------------------------------- %


% Local Influence Approach
%--------------------------------------------------------- %



% --------------------------------------------------------- %

Cook (1986) gave a completely general method for assessing influence of local departures from
assumptions in statistical models.

% Local Influence Approach

We describe a new method for assessment of model inadequacy in maximum-likelihood mixed-model analysis of variance. In particular, we discuss its use in diagnosing perturbations from the usual assumption of constant error variance and from the assumption that each realization of a given random factor has been drawn from the same normal population. Computer implementation of the procedure is described, and an example is presented, involving the analysis of filter cartridges used with commercial respirators.

\newpage
%---------------------------------------------------------- %
%Likelihood Displacement.
\[  LD(\boldsymbol{\omega}= 2[ L\boldsymbol{\hat{theta}} - \boldsymbol{\hat{theta}_\omega} \]

Large values indicate that $\boldsymbol{\hat{theta}}$ and $\boldsymbol{\hat{theta}_\omega}$ differ considerably.
  
\end{document}