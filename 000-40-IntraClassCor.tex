\section*{Intraclass Correlation as a Measure of Agreement}

% http%3A%2F%2Fwww.springer.com%2Fcda%2Fcontent%2Fdocument%2Fcda_downloaddocument%2F9783642371301-c2.pdf%3FSGWID%3D0-0-45-1402889-p174984284&ei=ik5oUo33DYW70QW0jICIBA&usg=AFQjCNHtxad27T-bNsQhZDC5jbiEkoTjJQ&sig2=Ro19Swq6aJ6IAu96YfvKoQ

Intraclass correlation is the strength of a linear relationship between subjects belonging to the same class or the same subgroup or the same family. In the agreement setup,
the two measurements obtained on the same subject by two observers or two methods is a subgroup. If they agree, the intraclass correlation will be high. This method of
assessing an agreement was advocated by Lee et al. (1989).

In the usual correlation setup, the values of two different variables are obtained on a series of subjects. For example, you can have the weight and height of 20 girls aged 5–7 years. You can also have the weight of the father and mother of 30 low
birthweight newborns. Both are weights and the product–moment correlation coefficient is a perfectly valid measure of the strength of the relationship in this case.
Now consider the weight of 15 persons obtained on two machines. 

Any person, say number 7, may be measured by machine 2 first and then by machine 1. Others may be measured by machine 1 then by machine 2. The order does not matter in this setup as
the interest is in finding whether the values are in agreement or not.

Statistically, intraclass correlation is that part of the total variance that is accounted for by the differences in the paired measurements obtained by two
methods.

%-------------------------------------------------%
