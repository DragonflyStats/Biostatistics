\documentclass[Main.tex]{subfiles}
\begin{document}
\newpage
\section{Influence analysis} %1.7

Likelihood based estimation methods, such as ML and REML, are sensitive to unusual observations. Influence diagnostics are formal techniques that assess the influence of observations on parameter estimates for $\beta$ and $\theta$. A common technique is to refit the model with an observation or group of observations omitted.

\citet{west} examines a group of methods that examine various aspects of influence diagnostics for LME models.
For overall influence, the most common approaches are the `likelihood distance' and the `restricted likelihood distance'.

\subsection{Cook's 1986 paper on Local Influence}%1.7.1
Cook 1986 introduced methods for local influence assessment. These methods provide a powerful tool for examining perturbations in the assumption of a model, particularly the effects of local perturbations of parameters of observations.

The local-influence approach to influence assessment is quitedifferent from the case deletion approach, comparisons are of
interest.

%---------------------------------------------------------------------------%

\subsection{Overall Influence}
An overall influence statistic measures the change in the objective function being minimized. For example, in
OLS regression, the residual sums of squares serves that purpose. In linear mixed models fit by
\index{maximum likelihood} maximum likelihood (ML) or \index{restricted maximum likelihood} restricted maximum likelihood (REML), an overall influence measure is the \index{likelihood distance} likelihood distance [Cook and Weisberg ].

%--------------------------------------------------------------------------%
\newpage
\section{Terminology for Case Deletion diagnostics} %1.8

\citet{preisser} describes two type of diagnostics. When the set consists of only one observation, the type is called
'observation-diagnostics'. For multiple observations, Preisser describes the diagnostics as 'cluster-deletion' diagnostics.



%---------------------------------------------------------------------------%
\newpage
\section{Cook's Distance} %1.9

\subsection{Cook's Distance}%1.19.1 
Cooks Distance ($D_{i}$) is an overall measure of the combined impact of the $i$th case of all estimated regression coefficients. It uses the same structure for measuring the combined impact of the differences in the estimated regression coefficients when the $k$th case is deleted. $D_{(k)}$ can be calculated without fitting
a new regression coefficient each time an observation is deleted.





\subsection{Cooks's Distance}%1.9.2
\index{Cook's distance} Cook's $D$ statistics (i.e. colloquially Cook's Distance) is a measure of the influence of observations in subset $U$ on a vector of parameter estimates \citep{cook77}.

\[ \delta_{(U)} = \hat{\beta} - \hat{\beta}_{(U)}\]

If V is known, Cook's D can be calibrated according to a chi-square distribution with degrees of freedom equal to the rank of $\boldsymbol{X}$ \citep{cpj92}.



\bibliography{DB-txfrbib}
\end{document}