\documentclass[12pt, a4paper]{report}
\usepackage{epsfig}
\usepackage{subfigure}
%\usepackage{amscd}
\usepackage{amssymb}
\usepackage{amsbsy}
\usepackage{amsthm}
%\usepackage[dvips]{graphicx}
\usepackage{natbib}
\bibliographystyle{chicago}
\usepackage{vmargin}
% left top textwidth textheight headheight
% headsep footheight footskip
\setmargins{3.0cm}{2.5cm}{15.5 cm}{22cm}{0.5cm}{0cm}{1cm}{1cm}
\renewcommand{\baselinestretch}{1.5}
\pagenumbering{arabic}
\theoremstyle{plain}
\newtheorem{theorem}{Theorem}[section]
\newtheorem{corollary}[theorem]{Corollary}
\newtheorem{ill}[theorem]{Example}
\newtheorem{lemma}[theorem]{Lemma}
\newtheorem{proposition}[theorem]{Proposition}
\newtheorem{conjecture}[theorem]{Conjecture}
\newtheorem{axiom}{Axiom}
\theoremstyle{definition}
\newtheorem{definition}{Definition}[section]
\newtheorem{notation}{Notation}
\theoremstyle{remark}
\newtheorem{remark}{Remark}[section]
\newtheorem{example}{Example}[section]
\renewcommand{\thenotation}{}
\renewcommand{\thetable}{\thesection.\arabic{table}}
\renewcommand{\thefigure}{\thesection.\arabic{figure}}
\title{Research notes: linear mixed effects models}
\author{ } \date{ }


\begin{document}
\author{Kevin O'Brien}
\title{Updating techniques for LME models}

\addcontentsline{toc}{section}{Bibliography}

%----------------------------------------------------------------------------------------%
\newpage
\chapter{Limits of Agreement}
\section{Computing LoAs from LME models}

\emph{
One important feature of replicate observations is that they should be independent
of each other. In essence, this is achieved by ensuring that the observer makes each
measurement independent of knowledge of the previous value(s). This may be difficult
to achieve in practice.}




\subsection{Carstensen's Model}

MORE

\subsection{Carstensen's LOAs}


Carstensen presents a model where the variation between items for
method $m$ is captured by $\sigma_m$ and the within item variation
by $\tau_m$.

Further to his model, Carstensen computes the limits of agreement
as

\[
\hat{\alpha}_1 - \hat{\alpha}_2 \pm \sqrt{2 \hat{\tau}^2 +
\hat{\sigma}^2_1 + \hat{\sigma}^2_2}
\]

\subsection{Roy's LOAs}

The limits of agreement computed by Roy's method are derived from the variance covariance matrix for overall variability.
This matrix is the sum of the between subject VC matrix and the within-subject VC matrix.

The standard deviation of the differences of methods $x$ and $y$ is computed using values from the overall VC matrix.
\[
\mbox{var}(x - y ) = \mbox{var} ( x )  + \mbox{var} ( y ) - 2\mbox{cov} ( x ,y )
\]


The respective estimates computed by both methods are tabulated as follows. Evidently there is close correspondence between both sets of estimates.

\citet{bxc2008} formulates an LME model, both in the absence and the presence of an interaction term.\citet{bxc} uses both to demonstrate the importance of using an interaction term. Failure to take the replication structure into
account results in over-estimation of the limits of agreement. For the Carstensen estimates below, an interaction term was included when computed.
\newpage



\citet{Roy2006} uses the ``Blood" data set, which featured in \citet{BA99}.


\newpage
\citet{bxc2008} describes the sampling method when discussing of a motivating example

Diabetes patients attending an outpatient clinic in Denmark have their $HbA_{1c}$ levels routinely measured at every visit.Venous and Capillary blood samples were obtained from all patients appearing at the clinic over two days.
Samples were measured on four consecutive days on each machines, hence there are five analysis days.

\citet{bxc2008} notes that every machine was calibrated every day to  the manufacturers guidelines.
Measurements are classified by method, individual and replicate. In this case the replicates are clearly not exchangeable, neither within patients nor simulataneously for all patients.


\section{Hamlett}
Hamlett re-analyses the data of \citet{lam} to generalize their model to cover other settings not covered by the Lam method.

In many cases, repeated observation are collected from each subject in sequence  and/or longitudinally.


\[ y_i = \alpha + \mu_i + \epsilon \]


\subsection
The classical model is based on measurements $y_{mi}$
by method $m=1,2$ on item $i = 1,2 \ldots$

\[y_{mi} + \alpha_{m} + \mu_{i} + e_{mi}\]

\[e_{mi} \sim \mathcal{n} (0,\sigma^2_m)\]

Even though the separate variances can not be
identified, their sum can be estimated by the empirical variance of the differences.

Like wise the separate $\alpha$ can not be
estimated, only theiir difference can be estimated as
$\bar{D}$





\bibliography{DB-txfrbib}
\end{document} 