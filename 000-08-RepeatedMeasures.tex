\documentclass[MAIN.tex]{subfiles}
\begin{document}

	
	
	
	\subsection{Replicate Measurements}
	
	Thus far, the formulation for comparison of two measurement
	methods is one where one measurement by each method is taken on
	each subject. Should there be two or more measurements by each
	methods, these measurement are known as `replicate measurements'.
	\citet{BXC2008} recommends the use of replicate measurements, but
	acknowledges that  additional computational complexity.
	
	\citet*{BA86} address this problem by offering two different
	approaches. The premise of the first approach is that replicate
	measurements can be treated as independent measurements. The
	second approach is based upon using the mean of the each group of
	replicates as a representative value of that group. Using either
	of these approaches will allow an analyst to estimate the inter
	method bias.
	
	%\subsubsection{Mean of Replicates Limits of Agreement}
	
	However, because of the removal of the effects of the replicate
	measurements error, this would cause the estimation of the
	standard deviation of the differences to be unduly small.
	\citet*{BA86} propose a correction for this.
	
	\citet{BXC2008} takes issue with the limits of agreement based on
	mean values, in that they can only be interpreted as prediction
	limits for difference between means of repeated measurements by
	both methods, as opposed to the difference of all measurements.
	Incorrect conclusions would be caused by such a misinterpretation.
	\citet{BXC2008} demonstrates how the limits of agreement
	calculated using the mean of replicates are `much too narrow as
	prediction limits for differences between future single
	measurements'. This paper also comments that, while treating the
	replicate measurements as independent will cause a downward bias
	on the limits of agreement calculation, this method is preferable
	to the `mean of replicates' approach.
	

	
	
	
	%This application of the
	%Grubbs method presumes the existence of this condition, and necessitates
	%replication of observations by means external to and independent of the first
	%means. The Grubbs estimators method is based on the laws of propagation of
	%error. By making three independent simultaneous measurements on the same
	%physical material, it is possible by appropriate mathematical manipulation of
	%the sums and differences of the associated variances to obtain a valid
	%estimate of the precision of the primary means. Application of the Grubbs
	%estimators procedure to estimation of the precision of an apparatus uses
	%the results of a physical test conducted in such a way as to obtain a series
	%of sets of three independent observations.
	
\section{Repeated Measurements}
	
In cases where there are repeated measurements by each of the two methods on the same subjects , Bland Altman suggest calculating	the mean for each method on each subject and use these pairs of means to compare the two methods.
The estimate of bias will be unaffected using this approach, but the estimate of the standard deviation of the differences will be too small, because of the reduction of the effect of repeated measurement error. Bland Altman propose a correction for this. Carstensen attends to this issue also, adding that another approach would be to treat each repeated measurement separately.
	
In this model , the variances of the random effects must depend on $m$, since the different methods do not necessarily measure on the
same scale, and different methods naturally must be assumed to have different variances. \citet{BXC2004} attends to the issue of comparative variances.

\section{Repeated measurements in LME models}

In many statistical analyzes, the need to determine parameter estimates where multiple measurements are available on each of a set of variables often arises. Further to \citet{lam}, \citet{hamlett} performs an analysis of the correlation of replicate measurements, for two variables of interest, using LME models.

Let $y_{Aij}$ and $y_{Bij}$ be the $j$th repeated observations of the variables of interest $A$ and $B$ taken on the $i$th subject. The number of repeated measurements for each variable may differ for each individual. Both variables are measured on each time points. Let $n_{i}$ be the number of observations for each variable, hence $2\times n_{i}$ observations in total.

It is assumed that the pair $y_{Aij}$ and $y_{Bij}$ follow a bivariate normal distribution.
\begin{eqnarray*}
	\left(
	\begin{array}{c}
		y_{Aij} \\
		y_{Bij} \\
	\end{array}
	\right) \sim \mathcal{N}(
	\boldsymbol{\mu}, \boldsymbol{\Sigma})\mbox{   where } \boldsymbol{\mu} = \left(
	\begin{array}{c}
		\mu_{A} \\
		\mu_{B} \\
	\end{array}
	\right)
\end{eqnarray*}

The matrix $\Sigma$ represents the variance component matrix between response variables at a given time point $j$.

\[
\boldsymbol{\Sigma} = \left( \begin{array}{cc}
\sigma^2_{A} & \sigma_{AB} \\
\sigma_{AB} & \sigma^2_{B}\\
\end{array}   \right)
\]

$\sigma^2_{A}$ is the variance of variable $A$, $\sigma^2_{B}$ is the variance of variable $B$ and $\sigma_{AB}$ is the covariance of the two variable. It is assumed that $\boldsymbol{\Sigma}$ does not depend on a particular time point, and is the same over all time points.

\addcontentsline{toc}{section}{Bibliography}

%--------------------------------------------------------------------------------------%

\bibliographystyle{chicago}
\bibliography{DB-txfrbib}
\end{document}
