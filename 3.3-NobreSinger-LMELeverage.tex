\documentclass[Main.tex]{subfiles}
\begin{document}
	
%http://www.ime.usp.br/~jmsinger/MAE0610/Mixedmodelresiduals.pdf

\subsection*{Nobre Singer :  Mixed Model Residuals }

%--------------------------------------------------------------%
% Slides
Usually one assumes
\begin{itemize}
\item $b_i \sim N_q(0, G) i = 1, ..., m$
\item $e_i  \sim N_{n_i} (0, \sigma_i)$
\item $b_i$ and $e_i$ independent
\item G and $\sigma_i$ are $(q \times q)$ and $(n_i \times n_i)$ positive definite matrices with
elements expressed as functions of a vector of covariance parameters $\theta$ not functionally related to $\beta$
\item If $\sigma_i = I_{n_i} \sigma^2$: homoskedastic conditional independence model
\end{itemize}
%-------------------------------------------------------------%

%Page 1064
\[  \left[ \begin{array}{c} \boldsymbol{b} \\ \boldsymbol{e} \end{array}\right] \sim \mathcal{N}_{qm+n} \] 

%-------------------------------------------------------------%
%Page 1065 Top

\[ \boldsymbol{Q} = \boldsymbol{V}^{-1} - \boldsymbol{V}^{-1}\boldsymbol{X} ( \boldsymbol{X}^{T} \boldsymbol{V}^{-1} \boldsymbol{X})^{-1} \]

%Papers  : Harville, Robinson, Banerjee

Sensitivity and residual analysis of the underlying assumptions constitute important tools for evaluating the fit of any model to given data.


%-------------------------------------------------------------%
%Page 1065 Bottom
\subsection*{Generalized Leverage}

%-------------------------------------------------------------%
%Page 1068


%%-------------------------------------------------------------%
%1 1.05 1.00 1.13 0.84
%2 1.07 0.62 0.92 0.62
%3 0.82 0.62 1.52 1.07
%4 1.37 0.90 1.65 1.20
%5 1.97 1.52 1.30 1.07
%6 1.30 0.82 1.17 0.70
%7 1.61 1.19 1.52 1.13
%8 1.02 0.73 1.08 0.64
%9 1.62 1.25 1.45 1.10
%10 1.65 1.22 1.57 1.22
%11 1.02 0.78 0.60 0.47
%12 0.68 0.60 1.13 0.39
%13 1.70 1.55 1.85 1.37
%14 1.30 1.02 1.65 0.97
%15 0.90 0.80 1.10 1.03
%16 1.40 1.12 1.25 0.67
%17 1.66 1.63 1.36 1.16
%18 1.02 0.80 0.92 0.82
%19 0.68 0.67 1.00 0.92
%20 1.29 1.23 0.91 0.76
%21 1.27 1.20 1.20 0.95
%22 1.07 0.85 1.39 1.25
%23 1.35 1.21 1.42 1.17
%24 1.32 1.02 1.60 1.40
%25 1.66 1.61 1.50 1.36
%26 1.30 1.07 0.84 0.61
%27 1.57 1.20 1.50 1.07
%28 1.67 1.50 1.47 1.32
%29 0.91 0.67 0.96 0.62
%30 1.06 0.70 1.00 0.85
%31 2.35 2.14 1.57 1.55
%32 1.15 1.00 1.23 1.11
%%-----------------------------------------------------------------%
%
%Comparison between the two generalized random component leverage matrics
%
%%-------------------------------------------------------------%
%Section 4
%Data Analysis

\end{document}
