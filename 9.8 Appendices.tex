\chapter{Appendices} % Chapter 5
%---------------------------------------------------------------------------------------------------------%
% Appendices
% - The Hat Matrix (5.1)
% - Sherman Morrison Woodbury Formula (5.2)
% -  Hat Matrix applied to MCS (5.3)
% - Cross Validation (Updating standard deviation) (5.4)
% - Updating Estimates (5.5)
% - Lesaffre's paper (5.6)
%---------------------------------------------------------------------------------------------------------%
%------------------------------------------------------------------------%
\newpage
\section{The Hat Matrix} %5.1

The projection matrix $H$ (also known as the hat matrix), is a
well known identity that maps the fitted values $\hat{Y}$ to the
observed values $Y$, i.e. $\hat{Y} = HY$.

\begin{equation}
H =\quad X(X^{T}X)^{-1}X^{T}
\end{equation}

$H$ describes the influence each observed value has on each fitted
value. The diagonal elements of the $H$ are the `leverages', which
describe the influence each observed value has on the fitted value
for that same observation. The residuals ($R$) are related to the
observed values by the following formula:
\begin{equation}
R = (I-H)Y
\end{equation}

The variances of $Y$ and $R$ can be expressed as:
\begin{eqnarray}
\mbox{var}(Y) = H\sigma^{2} \nonumber\\
\mbox{var}(R) = (I-H)\sigma^{2}
\end{eqnarray}

Updating techniques allow an economic approach to recalculating
the projection matrix, $H$, by removing the necessity to refit the
model each time it is updated. However this approach is known for
numerical instability in the case of down-dating.

\section{Sherman Morrison Woodbury Formula} % 5.2

The `Sherman Morrison Woodbury' Formula is a well known result in
linear algebra;
\begin{equation}
(A+a^{T}B)^{-1} \quad = \quad A^{-1}-
A^{-1}a^{T}(I-bA^{-1}a^{T})^{-1}bA^{-1}
\end{equation}

This result is highly useful for analyzing regression diagnostics,
and for matrices inverses in general. Consider a $p \times p$
matrix $X$, from which a row $x_{i}^{T}$ is to be added or
deleted. \citet{CookWeisberg} sets $A = X^{T}X$, $a=-x_{i}^{T}$
and $b=x_{i}^{T}$, and writes the above equation as

\begin{equation}
(X^{T}X \pm x_{i}x_{i}^{T})^{-1} = \quad(X^{T}X )^{-1} \mp \quad
\frac{(X^{T}X)^{-1}(x_{i}x_{i}^{T}(X^{T}X)^{-1}}{1-x_{i}^{T}(X^{T}X)^{-1}x_{i}}
\end{equation}

The projection matrix $H$ (also known as the hat matrix), is a
well known identity that maps the fitted values $\hat{Y}$ to the
observed values $Y$, i.e. $\hat{Y} = HY$.

\begin{equation}
H =\quad X(X^{T}X)^{-1}X^{T}
\end{equation}

$H$ describes the influence each observed value has on each fitted value. The diagonal elements of the $H$ are the `leverages', which describe the influence each observed value has on the fitted value for that same observation. The residuals ($R$) are related to the observed values by the following formula:
\begin{equation}
R = (I-H)Y
\end{equation}

The variances of $Y$ and $R$ can be expressed as:
\begin{eqnarray}
\mbox{var}(Y) = H\sigma^{2} \nonumber\\
\mbox{var}(R) = (I-H)\sigma^{2}
\end{eqnarray}

Updating techniques allow an economic approach to recalculating the projection matrix, $H$, by removing the necessity to refit the model each time it is updated. However this approach is known for
numerical instability in the case of down-dating.



\subsection{Hat Values for MCS regression}

With A as the averages and D as the casewise differences.
\begin{verbatim}
fit = lm(D~A)
\end{verbatim}

\begin{displaymath}
H = A \left(A^\top  A\right)^{-1} A^\top ,
\end{displaymath}

%------------------------------------------------------------------------%
\newpage
\section{Cross Validation} %5.4

Cross validation techniques for linear regression employ the use `leave one out' re-calculations. In such procedures the regression coefficients are estimated for $n-1$ covariates, with the $Q^{th}$ observation omitted.

Let $\hat{\beta}$ denote the least square estimate of $\beta$ based upon the full set of observations, and let
$\hat{\beta}^{-Q}$ denoted the estimate with the $Q^{th}$ case
excluded.


In leave-one-out cross validation, each observation is omitted in turn, and a regression model is fitted on the rest of the data. Cross validation is used to estimate the generalization error of a given model. alternatively it can be used for model selection by determining the candidate model that has the smallest generalization error.


Evidently leave-one-out cross validation has similarities with `jackknifing', a well known statistical technique. However cross validation is used to estimate generalization error, whereas the jackknife technique is used to estimate bias.

\subsection{Cross Validation: Updating standard deviation} %5.4.1

The variance of a data set can be calculated using the following formula.
\begin{equation}
S^{2}=\frac{\sum_{i=1}^{n}(x_{i}^{2})-\frac{(\sum_{i=1}^{n}x_{i})^{2}}{n}}{n-1}
\end{equation}

While using bivariate data, the notation $Sxx$ and $Syy$ shall apply to the variance of $x$ and of $y$ respectively. The covariance term $Sxy$ is given by

\begin{equation}
Sxy=\frac{\sum_{i=1}^{n}(x_{i}y_{i})-\frac{(\sum_{i=1}^{n}x_{i})(\sum_{i=1}^{n}y_{i})}{n}}{n-1}
\end{equation}

Let the observation $j$ be omitted from the data set. The estimates for the variance identities can be updating using minor adjustments to the full sample estimates. Where $(j)$ denotes that the $j$th has been omitted, these identities are

\begin{equation}
Sxx^{(j)}=\frac{\sum_{i=1}^{n}(x_{i}^{2})-(x_{j})^{2}-\frac{((\sum_{i=1}^{n}x_{i})-x_{j})^{2}}{n-1}}{n-2}
\end{equation}
\begin{equation}
Syy^{(j)}=\frac{\sum_{i=1}^{n}(y_{i}^{2})-(y_{j})^{2}-\frac{((\sum_{i=1}^{n}y_{i})-y_{j})^{2}}{n-1}}{n-2}
\end{equation}
\begin{equation}
Sxy^{(j)}=\frac{\sum_{i=1}^{n}(x_{i}y_{i})-(y_{j}x_{j})-\frac{((\sum_{i=1}^{n}x_{i})-x_{j})(\sum_{i=1}^{n}y_{i})-y_{k})}{n-1}}{n-2}
\end{equation}

The updated estimate for the slope is therefore
\begin{equation}
\hat{\beta}_{1}^{(j)}=\frac{Sxy^{(j)}}{Sxx^{(j)}}
\end{equation}

It is necessary to determine the mean for $x$ and $y$ of the
remaining $n-1$ terms
\begin{equation}
\bar{x}^{(j)}=\frac{(\sum_{i=1}^{n}x_{i})-(x_{j})}{n-1},
\end{equation}

\begin{equation}
\bar{y}^{(j)}=\frac{(\sum_{i=1}^{n}y_{i})-(y_{j})}{n-1}.
\end{equation}

The updated intercept estimate is therefore

\begin{equation}
\hat{\beta}_{0}^{(j)}=\bar{y}^{(j)}-\hat{\beta}_{1}^{(j)}\bar{x}^{(j)}.
\end{equation}

%------------------------------------------------------------------------%
\newpage
\section{Updating Estimates} %5.5

\subsection{Updating of Regression Estimates}
Updating techniques are used in regression analysis to add or delete rows from a model, allowing the analyst the effect of the observation associated with that row. In time series problems, there will be scientific interest in the changing relationship between variables. In cases where there a single row is to be added or deleted, the procedure used is equivalent to a geometric rotation of a plane.

Updating techniques are used in regression analysis to add or delete rows from a model, allowing the analyst the effect of the observation associated with that row.

\subsection{Updating Standard deviation}
A simple, but useful, example of updating is the updating of the standard deviation when an observation is omitted, as practised in statistical process control analyzes. From first principles, the variance of a data set can be calculated using the following formula.
\begin{equation}
S^{2}=\frac{\sum_{i=1}^{n}(x_{i}^{2})-\frac{(\sum_{i=1}^{n}x_{i})^{2}}{n}}{n-1}
\end{equation}

While using bivariate data, the notation $Sxx$ and $Syy$ shall apply hither to the variance of $x$ and of $y$ respectively. The covariance term $Sxy$ is given by

\begin{equation}
Sxy=\frac{\sum_{i=1}^{n}(x_{i}y_{i})-\frac{(\sum_{i=1}^{n}x_{i})(\sum_{i=1}^{n}y_{i})}{n}}{n-1}.
\end{equation}

\subsection{Updating of Regression Estimates}
Updating techniques are used in regression analysis to add or
delete rows from a model, allowing the analyst the effect of the
observation associated with that row. In time series problems,
there will be scientific interest in the changing relationship
between variables. In cases where there a single row is to be
added or deleted, the procedure used is equivalent to a geometric
rotation of a plane.

Consider a $p \times p$ matrix $X$, from which a row $x_{i}^{T}$
is to be added or deleted. \citet{CookWeisberg} sets $A = X^{T}X$,
$a=-x_{i}^{T}$ and $b=x_{i}^{T}$, and writes the above equation as

\begin{equation}
(X^{T}X \pm x_{i}x_{i}^{T})^{-1} = \quad(X^{T}X )^{-1} \mp \quad
\frac{(X^{T}X)^{-1}(x_{i}x_{i}^{T}(X^{T}X)^{-1}}{1-x_{i}^{T}(X^{T}X)^{-1}x_{i}}
\end{equation}

\subsection{Updating Regression Estimates}
Let the observation $j$ be omitted from the data set. The estimates for the variance identities can be updating using minor adjustments to the full sample estimates. Where $(j)$ denotes that the $j$th has been omitted, these identities are

\begin{equation}
Sxx^{(j)}=\frac{\sum_{i=1}^{n}(x_{i}^{2})-(x_{j})^{2}-\frac{((\sum_{i=1}^{n}x_{i})-x_{j})^{2}}{n-1}}{n-2}
\end{equation}
\begin{equation}
Syy^{(j)}=\frac{\sum_{i=1}^{n}(y_{i}^{2})-(y_{j})^{2}-\frac{((\sum_{i=1}^{n}y_{i})-y_{j})^{2}}{n-1}}{n-2}
\end{equation}
\begin{equation}
Sxy^{(j)}=\frac{\sum_{i=1}^{n}(x_{i}y_{i})-(y_{j}x_{j})-\frac{((\sum_{i=1}^{n}x_{i})-x_{j})(\sum_{i=1}^{n}y_{i})-y_{k})}{n-1}}{n-2}
\end{equation}

The updated estimate for the slope is therefore
\begin{equation}
\hat{\beta}_{1}^{(j)}=\frac{Sxy^{(j)}}{Sxx^{(j)}}
\end{equation}

It is necessary to determine the mean for $x$ and $y$ of the
remaining $n-1$ terms
\begin{equation}
\bar{x}^{(j)}=\frac{(\sum_{i=1}^{n}x_{i})-(x_{j})}{n-1},
\end{equation}

\begin{equation}
\bar{y}^{(j)}=\frac{(\sum_{i=1}^{n}y_{i})-(y_{j})}{n-1}.
\end{equation}

The updated intercept estimate is therefore

\begin{equation}
\hat{\beta}_{0}^{(j)}=\bar{y}^{(j)}-\hat{\beta}_{1}^{(j)}\bar{x}^{(j)}.
\end{equation}

\subsection{Inference on intercept and slope}
\begin{equation}
\hat{\beta_{1}} \pm t_{(\alpha, n-2) }
\sqrt{\frac{S^2}{(n-1)S^{2}_{x}}}
\end{equation}

\begin{equation}
\frac{\hat{\beta_{0}}-\beta_{0}}{SE(\hat{\beta_{0}})}
\end{equation}
\begin{equation}
\frac{\hat{\beta_{1}}-\beta_{1}}{SE(\hat{\beta_{0}})}
\end{equation}


\subsubsection{Inference on correlation coefficient} This test of
the slope is coincidentally the equivalent of a test of the
correlation of the $n$ observations of $X$ and $Y$.
\begin{eqnarray}
H_{0}: \rho_{XY} = 0 \nonumber \\
H_{A}: \rho_{XY} \ne 0 \nonumber \\
\end{eqnarray}

%---------------------------------------------------------%
\newpage


%-------------------------------------------------------------------------------------------------------%
\chapter{Appendices} % Chapter 5
%---------------------------------------------------------------------------------------------------------%
% Appendices
% - The Hat Matrix (5.1)
% - Sherman Morrison Woodbury Formula (5.2)
% -  Hat Matrix applied to MCS (5.3)
% - Cross Validation (Updating standard deviation) (5.4)
% - Updating Estimates (5.5)
% - Lesaffre's paper (5.6)
%---------------------------------------------------------------------------------------------------------%
%------------------------------------------------------------------------%
\newpage
\section{The Hat Matrix} %5.1


The projection matrix $H$ (also known as the hat matrix), is a
well known identity that maps the fitted values $\hat{Y}$ to the
observed values $Y$, i.e. $\hat{Y} = HY$.


\begin{equation}
H =\quad X(X^{T}X)^{-1}X^{T}
\end{equation}


$H$ describes the influence each observed value has on each fitted
value. The diagonal elements of the $H$ are the `leverages', which
describe the influence each observed value has on the fitted value
for that same observation. The residuals ($R$) are related to the
observed values by the following formula:
\begin{equation}
R = (I-H)Y
\end{equation}


The variances of $Y$ and $R$ can be expressed as:
\begin{eqnarray}
\mbox{var}(Y) = H\sigma^{2} \nonumber\\
\mbox{var}(R) = (I-H)\sigma^{2}
\end{eqnarray}


Updating techniques allow an economic approach to recalculating
the projection matrix, $H$, by removing the necessity to refit the
model each time it is updated. However this approach is known for
numerical instability in the case of down-dating.


\section{Sherman Morrison Woodbury Formula} % 5.2


The `Sherman Morrison Woodbury' Formula is a well known result in
linear algebra;
\begin{equation}
(A+a^{T}B)^{-1} \quad = \quad A^{-1}-
A^{-1}a^{T}(I-bA^{-1}a^{T})^{-1}bA^{-1}
\end{equation}


This result is highly useful for analyzing regression diagnostics,
and for matrices inverses in general. Consider a $p \times p$
matrix $X$, from which a row $x_{i}^{T}$ is to be added or
deleted. \citet{CookWeisberg} sets $A = X^{T}X$, $a=-x_{i}^{T}$
and $b=x_{i}^{T}$, and writes the above equation as


\begin{equation}
(X^{T}X \pm x_{i}x_{i}^{T})^{-1} = \quad(X^{T}X )^{-1} \mp \quad
\frac{(X^{T}X)^{-1}(x_{i}x_{i}^{T}(X^{T}X)^{-1}}{1-x_{i}^{T}(X^{T}X)^{-1}x_{i}}
\end{equation}


The projection matrix $H$ (also known as the hat matrix), is a
well known identity that maps the fitted values $\hat{Y}$ to the
observed values $Y$, i.e. $\hat{Y} = HY$.


\begin{equation}
H =\quad X(X^{T}X)^{-1}X^{T}
\end{equation}


$H$ describes the influence each observed value has on each fitted value. The diagonal elements of the $H$ are the `leverages', which describe the influence each observed value has on the fitted value for that same observation. The residuals ($R$) are related to the observed values by the following formula:
\begin{equation}
R = (I-H)Y
\end{equation}


The variances of $Y$ and $R$ can be expressed as:
\begin{eqnarray}
\mbox{var}(Y) = H\sigma^{2} \nonumber\\
\mbox{var}(R) = (I-H)\sigma^{2}
\end{eqnarray}


Updating techniques allow an economic approach to recalculating the projection matrix, $H$, by removing the necessity to refit the model each time it is updated. However this approach is known for
numerical instability in the case of down-dating.






\subsection{Hat Values for MCS regression}


With A as the averages and D as the casewise differences.
\begin{verbatim}
fit = lm(D~A)
\end{verbatim}


\begin{displaymath}
H = A \left(A^\top  A\right)^{-1} A^\top ,
\end{displaymath}

\newpage



%-------------------------------------------------------------------------------------------------------%
\chapter{Appendices} % Chapter 5
%---------------------------------------------------------------------------------------------------------%
% Appendices
% - The Hat Matrix (5.1)
% - Sherman Morrison Woodbury Formula (5.2)
% -  Hat Matrix applied to MCS (5.3)
% - Cross Validation (Updating standard deviation) (5.4)
% - Updating Estimates (5.5)
% - Lesaffre's paper (5.6)
%---------------------------------------------------------------------------------------------------------%
%------------------------------------------------------------------------%
\newpage
\section{The Hat Matrix} %5.1


The projection matrix $H$ (also known as the hat matrix), is a
well known identity that maps the fitted values $\hat{Y}$ to the
observed values $Y$, i.e. $\hat{Y} = HY$.


\begin{equation}
H =\quad X(X^{T}X)^{-1}X^{T}
\end{equation}


$H$ describes the influence each observed value has on each fitted
value. The diagonal elements of the $H$ are the `leverages', which
describe the influence each observed value has on the fitted value
for that same observation. The residuals ($R$) are related to the
observed values by the following formula:
\begin{equation}
R = (I-H)Y
\end{equation}


The variances of $Y$ and $R$ can be expressed as:
\begin{eqnarray}
\mbox{var}(Y) = H\sigma^{2} \nonumber\\
\mbox{var}(R) = (I-H)\sigma^{2}
\end{eqnarray}


Updating techniques allow an economic approach to recalculating
the projection matrix, $H$, by removing the necessity to refit the
model each time it is updated. However this approach is known for
numerical instability in the case of down-dating.


\section{Sherman Morrison Woodbury Formula} % 5.2


The `Sherman Morrison Woodbury' Formula is a well known result in
linear algebra;
\begin{equation}
(A+a^{T}B)^{-1} \quad = \quad A^{-1}-
A^{-1}a^{T}(I-bA^{-1}a^{T})^{-1}bA^{-1}
\end{equation}


This result is highly useful for analyzing regression diagnostics,
and for matrices inverses in general. Consider a $p \times p$
matrix $X$, from which a row $x_{i}^{T}$ is to be added or
deleted. \citet{CookWeisberg} sets $A = X^{T}X$, $a=-x_{i}^{T}$
and $b=x_{i}^{T}$, and writes the above equation as


\begin{equation}
(X^{T}X \pm x_{i}x_{i}^{T})^{-1} = \quad(X^{T}X )^{-1} \mp \quad
\frac{(X^{T}X)^{-1}(x_{i}x_{i}^{T}(X^{T}X)^{-1}}{1-x_{i}^{T}(X^{T}X)^{-1}x_{i}}
\end{equation}


The projection matrix $H$ (also known as the hat matrix), is a
well known identity that maps the fitted values $\hat{Y}$ to the
observed values $Y$, i.e. $\hat{Y} = HY$.


\begin{equation}
H =\quad X(X^{T}X)^{-1}X^{T}
\end{equation}


$H$ describes the influence each observed value has on each fitted value. The diagonal elements of the $H$ are the `leverages', which describe the influence each observed value has on the fitted value for that same observation. The residuals ($R$) are related to the observed values by the following formula:
\begin{equation}
R = (I-H)Y
\end{equation}


The variances of $Y$ and $R$ can be expressed as:
\begin{eqnarray}
\mbox{var}(Y) = H\sigma^{2} \nonumber\\
\mbox{var}(R) = (I-H)\sigma^{2}
\end{eqnarray}


Updating techniques allow an economic approach to recalculating the projection matrix, $H$, by removing the necessity to refit the model each time it is updated. However this approach is known for
numerical instability in the case of down-dating.






\subsection{Hat Values for MCS regression}


With A as the averages and D as the casewise differences.
\begin{verbatim}
fit = lm(D~A)
\end{verbatim}


\begin{displaymath}
H = A \left(A^\top  A\right)^{-1} A^\top ,
\end{displaymath}



%-------------------------------------------------------------------------------------------------------%
\chapter{Appendices} % Chapter 5
%---------------------------------------------------------------------------------------------------------%
% Appendices
% - The Hat Matrix (5.1)
% - Sherman Morrison Woodbury Formula (5.2)
% -  Hat Matrix applied to MCS (5.3)
% - Cross Validation (Updating standard deviation) (5.4)
% - Updating Estimates (5.5)
% - Lesaffre's paper (5.6)
%---------------------------------------------------------------------------------------------------------%
%------------------------------------------------------------------------%
\newpage
\section{The Hat Matrix} %5.1

The projection matrix $H$ (also known as the hat matrix), is a
well known identity that maps the fitted values $\hat{Y}$ to the
observed values $Y$, i.e. $\hat{Y} = HY$.

\begin{equation}
H =\quad X(X^{T}X)^{-1}X^{T}
\end{equation}

$H$ describes the influence each observed value has on each fitted
value. The diagonal elements of the $H$ are the `leverages', which
describe the influence each observed value has on the fitted value
for that same observation. The residuals ($R$) are related to the
observed values by the following formula:
\begin{equation}
R = (I-H)Y
\end{equation}

The variances of $Y$ and $R$ can be expressed as:
\begin{eqnarray}
\mbox{var}(Y) = H\sigma^{2} \nonumber\\
\mbox{var}(R) = (I-H)\sigma^{2}
\end{eqnarray}

Updating techniques allow an economic approach to recalculating
the projection matrix, $H$, by removing the necessity to refit the
model each time it is updated. However this approach is known for
numerical instability in the case of down-dating.

\section{Sherman Morrison Woodbury Formula} % 5.2

The `Sherman Morrison Woodbury' Formula is a well known result in
linear algebra;
\begin{equation}
(A+a^{T}B)^{-1} \quad = \quad A^{-1}-
A^{-1}a^{T}(I-bA^{-1}a^{T})^{-1}bA^{-1}
\end{equation}

This result is highly useful for analyzing regression diagnostics,
and for matrices inverses in general. Consider a $p \times p$
matrix $X$, from which a row $x_{i}^{T}$ is to be added or
deleted. \citet{CookWeisberg} sets $A = X^{T}X$, $a=-x_{i}^{T}$
and $b=x_{i}^{T}$, and writes the above equation as

\begin{equation}
(X^{T}X \pm x_{i}x_{i}^{T})^{-1} = \quad(X^{T}X )^{-1} \mp \quad
\frac{(X^{T}X)^{-1}(x_{i}x_{i}^{T}(X^{T}X)^{-1}}{1-x_{i}^{T}(X^{T}X)^{-1}x_{i}}
\end{equation}

The projection matrix $H$ (also known as the hat matrix), is a
well known identity that maps the fitted values $\hat{Y}$ to the
observed values $Y$, i.e. $\hat{Y} = HY$.

\begin{equation}
H =\quad X(X^{T}X)^{-1}X^{T}
\end{equation}

$H$ describes the influence each observed value has on each fitted value. The diagonal elements of the $H$ are the `leverages', which describe the influence each observed value has on the fitted value for that same observation. The residuals ($R$) are related to the observed values by the following formula:
\begin{equation}
R = (I-H)Y
\end{equation}

The variances of $Y$ and $R$ can be expressed as:
\begin{eqnarray}
\mbox{var}(Y) = H\sigma^{2} \nonumber\\
\mbox{var}(R) = (I-H)\sigma^{2}
\end{eqnarray}

Updating techniques allow an economic approach to recalculating the projection matrix, $H$, by removing the necessity to refit the model each time it is updated. However this approach is known for
numerical instability in the case of down-dating.



\subsection{Hat Values for MCS regression}

With A as the averages and D as the casewise differences.
\begin{verbatim}
fit = lm(D~A)
\end{verbatim}

\begin{displaymath}
H = A \left(A^\top  A\right)^{-1} A^\top ,
\end{displaymath}

%------------------------------------------------------------------------%
\newpage
\section{Cross Validation} %5.4

Cross validation techniques for linear regression employ the use `leave one out' re-calculations. In such procedures the regression coefficients are estimated for $n-1$ covariates, with the $Q^{th}$ observation omitted.

Let $\hat{\beta}$ denote the least square estimate of $\beta$ based upon the full set of observations, and let
$\hat{\beta}^{-Q}$ denoted the estimate with the $Q^{th}$ case
excluded.


In leave-one-out cross validation, each observation is omitted in turn, and a regression model is fitted on the rest of the data. Cross validation is used to estimate the generalization error of a given model. alternatively it can be used for model selection by determining the candidate model that has the smallest generalization error.


Evidently leave-one-out cross validation has similarities with `jackknifing', a well known statistical technique. However cross validation is used to estimate generalization error, whereas the jackknife technique is used to estimate bias.

\subsection{Cross Validation: Updating standard deviation} %5.4.1

The variance of a data set can be calculated using the following formula.
\begin{equation}
S^{2}=\frac{\sum_{i=1}^{n}(x_{i}^{2})-\frac{(\sum_{i=1}^{n}x_{i})^{2}}{n}}{n-1}
\end{equation}

While using bivariate data, the notation $Sxx$ and $Syy$ shall apply to the variance of $x$ and of $y$ respectively. The covariance term $Sxy$ is given by

\begin{equation}
Sxy=\frac{\sum_{i=1}^{n}(x_{i}y_{i})-\frac{(\sum_{i=1}^{n}x_{i})(\sum_{i=1}^{n}y_{i})}{n}}{n-1}
\end{equation}

Let the observation $j$ be omitted from the data set. The estimates for the variance identities can be updating using minor adjustments to the full sample estimates. Where $(j)$ denotes that the $j$th has been omitted, these identities are

\begin{equation}
Sxx^{(j)}=\frac{\sum_{i=1}^{n}(x_{i}^{2})-(x_{j})^{2}-\frac{((\sum_{i=1}^{n}x_{i})-x_{j})^{2}}{n-1}}{n-2}
\end{equation}
\begin{equation}
Syy^{(j)}=\frac{\sum_{i=1}^{n}(y_{i}^{2})-(y_{j})^{2}-\frac{((\sum_{i=1}^{n}y_{i})-y_{j})^{2}}{n-1}}{n-2}
\end{equation}
\begin{equation}
Sxy^{(j)}=\frac{\sum_{i=1}^{n}(x_{i}y_{i})-(y_{j}x_{j})-\frac{((\sum_{i=1}^{n}x_{i})-x_{j})(\sum_{i=1}^{n}y_{i})-y_{k})}{n-1}}{n-2}
\end{equation}

The updated estimate for the slope is therefore
\begin{equation}
\hat{\beta}_{1}^{(j)}=\frac{Sxy^{(j)}}{Sxx^{(j)}}
\end{equation}

It is necessary to determine the mean for $x$ and $y$ of the
remaining $n-1$ terms
\begin{equation}
\bar{x}^{(j)}=\frac{(\sum_{i=1}^{n}x_{i})-(x_{j})}{n-1},
\end{equation}

\begin{equation}
\bar{y}^{(j)}=\frac{(\sum_{i=1}^{n}y_{i})-(y_{j})}{n-1}.
\end{equation}

The updated intercept estimate is therefore

\begin{equation}
\hat{\beta}_{0}^{(j)}=\bar{y}^{(j)}-\hat{\beta}_{1}^{(j)}\bar{x}^{(j)}.
\end{equation}

%------------------------------------------------------------------------%
\newpage
\section{Updating Estimates} %5.5

\subsection{Updating of Regression Estimates}
Updating techniques are used in regression analysis to add or delete rows from a model, allowing the analyst the effect of the observation associated with that row. In time series problems, there will be scientific interest in the changing relationship between variables. In cases where there a single row is to be added or deleted, the procedure used is equivalent to a geometric rotation of a plane.

Updating techniques are used in regression analysis to add or delete rows from a model, allowing the analyst the effect of the observation associated with that row.

\subsection{Updating Standard deviation}
A simple, but useful, example of updating is the updating of the standard deviation when an observation is omitted, as practised in statistical process control analyzes. From first principles, the variance of a data set can be calculated using the following formula.
\begin{equation}
S^{2}=\frac{\sum_{i=1}^{n}(x_{i}^{2})-\frac{(\sum_{i=1}^{n}x_{i})^{2}}{n}}{n-1}
\end{equation}

While using bivariate data, the notation $Sxx$ and $Syy$ shall apply hither to the variance of $x$ and of $y$ respectively. The covariance term $Sxy$ is given by

\begin{equation}
Sxy=\frac{\sum_{i=1}^{n}(x_{i}y_{i})-\frac{(\sum_{i=1}^{n}x_{i})(\sum_{i=1}^{n}y_{i})}{n}}{n-1}.
\end{equation}

\subsection{Updating of Regression Estimates}
Updating techniques are used in regression analysis to add or
delete rows from a model, allowing the analyst the effect of the
observation associated with that row. In time series problems,
there will be scientific interest in the changing relationship
between variables. In cases where there a single row is to be
added or deleted, the procedure used is equivalent to a geometric
rotation of a plane.

Consider a $p \times p$ matrix $X$, from which a row $x_{i}^{T}$
is to be added or deleted. \citet{CookWeisberg} sets $A = X^{T}X$,
$a=-x_{i}^{T}$ and $b=x_{i}^{T}$, and writes the above equation as

\begin{equation}
(X^{T}X \pm x_{i}x_{i}^{T})^{-1} = \quad(X^{T}X )^{-1} \mp \quad
\frac{(X^{T}X)^{-1}(x_{i}x_{i}^{T}(X^{T}X)^{-1}}{1-x_{i}^{T}(X^{T}X)^{-1}x_{i}}
\end{equation}

\subsection{Updating Regression Estimates}
Let the observation $j$ be omitted from the data set. The estimates for the variance identities can be updating using minor adjustments to the full sample estimates. Where $(j)$ denotes that the $j$th has been omitted, these identities are

\begin{equation}
Sxx^{(j)}=\frac{\sum_{i=1}^{n}(x_{i}^{2})-(x_{j})^{2}-\frac{((\sum_{i=1}^{n}x_{i})-x_{j})^{2}}{n-1}}{n-2}
\end{equation}
\begin{equation}
Syy^{(j)}=\frac{\sum_{i=1}^{n}(y_{i}^{2})-(y_{j})^{2}-\frac{((\sum_{i=1}^{n}y_{i})-y_{j})^{2}}{n-1}}{n-2}
\end{equation}
\begin{equation}
Sxy^{(j)}=\frac{\sum_{i=1}^{n}(x_{i}y_{i})-(y_{j}x_{j})-\frac{((\sum_{i=1}^{n}x_{i})-x_{j})(\sum_{i=1}^{n}y_{i})-y_{k})}{n-1}}{n-2}
\end{equation}

The updated estimate for the slope is therefore
\begin{equation}
\hat{\beta}_{1}^{(j)}=\frac{Sxy^{(j)}}{Sxx^{(j)}}
\end{equation}

It is necessary to determine the mean for $x$ and $y$ of the
remaining $n-1$ terms
\begin{equation}
\bar{x}^{(j)}=\frac{(\sum_{i=1}^{n}x_{i})-(x_{j})}{n-1},
\end{equation}

\begin{equation}
\bar{y}^{(j)}=\frac{(\sum_{i=1}^{n}y_{i})-(y_{j})}{n-1}.
\end{equation}

The updated intercept estimate is therefore

\begin{equation}
\hat{\beta}_{0}^{(j)}=\bar{y}^{(j)}-\hat{\beta}_{1}^{(j)}\bar{x}^{(j)}.
\end{equation}

\subsection{Inference on intercept and slope}
\begin{equation}
\hat{\beta_{1}} \pm t_{(\alpha, n-2) }
\sqrt{\frac{S^2}{(n-1)S^{2}_{x}}}
\end{equation}

\begin{equation}
\frac{\hat{\beta_{0}}-\beta_{0}}{SE(\hat{\beta_{0}})}
\end{equation}
\begin{equation}
\frac{\hat{\beta_{1}}-\beta_{1}}{SE(\hat{\beta_{0}})}
\end{equation}


\subsubsection{Inference on correlation coefficient} This test of
the slope is coincidentally the equivalent of a test of the
correlation of the $n$ observations of $X$ and $Y$.
\begin{eqnarray}
H_{0}: \rho_{XY} = 0 \nonumber \\
H_{A}: \rho_{XY} \ne 0 \nonumber \\
\end{eqnarray}

%---------------------------------------------------------%
\newpage
\section{Lesaffre's paper.} %5.6

Lesaffre considers the case-weight perturbation approach.


%\citep{cook86}
Cook's 86 describes a local approach wherein each case is given a weight $w_{i}$ and the effect on the parameter estimation is measured by perturbing these weights. Choosing weights close to zero or one corresponds to the global case-deletion approach.

Lesaffre  describes the displacement in log-likelihood as a useful metric to evaluate local influence %\citep{cook86}.


%\citet{lesaffre}
Lesaffre describes a framework to detect outlying observations that matter in an LME model. Detection should be carried out by evaluating diagnostics $C_{i}$ , $C_{i}(\alpha)$ and $C_{i}(D,\sigma^2)$.


Lesaffre defines the total local influence of individual $i$ as
\begin{equation}
C_{i} = 2 | \triangle \prime _{i} L^{-1} \triangle_{i}|.
\end{equation}


The influence function of the MLEs evaluated at the $i$th point $IF_{i}$, given by
\begin{equation}
IF_{i} = -L^{-1}\triangle _{i}
\end{equation}
can indicate how $\hat{theta}$ changes as the weight of the $i$th
subject changes.

The manner by which influential observations distort the estimation process can be determined by inspecting the
interpretable components in the decomposition of the above measures of local influence.


Lesaffre comments that there is no clear way of interpreting the information contained in the angles, but that this doesn't mean the information should be ignored.


%-------------------------------------------------------------------------------------------------------%
\chapter{Appendices} % Chapter 5
%---------------------------------------------------------------------------------------------------------%
% Appendices
% - The Hat Matrix (5.1)
% - Sherman Morrison Woodbury Formula (5.2)
% -  Hat Matrix applied to MCS (5.3)
% - Cross Validation (Updating standard deviation) (5.4)
% - Updating Estimates (5.5)
% - Lesaffre's paper (5.6)
%---------------------------------------------------------------------------------------------------------%
%------------------------------------------------------------------------%
\newpage
\section{The Hat Matrix} %5.1

The projection matrix $H$ (also known as the hat matrix), is a
well known identity that maps the fitted values $\hat{Y}$ to the
observed values $Y$, i.e. $\hat{Y} = HY$.

\begin{equation}
H =\quad X(X^{T}X)^{-1}X^{T}
\end{equation}

$H$ describes the influence each observed value has on each fitted
value. The diagonal elements of the $H$ are the `leverages', which
describe the influence each observed value has on the fitted value
for that same observation. The residuals ($R$) are related to the
observed values by the following formula:
\begin{equation}
R = (I-H)Y
\end{equation}

The variances of $Y$ and $R$ can be expressed as:
\begin{eqnarray}
\mbox{var}(Y) = H\sigma^{2} \nonumber\\
\mbox{var}(R) = (I-H)\sigma^{2}
\end{eqnarray}

Updating techniques allow an economic approach to recalculating
the projection matrix, $H$, by removing the necessity to refit the
model each time it is updated. However this approach is known for
numerical instability in the case of down-dating.

\section{Sherman Morrison Woodbury Formula} % 5.2

The `Sherman Morrison Woodbury' Formula is a well known result in
linear algebra;
\begin{equation}
(A+a^{T}B)^{-1} \quad = \quad A^{-1}-
A^{-1}a^{T}(I-bA^{-1}a^{T})^{-1}bA^{-1}
\end{equation}

This result is highly useful for analyzing regression diagnostics,
and for matrices inverses in general. Consider a $p \times p$
matrix $X$, from which a row $x_{i}^{T}$ is to be added or
deleted. \citet{CookWeisberg} sets $A = X^{T}X$, $a=-x_{i}^{T}$
and $b=x_{i}^{T}$, and writes the above equation as

\begin{equation}
(X^{T}X \pm x_{i}x_{i}^{T})^{-1} = \quad(X^{T}X )^{-1} \mp \quad
\frac{(X^{T}X)^{-1}(x_{i}x_{i}^{T}(X^{T}X)^{-1}}{1-x_{i}^{T}(X^{T}X)^{-1}x_{i}}
\end{equation}

The projection matrix $H$ (also known as the hat matrix), is a
well known identity that maps the fitted values $\hat{Y}$ to the
observed values $Y$, i.e. $\hat{Y} = HY$.

\begin{equation}
H =\quad X(X^{T}X)^{-1}X^{T}
\end{equation}

$H$ describes the influence each observed value has on each fitted value. The diagonal elements of the $H$ are the `leverages', which describe the influence each observed value has on the fitted value for that same observation. The residuals ($R$) are related to the observed values by the following formula:
\begin{equation}
R = (I-H)Y
\end{equation}

The variances of $Y$ and $R$ can be expressed as:
\begin{eqnarray}
\mbox{var}(Y) = H\sigma^{2} \nonumber\\
\mbox{var}(R) = (I-H)\sigma^{2}
\end{eqnarray}

Updating techniques allow an economic approach to recalculating the projection matrix, $H$, by removing the necessity to refit the model each time it is updated. However this approach is known for
numerical instability in the case of down-dating.



\subsection{Hat Values for MCS regression}

With A as the averages and D as the casewise differences.
\begin{verbatim}
fit = lm(D~A)
\end{verbatim}

\begin{displaymath}
H = A \left(A^\top  A\right)^{-1} A^\top ,
\end{displaymath}

%------------------------------------------------------------------------%
\newpage
\section{Cross Validation} %5.4

Cross validation techniques for linear regression employ the use `leave one out' re-calculations. In such procedures the regression coefficients are estimated for $n-1$ covariates, with the $Q^{th}$ observation omitted.

Let $\hat{\beta}$ denote the least square estimate of $\beta$ based upon the full set of observations, and let
$\hat{\beta}^{-Q}$ denoted the estimate with the $Q^{th}$ case
excluded.


In leave-one-out cross validation, each observation is omitted in turn, and a regression model is fitted on the rest of the data. Cross validation is used to estimate the generalization error of a given model. alternatively it can be used for model selection by determining the candidate model that has the smallest generalization error.


Evidently leave-one-out cross validation has similarities with `jackknifing', a well known statistical technique. However cross validation is used to estimate generalization error, whereas the jackknife technique is used to estimate bias.

\subsection{Cross Validation: Updating standard deviation} %5.4.1

The variance of a data set can be calculated using the following formula.
\begin{equation}
S^{2}=\frac{\sum_{i=1}^{n}(x_{i}^{2})-\frac{(\sum_{i=1}^{n}x_{i})^{2}}{n}}{n-1}
\end{equation}

While using bivariate data, the notation $Sxx$ and $Syy$ shall apply to the variance of $x$ and of $y$ respectively. The covariance term $Sxy$ is given by

\begin{equation}
Sxy=\frac{\sum_{i=1}^{n}(x_{i}y_{i})-\frac{(\sum_{i=1}^{n}x_{i})(\sum_{i=1}^{n}y_{i})}{n}}{n-1}
\end{equation}

Let the observation $j$ be omitted from the data set. The estimates for the variance identities can be updating using minor adjustments to the full sample estimates. Where $(j)$ denotes that the $j$th has been omitted, these identities are

\begin{equation}
Sxx^{(j)}=\frac{\sum_{i=1}^{n}(x_{i}^{2})-(x_{j})^{2}-\frac{((\sum_{i=1}^{n}x_{i})-x_{j})^{2}}{n-1}}{n-2}
\end{equation}
\begin{equation}
Syy^{(j)}=\frac{\sum_{i=1}^{n}(y_{i}^{2})-(y_{j})^{2}-\frac{((\sum_{i=1}^{n}y_{i})-y_{j})^{2}}{n-1}}{n-2}
\end{equation}
\begin{equation}
Sxy^{(j)}=\frac{\sum_{i=1}^{n}(x_{i}y_{i})-(y_{j}x_{j})-\frac{((\sum_{i=1}^{n}x_{i})-x_{j})(\sum_{i=1}^{n}y_{i})-y_{k})}{n-1}}{n-2}
\end{equation}

The updated estimate for the slope is therefore
\begin{equation}
\hat{\beta}_{1}^{(j)}=\frac{Sxy^{(j)}}{Sxx^{(j)}}
\end{equation}

It is necessary to determine the mean for $x$ and $y$ of the
remaining $n-1$ terms
\begin{equation}
\bar{x}^{(j)}=\frac{(\sum_{i=1}^{n}x_{i})-(x_{j})}{n-1},
\end{equation}

\begin{equation}
\bar{y}^{(j)}=\frac{(\sum_{i=1}^{n}y_{i})-(y_{j})}{n-1}.
\end{equation}

The updated intercept estimate is therefore

\begin{equation}
\hat{\beta}_{0}^{(j)}=\bar{y}^{(j)}-\hat{\beta}_{1}^{(j)}\bar{x}^{(j)}.
\end{equation}

%------------------------------------------------------------------------%
\newpage
\section{Updating Estimates} %5.5

\subsection{Updating of Regression Estimates}
Updating techniques are used in regression analysis to add or delete rows from a model, allowing the analyst the effect of the observation associated with that row. In time series problems, there will be scientific interest in the changing relationship between variables. In cases where there a single row is to be added or deleted, the procedure used is equivalent to a geometric rotation of a plane.

Updating techniques are used in regression analysis to add or delete rows from a model, allowing the analyst the effect of the observation associated with that row.

\subsection{Updating Standard deviation}
A simple, but useful, example of updating is the updating of the standard deviation when an observation is omitted, as practised in statistical process control analyzes. From first principles, the variance of a data set can be calculated using the following formula.
\begin{equation}
S^{2}=\frac{\sum_{i=1}^{n}(x_{i}^{2})-\frac{(\sum_{i=1}^{n}x_{i})^{2}}{n}}{n-1}
\end{equation}

While using bivariate data, the notation $Sxx$ and $Syy$ shall apply hither to the variance of $x$ and of $y$ respectively. The covariance term $Sxy$ is given by

\begin{equation}
Sxy=\frac{\sum_{i=1}^{n}(x_{i}y_{i})-\frac{(\sum_{i=1}^{n}x_{i})(\sum_{i=1}^{n}y_{i})}{n}}{n-1}.
\end{equation}

\subsection{Updating of Regression Estimates}
Updating techniques are used in regression analysis to add or
delete rows from a model, allowing the analyst the effect of the
observation associated with that row. In time series problems,
there will be scientific interest in the changing relationship
between variables. In cases where there a single row is to be
added or deleted, the procedure used is equivalent to a geometric
rotation of a plane.

Consider a $p \times p$ matrix $X$, from which a row $x_{i}^{T}$
is to be added or deleted. \citet{CookWeisberg} sets $A = X^{T}X$,
$a=-x_{i}^{T}$ and $b=x_{i}^{T}$, and writes the above equation as

\begin{equation}
(X^{T}X \pm x_{i}x_{i}^{T})^{-1} = \quad(X^{T}X )^{-1} \mp \quad
\frac{(X^{T}X)^{-1}(x_{i}x_{i}^{T}(X^{T}X)^{-1}}{1-x_{i}^{T}(X^{T}X)^{-1}x_{i}}
\end{equation}

\subsection{Updating Regression Estimates}
Let the observation $j$ be omitted from the data set. The estimates for the variance identities can be updating using minor adjustments to the full sample estimates. Where $(j)$ denotes that the $j$th has been omitted, these identities are

\begin{equation}
Sxx^{(j)}=\frac{\sum_{i=1}^{n}(x_{i}^{2})-(x_{j})^{2}-\frac{((\sum_{i=1}^{n}x_{i})-x_{j})^{2}}{n-1}}{n-2}
\end{equation}
\begin{equation}
Syy^{(j)}=\frac{\sum_{i=1}^{n}(y_{i}^{2})-(y_{j})^{2}-\frac{((\sum_{i=1}^{n}y_{i})-y_{j})^{2}}{n-1}}{n-2}
\end{equation}
\begin{equation}
Sxy^{(j)}=\frac{\sum_{i=1}^{n}(x_{i}y_{i})-(y_{j}x_{j})-\frac{((\sum_{i=1}^{n}x_{i})-x_{j})(\sum_{i=1}^{n}y_{i})-y_{k})}{n-1}}{n-2}
\end{equation}

The updated estimate for the slope is therefore
\begin{equation}
\hat{\beta}_{1}^{(j)}=\frac{Sxy^{(j)}}{Sxx^{(j)}}
\end{equation}

It is necessary to determine the mean for $x$ and $y$ of the
remaining $n-1$ terms
\begin{equation}
\bar{x}^{(j)}=\frac{(\sum_{i=1}^{n}x_{i})-(x_{j})}{n-1},
\end{equation}

\begin{equation}
\bar{y}^{(j)}=\frac{(\sum_{i=1}^{n}y_{i})-(y_{j})}{n-1}.
\end{equation}

The updated intercept estimate is therefore

\begin{equation}
\hat{\beta}_{0}^{(j)}=\bar{y}^{(j)}-\hat{\beta}_{1}^{(j)}\bar{x}^{(j)}.
\end{equation}

\subsection{Inference on intercept and slope}
\begin{equation}
\hat{\beta_{1}} \pm t_{(\alpha, n-2) }
\sqrt{\frac{S^2}{(n-1)S^{2}_{x}}}
\end{equation}

\begin{equation}
\frac{\hat{\beta_{0}}-\beta_{0}}{SE(\hat{\beta_{0}})}
\end{equation}
\begin{equation}
\frac{\hat{\beta_{1}}-\beta_{1}}{SE(\hat{\beta_{0}})}
\end{equation}


\subsubsection{Inference on correlation coefficient} This test of
the slope is coincidentally the equivalent of a test of the
correlation of the $n$ observations of $X$ and $Y$.
\begin{eqnarray}
H_{0}: \rho_{XY} = 0 \nonumber \\
H_{A}: \rho_{XY} \ne 0 \nonumber \\
\end{eqnarray}



\end{document} 
