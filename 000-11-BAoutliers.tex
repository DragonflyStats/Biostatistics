\documentclass[Main.tex]{subfiles}
\begin{document}

\section{Treatment of Outliers}
Bland and Altman attend to the issue of outliers in their 1986
paper, wherein they present a data set with an extreme outlier


	\subsection{Effect of Outliers} Another argument against
	the use of model I regression is based on outliers. Outliers can
	adversely influence the fitting of a regression model. Cornbleet
	and Cochrane compare a regression model influenced by an outlier
	with a model for the same data set, with the outlier excluded from
	the data set. A demonstration of the effect of outliers was made
	in Bland Altman's 1986 paper. However they discourage the
	exclusion of outliers.
	
	\section{Indications on how to deal with outliers in Bland Altman plots}
	
	We wish to determine how outliers should be treated in a Bland
	Altman Plot
	
	In their 1983 paper they merely state that the plot can be used to
	'spot outliers'.
	
	In  their 1986 paper, Bland and Altman give an example of an
	outlier. They state that it could be omitted in practice, but make
	no further comments on the matter.
	\\
	In Bland and Altmans 1999 paper, we get the clearest indication of
	what Bland and Altman suggest on how to react to the presence of
	outliers. Their recommendation is to recalculate the limits
	without them, in order to test the difference with the calculation
	where outliers are retained.\\
	
	The span has reduced from 77 to 59 mmHg, a noticeable but not
	particularly large reduction.
	\\
	However, they do not recommend removing outliers. Furthermore,
	they say:
	\\
	We usually find that this method of analysis is not too sensitive
	to one or two large outlying differences.
	\\
	We ask if this would be so in all cases. Given that the limits of
	agreement may or may not be disregarded, depending on their
	perceived suitability, we examine whether it would possible that
	the deletion of an outlier may lead to a calculation of limits of
	agreement that are usable in all cases?
	\\
	Should an Outlying Observation be omitted from a data set? In
	general, this is not considered prudent.
	\\
	Also, it may be required that the outliers are worthy of
	particular attention themselves.
	\\
	Classifying outliers and recalculating We opted to examine this
	matter in more detail. The following points have to be considered
	\\how to suitably identify an outlier (in a generalized sense)
	\\Would a recalculation of the limits of agreement generally
	results in  a compacted range between the upper and lower limits
	of agreement?
	%=====================================================================%	
	\addcontentsline{toc}{section}{Bibliography}
	\bibliography{transferbib}	
\end{document}
