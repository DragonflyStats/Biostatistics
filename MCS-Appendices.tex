\documentclass[12pt, a4paper]{report}

\usepackage{epsfig}
\usepackage{subfigure}
%\usepackage{amscd}
\usepackage{amssymb}
\usepackage{graphicx}
%\usepackage{amscd}
\usepackage{amssymb}
\usepackage{subfiles}
\usepackage{framed}
\usepackage{subfiles}
\usepackage{amsthm, amsmath}
\usepackage{amsbsy}
\usepackage{framed}
\usepackage[usenames]{color}
\usepackage{listings}
\lstset{% general command to set parameter(s)
	basicstyle=\small, % print whole listing small
	keywordstyle=\color{red}\itshape,
	% underlined bold black keywords
	commentstyle=\color{blue}, % white comments
	stringstyle=\ttfamily, % typewriter type for strings
	showstringspaces=false,
	numbers=left, numberstyle=\tiny, stepnumber=1, numbersep=5pt, %
	frame=shadowbox,
	rulesepcolor=\color{black},
	,columns=fullflexible
} %
%\usepackage[dvips]{graphicx}
\usepackage{natbib}
\bibliographystyle{chicago}
\usepackage{vmargin}
% left top textwidth textheight headheight
% headsep footheight footskip
\setmargins{1.0cm}{0.75cm}{18.5 cm}{22cm}{0.5cm}{0cm}{1cm}{1cm}
%\voffset=-2.5cm
%\oddsidemargin=1cm
%\textwidth = 520pt

\renewcommand{\baselinestretch}{1.5}
\pagenumbering{arabic}
\theoremstyle{plain}
\newtheorem{theorem}{Theorem}[section]
\newtheorem{corollary}[theorem]{Corollary}
\newtheorem{ill}[theorem]{Example}
\newtheorem{lemma}[theorem]{Lemma}
\newtheorem{proposition}[theorem]{Proposition}
\newtheorem{conjecture}[theorem]{Conjecture}
\newtheorem{axiom}{Axiom}
\theoremstyle{definition}
\newtheorem{definition}{Definition}[section]
\newtheorem{notation}{Notation}
\theoremstyle{remark}
\newtheorem{remark}{Remark}[section]
\newtheorem{example}{Example}[section]
\renewcommand{\thenotation}{}
\renewcommand{\thetable}{\thesection.\arabic{table}}
\renewcommand{\thefigure}{\thesection.\arabic{figure}}
\title{Research notes: linear mixed effects models}
\author{ } \date{ }


\begin{document}
	\author{Kevin O'Brien}
	\title{Mixed Models for Method Comparison Studies}
	\tableofcontents




\chapter{Method Comparison Studies}

\section{Introduction}
The problem of assessing the agreement between two or more methods
of measurement is ubiquitous in scientific research, and is
commonly referred to as a `method comparison study'. Published
examples of method comparison studies can be found in disciplines
as diverse as pharmacology \citep{ludbrook97}, anaesthesia
\citep{Myles}, and cardiac imaging methods \citep{Krumm}.
\smallskip

To illustrate the characteristics of a typical method comparison
study consider the data in Table I \citep{Grubbs73}. In each of
twelve experimental trials, a single round of ammunition was fired
from a 155mm gun and its velocity was measured simultaneously (and
independently) by three chronographs devices, identified here by
the labels `Fotobalk', `Counter' and `Terma'.
\smallskip


\newpage

\begin{table}[ht]
	\begin{center}
		\begin{tabular}{rrrr}
			\hline
			Round& Fotobalk [F] & Counter [C]& Terma [T]\\
			\hline
			1 & 793.8 & 794.6 & 793.2 \\
			2 & 793.1 & 793.9 & 793.3 \\
			3 & 792.4 & 793.2 & 792.6 \\
			4 & 794.0 & 794.0 & 793.8 \\
			5 & 791.4 & 792.2 & 791.6 \\
			6 & 792.4 & 793.1 & 791.6 \\
			7 & 791.7 & 792.4 & 791.6 \\
			8 & 792.3 & 792.8 & 792.4 \\
			9 & 789.6 & 790.2 & 788.5 \\
			10 & 794.4 & 795.0 & 794.7 \\
			11 & 790.9 & 791.6 & 791.3 \\
			12 & 793.5 & 793.8 & 793.5 \\
			\hline
		\end{tabular}
		\caption{Velocity measurement from the three chronographs (Grubbs
			1973).}
	\end{center}
\end{table}

An important aspect of the these data is that all three methods of
measurement are assumed to have an attended measurement error, and
the velocities reported in Table 1.1 can not be assumed to be
`true values' in any absolute sense.

%While lack of
%agreement between two methods is inevitable, the question , as
%posed by \citet{BA83}, is 'do the two methods of measurement agree
%sufficiently closely?'

A method of measurement should ideally be both accurate and
precise. \citet{Barnhart} describes agreement as being a broader
term that contains both of those qualities. An accurate
measurement method will give results close to the unknown `true
value'. The precision of a method is indicated by how tightly
measurements obtained under identical conditions are distributed
around their mean measurement value. A precise and accurate method
will yield results consistently close to the true value. Of course
a method may be accurate, but not precise, if the average of its
measurements is close to the true value, but those measurements
are highly dispersed. Conversely a method that is not accurate may
be quite precise, as it consistently indicates the same level of
inaccuracy. The tendency of a method of measurement to
consistently give results above or below the true value is a
source of systematic bias. The smaller the systematic bias, the
greater the accuracy of the method.

% The FDA define precision as the closeness of agreement (degree of
% scatter) between a series of measurements obtained from multiple
% sampling of the same homogeneous sample under prescribed
% conditions. \citet{Barnhart} describes precision as being further
% subdivided as either within-run, intra-batch precision or
% repeatability (which assesses precision during a single analytical
% run), or between-run, inter-batch precision or repeatability
%(which measures precision over time).

In the context of the agreement of two methods, there is also a
tendency of one measurement method to consistently give results
above or below the other method. Lack of agreement is a
consequence of the existence of `inter-method bias'. For two
methods to be considered in good agreement, the inter-method bias
should be in the region of zero. A simple estimate of the
inter-method bias is given by the differences between pairs of measurements, for example,  Table~\ref{FCTdata} is a good example of
possible inter-method bias; the `Fotobalk' consistently recording
smaller velocities than the `Counter' method. A cursory inspection of the table will indicate a systematic tendency for the Counter method to result in higher measurements than the Fotobalk method. % Consequently one would conclude that there is lack of agreement % between the two methods.

The absence of inter-method bias is, by itself, not sufficient to
establish that two measurement methods agree. The two methods
must also have equivalent levels of precision. Should one method
yield results considerably more variable than those of the other,
they can not be considered to be in agreement. Hence, method comparison studies are required to take account of both inter-method bias and difference in precision of measurements.
\newpage
% latex table generated in R 2.6.0 by xtable 1.5-5 package
% Wed Aug 26 15:22:41 2009
\begin{table}[h!]
	
	\begin{center}
		
		\begin{tabular}{rrrr}
			\hline
			Round& Fotobalk (F) & Counter (C) & Difference (F-C) \\
			\hline
			1 & 793.8& 794.6 & -0.8 \\
			2 & 793.1 & 793.9 & -0.8 \\
			3 & 792.4 & 793.2 & -0.8 \\
			4 & 794.0 & 794.0 & 0.0 \\
			5 & 791.4 & 792.2 & -0.8 \\
			6 & 792.4 & 793.1 & -0.7 \\
			7 & 791.7 & 792.4 & -0.7 \\
			8 & 792.3 & 792.8 & -0.5 \\
			9 & 789.6 & 790.2 & -0.6 \\
			10 & 794.4 & 795.0 & -0.6 \\
			11 & 790.9 & 791.6 & -0.7 \\
			12 & 793.5 & 793.8 & -0.3 \\
			\hline
		\end{tabular}
		\caption{Difference between Fotobalk and Counter measurements.}
		\label{FCTdata}\end{center}
\end{table}




\section*{Grubbs Data}

% latex table generated in R 2.6.0 by xtable 1.5-5 package
% Wed Aug 26 15:22:41 2009
\begin{table}[h!]
	
	\begin{center}
		
		\begin{tabular}{rrrr}
			\hline
			Round& Fotobalk (F) & Counter (C) & F-C \\
			\hline
			1 & 793.8& 794.6 & -0.8 \\
			2 & 793.1 & 793.9 & -0.8 \\
			3 & 792.4 & 793.2 & -0.8 \\
			4 & 794.0 & 794.0 & 0.0 \\
			5 & 791.4 & 792.2 & -0.8 \\
			6 & 792.4 & 793.1 & -0.7 \\
			7 & 791.7 & 792.4 & -0.7 \\
			8 & 792.3 & 792.8 & -0.5 \\
			9 & 789.6 & 790.2 & -0.6 \\
			10 & 794.4 & 795.0 & -0.6 \\
			11 & 790.9 & 791.6 & -0.7 \\
			12 & 793.5 & 793.8 & -0.3 \\
			\hline
		\end{tabular}
		\caption{Difference between Fotobalk and Counter measurements.}
	\end{center}
\end{table}


\section{BXC - Model Terms}

\begin{itemize}
	\item Let $y_{mir}$ be the response of method $m$ on the $i$th subject
	at the $r-$th replicate.
	\item Let $\boldsymbol{y}_{ir}$ be the $2 \times 1$ vector of measurements
	corresponding to the $i-$th subject at the $r-$th replicate.
	\item Let $\boldsymbol{y}_{i}$ be the $R_i \times 1$ vector of
	measurements corresponding to the $i-$th subject, where $R_i$ is number of replicate measurements taken on item $i$.
	\item Let $\alpha_mi$ be the fixed effect parameter for method for subject $i$.
	\item Formally Roy uses a separate fixed effect parameter to describe the true value $\mu_i$, but later combines it with the other fixed effects when implementing the model.
	\item Let $u_{1i}$ and $u_{2i}$ be the random effects corresponding to methods for item $i$.
	
	\item $\boldsymbol{\epsilon}_{i}$ is a $n_{i}$-dimensional vector
	comprised of residual components. For the blood pressure data $n_{i} = 85$.
	
	\item $\boldsymbol{\beta}$ is the solutions of the means of the two methods. In the LME output, the bias ad corresponding
	t-value and p-values are presented. This is relevant to Roy's first test.\end{itemize}
%-----------------------------------------------------------------------------------%



\subsection{Test for inter-method bias}
Bias is determinable by examination of the 't-table'. Estimate for both methods are given, and the bias is simply the difference between the two. Because the $R$ implementation does not account for an intercept term, a $p-$value is not given. Should a $p-$value be required specifically for the bias, and simple restructuring of the model is required wherein an intercept term is included. Output from a second implementation will yield a $p-$value.
\newpage

\section{LME}
Consistent with the conventions of mixed models, \citet{pkc}
formulates the measurement $y_{ij} $from method $i$ on individual
$j$ as follows;
\begin{equation}
y_{ij} =P_{ij}\theta + W_{ij}v_{i} + X_{ij}b_{j} + Z_{ij}u_{j} +
\epsilon_{ij},     (j=1,2, i=1,2....n)
\end{equation}
The design matrix $P_{ij}$ , with its associated column vector
$\theta$, specifies the fixed effects common to both methods. The
fixed effect specific to the $j$th method is articulated by the
design matrix $W_{ij}$ and its column vector $v_{i}$. The random
effects common to both methods is specified in the design matrix
$X_{ij}$, with vector $b_{j}$ whereas the random effects specific
to the $i$th subject by the $j$th method is expressed by $Z_{ij}$,
and vector $u_{j}$. Noticeably this notation is not consistent
with that described previously.  The design matrices are specified
so as to includes a fixed intercept for each method, and a random
intercept for each individual. Additional assumptions must also be
specified;
\begin{equation}
v_{ij} \sim N(0,\Sigma),
\end{equation}
These vectors are assumed to be independent for different $i$s,
and are also mutually independent. All Covariance matrices are
positive definite.  In the above model effects can be classed as
those common to both methods, and those that vary with method.
When considering differences, the effects common to both
effectively cancel each other out. The differences of each pair of
measurements can be specified as following;
\begin{equation}
d_{ij} = X_{ij}b_{j} + Z_{ij}u_{j} + \epsilon_{ij},     (j=1,2,
i=1,2....n)
\end{equation}
This formulation has seperate distributional assumption from the
model stated previously.

This agreement covariate $x$ is the key step in how this
methodology assesses agreement.
%%%%%%%%%%%%%%%%%%%%%%%%%%%%%%%%%%%%%%%%%%%%%%%%%%%%%%%%%%%%%%%%%%%%%%%%%%%%%%%%%%%%%%%%%%%%%%%%%%%%%%%5






\section{Remarks}
The relationship between precision and the within-item and between-item variability must be established. Roy establishes the equivalence of repeatability and within-item variability, and hence precision.  The method with the smaller within-item variability can be deemed to be the more precise.

A useful approach is to compute the confidence intervals for the ratio of within-item standard deviations (equivalent to the ratio of repeatability coefficients), which can be interpreted in the usual manner.  

In fact, the ratio of within-item standard deviations, with the attendant confidence interval,  can be determined using a single R command: \texttt{intervals()}.

Pinheiro and Bates (pg 93-95) give a description of how confidence intervals for the variance components are computed. Furthermore a complete set of confidence intervals can be computed to complement the variance component estimates. 

What is required is the computation of the variance ratios of within-item and between-item standard deviations.  

A naive  approach would be to compute the variance ratios by relevant F distribution quantiles. However, the question arises as to the appropriate degrees of freedom.
Limits of agreement are easily computable using the LME framework. While we will not be considering this analysis, a demonstration will be provided in the example.

%============================================================== %


	\chapter{LME Likelihood}

\section{PRESS}
	% The DFBETA for a particular observation is the difference between the regression coefficient for an included variable calculated for all of the data and the regression coefficient calculated with the observation deleted, scaled by the standard error calculated with the observation deleted.
	
	
	
	%===================================%
	
	
	% % - http://support.sas.com/documentation/cdl/en/statug/63347/HTML/default/viewer.htm#statug_mixed_sect027.htm
	An (unconditional) predicted value is $\hat{y}_i = x^{\prime}_i \boldsymbol{\hat{\beta}}$, where
	the vector $x_i$ is the $i$th row of $\boldsymbol{X}$.
	
	%	For an \texttt{lme} object, such as our fitted model \texttt{JS.roy1}, the predicted values for each subject can be determined using the \texttt{coef.lme} function.
	%	\begin{framed}
	%		\begin{verbatim}
	%		> JS.roy1 %>% coef %>% head(5)
	%		methodJ   methodS
	%		74     84.31724  91.08404
	%		36     91.54994  97.05548
	%		3      81.16581  96.48653
	%		62     92.09493  90.89073
	%		31     88.41411 103.38802
	%		\end{verbatim}
	%	\end{framed}
	
 %http://support.sas.com/documentation/cdl/en/statug/63347/HTML/default/viewer.htm#statug_mixed_sect027.htm

	An (unconditional) predicted value is $\hat{y}_i = x^{\prime}_i \boldsymbol{\hat{\beta}}$, where
	the vector $x_i$ is the $i$th row of $\boldsymbol{X}$.
	The (raw) residual is given as $\varepsilon_i = y_i - \hat{y}_i$. The PRESS residual is
	similarly constructed, using the predicted value for observation $i$ with a model fitted from reduced data.
	\[ \varepsilon_{i(U)} = y_i - x^{\prime}_i \boldsymbol{\hat{\beta}}_{(U)} \]
	
	
	%---------------------------------------------------------------------------%
	
	


	\section{One Way ANOVA}
	\subsection{Page 448}
	Computing the variance of $\hat{\beta}$
	\begin{eqnarray}
	\mbox{var}(\hat{\beta}) = (X^{\prime}V^{-1}X)^-1
	\end{eqnarray}
	It is not necessary to compute $V^{-1}$ explicitly.
	
	\begin{eqnarray}
	V^{-1}X &= \Sigma^{1}{X-Z()Z^{\prime}\Sigma^{-1}X} \\
	&= \Sigma^{-1}(X-Zb_{x})
	\end{eqnarray}
	
	The estimate $b_{x}$ is the same term obtained from the random effects model; $X = Zb_{x} + e$, using $X$ as an outcome variable.
	This formula is convenient in applications where $b_{x}$ can be easily computed. Since $X$ is a matrix of $p$ columns, $b_{x}$ can simple be computed column by column. according to the columns of $X$.
	\subsection{Page 448- simple example}
	Consider a simple model of the form;
	\begin{equation*}
	y_{ij} = \mu + \beta_{i} + \epsilon_{ij}.
	\end{equation*}
	
	The iterative procedure is as follows Evaluate the individual group mean $\bar{y_{i}}$ and variance $\hat{Sigma^2}_{i}$. Then use the variance of the group means as an estimate of the $\sigma^2_{b}$. The average of the the variances of the groups is the initial estimate of the $\sigma^2_{e}$.
	\subsubsection{Iterative procedure}
	
	The iterative procedure comprises two steps, with $0$ as the first approximation of $b_{i}$.
	
	The first step is to compute $\lambda$, the ratio of variabilities,
	
	\begin{equation*}
	\lambda = \frac{\sigma^2_{b}}{\sigma^2_{e}}
	\end{equation*}
	
	\begin{eqnarray*}
		\mu = \frac{1}{N} \sum_{ij} (y_{ij} - b_{i}) \\
		b_{i} = \frac{n(\bar{y_{i}}-\mu)}{n+ \lambda} \\
	\end{eqnarray*}
	
	
	The second step is to updat $sigma^2_{e}$
	
	\begin{equation}
	\sigma^2_{e} = \frac{e^{\prime}e}{N-df}
	\end{equation}
	
	where $e$ is the vector of $e_{ij} = y_{ij}-\mu-b_{i}$ and $df =
	qn / n+\lambda$ and
	\begin{equation}
	\sigma^{2}_{b} = \frac{1}{q} \sum_{i=1}^{q} b_{1}^2 +
	(\frac{n}{\sigma^2_{e}}+\frac{1}{\sigma^2_{b}})^{-1}
	\end{equation}
	
	\subsubsection{Worked Example}
	
	Further to [pawitan 17.1] the initial estimates for variability
	are $\sigma^{2}_{b} = 1.7698$ and $\sigma^{2}_{e} = 0.3254$. At
	convergence the following results are obtained.
	\\
	n=16, q=5
	\begin{eqnarray*}
		\hat{\mu} = \bar{y} = 14.175 \\
		\hat{\sigma}^2 = 0.325\\
		\hat{\sigma}^2_{b} = 1.395\\
		\sigma  = 0.986 \\
	\end{eqnarray*}
	At convergene the following estimates are obtained,
	\begin{eqnarray*}
		\hat{\mu} = 14.1751 \\
		\hat{b}= (-0.6211, 0.2683,1.4389,-1.914,0.8279)\\
		\hat{\sigma}^2_{b} = 1.3955\\
		\hat{\sigma}^2_{e} = 0.3254\\
	\end{eqnarray*}
	

	
	

	\section{Sampling}
	\emph{
		One important feature of replicate observations is that they should be independent
		of each other. In essence, this is achieved by ensuring that the observer makes each
		measurement independent of knowledge of the previous value(s). This may be difficult
		to achieve in practice.} (Check who said this
	)
	
	
	
	
	
	

	%=========================================================================================================================================== %
	%=========================================================================================================================================== %
	
	\section*{Permutation Test, Power Tests and Missing Data }
	
	This section explores topics such as dependent variable simulation and power analysis, introduced by Galecki \& Burzykowski (2013), and implementable with their \textbf{\textit{nlmeU}} \texttt{R} package.
	
	Using the \textbf{\textit{predictmeans}} \texttt{R} package, it is possible to perform permutation t-tests for coefficients of (fixed) effects and permutation F-tests.
	
	The matter of missing data has not been commonly encountered in either Method Comparison Studies or Linear Mixed Effects Modelling. However ARoy2009 (2009) deals with the relevant assumptions regrading missing data. 
	
	Galecki \& Burzykowski (2013) approaches the subject of missing data in LME Modelling. The \textbf{\textit{nlmeU}} package includes the \texttt{patMiss} function, which ``\textit{allows to compactly present pattern of missing data in a given vector/matrix/data
		frame or combination of thereof}".
	
	
	%================================================%
	
\subsection{EBLUPS-Diagnostics for Random Effects}
%% West Page 42 Section 2.8.3
\citet{west} recommends the empirical Bayes predictor, also known as EBLUPS as a diagnostic tool for Random effects. Checking EBLUPS for normality is of limited value.


%--------------------------------------------------------------------------------------------------------%
%%EBLUP

%-http://chjs.deuv.cl/Vol3N1/ChJS-03-01-05.pdf

The EBLUP is useful to identify outlier subjects given that it represents the distance
between the population mean value and the value predicted for the ith subject. A way of
using the EBLUP to search for outliers subjects is to use the Mahalanobis distance (see
Waternaux et al., 1989), FORMULA

. It is also possible to use the EBLUP
to verify the random effects normality assumption. For more information; see Nobre and
Singer (2007). In Table 2 we summarize diagnostic techniques involving residuals discussed
in Nobre and Singer (2007).

%----------------------------------------------------%



	\chapter{General Appendices}
	$\Lambda = \frac{\mbox{max}_{H_{0}}L}{\mbox{max}_{H_{1}}L}$
	

\section{Unknown Material}


To standardize the assessment of how influential data is, several measures of influence are commonly used, such as DFBETAS 
and Cook’s Distance.





Although influential cases thus have extreme values on one or more of the variables, they can be onliers
rather than outliers. 

To account for this, the (standardized) deleted residual is defined as the difference between
the observed score of a case on the dependent variable, and the predicted score from the regression
model fitted from data when that case is omitted.


Just as influential cases are not necessarily outliers, outliers are not necessarily influential cases. 

This also holds for deleted residuals. The reason for this is that the amount of influence a case exerts on the regression slope is not only determined by how well its (observed) score is fitted by the specified
regression model, but also by its score(s) on the independent variable(s). The degree to which the scores of a case on the independent variable(s) are extreme is indicated by the leverage of this case. 


%=========================================================%



%\subsubsection{Deviance}
%In statistics, deviance is a quality of fit statistic for a model that is often used for statistical hypothesis testing. It is a generalization of the idea of using the sum of squares of residuals in ordinary least squares to cases where model-fitting is achieved by maximum likelihood.
%
\subsection{Estimation}

\begin{eqnarray}
\hat{\beta} &=& X^{T} \\
\hat{\gamma} &=& G(\hat{\theta})Z^{T}
\end{eqnarray}

The difference between perturbation and residual analysis between the linear and LME models.
The estimates of the fixed effects $\beta$ depend on the estimates of the covariance parameters.
\subsection{Zewotir-Cook's Distance}

Diagnostic tool for variance components
\[ C_{\theta i} =(\hat(\theta)_{[i]} - \hat(\theta))^{T}\mbox{cov}( \hat(\theta))^{-1}(\hat(\theta)_{[i]} - \hat(\theta))\]

\begin{description}
	\item[linear functions]
	$CD(\psi)_i$ does not have to be calculated unless $CD(\beta)_i$ is large.
\end{description}


It is also desirable to measure the influence of the case deletions on the covariance matrix of $\hat{\beta}$.
% % - Fixed Effects Parameter		

For fixed effects parameter estimates in LME models, the \index{Cook's distance} Cook's distance can be extended to measure influence on these fixed effects.

\[
\mbox{CD}_{i}(\beta) = \frac{(c_{ii} - r_{ii}) \times t^2_{i}}{r_{ii} \times p}
\]


For random effect estimates, the \index{Cook's distance} Cook's distance is

\[
\mbox{CD}_{i}(b) = g{\prime}_{(i)} (I_{r} + \mbox{var}(\hat{b})D)^{-2}\mbox{var}(\hat{b})g_{(i)}.
\]

Large values for Cook's distance indicate observations for special attention.


For linear functions, $CD(\psi)_i$ does not have to be calculated unless $CD(\beta)_i$ is large.
%===================================================== %
\subsubsection{Mean Square Prediction Error}
\begin{equation}
MSPR = \frac{\sum (y_{i}-\hat{y}_{i})^2}{n^*}
\end{equation}

%===================================================== %
\subsection{Leverage}
Leverage can be defined through the projection matrix that results from a transformation of the model with the inverse of the Cholesky decomposition of $\boldsymbol{V}$, or an oblique projector:	$\boldsymbol{Y} = \boldsymbol{H}\boldsymbol{\hat{Y}}$.

While $H$ is idempotent, it is generally not symmetric and thus not a projection matrix in the narrow sense.
\[ h_{ii} = x^{\prime}_{i}(X^{\prime}X)^{-1}x_{i} \]
The trace of $\boldsymbol{H}$ equals the rank of $\boldsymbol{X}$.
If $V_{ij}$ denotes the element in row $i$, column $j$ of $\boldsymbol{V}^{-1}$, then for a model containing only an intercept the diagonal elements of $\boldsymbol{H}$.

\[ h_{ii} = \frac{\sum v_{ij}}{\sum \sum v_{ij}} \]


\subsection{Local Influence}

\citet{Christensen} developed their global influences for the deletion of single observations in two steps: a one-step estimate for the REML (or ML) estimate of the variance components, and an ordinary case-deletion diagnostic for a weighted regression problem (conditional on the estimated covariance matrix) for fixed effects.


	
	%http://blog.minitab.com/blog/adventures-in-statistics/why-you-need-to-check-your-residual-plots-for-regression-analysis
	In the graph above, you can predict non-zero values for the residuals based on the fitted value. For example, a fitted value of 8 has an expected residual that is negative. Conversely, a fitted value of 5 or 11 has an expected residual that is positive.
	
	The non-random pattern in the residuals indicates that the deterministic portion (predictor variables) of the model is not capturing some explanatory information that is “leaking” into the residuals. The graph could represent several ways in which the model is not explaining all that is possible. 
	
	Possibilities include:
	
	\begin{itemize}
		\item A missing variable
		\item A missing higher-order term of a variable in the model to explain the curvature
		\item A missing interction between terms already in the model
	\end{itemize}
	
	
	Identifying and fixing the problem so that the predictors now explain the information that they missed before should produce a good-looking set of residuals.
	
	In addition to the above, here are two more specific ways that predictive information can sneak into the residuals:
	
	The residuals should not be correlated with another variable. If you can predict the residuals with another variable, that variable should be included in the model. In Minitab’s regression, you can plot the residuals by other variables to look for this problem.
	
	\noindent \textbf{Autocorrelation} \\
	Adjacent residuals should not be correlated with each other (\textbf{autocorrelation}). If you can use one residual to predict the next residual, there is some predictive information present that is not captured by the predictors. Typically, this situation involves time-ordered observations. For example, if a residual is more likely to be followed by another residual that has the same sign, adjacent residuals are positively correlated. You can include a variable that captures the relevant time-related information, or use a time series analysis. 
	
	In Minitab’s regression, you can perform the \textbf{\textit{Durbin-Watson} }test to test for autocorrelation.
	

	
	
	

	





	\section{RSquared for LME models}
	
	As a complement to this, one can also consider how to properly employ the $R^2$ measure, in the context of Methoc Comparison Studies, further to the work by Edwards et al, namely ``An $R^2$ statistic for fixed effects in the linear mixed model".
	
	\begin{framed}
		
		\begin{quote}
			\textbf{Abstract for ``An $R^2$ statistic for fixed effects in the linear mixed model"}
			Statisticians most often use the linear mixed model to analyze Gaussian longitudinal data. 
			
			The value and familiarity of the R2 statistic in the linear univariate model naturally creates great interest in extending it to the linear mixed model. We define and describe how to compute a model R2 statistic for the linear mixed model by using only a single model. 
			
			The proposed R2 statistic measures multivariate association between the repeated outcomes and the fixed effects in the linear mixed model. The R2 statistic arises as a 1–1 function of an appropriate F statistic for testing all fixed effects (except typically the intercept) in a full model. 
			
			The statistic compares the full model with a null model with all fixed effects deleted (except typically the intercept) while retaining exactly the same covariance structure. 
			
			Furthermore, the R2 statistic leads immediately to a natural definition of a partial R2 statistic. A mixed model in which ethnicity gives a very small p-value as a longitudinal predictor of blood pressure (BP) compellingly illustrates the value of the statistic. 
			
			In sharp contrast to the extreme p-value, a very small $R^2$ , a measure of statistical and scientific importance, indicates that ethnicity has an almost negligible association with the repeated BP outcomes for the study.
		\end{quote}
	\end{framed}
	

	\chapter{Residual Diagnostics}

\title{Roy's Candidate models}
The original Bland Altman Method was developed for two sets of
measurements done on one occasion (i.e. independent data), and so
this approach is not suitable for repeated measures data. However,
as a naïve analysis, it may be used to explore the data because of
the simplicity of the method. Myles states that such misuse of the
standards Bland Altman method is widespread in Anaesthetic and
critical care literature.
\\
\\
Bland and Altman have provided a modification for analysing
repeated measures under stable or chaninging conditions, where
repeated data is collected over a period of time. Myers proposes
an alternative Random effects model for this purpose.
\\
\\
with repeated measures data, we can
calculate the mean of the repeated measurements by each method on
each individuals. \emph{ The pairs of means can then be used to
	compare the two methods based on the 95\% limits of agreement for
	the difference of means. The bias between the two methods will not
	be affected by averaging the repeated measurements.}.However the
variation of the differences will be underestimated by this
practice because the measurement error is, to some extent,
removed. Some advanced statistical calculations are needed to take
into account these measurement errors. \emph{Random effects models
	can be used to estimate the within-subject variation after
	accounting for other observed and unobserved variations, in which
	each subject has a different intercept and slope over the
	observation period .On the basis of the within-subject variance
	estimated by the random effects model, we can then create an
	appropriate Bland Altman Plot.}The sequence or the time of the
measurement over the observation period can be taken as a random
effect.








\chapter{BA99}



\chapter{Random Effects and MCS}
\section{Random Effects and MCS}
The methodology comprises two calculations. The second calculation
is for the standard deviation of means Before the modified Bland
and Altman method can be applied for repeated measurement data, a
check of the assumption that the variance of the repeated
measurements for each subject by each method is independent of the
mean of the repeated measures. This can be done by plotting the
within-subject standard deviation against the mean of each subject
by each method. Mean Square deviation measures the total deviation
of a


\subsection{Random coefficient growth curve model} (Chincilli
1996) Random coefficient growth curve model, a special type of
mixed model have been proposed a single measure of agreement for
repeated measurements.
\begin{equation}
\textbf{d}= \textbf{Xb} + \textbf{Zu} + \textbf{e}
\end{equation}
The distributional asummptions also require \textbf{d} to
\textbf{N}


\section{Random effects Model} \citet{Myles} proposes the use of
Random effects models to address the issue of repeated
measurement. 

Myles proposes a formulation of the Bland–Altman
plot, using the within-subject variance estimated by the random
effects model, with the time of the measurement taken as a random
effect. He states that \emph{random effects models account for the
	dependent nature of the data, and additional explanatory
	variables, to provide reliable estimates of agreement in this
	setting.}
\\
Agreement between methods is reflected by the between-subject
variation.The Random Effects Model takes this into account before
calculating the within-subject standard deviation.

\subsection{Myers Random Effects Model} The presentation of the
95\% limits of agreement is for visual judgement of how well two
methods of measurement agree. The smaller the range between the
two, the better the agreement is The question of small is small is
a question of clinical judgement


Repeated measurements for each subjects are often used in clinical
research.



\subsection{Random Effects Modelling}
Random effects models are used to examine the within-subject
variation after adjusting for known and unknown variables, in
which each subject has a different intercept and slope over a time
period period.


\citet{Myles} remarks that the random effects model is an
extension of the analysis of variance method, accounting for more
covariates.

A random effect (in Myles's case, time of measurement) is chosen
to reflect the different intercept and slope for each subject with
respect to their change of measurements over the time period.

In Myles's methodology, the standard deviation of difference
between the means of the repeated measurements can be calculated
based on the within-subject standard deviation estimates.

A random effects model (also variance components model)is a type
of hierarchical linear model. Hierarchical linear modelling (HLM)
is a more advanced form of simple linear regression and multiple
linear regression. HLM is appropriate for use with nested
data.\\Faraway comments that the random effects approach is
\emph{more ambitious than the LME model in that it attempts to say
	something about the wider population beyond the particular
	sample}.

\section{Other Approaches : Marginal Modelling}
(Diggle 2002) proposes the use of marginal models as an
alternative to mixed models.m Marginal models are appropriate when
interences about the mean response are of specific interest.

\section{Other Approaches}


\citet{pkcng} generalize this approach to account for situations
where the distributions are not identical, which is commonly the
case. The TDI is not consistent and may not preserve its
asymptotic nominal level, and that the coverage probability
approach of \citet{lin2002} is overly conservative for moderate
sample sizes. This methodology proposed by \citet{pkcng} is a
regression based approach that models the mean and the variance of
differences as functions of observed values of the average of the
paired measurements.



%-------------------------------------------------

\begin{enumerate}
	\item Agreement and Method Comparison Studies
	\begin{enumerate}
		\item What is Agreement?
		\item Repeatability
		\item
		\item
		\item
	\end{enumerate}
	\item Bland Altman Single Observations
	\begin{enumerate}
		\item
		\item
	\end{enumerate}
	\item Alternative Methods
	\begin{enumerate}
		\item Deming Regression
		\item Mountain Plot
		\item Bartko's Ellipse
		\item Formal Tests and Procedures
	\end{enumerate}
	\item Replicate Observations
	
	\item LME models
	
	\item Estimation and Algorithms
	\begin{enumerate}
		\item ML and REML estimation
		\item MINQUE
		\item
	\end{enumerate}
	\item Residual Diagnostics
	\begin{enumerate}
		\item Marginal and Conditional Diagnostics
		\item Scaled Residuals
	\end{enumerate}
	
	\item Influence Diagnostics
	\begin{enumerate}
		\item Underlying Concepts
		\item Managing the Covariance Parameters
		\item Predicted Values, PRESS Residual and the PRESS Statistic
		\item Leverage
		\item Internally and Externally Studentized Residuals
		\item DFFITs and MDFFITs
		\item Covariance Ratio and Trace
		\item Likelihood Distance
		\item Non-iterative Update Procedures
	\end{enumerate}
\end{enumerate}
\newpage

\section{MCS Data Sets}
\begin{enumerate}
	\item Blood Data
	\item Cardiac Data
	\item Nadler Hurley 
\end{enumerate}

%%%%%%%%%%%%%
%1 Method Comparison Studies            %%%%%%%%%%%%%%%%%%%%%%%%%%%%%%%%%
%%%%%%%%%%%%%%%%%%%%%%%%%%%%%%%%%%%%%%%%%%%%%%%%%%%%%%%%%%%%%%%%%%%%%%%%%
\begin{itemize}		
	%-------------------------------------------------%
	% Chapter 1
	
	\item	Introduction to Method Comparison Studies	
	\begin{itemize}	
		\item	Accuracy and Precision
		\item	Repeatability (Bland Altman 1999)
		\item	Barnharts Paper
		\item	
	\end{itemize}	
	
	%-------------------------------------------------%
	% Chapter 2
	
	\item	Bland and Altman Plot	
	\begin{itemize}	
		\item	Bland and Altman 1983 and 86
		\item	Limits of Agreement
		\item	
		\item	
	\end{itemize}
	
	
\end{itemize}		

%====================================================================%

\section{Introduction}

Outliers and detection of influent observations is an important step in the analysis of a data set. There are several ways of evaluating the influence of perturbations in the data set and in the model given the parameter estimates. 

\subsection{Overview of R implementations}
Further to previous material, an appraisal of the current state of development (or lack thereof) for current implemenations for LME models, particularly for \texttt{nlme} and \texttt{lme4} fitted models.

Crucially, a review of internet resources indicates that almost all of the progress in this regard has been done for \texttt{lme4} fitted models, specifically the \textit{Influence.ME} \texttt{R} package. (Nieuwenhuis et al 2014)
Conversely there is very little for \texttt{nlme} models. One would immediately look at the current development workflow for both packages.

%======================%
% Douglas Bates

As an aside, Douglas Bates was arguably the most prominent \texttt{R} developer working in the LME area. 
However Bates has now prioritised the development of LME models in another computing environment , i.e Julia. 
% The current version of this is XXXX

%======================%
% nlme

With regards to \texttt{nlme}, the package is now maintained by the \texttt{R} core development team. The most recent major text is by Galecki \& Burzykowski, who have published \textit{ Linear Mixed Effects Models using \texttt{R}. }
Also, the accompanying \texttt{R} package, nlmeU package is under current development, with a version being released $0.70-3$.


%======================%
% lme4 and influence.ME

The \textbf{lme4} pacakge is used to fit linear and generalized linear mixed-effects models in the R environment.
The \textbf{lme4} package is also under active development, under the leadership of Ben Bolker (McMaster Uni., Canada).


%=====================%
\subsection*{Important Consideration for MCS}

The key issue is that \texttt{nlme} allows for the particular specification of Roy's Model, speciifically direct specification of the VC matrices for within subject and between subject residuals.
The \texttt{lme4} package does not allow for Roy's Model, for reasons that will identified shortly.
To advance the ideas that eminate from Roys' paper, one is required to use the \texttt{nlme} context. However, to take advantage of the infrastructure already provided for \texttt{lme4} models, one may change the research question away from that of Roy's paper. 
To this end, an exploration of what textbf{influence.ME} can accomplished is merited.




%--------------------------------------------------------------%
\newpage
\section{Computation and Notation } %2.3
with $\boldsymbol{V}$ unknown, a standard practice for estimating $\boldsymbol{X \beta}$ is the estime the variance components $\sigma^2_j$,
compute an estimate for $\boldsymbol{V}$ and then compute the projector matrix $A$, $\boldsymbol{X \hat{\beta}}  = \boldsymbol{AY}$.


Zewotir remarks that $\boldsymbol{D}$ is a block diagonal with the $i-$th block being $u \boldsymbol{I}$

\section{Lai Shiao}
\citet{LaiShiao} use mixed models to determine the factors that
affect the difference of two methods of measurement using the
conventional formulation of linear mixed effects models.

If the parameter \textbf{b}, and the variance components are not
significantly different from zero, the conclusion that there is no
inter-method bias can be drawn. If the fixed effects component
contains only the intercept, and a simple correlation coefficient
is used, then the estimate of the intercept in the model is the
inter-method bias. Conversely the estimates for the fixed effects
factors can advise the respective influences each factor has on
the differences. The Proc Mixed package allows users to specify
different correlation structures of the variance components
\textbf{G} and \textbf{R}.


Oxygen saturation is one of the most frequently measured variables
in clinical nursing studies. `Fractional saturation' ($HbO_{2}$)
is considered to be the gold standard method of measurement, with
`functional saturation' ($SO_{2}$) being an alternative method.
The method of examining the causes of differences between these
two methods is applied to a clinical study conducted by
\citet{Shiao}. This experiment was conducted by 8 lab
practitioners on blood samples, with varying levels of
haemoglobin, from two donors. The samples have been in storage for
varying periods ( described by the variable `Bloodage') and are
categorized according to haemoglobin percentages(i.e
$0\%$,$20\%$,$40\%$,$60\%$,$80\%$,$100\%$). There are 625
observations in all.

\citet{LaiShiao} fits two models on this data, with the lab
technicians and the replicate measurements as the random effects
in both models. The first model uses haemoglobin level as a fixed
effects component. For the second model, blood age is added as a
second fixed factor.

\subsubsection{Single fixed effect} The first model fitted by \citet{LaiShiao} takes the
blood level as the sole fixed effect to be analyzed. The following
coefficient estimates are estimated by `Proc Mixed';
\begin{eqnarray}
\mbox{fixed effects :   } 2.5056 - 0.0263\mbox{Fhbperct}_{ijtl} \\
(\mbox{p-values :   } = 0.0054, <0.0001, <0.0001)\nonumber\\\nonumber\\
\mbox{random effects :   } u(\sigma^{2}=3.1826) + e_{ijtl}
(\sigma^{2}_{e}=0.1525, \rho= 0.6978) \nonumber\\
(\mbox{p-values :   } = 0.8113, <0.0001, <0.0001)\nonumber
\end{eqnarray}

With the intercept estimate being both non-zero and statistically
significant ($p=0.0054$), this models supports the presence
inter-method bias is $2.5\%$ in favour of $SO_{2}$. Also, the
negative value of the haemoglobin level coefficient indicate that
differences will decrease by $0.0263\%$ for every percentage
increase in the haemoglobin .

In the random effects estimates, the variance due to the
practitioners is $3.1826$, indicating that there is a significant
variation due to technicians ($p=0.0311$) affecting the
differences. The variance for the estimates is given as $0.1525$,
($p<0.0001$).

\subsubsection{Two fixed effects}
Blood age is added as a second fixed factor to the model,
whereupon new estimates are calculated;
\begin{eqnarray}
\mbox{fixed effects :   } -0.2866 + 0.1072 \mbox{Bloodage}_{ijtl}
- 0.0264\mbox{Fhbperct}_{ijtl}\nonumber\\
( \mbox{p-values :   } = 0.8113, <0.0001, <0.0001)\nonumber\\\nonumber\\
\mbox{random effects :   } u(\sigma^{2}=10.2346) + e_{ijtl}
(\sigma^{2}_{e}=0.0920, \rho= 0.5577) \nonumber\\
(\mbox{p-values :   } = 0.0446, <0.0001, <0.0001)
\end{eqnarray}


With this extra fixed effect added to the model, the intercept
term is no longer statistically significant. Therefore, with the
presence of the second fixed factor, the model is no longer
supporting the presence of inter-method bias. Furthermore, the
second coefficient indicates that the blood age of the observation
has a significant bearing on the size of the difference between
both methods ($p <0.0001$). Longer storage times for blood will
lead to higher levels of particular blood factors such as MetHb
and HbCO (due to the breakdown and oxidisation of the
haemoglobin). Increased levels of MetHb and HbCO are concluded to
be the cause of the differences. The coefficient for the
haemoglobin level doesn't differ greatly from the single fixed
factor model, and has a much smaller effect on the differences.
The random effects estimates also indicate significant variation
for the various technicians; $10.2346$ with $p=0.0446$.

\citet{LaiShiao} demonstrates how that linear mixed effects models
can be used to provide greater insight into the cause of the
differences. Naturally the addition of further factors to the
model provides for more insight into the behavior of the data.



\newpage
\section{ Liao Shaio}

Lai et Shiao is interesting in that it extends the usual method comparison study question. It correctly identifies LME models as a methodoloy that can used to make such questions tractable.
The Data Set used in their examples is unavailable for independent use. Therefore, for the sake of consistency, a data set will be simulated based on the Blood Data that will allow for extra variables.



%============================================================================%

% - http://journal.r-project.org/archive/2012-2/RJournal_2012-2_Nieuwenhuis~et~al.pdf

% - http://www.rensenieuwenhuis.nl/tag/lme4/

% - http://lme4.r-forge.r-project.org/lMMwR/lrgprt.pdf

%============================================================================%

A Study of the Bland-Altman Plot and its Associated Methodology

Joseph G. Voelkel Bruce E. Siskowski 

% - https://www.rit.edu/kgcoe/cqas/sites/rit.edu.kgcoe.cqas/files/docs/TR%202005-3.pdf

%============================================================================%

% - http://sprouts.aisnet.org/785/1/TAMReview.pdf
% - http://organizacija.fov.uni-mb.si/index.php/organizacija/article/viewFile/557/999

\section{Hamlett and Lam}
The methodology proposed by \citet{Roy2009} is largely based on \citet{hamlett}, which in turn follows on from \citet{lam}.

%Lam 99
%In many cases, repeated observation are collected from each subject in sequence  and/or longitudinally.

%Hamlett
%Hamlett re-analyses the data of lam et al to generalize their model to cover other settings not covered by the Lam %method.






The desired outcome of this research is to

\begin{itemize}
	\item Formulate a methodology that represents Best practice in Method Comparison Studies. Indeed the methodology is envsiaged to advance what is considered best practice, inter alia, by making diagnostics procedures a standard part of MCS. 
	
	\item Provide for ease of use such that non-statisticians can master and implement the method, with a level of training that one would expect 
	as part of a Professional CPD programe.
	
\end{itemize}

Apropos of the matter of ease-of-use, certain assumptions must be made.


The user has a reasonable amount of computer literacy.
The user would have a reasonable understanding of statistics, consistent with an undergraduate statistics module. 
That is to say, that the user is acquainted with the idea of $p-$values.

Easy to follow set of instructions to properly implement the method.

%==============================================================================%

Linear Mixed Effects Models can be implemented by using one of the following R packages.
lme4
nlme

The first package to be introducted was nlme, developled by Jose Pinheiro and Douglas Bates ( Authors of the the companion textbook, NAME)

As this package has been under ongoing development for quite a long time, it is now allows for a lot of complex LME implementations. 
Furthermore, nlme is one of the base R packages.  That is to say, when one downloads and installs R, nlme is automatically installed also, and can be called immediately.

Having said that, the authors have pointed to several limitations of the overall methodology thrugh R.
The original developers have both left the project, but other statisticians have taken over the development, and indeed a new version of nlme was released.

LME4 is a more recent package. at a glance, the syntax is easier, but the development is less advanced. There are several functionalities that can not be implemented with lme4 yet. 
As an example - CHAP5 in PB - has no equivalent in LME4. Indeed no textbook exists to co-incide with LME4.

The main author, Douglas Bates, has turned his attention to development of LME models in the Julia programming language.

The nlmeU package is described by its authors as an extesntion of the nlme package, and indeed provides for additionally functionality. The package is also useful as it serves as a companion piece to the 
book by Galecki and Burzwhatski.

The nlme package also allows for the specification of GLS models.

%-----------------------------------------------------------------------------%
\subsection*{Objects and Classes}

The main nlme object is an \texttt{nlme} model.

The main lme4 object is called an \texttt{lmer} model

The lattice package is used for graphical methods.

%=============================================================================%

Model Diagnostics with \texttt{nlme}





\subsection{Inappropriate Techniques for MCS}


\subsection{Links and Papers}
\begin{verbatim}

Westgard Statistics  - http://www.westgard.com/lesson23.htm
\end{verbatim}





%==============================================================================%
\subsection*{Measurement Systems Analysis}The topic of measurement sensitivity anaylysis (MSA, also known as Gauge R\&R) is prevalent in industrial statistics (i.e Six Sigma).

There is extensive literature that covers the area. For the sake of brevity, we will use Cano et al.

For sake of clarity, Cano's definitions of repeatability and reproducibility are listed, with added emphasis.

Reproducibility is rarely, if ever, discussed in the domain of Method Comparison Studies. This may be due to the fact that prevalent methodologies can be used for the problem.However
the methodologies proposed by this research can easily be extended.






\section*{Bayesian BA - Philip J Schluter}
Bayesian Bland Altman Approaches
%================================%
A multivariate hierarchical Bayesian approach to measuring agreement in repeated
measurement method comparison studies

*http://www.biomedcentral.com/1471-2288/9/6*



\subsection*{Background}
Assessing agreement in method comparison studies depends on two fundamentally important components; validity (the between method agreement) and reproducibility (the within method agreement). 

The Bland-Altman limits of agreement technique is one of the favoured approaches in medical literature for assessing between method validity. However, few researchers have adopted this approach for the assessment of both validity and reproducibility. 

This may be partly due to a lack of a flexible, easily implemented and readily available statistical machinery to analyse repeated measurement method comparison data.

\textbf{Methods}\\
Adopting the Bland-Altman framework, but using Bayesian methods, we present this statistical machinery. Two multivariate hierarchical Bayesian models are advocated, one which assumes that the underlying values for subjects remain static (exchangeable replicates) and one which assumes that the underlying values can change between repeated measurements (non-exchangeable replicates).

\textbf{Results}\\
We illustrate the salient advantages of these models using two separate datasets that have been previously analysed and presented; 
(i) assuming static underlying values analysed using both multivariate hierarchical Bayesian models,  
(ii) assuming each subject's underlying value is continually changing quantity and analysed using the non-exchangeable replicate multivariate hierarchical Bayesian model.  

\textbf{Conclusion}
These easily implemented models allow for full parameter uncertainty, simultaneous method comparison, handle unbalanced or missing data, and provide estimates and credible regions for all the parameters of interest. Computer code for the analyses in also presented, provided in the freely available and currently cost free software package WinBUGS.
<hr>

\section*{Bayesian Approach}
A multivariate hierarchical Bayesian approach to measuring agreement in repeated measurement method comparison studies
PJ Schluter - BMC medical research methodology, 2009 - biomedcentral.com

\begin{itemize}
	\item Assessing agreement in method comparison studies depends on two fundamentally important 
	components; validity (the between method agreement) and reproducibility (the within method 
	agreement). 
	\item The Bland-Altman limits of agreement technique is one of the f
\end{itemize}

%========================%
\section{Escaramis}
% Escaramis et al 2010
% http://www.biomedcentral.com/1471-2288/10/31/
%-----------------------------------------------------%
\subsection{Background}
In an agreement assay, it is of interest to evaluate the degree of agreement between the different methods (devices, instruments or observers) used to measure the same characteristic. We propose in this study a technical simplification for inference about the total deviation index (TDI) estimate to assess agreement between two devices of normally-distributed measurements and describe its utility to evaluate inter- and intra-rater agreement if more than one reading per subject is available for each device.

\subsection{Methods}
We propose to estimate the TDI by constructing a probability interval of the difference in paired measurements between devices, and thereafter, we derive a tolerance interval (TI) procedure as a natural way to make inferences about probability limit estimates. We also describe how the proposed method can be used to compute bounds of the coverage probability.

\subsection{Results}
The approach is illustrated in a real case example where the agreement between two instruments, a handle mercury sphygmomanometer device and an OMRON 711 automatic device, is assessed in a sample of 384 subjects where measures of systolic blood pressure were taken twice by each device. A simulation study procedure is implemented to evaluate and compare the accuracy of the approach to two already established methods, showing that the TI approximation produces accurate empirical confidence levels which are reasonably close to the nominal confidence level.

\subsection{Conclusions}
The method proposed is straightforward since the TDI estimate is derived directly from a probability interval of a normally-distributed variable in its original scale, without further transformations. Thereafter, a natural way of making inferences about this estimate is to derive the appropriate TI. Constructions of TI based on normal populations are implemented in most standard statistical packages, thus making it simpler for any practitioner to implement our proposal to assess agreement.

%-----------------------------------------------------%
%PAGE 3 of 12 COLUMN 1
Lin defined the TDI as the boundary, $\kappa_P$ which capyures a large proportion $p$ of paired based differences from 
two devices or observers within the boundary.

The value of $\kappa_P$ that yeilds $P(|D| <\kappa_p) = p$ where D is the paired-difference variate.

%-----------------------------------------------------%
%PAGE 3 of 12 COLUMN 2

\[ \kappa_P = F^{-1}(p) = \sigma_D \sqrt{\chi^2(p,1,\mu^2_D/\sigma^2_d) }  \]

\[ \kappa_P = Z_{\frac{1+p}{2}} \| \varepsilon\| \]

%-----------------------------------------------------%

%PAGE 5 of 12 COLUMN 1

Tolerance Interval around the TDI estimate

\[ \hat{\kappa_p} = \hat{\mu}_D = Z_{p_i}\sigma_d \]

%-----------------------------------------------------%

%PAGE 5 of 12 COLUMN 2
Coverage Probability is another user friendly measure of agrre,ment which is related to the computation of the TDI.

\section{Schabenberger}

\emph{schab} examines the use and implementation of
influence measures in LME models.

Influence is understood to be the ability of a single or multiple
data points, through their presences or absence in the data, to
alter important aspects of the analysis, yield qualitatively
different inferences, or violate assumptions of the statistical
model (\textit{schabenberger}).

Outliers are the most noteworthy data points in an analysis, and
an objective of influence analysis is how influential they are,
and the manner in which they are influential.

\emph{schab} describes a simple procedure for quantifying
influence. Firstly a model should be fitted to the data, and
estimates of the parameters should be obtained. The second step is
that either single of multiple data points, specifically outliers,
should be omitted from the analysis, with the original parameter
estimates being updated. 

This is known as `\textit{leave one out \ leave k
	out}' analysis. The final step of the procedure is comparing the
sets of estimates computed from the entire and reduced data sets
to determine whether the absence of observations changed the
analysis.

\textit{schabenberger} notes that it is not always possible to
derive influence statistics necessary for comparing full- and
reduced-data parameter estimates. 

%
%\begin{abstract}
%	\noindent This paper reviews the use of diagnostic measures for LME models in SAS. This text has been widely cited by texts that don't deal with SAS implementations.
%\end{abstract}
%

%==================================================================================================== %

In recent years, mixed models have become invaluable tools in the analysis of experimental and observational
data. In these models, more than one term can be subject to random variation. Mixed model
technology enables you to analyze complex experimental data with hierarchical random processes, temporal,
longitudinal, and spatial data, to name just a few important applications. 
%
%\subsection{Stating the LME Model}
%The general linear mixed
%model is
%\[
%Y = X\beta + Zu + \varepsilon\]
%where Y is a $(n\times1)$ vector of observed data, X is an $(n\times p)$ fixed-effects design or regressor matrix of rank
%k, Z is a $(n \times g)$ random-effects design or regressor matrix, $u$ is a $(g \times 1)$ vector of random effects, and $\varepsilon$ is
%an $(n\times1)$ vector of model errors (also random effects). The distributional assumptions made by the MIXED
%procedure are as follows: γ is normal with mean 0 and variance G; $\varepsilon$ is normal with mean 0 and variance
%R; the random components $u$ and $\varepsilon$ are independent. Parameters of this model are the fixed-effects β and
%all unknowns in the variance matrices G and R. The unknown variance elements are referred to as the
%covariance parameters and collected in the vector $theta$.
%===========================================================================%

\emph{schab} remarks that the concept of critiquing the model-data agreement applies in mixed models in the same way as in linear
fixed-effects models. In fact, because of the more complex model structure, you can argue that model and
data diagnostics are even more important. For example, you are not only concerned with capturing the
important variables in the model. You are also concerned with ``distributing” them correctly between the
fixed and random components of the model. The mixed model structure presents unique and interesting
challenges that prompt us to reexamine the traditional ideas of influence and residual analysis.
%==========================================================================%
%This paper presents the extension of traditional tools and statistical measures for influence and residual
%analysis to the linear mixed model and demonstrates their implementation in the MIXED procedure (experimental
%features in SAS 9.1). The remainder of this paper is organized as follows. The “Background” section
%briefly discusses some mixed model estimation theory and the challenges to model diagnosis that result
%from it.

%	 The diagnostics implemented in the MIXED procedure are discussed in the “Residual Diagnostics
%	in the MIXED Procedure” section (page 3) and the “Influence Diagnostics in the MIXED Procedure” section
%	(page 5). The syntax options and suboptions you use to request the various diagnostics are briefly sketched
%	in the “Syntax” section (page 9). The presentation concludes with an example.
%	
%	
%====================================================================================================================%



\section{Hawkins : Diagnostics for conformity of paired quantitative measurements}

\begin{itemize}
	\item Matched pairs data arise in many contexts – in case-control clinical trials, for example, and from cross-over designs. They also arise in experiments to verify the equivalence of quantitative assays. This latter use (which is the main focus of this paper) raises difficulties not always seen in other matched pairs applications. 
	
	\item Since the designs deliberately vary the analyte levels over a wide range, issues of variance dependent on mean, calibrations of differing slopes, and curvature all need to be added to the usual model assumptions such as normality. 
	
	\item Violations in any of these assumptions invalidate the conventional matched pairs analysis. 
	
	\item A graphical method, due to Bland and Altman, of looking at the relationship between the average and the difference of the members of the pairs is shown to correspond to a formal testable regression model. 
	
	\item Using standard regression diagnostics, one may detect and diagnose departures from the model assumptions and remedy them – for example using variable transformations. Examples of different common scenarios and possible approaches to handling them are shown.
\end{itemize}
%====================================%

% 1. Problem Description
% 2. A Statistical Formulation
%    2.1 Regression Methods for Testing Equivalence
%    2.2 Matched Pair Analysis
% 3. Regression of Differences on Sums
%    3.1 Use of Formal regression Diagnostics
% 4. Examples
%    4.1 Example 1
%    4.2 Detection and Cure of Heteroscedascity
%    4.3 Example 2
%    4.4 Example 3
%    4.5 Example 4
%    4.6 Example 5
% 5. Precepts for Design
% 6. Conclusion


%====================================%
A multi-Rate nonparametric test of agreement and corresponding agreement plot

- Published in: Computational Statistics and Data Analysis 54(2010)109-119
- Author: Alan D. Hutson, University of Buffalo



This approach takes advantage of readily avilable tests of uniformity found in most statistical software packages.
Such tests include the KS d statistic, the Anderson Darling Statistic and the Cramer-Von Mises statistical test for univariate data.

An important aspect of this approach is the "Agreement Region".




\section{Turkan's LMEs}
% Influence Analysis in the LME Models
% Turkan and Toktamus
% Pakistan Journal of Statistics;2012, Vol. 28 Issue 3, p341

% - http://www.tandfonline.com/doi/abs/10.1080/03610920903564727?journalCode=lsta20#.VHj9vfmsXE4

The linear mixed model is considerably sensitive to outliers and influential observations. 
It is known that outliers and influential observations affect substantially the results of analysis. 
So it is very important to be aware of these observations. 

Some diagnostics which are analogue of diagnostics in multiple linear regression were developed to detect 
outliers and influential observations in the linear mixed model. 
\emph{
	In this paper, the new diagnostic measure which is analogue of the Pena's influence statistic is developed for 
	the linear mixed model.
}
\newpage
%-----------------------------------------------------------------------------------------%

%page 341

%-----------------------------------------------------------------------------------------%
%page 342
Estimation and Building blacks in LME models

%-----------------------------------------------------------------------------------------%
%page 343

\[ \hat{u} = DZ^{T}H^{-1}(y-X\hat{\beta}) \]

\[ \hat{y} = (I_n -  H^{-1})y + H^{-1}X\hat{\beta}\]

%-----------------------------------------------------------------------------------------%
%page 345

The proposed diagnostic Measure.

%-----------------------------------------------------------------------------------------%
% http://halweb.uc3m.es/esp/Personal/personas/dpena/articles/TECHanews2005.pdf

\subsection{Ordinary Least Product Regression}
\citet{ludbrook97} states that the grouping structure can be
straightforward, but there are more complex data sets that have a
hierarchical(nested) model.
\\
\\
Observations between groups are independent, but observations
within each groups are dependent because they belong to the same
subpopulation. Therefore there are two sources of variation:
between-group and within-group variance.
\vspace{5 mm} \noindent Mean correction is a method of reducing
bias.







\subsection{A regression based approach based on Bland Altman Analysis}
Lu et al used such a technique in their comparison of DXA
scanners. They also used the Blackwood Bradley test. However it
was shown that, for particular comparisons,  agreement between
methods was indicated according to one test, but lack of agreement
was indicated by the other.



\section{Work List}
\begin{enumerate}
	\item ML v REML
	\item Nested Models and LRTs
	\item Generalized Lease Squares
	\item Diagnostics
	\item Simplifying GLS
	\item Paper progression
\end{enumerate}





\newpage
%--------------------------------------------------------------------Diagnostics%
\section{Diagnostics}

%http://www.artifex.org/~meiercl/R_statistics_guide.pdf
\subsection{Identifying outliers with a LME model object}

The process is slightly different than with standard LME model objects, since the \textbf{\emph{influence}}
function does not work on lme model objects. Given \textbf{\emph{mod.lme}}, we can use the plot function to
identify outliers.
%----------------------%
\subsection{Diagnostics for Random Effects}
Empirical best linear unbiased predictors EBLUPS provide the a useful way of diagnosing random effects.

EBLUPs are also known as ``shrinkage estimators" because they tend to be smaller than the estimated effects would be if they were computed by treating a random factor as if it was fixed (West etal )








\section{Covariance Parameters} %1.5
The unknown variance elements are referred to as the covariance parameters and collected in the vector $\theta$.
% - where is this coming from?
% - where is it used again?
% - Has this got anything to do with CovTrace etc?
%---------------------------------------------------------------------------%




\section{Carstensen Model (mir model)}

A measurement $y_{mi}$ by method $m$ on individual $i$ is formulated as follows;
\begin{equation}
y_{mi}  = \alpha_{m} + \mu_{i} + e_{mi} \qquad  e_{mi} \sim
\mathcal{N}(0,\sigma^{2}_{m})
\end{equation}

The differences are expressed as $d_{i} = y_{1i} - y_{2i}$. For the replicate case, an interaction term $c$ is added to the model, with an associated variance component. All the random effects are assumed independent, and that all replicate measurements are assumed to be exchangeable within each method.

%----

The following model (in the authors own notation) is
formulated as follows, where $y_{mir}$ is the $r$th replicate measurement on subject $i$ with method $m$.


Using Carstensen's notation, a measurement $y_{mi}$ by method $m$ on individual $i$ the measurement $y_{mir} $ is the $r$th replicate measurement on the $i$th item by the $m$th method, where $m=1,2,$ $i=1,\ldots,N,$ and $r = 1,\ldots,n_i$ is formulated as follows;

\begin{equation}
y_{mir}  = \alpha_{m} + \mu_{i} + c_{mi} + \epsilon_{mir}, \qquad  e_{mi}
\sim \mathcal{N}(0,\sigma^{2}_{m}), \quad c_{mi} \sim \mathcal{N}(0,\tau^{2}_{m}).
\end{equation}

Let $y_{mir} $ be the $r$th replicate measurement on the $i$th item by the $m$th method, where $m=1,2,$ $i=1,\ldots,N,$ and $r = 1,\ldots,n_i.$ When the design is balanced and there is no ambiguity we can set $n_i=n.$ The LME model can be written
\begin{equation}
y_{mir} = \beta_{0} + \beta_{m} + b_{mi} + \epsilon_{mir}.
\end{equation}
Here $\beta_0$ and $\beta_m$ are fixed-effect terms representing, respectively, a model intercept and an overall effect for method $m.$ The model can be reparameterized by gathering the $\beta$ terms together into (fixed effect) intercept terms $\alpha_m=\beta_0+\beta_m.$ The $b_{1i}$ and $b_{2i}$ terms are correlated random effect parameters having $\mathrm{E}(b_{mi})=0$ with $\mathrm{Var}(b_{mi})=d^2_m$ and $\mathrm{Cov}(b_{1i}, b_{2 i})=d_{12}.$ 

%Here $\beta_0$ and $\beta_m$ are fixed-effect terms representing, respectively, a model intercept and an overall effect for method $m.$ The $b_{1i}$ and $b_{2i}$ terms represent random effect parameters corresponding to the two methods, having $\mathrm{E}(b_{mi})=0$ with $\mathrm{Var}(b_{mi})=d^2_m$ and $\mathrm{Cov}(b_{mi}, b_{m^\prime i})=g_{12}.$ 

%The random error term for each response is denoted $\epsilon_{mir}$ having $\mathrm{E}(\epsilon_{mir})=0$, $\mathrm{Var}(\epsilon_{mir})=\sigma^2_m$, $\mathrm{Cov}(\epsilon_{1ir}, \epsilon_{2 ir})=\sigma_{12}$, $\mathrm{Cov}(\epsilon_{mir}, \epsilon_{mir^\prime})= 0$ and $\mathrm{Cov}(\epsilon_{1ir}, \epsilon_{2 ir^\prime})= 0.$ 
The random error term for each response is denoted $\epsilon_{mir}$ having $\mathrm{E}(\epsilon_{mir})=0$, $\mathrm{Var}(\epsilon_{mir})=\sigma^2_m$, $\mathrm{Cov}(b_{mir}, b_{m^\prime ir})=\sigma_{12}$, $\mathrm{Cov}(\epsilon_{mir}, \epsilon_{mir^\prime})= 0$ and $\mathrm{Cov}(\epsilon_{mir}, \epsilon_{m^\prime ir^\prime})= 0.$
When two methods of measurement are in agreement, there is no significant differences between $\beta_1$ and $\beta_2,$ $d^2_1 $ and$ d^2_2$, and $\sigma^2_1 $ and$ \sigma^2_2$.
\bigskip

% Complete paragraph by specifying variances and covariances for epsilons.
% I thing that these are your sigmas?
% Also, state equality of the parameters in this model when each of the three hypotheses above are true.


Additionally these parameter are assumed to have Gaussian distribution. Two methods of measurement are in complete agreement if the null hypotheses $\mathrm{H}_1\colon \alpha_1 = \alpha_2$ and $\mathrm{H}_2\colon \sigma^2_1 = \sigma^2_2 $ and $\mathrm{H}_3\colon d^2_1= d^2_2$ hold simultaneously. \citet{ARoy2009} uses a Bonferroni correction to control the familywise error rate for tests of $\{\mathrm{H}_1, \mathrm{H}_2, \mathrm{H}_3\}$ and account for difficulties arising due to multiple testing. Additionally, Roy combines $\mathrm{H}_2$ and $\mathrm{H}_3$ into a single testable hypothesis $\mathrm{H}_4\colon \omega^2_1=\omega^2_2,$ where $\omega^2_m = \sigma^2_m + d^2_m$ represent the overall variability of method $m.$
%Disagreement in overall variability may be caused by different between-item variabilities, by different within-item variabilities, or by both.



Here the terms $\alpha_{m}$ and $\mu_{i}$ represent the fixed effect for method $m$ and a true value for item $i$ respectively. The random effect terms comprise an interaction term $c_{mi}$ and the residuals $\varepsilon_{mir}$.
The $c_{mi}$ term represent random effect parameters corresponding to the two methods, having $\mathrm{E}(c_{mi})= 0$ with $\mathrm{Var}(c_{mi})=\tau^2_m$.  

Carstensen specifies the variance of the interaction terms as being univariate normally distributed. As such, $\mathrm{Cov}(c_{mi}, c_{m^\prime i})= 0.$ All the random effects are assumed independent, and that all replicate measurements are assumed to be exchangeable within each method.




%---Key difference 1---The True Value
%---Colollary -- Difference in model types
The presence of the true value term $\mu_i$ gives rise to an important difference between Carstensen's and Roy's models. Of particular importance is terms of the model, a true value for item $i$ ($\mu_{i}$).  The fixed effect of Roy's model comprise of an intercept term and fixed effect terms for both methods, with no reference to the true value of any individual item. In other words, Roy considers the group of items being measured as a sample taken from a population. Therefore a distinction can be made between the two models: Roy's model is a standard LME model, whereas Carstensen's model is a more complex additive model.


%======================================================================================= %







%\emph{The formulation of this model is general and refers to comparison
%	of any number of methods — however, if only two methods are
%	compared, separate values of $\tau^2_1$ and $\tau^2_2$ cannot be
%	estimated, only their average value $\tau$, so in the case of only
%	two methods we are forced to assume that $\tau_1 = \tau_2 = \tau$} \citep{BXC2008}.





\section{Carstensen's Mixed Models}

\citet{BXC2004} proposes linear mixed effects models for deriving
conversion calculations similar to Deming's regression, and for
estimating variance components for measurements by different
methods. The following model ( in the authors own notation) is
formulated as follows, where $y_{mir}$ is the $r$th replicate
measurement on subject $i$ with method $m$.

\begin{equation}
y_{mir}  = \alpha_{m} + \beta_{m}\mu_{i} + c_{mi} + e_{mir} \qquad
( e_{mi} \sim N(0,\sigma^{2}_{m}), c_{mi} \sim N(0,\tau^{2}_{m}))
\end{equation}
The intercept term $\alpha$ and the $\beta_{m}\mu_{i}$ term follow
from \citet{DunnSEME}, expressing constant and proportional bias
respectively , in the presence of a real value $\mu_{i}.$
$c_{mi}$ is a interaction term to account for replicate, and
$e_{mir}$ is the residual associated with each observation.
Since variances are specific to each method, this model can be
fitted separately for each method.

The above formulation doesn't require the data set to be balanced.
However, it does require a sufficient large number of replicates
and measurements to overcome the problem of identifiability. The
import of which is that more than two methods of measurement may
be required to carry out the analysis. There is also the
assumptions that observations of measurements by particular
methods are exchangeable within subjects. (Exchangeability means
that future samples from a population behaves like earlier
samples).

%\citet{BXC2004} describes the above model as a `functional model',
%similar to models described by \citet{Kimura}, but without any
%assumptions on variance ratios. A functional model is . An
%alternative to functional models is structural modelling

\citet{BXC2004} uses the above formula to predict observations for
a specific individual $i$ by method $m$;

\begin{equation}BLUP_{mir} = \hat{\alpha_{m}} + \hat{\beta_{m}}\mu_{i} +
c_{mi} \end{equation}. Under the assumption that the $\mu$s are
the true item values, this would be sufficient to estimate
parameters. When that assumption doesn't hold, regression
techniques (known as updating techniques) can be used additionally
to determine the estimates. The assumption of exchangeability can
be unrealistic in certain situations. \citet{BXC2004} provides an
amended formulation which includes an extra interaction term ($
d_{mr} \sim N(0,\omega^{2}_{m}$)to account for this.


\newpage
\citet{BXC2008} sets out a methodology of computing the limits of
agreement based upon variance component estimates derived using
linear mixed effects models. Measures of repeatability, a
characteristic of individual methods of measurements, are also
derived using this method.

\subsection{Using LME models to create Prediction Intervals}
\citet{BXC2004} also advocates the use of linear mixed models in
the study of method comparisons. The model is constructed to
describe the relationship between a value of measurement and its
real value. The non-replicate case is considered first, as it is
the context of the Bland-Altman plots. This model assumes that
inter-method bias is the only difference between the two methods.
A measurement $y_{mi}$ by method $m$ on individual $i$ is
formulated as follows;
\begin{equation}
y_{mi}  = \alpha_{m} + \mu_{i} + e_{mi} \qquad ( e_{mi} \sim
N(0,\sigma^{2}_{m}))
\end{equation}
The differences are expressed as $d_{i} = y_{1i} - y_{2i}$ For the
replicate case, an interaction term $c$ is added to the model,
with an associated variance component. All the random effects are
assumed independent, and that all replicate measurements are
assumed to be exchangeable within each method.

\begin{equation}
y_{mir}  = \alpha_{m} + \mu_{i} + c_{mi} + e_{mir} \qquad ( e_{mi}
\sim N(0,\sigma^{2}_{m}), c_{mi} \sim N(0,\tau^{2}_{m}))
\end{equation}





%-------------------------------------------------------------------------------------------------------%

\chapter{Model Diagnostics}
%---------------------------------------------------------------------------%
%1.1 Introduction to Influence Analysis
%1.2 Extension of techniques to LME Models
%1.3 Residual Diagnostics
%1.4 Standardized and studentized residuals
%1.5 Covariance Parameters
%1.6 Case Deletion Diagnostics
%1.7 Influence Analysis
%1.8 Terminology for Case Deletion
%1.9 Cook's Distance (Classical Case)
%1.10 Cook's Distance (LME Case)
%1.11 Likelihood Distance
%1.12 Other Measures
%1.13 CPJ Paper
%1.14 Matrix Notation of Case Deletion
%1.15 CPJ's Three Propositions
%1.16 Other measures of Influence
\tableofcontents
%===========================================================================%
\newpage

\subsection*{Abstract}
This chapter is broken into two parts. The first part is a review of diagnostics methods for linear models, intended to acquaint the reader with the subject, and also to provide a basis for material covered in the second part. Particular attention is drawn to graphical methods.

The second part of the chapter looks at diagnostics techniques for LME models, firsly covering the theory, then proceeding to a discussion on 
implementing these using \texttt{R} code.

While a substantial body of work has been developed in this area, there is still areas worth exploring. 
In particular the development of graphical techniques pertinent to LME models should be looked at.




%\section{Introduction (Page 1)}
%
%Linear models for uncorrelated data have well established measures to gauge the influence of one or more
%observations on the analysis. For such models, closed-form update expressions allow efficient computations
%without refitting the model. 
%
%
%When similar notions of statistical influence are applied to mixed models,
%things are more complicated. Removing data points affects fixed effects and covariance parameter estimates.
%Update formulas for “\textit{leave-one-out}” estimates typically fail to account for changes in covariance
%parameters. 
%
%Moreover, in repeated measures or longitudinal studies, one is often interested in multivariate
%influence, rather than the impact of isolated points. 

% This paper examines extensions of influence measures
% in linear mixed models and their implementation in the MIXED procedure.









\newpage
%=========================================================================%

\section*{Cook's distance}
In the study of Linear model diagnostics, Cook proposed a measure that combines the information of leverage and residual of the observation, now known simply as the Cook's Distance. \citet{CPJ} would later adapt the Cook's distance measure for the analysis of LME models.



%---------------------------------------------------------------------------%



\section{Matrix Notation for Case Deletion} %1.14

\subsection{Case deletion notation} %1.14.1

For notational simplicity, $\boldsymbol{A}(i)$ denotes an $n \times m$ matrix $\boldsymbol{A}$ with the $i$-th row
removed, $a_i$ denotes the $i$-th row of $\boldsymbol{A}$, and $a_{ij}$ denotes the $(i, j)-$th element of $\boldsymbol{A}$.
%
%\subsection{Partitioning Matrices} %1.14.2
%Without loss of generality, matrices can be partitioned as if the $i-$th omitted observation is the first row; i.e. $i=1$.



\subsection{Further Assumptions of Linear Models}

As with fitted models, the assumption of normality of residuals and homogeneity of variance is applicable to LMEs also. 

%--------------------------------------%


Homoscedascity is the technical term to describe the variance of the
residuals being constant across the range of predicted values.
Heteroscedascity is the converse scenario : the variance differs along
the range of values.

%--Marginal and Conditional Residuals

%\subsection{INFLUENCE DIAGNOSTICS IN THE MIXED PROCEDURE}
%Key to the implementations of influence diagnostics in the MIXED procedure is the attempt to quantify
%influence, where possible, by drawing on the basic definitions of the various statistics in the classical linear
%model. 

On occasion, quantification is not possible. Assume, for example, that a data point is removed
and the new estimate of the G matrix is not positive definite. This may occur if a variance component
estimate now falls on the boundary of the parameter space. Thus, it may not be possible to compute certain
influence statistics comparing the full-data and reduced-data parameter estimates. However, knowing that
a new singularity was encountered is important qualitative information about the data point’s influence on
the analysis.

The basic procedure for quantifying influence is simple:

\begin{enumerate}
	\item Fit the model to the data and obtain estimates of all parameters.
	\item Remove one or more data points from the analysis and compute updated estimates of model parameters.
	\item Based on full- and reduced-data estimates, contrast quantities of interest to determine how the absence
	of the observations changes the analysis.
\end{enumerate}
We use the subscript (U) to denote quantities obtained without the observations in the set U. For example,
%βb
(U) denotes the fixed-effects “\textit{\textbf{leave-U-out}}” estimates. Note that the set U can contain multiple observations.


%===================================================================================
If the global measure suggests that the points in U are influential, you should next determine the nature of
that influence. In particular, the points can affect
\begin{itemize}
	\item the estimates of fixed effects
	\item the estimates of the precision of the fixed effects
	\item the estimates of the covariance parameters
	\item the estimates of the precision of the covariance parameters
	\item fitted and predicted values
\end{itemize}

It is important to further decompose the initial finding to determine whether data points are actually troublesome.
Simply because they are influential “somehow”, should not trigger their removal from the analysis or
a change in the model. For example, if points primarily affect the precision of the covariance parameters
without exerting much influence on the fixed effects, then their presence in the data may not distort hypothesis
tests or confidence intervals about $\beta$.
%They will only do so if your inference depends on an estimate of the
%precision of the covariance parameter estimates, as is the case for the Satterthwaite and Kenward-Roger
%degrees of freedom methods and the standard error adjustment associated with the DDFM=KR option.

%------------------------------------------------------------%
\subsection{Summary of Paper}
%Summary of Schabenberger
Standard residual and influence diagnostics for linear models can be extended to LME models.
The dependence of the fixed effects solutions on the covariance parameters has important ramifications on the perturbation analysis.	
Calculating the studentized residuals-And influence statistics whereas each software procedure can calculate both conditional and marginal raw residuals, only SAs Proc Mixed is currently the only program that provide studentized residuals Which ave preferred for model diagnostics. The conditional Raw residuals ave not well suited to detecting outliers as are the studentized conditional residuals. (schabenbege r)


LME are flexible tools for the analysis of clustered and repeated measurement data. LME extend the capabilities of standard linear models by allowing unbalanced and missing data, as long as the missing data are MAR. Structured covariance matrices for both the random effects G and the residuals R. missing at Random.

A conditional residual is the difference between the observed valve and the predicted valve of a dependent variable- Influence diagnostics are formal techniques that allow the identification observation that heavily influence estimates of parameters.
To alleviate the problems with the interpretation of conditional residuals that may have unequal variances, we consider sealing.
Residuals obtained in this manner ave called studentized residuals.



%---------------------------------------------------------------%
\section{Schabenberger: Summary and Conclusions}
\begin{itemize}
	\item Standard residual and influence diagnostics for linear models can be extended to linear mixed models. The dependence of fixed-effects solutions on the covariance parameter estimates has important ramifications in perturbation analysis. 
	\item To gauge the full impact of a set of observations on the analysis, covariance parameters need to be updated, which requires refitting of the model. 
	\item The experimental INFLUENCE option of the MODEL statement in the MIXED procedure (SAS 9.1) enables you to perform iterative and noniterative influence analysis for individual observations and sets of observations.
	
	\item The conditional (subject-specific) and marginal (population-averaged) formulations in the linear mixed model enable you to consider conditional residuals that use the estimated BLUPs of the random effects, and marginal residuals which are deviations from the overall mean. 
	\item Residuals using the BLUPs are useful to diagnose whether the random effects components in the model are specified correctly, marginal residuals are useful to diagnose the fixed-effects components. 
	\item Both types of residuals are available in SAS 9.1 as an experimental option of the MODEL statement in the MIXED procedure.
	
	\item It is important to note that influence analyses are performed under the assumption that the chosen model is correct. Changing the model structure can alter the conclusions. Many other variance models have been fit to the data presented in the repeated measures example. You need to see the conclusions about which model component is affected in light of the model being fit.
	%	\item  For example, modeling these data with a random intercept and random slope for each child or an unstructured covariance matrix will affect your conclusions about which children are influential on the analysis and how this influence manifests itself.
\end{itemize}











% Diagnostics with nlmeU

\section*{Leave-One-Out Diagnostics with \texttt{lmeU}}
Galecki et al provide a brief the matter of LME influence diagnostics in their book.

The command \texttt{lmeU} fits a model with a particular subject removed. The identifier of the subject to be removed is passed as the only argument

A plot ofthe per-observation diagnostics individual subject log-likelihood contributions can be rendered.

\subsubsection*{The addition of an extra factor}


%=========================================================================%


Interaction terms are featured in ANOVA designs.

%=========================================================================%
My search just now found no mention of Cook's distance or influence measures.  

The closest I found was an unanswered question on this from 
April 2003 (http://finzi.psych.upenn.edu/R/Rhelp02a/archive/4797.html).

Beyond that, there is an excellent discussion of "Examining a Fitted Model" in Sec. 4.3 (pp. 174-197) of Pinheiro and Bates (2000) 
Mixed-Effects Models in S and S-Plus (Springer).  

Pinheiro and Bates decided NOT to include plots of Cook's distance among the many diagnostics they did provide.  
However, 'plot(fit.lme)' plots 'standardized residuals' vs. predicted or 'fitted values'.  
Wouldn't points with large influence stand apart from the crowd in terms of 'fitted value'?

Of course, there are many things other one could do to get at related information, including reading the code for 'influence' and 'lme', and 
figure out from that how to write an 'influence' method for an 'lme' object. 








\section{Outline of Thesis}
In the first chapter the study of method comparison is introduced, while the second chapter provides a review of current methodologies. The intention of this thesis is to progress the
study of method comparison studies, using a statistical method known as Linear mixed effects models.
Chapter three shall describes linear mixed effects models, and how the use of the linear mixed
effects models have so far extended to method comparison studies. Implementations of important existing work shall be presented, using the \texttt{R} programming language.

Model diagnostics are an integral component of a complete statistical analysis.
In chapter three model diagnostics shall be described in depth, with particular
emphasis on linear mixed effects models, further to chapter two.

For the fourth chapter, important linear mixed effects model diagnostic methods shall be extended to method comparison studies, and proposed methods shall be demonstrated on data sets that have become well known in literature on method comparison. The purpose is to both calibrate these methods and to demonstrate applications for them.
The last chapter shall focus on robust measures of important parameters such as agreement.

%======================================================================%



\section{Covariance Parameters} %1.18
The unknown variance elements are referred to as the covariance parameters and collected in the vector $\theta$.




\chapter{Roy2013}

http://business.utsa.edu/wps/MSS/0017MSS-253-2013.pdf


Testing the Equality of Mean Vectors for Paired Doubly Multivariate Observations 


Example 2. (Mineral Data): This data set is taken from Johnson and Wichern (2007, p. 43).
An investigator measured the mineral content of bones (radius, humerus and ulna) by photon
absorptiometry to examine whether dietary supplements would slow bone loss in 25 older women.
Measurements were recorded for three bones on the dominant and nondominant sides. Thus,
the data is doubly multivariate and clearly u = 2 and q = 3.
The bone mineral contents for the first 24 women one year after their participation in an
experimental program is given in Johnson and Wichern (2007, p. 353). 



Thus, for our analysis
we take only first 24 women in the first data set. We test whether there has been a bone loss
considering the data as doubly multivariate and has BCS structure. We rearrange the variables
in the data set by grouping together the mineral content of the dominant sides of radius, humerus
and ulna as the first three variables, that is, the variables in the first location (u = 1) and then
the mineral contents for the non-dominant side of the same bones (u = 2)



\section{Outlier Testing} 
A new outlier identification test for method comparison studies based on robust regression.

The identification of outliers in method comparison studies (MCS) is an important part of data analysis, as outliers can indicate serious errors in the measurement process. Common outlier tests proposed in the literature usually require a homogeneous sample distribution and homoscedastic random error variances. However, datasets in MCS usually do not meet these assumptions. In this work, a new outlier test based on robust linear regression is proposed to overcome these special problems. The LORELIA (local reliability) residual test is based on a local, robust residual variance estimator, given as a weighted sum of the observed residuals. The new test is compared to a standard test proposed in the literature by a Monte Carlo simulation. Its performance is illustrated in examples.

\section{Lorelia}


Method comparison studies are performed in order to prove equivalence between two measurement methods or instruments. The identification of outliers is an important part of data analysis as outliers can indicate serious errors in the measurement process. Common outlier tests 
proposed in the literature require a homogeneous sample distribution and homoscedastic random error variances. However, datasets in method comparison studies usually do not meet these assumptions. To overcome this problem, different data transformation methods are proposed in the literature. However, they will only be applicable if the random errors can be described by simple additive or multiplicative models. In this work, a new outlier test based on robust linear regression is proposed which provides a general solution to the above problem. The LORELIA (LOcal RELIAbility) residual test is based on a local, robust residual variance estimator, given as a weighted sum of the observed residuals. Outlier limits are estimated from the actual data situation without making assumptions on the underlying error variance model. The performance of the new test is demonstrated in examples and simulations.

\section{Note on Roy's paper}
\begin{enumerate}
	
	
	\item Basic model:
	\begin{center}
		$ \boldsymbol{y_{i}} = \boldsymbol{X_{i}\beta}
		+ \boldsymbol{Z_{i}b_{i}} + \boldsymbol{\epsilon_{i}}, \qquad i=1,\dots,n$ \\
		$\boldsymbol{Z_{i}} \sim \mathcal{N}(\boldsymbol{0,\Sigma}),\quad
		\boldsymbol{\epsilon_{i}} \sim \mathcal{N}(\boldsymbol{0, \sigma^2
			\boldsymbol{I} })$
	\end{center}
	
	Assumptions are made about homoskedasticity.
	
	\item General model:
	\begin{center}
		$ \boldsymbol{y_{i}} = \boldsymbol{X_{i}\beta}
		+ \boldsymbol{Z_{i}b_{i}} + \boldsymbol{\epsilon_{i}}, \qquad i=1,\dots,n$ \\
		$\boldsymbol{Z_{i}} \sim \mathcal{N}(\boldsymbol{0,\Psi}),\quad
		\boldsymbol{\epsilon_{i}} \sim \mathcal{N}(\boldsymbol{0,\sigma^2 \boldsymbol{\Lambda} })$
	\end{center}
	
	Assumptions about homoskedasticity are relaxed \cite[pg.202]{pb}.
	
	
	
	
	
	\item $\sigma^2 \boldsymbol{\Lambda}$ is the general form for the VC structure for residuals.
	
	\item The response vector $\boldsymbol{y}_{i}$ comprises the observations of
	the subject, as measured by two methods, taking three measurements each.
	Hence a $6 \times 1$ random vector corresponding to the $i$th subject.
	\begin{equation}
	\boldsymbol{y}_{i} = (y_{i}^{j1},y_{i}^{Jj2},y_{i}^{j3},y_{i}^{s1},y_{i}^{s2},y_{i}^{s3}) \prime
	\end{equation}
	
	\item The number of replicates is $p$. A subject will have up to
	$2p$ measurements, for the two instrument case, i.e. $Max(n_{i}) = 2p$.
	(Let $k$ denote number of instruments, which is assumed to be $2$
	unless stated otherwise.) For the blood pressure data $p=3$.
	
	
\end{enumerate}




%\section{Introduction (Page 1)}
%
%Linear models for uncorrelated data have well established measures to gauge the influence of one or more
%observations on the analysis. For such models, closed-form update expressions allow efficient computations
%without refitting the model. 
%
%
%When similar notions of statistical influence are applied to mixed models,
%things are more complicated. Removing data points affects fixed effects and covariance parameter estimates.
%Update formulas for “\textit{leave-one-out}” estimates typically fail to account for changes in covariance
%parameters. 
%
%Moreover, in repeated measures or longitudinal studies, one is often interested in multivariate
%influence, rather than the impact of isolated points. 

% This paper examines extensions of influence measures
% in linear mixed models and their implementation in the MIXED procedure.















%========================================================================================================= %
\subsection{Outliers and Leverage}



The question of whether or not a point should be considered an outlier must also be addressed. An outlier is an observation whose true value is unusual given its value on the predictor variables. The leverage of an observation is a further consideration. Leverage describes an observation with an extreme value on a predictor variable is a point with high leverage. High leverage points can have a great amount of effect on the estimate of regression coefficients.
% - Leverage is a measure of how far an independent variable deviates from its mean.

Influence can be thought of as the product of leverage and outlierness. An observation is said to be influential if removing the observation substantially changes the estimate of the regression coefficients. The \texttt{R} programming language has a variety of methods used to study each of the aspects for a linear model. While linear models and GLMS can be studied with a wide range of well-established diagnostic technqiues, the choice of methodology is much more restricted for the case of LMEs.

%---------------------------------------------------------------------------%
%\newpage
%\section{Residual diagnostics} %1.3
For classical linear models, residual diagnostics are typically conducted using a plot of the observed residuals and the predicted values. A visual inspection for the presence of trends inform the analyst on the validity of distributional assumptions, and to detect outliers and influential observations.

%\section{Case Deletion Diagnostics}
%
%
%Linear models for uncorrelated data have well established measures to gauge the influence of one or more
%observations on the analysis. For such models, closed-form update expressions allow efficient computations
%without refitting the model. 
%
%
%Since the pioneering work of Cook in 1977, deletion measures have been applied to many statistical models for identifying influential observations. Case-deletion diagnostics provide a useful tool for identifying influential observations and outliers.
%
%The key to making deletion diagnostics useable is the development of efficient computational formulas, allowing one to obtain the \index{case deletion diagnostics} case deletion diagnostics by making use of basic building blocks, computed only once for the full model.
%
%The computation of case deletion diagnostics in the classical model is made simple by the fact that estimates of $\beta$ and $\sigma^2$, which exclude the $i-$th observation, can be computed without re-fitting the model. %\subsection{Terminology for Case Deletion diagnostics} %1.8
%
%\citet{preisser} describes two type of diagnostics. When the set consists of only one observation, the type is called
%`\textit{observation-diagnostics}'. For multiple observations, Preisser describes the diagnostics as `\textit{cluster-deletion}' diagnostics. When applied to LME models, such update formulas are available only if one assumes that the covariance parameters are not affected by the removal of the observation in question. However, this is rarely a reasonable assumption.
%
%
%
%
%%---------------------------------------------------------------------------%
\subsection{Matrix Notation for Case Deletion} %1.14

%\subsection{Case deletion notation} %1.14.1

For notational simplicity, $\boldsymbol{A}(i)$ denotes an $n \times m$ matrix $\boldsymbol{A}$ with the $i$-th row
removed, $a_i$ denotes the $i$-th row of $\boldsymbol{A}$, and $a_{ij}$ denotes the $(i, j)-$th element of $\boldsymbol{A}$.
%
%\subsection{Partitioning Matrices} %1.14.2
%Without loss of generality, matrices can be partitioned as if the $i-$th omitted observation is the first row; i.e. $i=1$.



%-------------------------------------------------------------------------------------------------------------------------------------%
%--------------------------------------%
\subsection{Extension of Diagnostic Methods to LME models}


When similar notions of statistical influence are applied to mixed models,
things are more complicated. Removing data points affects fixed effects and covariance parameter estimates.
Update formulas for “\textit{leave-one-out}” estimates typically fail to account for changes in covariance
parameters. 
%
%
%In LME models, there are two types of residuals, marginal residuals and conditional residuals. A
%marginal residual is the difference between the observed data and the estimated marginal mean. A conditional residual is the
%difference between the observed data and the predicted value of the observation. In a model without random effects, both sets of residuals coincide \citep{schab}.

\citet{Christiansen} noted the case deletion diagnostics techniques have not been applied to linear mixed effects models and seeks to develop methodologies in that respect. \citet{Christiansen} develops these techniques in the context of REML.

\citet{CPJ} noted the case deletion diagnostics techniques had not been applied to linear mixed effects models and seeks to develop methodologies in that respect. \citet{CPJ} develops these techniques in the context of REML.
>>>>>>> origin/master

%\citet{CPJ} develops \index{case deletion diagnostics} case deletion diagnostics, in particular the equivalent of \index{Cook's distance} Cook's distance, a well-known metric, for diagnosing influential observations when estimating the fixed effect parameters and variance components. Deletion diagnostics provide a means of assessing the influence of an observation (or groups of observations) on inference on the estimated parameters of LME models. We shall provide a fuller discussion of Cook's distance in due course.


\citet{Demi} extends several regression diagnostic techniques commonly used in linear regression, such as leverage, infinitesimal influence, case deletion diagnostics, Cook's distance, and local influence to the linear mixed-effects model. In each case, the proposed new measure has a direct interpretation in terms of the effects on a parameter of interest, and reduces to the familiar linear regression measure when there are no random effects. 

The new measures that are proposed by \citet{Demi} are explicitly defined functions and do not require re-estimation of the model, especially for cluster deletion diagnostics. The basis for both the cluster deletion diagnostics and Cook's distance is a generalization of Miller's simple update formula for case deletion for linear models. Furthermore \citet{Demi} shows how Pregibon's infinitesimal case deletion diagnostics is adapted to the linear mixed-effects model. 
%A simple compact matrix formula is derived to assess the local influence of the fixed-effects regression coefficients. 


%
%
%\section{Case Deletion Diagnostics for LME models} %1.6
%
%Data from single individuals, or a small group of subjects may influence non-linear mixed effects model selection. Diagnostics routinely applied in model building may identify such individuals, but these methods are not specifically designed for that purpose and are, therefore, not optimal. 

\citet{Demi} proposes two likelihood-based diagnostics for identifying individuals that can influence the choice between two competing models.


\newpage





\section{Regression Of Differences On Averages}
Further to Carstensen, we can formulate the two measurements
$y_{1}$ and $y_{2}$ as follows:
\\
$y_{1} = \alpha + \beta\mu + \epsilon_{1}$
\\
$y_{2} = \alpha + \beta\mu + \epsilon_{2}$








\subsection{Note 1: Coefficient of Repeatability}
The coefficient of repeatability is a measure of how well a
measurement method agrees with itself over replicate measurements
\citep{BA99}. Once the within-item variability is known, the
computation of the coefficients of repeatability for both methods
is straightforward.



\subsection{Note 2: Carstensen model in the single measurement case}
\citet{BXC2004} presents a model to describe the relationship between a value of measurement and its real value.
The non-replicate case is considered first, as it is the context of the Bland-Altman plots.
This model assumes that inter-method bias is the only difference between the two methods.


\begin{equation}
y_{mi}  = \alpha_{m} + \mu_{i} + e_{mi} \qquad  e_{mi} \sim \mathcal{N}(0,\sigma^{2}_{m})
\end{equation}

The differences are expressed as $d_{i} = y_{1i} - y_{2i}$.

For the replicate case, an interaction term $c$ is added to the model, with an associated variance component.




\subsection{Note 3: Model terms}
It is important to note the following characteristics of this model.
\begin{itemize}
	\item Let the number of replicate measurements on each item $i$ for both methods be $n_i$, hence $2 \times n_i$ responses. However, it is assumed that there may be a different number of replicates made for different items. Let the maximum number of replicates be $p$. An item will have up to $2p$ measurements, i.e. $\max(n_{i}) = 2p$.
	
	% \item $\boldsymbol{y}_i$ is the $2n_i \times 1$ response vector for measurements on the $i-$th item.
	% \item $\boldsymbol{X}_i$ is the $2n_i \times  3$ model matrix for the fixed effects for observations on item $i$.
	% \item $\boldsymbol{\beta}$ is the $3 \times  1$ vector of fixed-effect coefficients, one for the true value for item $i$, and one effect each for both methods.
	
	\item Later on $\boldsymbol{X}_i$ will be reduced to a $2 \times 1$ matrix, to allow estimation of terms. This is due to a shortage of rank. The fixed effects vector can be modified accordingly.
	\item $\boldsymbol{Z}_i$ is the $2n_i \times  2$ model matrix for the random effects for measurement methods on item $i$.
	\item $\boldsymbol{b}_i$ is the $2 \times  1$ vector of random-effect coefficients on item $i$, one for each method.
	\item $\boldsymbol{\epsilon}$  is the $2n_i \times  1$ vector of residuals for measurements on item $i$.
	\item $\boldsymbol{G}$ is the $2 \times  2$ covariance matrix for the random effects.
	\item $\boldsymbol{R}_i$ is the $2n_i \times  2n_i$ covariance matrix for the residuals on item $i$.
	\item The expected value is given as $\mbox{E}(\boldsymbol{y}_i) = \boldsymbol{X}_i\boldsymbol{\beta}.$ \citep{hamlett}
	\item The variance of the response vector is given by $\mbox{Var}(\boldsymbol{y}_i)  = \boldsymbol{Z}_i \boldsymbol{G} \boldsymbol{Z}_i^{\prime} + \boldsymbol{R}_i$ \citep{hamlett}.
\end{itemize}
\newpage

%\chapter{Limits of Agreement}

%\section{Modelling Agreement with LME Models}

% Carstensen pages 22-23


Roys uses and LME model approach to provide a set of formal tests for method comparison studies.\\

Four candidates models are fitted to the data.\\

These models are similar to one another, but for the imposition of equality constraints.\\

These tests are the pairwise comparison of candidate models, one formulated without constraints, the other with a constraint.\\


Roy's model uses fixed effects $\beta_0 + \beta_1$ and $\beta_0 + \beta_1$ to specify the mean of all observationsby \\ methods 1 and 2 respectuively.





Roy adheres to Random Effect ideas in ANOVA

Roy treats items as a sample from a population.\\

Allocation of fixed effects and random effects are very different in each model\\

Carstensen's interest lies in the difference between the population from which they were drawn.\\

Carstensen's model is a mixed effects ANOVA.\\

\[
Y_{mir}  =  \alpha_m + \mu_i + c_{mi} + e_{mir}, \qquad c_{mi} \sim \mathcal{\tau^2_m}, \qquad e_{mir} \sim \mathcal{\sigma^2_m},
\]

This model includes a method by item iteration term.\\

Carstensen presents two models. One for the case where the replicates, and a second for when they are linked.\\

Carstensen's model does not take into account either between-item or within-item covariance between methods.\\


In the presented example, it is shown that Roy's LoAs are lower than those of Carstensen.
Carstensen makes some interesting remarks in this regard.

\begin{quote}
	The only slightly non-standard (meaning "not often used") feature is the differing residual variances between methods.
\end{quote}
\newpage


\subsection{Note 3: Model terms}
It is important to note the following characteristics of this model.
\begin{itemize}
	\item Let the number of replicate measurements on each item $i$ for both methods be $n_i$, hence $2 \times n_i$ responses. However, it is assumed that there may be a different number of replicates made for different items. Let the maximum number of replicates be $p$. An item will have up to $2p$ measurements, i.e. $\max(n_{i}) = 2p$.
	
	% \item $\boldsymbol{y}_i$ is the $2n_i \times 1$ response vector for measurements on the $i-$th item.
	% \item $\boldsymbol{X}_i$ is the $2n_i \times  3$ model matrix for the fixed effects for observations on item $i$.
	% \item $\boldsymbol{\beta}$ is the $3 \times  1$ vector of fixed-effect coefficients, one for the true value for item $i$, and one effect each for both methods.
	
	\item Later on $\boldsymbol{X}_i$ will be reduced to a $2 \times 1$ matrix, to allow estimation of terms. This is due to a shortage of rank. The fixed effects vector can be modified accordingly.
	\item $\boldsymbol{Z}_i$ is the $2n_i \times  2$ model matrix for the random effects for measurement methods on item $i$.
	\item $\boldsymbol{b}_i$ is the $2 \times  1$ vector of random-effect coefficients on item $i$, one for each method.
	\item $\boldsymbol{\epsilon}$  is the $2n_i \times  1$ vector of residuals for measurements on item $i$.
	\item $\boldsymbol{G}$ is the $2 \times  2$ covariance matrix for the random effects.
	\item $\boldsymbol{R}_i$ is the $2n_i \times  2n_i$ covariance matrix for the residuals on item $i$.
	\item The expected value is given as $\mbox{E}(\boldsymbol{y}_i) = \boldsymbol{X}_i\boldsymbol{\beta}.$ \citep{hamlett}
	\item The variance of the response vector is given by $\mbox{Var}(\boldsymbol{y}_i)  = \boldsymbol{Z}_i \boldsymbol{G} \boldsymbol{Z}_i^{\prime} + \boldsymbol{R}_i$ \citep{hamlett}.
\end{itemize}
\newpage
\newpage
\section{Regression Of Differences On Averages}
Further to Carstensen, we can formulate the two measurements
$y_{1}$ and $y_{2}$ as follows:
\\
$y_{1} = \alpha + \beta\mu + \epsilon_{1}$
\\
$y_{2} = \alpha + \beta\mu + \epsilon_{2}$








%---------------------------------------------------------------------------%
%\newpage
%\section{Residual diagnostics} %1.3
For classical linear models, residual diagnostics are typically conducted using a plot of the observed residuals and the predicted values. A visual inspection for the presence of trends inform the analyst on the validity of distributional assumptions, and to detect outliers and influential observations.

%\section{Case Deletion Diagnostics}
%
%
%Linear models for uncorrelated data have well established measures to gauge the influence of one or more
%observations on the analysis. For such models, closed-form update expressions allow efficient computations
%without refitting the model. 
%
%
%Since the pioneering work of Cook in 1977, deletion measures have been applied to many statistical models for identifying influential observations. Case-deletion diagnostics provide a useful tool for identifying influential observations and outliers.
%
%The key to making deletion diagnostics useable is the development of efficient computational formulas, allowing one to obtain the \index{case deletion diagnostics} case deletion diagnostics by making use of basic building blocks, computed only once for the full model.
%
%The computation of case deletion diagnostics in the classical model is made simple by the fact that estimates of $\beta$ and $\sigma^2$, which exclude the $i-$th observation, can be computed without re-fitting the model. %\subsection{Terminology for Case Deletion diagnostics} %1.8
%
%\citet{preisser} describes two type of diagnostics. When the set consists of only one observation, the type is called
%`\textit{observation-diagnostics}'. For multiple observations, Preisser describes the diagnostics as `\textit{cluster-deletion}' diagnostics. When applied to LME models, such update formulas are available only if one assumes that the covariance parameters are not affected by the removal of the observation in question. However, this is rarely a reasonable assumption.
%
\subsection*{Appendix to Section 4}




As an appendix to section 4, an appraisal of the current state of development (or lack thereof) for current implemenations for LME models, particularly for
\texttt{nlme} and \texttt{lme4} fitted models.

Crucially, a review of internet resources indicates that almost all of the progress in this regard has been done for \texttt{lme4} fitted models, specifically the \textit{Influence.ME} \texttt{R} package. (Nieuwenhuis et 2012)

Conversely there is very little for \texttt{nlme} models. To delve into this mor, one would immediately investigate the current development workflow for both packages.

As an aside, Douglas Bates was arguably the most prominent \texttt{R} developer working in the LME area. 
However Bates has now prioritised the development of LME models in another computing environment , i.e Julia. 


\subsubsection*{Important Consideration for MCS}

The key issue is that \texttt{nlme} allows for the particular specification of Roy's Model, speciifically direct spefiication of the VC matrices for within subject and between subject residuals.
The \texttt{lme4} package does not allow for this.
To advance the ideas that eminate from Roys' paper, one is required to use the \texttt{nlme} context. However, to take advantage of the infrastructure already provided for \texttt{lme4} models, one may change the research question away from that of Roy's paper. 
To this end, an exploration of what textit{influence.ME} can accomplished is merited.
As a complement to this, one can also consider how to properly employ the $R^2$ measure, in the context of Methoc Comparison Studies, further to the work by Edwards et al, namely ``An $R^2$ statistic for fixed effects in the linear mixed model".
%================================================= %

\begin{framed}
	
	\begin{quote}
		\textbf{Abstract for ``An $R^2$ statistic for fixed effects in the linear mixed model"}
		Statisticians most often use the linear mixed model to analyze Gaussian longitudinal data. 
		
		The value and familiarity of the R2 statistic in the linear univariate model naturally creates great interest in extending it to the linear mixed model. We define and describe how to compute a model R2 statistic for the linear mixed model by using only a single model. 
		
		The proposed R2 statistic measures multivariate association between the repeated outcomes and the fixed effects in the linear mixed model. The R2 statistic arises as a 1–1 function of an appropriate F statistic for testing all fixed effects (except typically the intercept) in a full model. 
		
		The statistic compares the full model with a null model with all fixed effects deleted (except typically the intercept) while retaining exactly the same covariance structure. 
		
		Furthermore, the R2 statistic leads immediately to a natural definition of a partial R2 statistic. A mixed model in which ethnicity gives a very small p-value as a longitudinal predictor of blood pressure (BP) compellingly illustrates the value of the statistic. 
		
		In sharp contrast to the extreme p-value, a very small $R^2$ , a measure of statistical and scientific importance, indicates that ethnicity has an almost negligible association with the repeated BP outcomes for the study.
	\end{quote}
\end{framed}

%======================%
% nlme
\subsubsection*{The \texttt{nlme} package}

With regards to \texttt{nlme}, the torch has been passed to Galecki Galecki \& Burzykowski (UMich. and Hasselt respecitely).  Galecki \& Burzykowski published \textit{Linear Mixed Effects Models using \texttt{R}}. 
Also, the accompanying \texttt{R} package, nlmeU package is under current development, with a version being released XXXX.





%======================%
% lme4 and influence.ME
\subsubsection*{The \texttt{lme4} package}

The \texttt{lme4} package is also under active development, under the leadership of Ben Bolker (McMaster University). According to CRAN, the LME4 package, fits linear and generalized linear mixed-effects models

\begin{quote}
	The models and their components are represented using S4 classes and methods. The core computational algorithms are implemented using the Eigen C++ library for numerical linear algebra and RcppEigen "glue".
	(CRAN)
\end{quote}


%=====================%
% Important Consideration for MCS

The key issue is that \texttt{nlme} allows for the particular specification of Roy's Model, speciifically direct spefiication of the VC matrices for within subject and between subject residuals.
The \texttt{lme4} package does not allow for this.
To advance the ideas that eminate from Roys' paper, one is required to use the \texttt{nlme} context. However, to take advantage of the infrastructure already provided for \texttt{lme4} models, one may change the research question away from that of Roy's paper. 
To this end, an exploration of what textit{influence.ME} can accomplished is merited.
As a complement to this, one can also consider how to properly employ the $R^2$ measure, in the context of Methoc Comparison Studies, further to the work by Edwards et al, namely ``An $R^2$ statistic for fixed effects in the linear mixed model".
%================================================= %
\newpage
\begin{framed}
	
	\begin{quote}
		\textbf{Abstract for ``An $R^2$ statistic for fixed effects in the linear mixed model"}
		Statisticians most often use the linear mixed model to analyze Gaussian longitudinal data. 
		
		The value and familiarity of the R2 statistic in the linear univariate model naturally creates great interest in extending it to the linear mixed model. We define and describe how to compute a model R2 statistic for the linear mixed model by using only a single model. 
		
		The proposed R2 statistic measures multivariate association between the repeated outcomes and the fixed effects in the linear mixed model. The R2 statistic arises as a 1–1 function of an appropriate F statistic for testing all fixed effects (except typically the intercept) in a full model. 
		
		The statistic compares the full model with a null model with all fixed effects deleted (except typically the intercept) while retaining exactly the same covariance structure. 
		
		Furthermore, the R2 statistic leads immediately to a natural definition of a partial R2 statistic. A mixed model in which ethnicity gives a very small p-value as a longitudinal predictor of blood pressure (BP) compellingly illustrates the value of the statistic. 
		
		In sharp contrast to the extreme p-value, a very small $R^2$ , a measure of statistical and scientific importance, indicates that ethnicity has an almost negligible association with the repeated BP outcomes for the study.
	\end{quote}
\end{framed}



\begin{equation}
r_{mi}=x^{T}_{i}\hat{\beta}
\end{equation}

\subsection{Marginal Residuals}
\begin{eqnarray}
\hat{\beta} &=& (X^{T}R^{-1}X)^{-1}X^{T}R^{-1}Y \nonumber \\
&=& BY \nonumber
\end{eqnarray}


\section{Covariance Parameters} %1.5
The unknown variance elements are referred to as the covariance parameters and collected in the vector $\theta$.
% - where is this coming from?
% - where is it used again?
% - Has this got anything to do with CovTrace etc?


\subsection{Methods and Measures}
The key to making deletion diagnostics useable is the development of efficient computational formulas, allowing one to obtain the \index{case deletion diagnostics} case deletion diagnostics by making use of basic building blocks, computed only once for the full model.

\citet{Zewotir} lists several established methods of analyzing influence in LME models. These methods include \begin{itemize}
	\item Cook's distance for LME models,
	\item \index{likelihood distance} likelihood distance,
	\item the variance (information) ration,
	\item the \index{Cook-Weisberg statistic} Cook-Weisberg statistic,
	\item the \index{Andrews-Prebigon statistic} Andrews-Prebigon statistic.
\end{itemize}


\section{Missing Data in Method Comparison Studies}

The matter of missing data has not been commonly encountered in either Method Comparison Studies or Linear Mixed Effects Modelling. However Roy (2009) deals with the relevant assumptions regrading missing data.

Galecki \& Burzykowski (2013) tackles the subject of missing data in LME Modelling.

Furthermore the nlmeU package includes the \texttt{patMiss} function, which ``allows to compactly present pattern of missing data in a given vector/matrix/data
frame or combination of thereof".


\section{Leave-One-Out Diagnostics with \texttt{lmeU}}
Galecki et al discuss the matter of LME influence diagnostics in their book, although not into great detail.


The command \texttt{lmeU} fits a model with a particular subject removed. The identifier of the subject to be removed is passed as the only argument

A plot ofthe per-observation diagnostics individual subject log-likelihood contributions can be rendered.


\begin{framed} 
	\begin{itemize}
		\item \texttt{R} command and \texttt{R} object - Typewriter Font
		\item \texttt{R} Package name - Italics
		\item Selected Acronyms and Proper Nouns - Italics
	\end{itemize}
\end{framed}
\medskip

\begin{itemize}	
	\item This chapter is broken into two parts. The first part is a review of diagnostics methods for linear models, intended to acquaint the
	reader with the subject, and also to provide a basis for material covered in the second part. Particular attention is drawn to graphical methods.
	
	\item The second part of the chapter looks at diagnostics techniques for LME models, firsly covering the theory, then proceeding to a discussion on 
	implementing these using \texttt{R} code.
	\item While a substantial body of work has been developed in this area, ther are still area worth exploring. 
	In particular the development of graphical techniques pertinent to LME models should be looked at.
\end{itemize}





\chapter{Model Diagnostics}








\section{Carstensen}
Let $y_{mir} $ be the $r$th replicate measurement on the $i$th item by the $m$th method, where $m=1,2,$ $i=1,\ldots,N,$ and $r = 1,\ldots,n_i.$ When the design is balanced and there is no ambiguity we can set $n_i=n.$ The LME model underpinning Roy's approach can be written
\begin{equation}
y_{mir} = \beta_{0} + \beta_{m} + b_{mi} + \epsilon_{mir}.
\end{equation}
Here $\beta_0$ and $\beta_m$ are fixed-effect terms representing, respectively, a model intercept and an overall effect for method $m.$
The $\beta$ terms can be gathered together into (fixed effect) intercept terms $\alpha_m=\beta_0+\beta_m.$ The $b_{1i}$ and $b_{2i}$ terms are correlated random effect parameters having $\mathrm{E}(b_{mi})=0$ with $\mathrm{Var}(b_{mi})=g^2_m$ and $\mathrm{Cov}(b_{mi}, b_{m^\prime i})=g_{12}.$ The random error term for each response is denoted $\epsilon_{mir}$ having $\mathrm{E}(\epsilon_{mir})=0$, $\mathrm{Var}(\epsilon_{mir})=\sigma^2_m$, $\mathrm{Cov}(b_{mir}, b_{m^\prime ir})=\sigma_{12}$, $\mathrm{Cov}(\epsilon_{mir}, \epsilon_{mir^\prime})= 0$ and $\mathrm{Cov}(\epsilon_{mir}, \epsilon_{m^\prime ir^\prime})= 0.$ Two methods of measurement are in complete agreement if the null hypotheses $\mathrm{H}_1\colon \beta_1 = \beta_2$ and $\mathrm{H}_2\colon \sigma^2_1 = \sigma^2_2 $ and $\mathrm{H}_3\colon g^2_1= g^2_2$ hold simultaneously. \citet{roy} proposes a Bonferroni correction to control the familywise error rate for tests of $\{\mathrm{H}_1, \mathrm{H}_2, \mathrm{H}_3\}$ and account for difficulties arising due to multiple testing. Let $\omega^2_m = \sigma^2_m + g^2_m$ represent the overall variability of method $m.$  Roy also integrates $\mathrm{H}_2$ and $\mathrm{H}_3$ into a single testable hypothesis $\mathrm{H}_4\colon \omega^2_1=\omega^2_2.$ CONCERNS?

\bigskip

% Complete paragraph by specifying variances and covariances for epsilons.
% I thing that these are your sigmas?
% Also, state equality of the parameters in this model when each of the three hypotheses above are true.
\citet{Roy} demonstrates how to implement a method comparison study further to model (1) using the SAS proc mixed package.
%------------------------------------------------------------------------------------------------%
\citet{BXC2008} demonstrates how to construct limits of agreement using SAS, STATA and R. In the case of SAS, the PROC MIXED procedure is used.
Implementation in R is performed using the nlme package \citep{pb2000}.

\citet{BXC2008} remarks that the implementation using R is quite ``arcane".

As R is freely available, this paper demonstrates an implementation of Roy's model using R.

The R statistical software package is freely available.

%------------------------------------------------------------------------------------------------%
The LME model is very easy to implement using PROC MIXED of SAS and the results are also easy to interpret.
The SAS proc mixed procedure has very simple syntax.

As the required code to fit the models is complex, R code necessary to fit the models is provided. 

A demonstration is provided on how to use the output to perform the tests, and to compute limits of agreement.



We assume the data are formatted as a dataset with four columns named:

meth, method of measurement, the number of methods being M,
item, items (persons, samples) measured by each method, of which there are I,
repl, replicate indicating repeated measurement of the same item by the same method, and
y, the measurement.







%=======================================================================================%

\subsection{Remarks on the Multivariate Normal Distribution}

Diligence is required when considering the models. Carstensen specifies his models in terms of the univariate normal distribution. Roy's model is specified using the bivariate normal distribution.
This gives rises to a key difference between the two model, in that a bivariate model accounts for covariance between the variables of interest.
The multivariate normal distribution of a $k$-dimensional random vector $X = [X_1, X_2, \ldots, X_k]$
can be written in the following notation:
\[
X\ \sim\ \mathcal{N}(\mu,\, \Sigma),
\]
or to make it explicitly known that $X$ is $k$-dimensional,
\[
X\ \sim\ \mathcal{N}_k(\mu,\, \Sigma).
\]
with $k$-dimensional mean vector
\[ \mu = [ \operatorname{E}[X_1], \operatorname{E}[X_2], \ldots, \operatorname{E}[X_k]] \]
and $k \times k$ covariance matrix
\[ \Sigma = [\operatorname{Cov}[X_i, X_j]], \; i=1,2,\ldots,k; \; j=1,2,\ldots,k \]

\bigskip

\begin{enumerate}
	\item Univariate Normal Distribution
	
	\[
	X\ \sim\ \mathcal{N}(\mu,\, \sigma^2),
	\]
	
	\item Bivariate Normal Distribution
	
	\begin{itemize}
		\item[(a)] \[  X\ \sim\ \mathcal{N}_2(\mu,\, \Sigma), \vspace{1cm}\]
		\item[(b)] \[    \mu = \begin{pmatrix} \mu_x \\ \mu_y \end{pmatrix}, \quad
		\Sigma = \begin{pmatrix} \sigma_x^2 & \rho \sigma_x \sigma_y \\
		\rho \sigma_x \sigma_y  & \sigma_y^2 \end{pmatrix}.\]
	\end{itemize}
\end{enumerate}


%\chapter{Limits of Agreement}

\section{Modelling Agreement with LME Models}

% Carstensen pages 22-23


Roys uses and LME model approach to provide a set of formal tests for method comparison studies.\\

Four candidates models are fitted to the data.\\

These models are similar to one another, but for the imposition of equality constraints.\\

These tests are the pairwise comparison of candidate models, one formulated without constraints, the other with a constraint.\\


Roy's model uses fixed effects $\beta_0 + \beta_1$ and $\beta_0 + \beta_1$ to specify the mean of all observationsby \\ methods 1 and 2 respectuively.





Roy adheres to Random Effect ideas in ANOVA

Roy treats items as a sample from a population.\\

Allocation of fixed effects and random effects are very different in each model\\

Carstensen's interest lies in the difference between the population from which they were drawn.\\

Carstensen's model is a mixed effects ANOVA.\\

\[
Y_{mir}  =  \alpha_m + \mu_i + c_{mi} + e_{mir}, \qquad c_{mi} \sim \mathcal{\tau^2_m}, \qquad e_{mir} \sim \mathcal{\sigma^2_m},
\]

This model includes a method by item iteration term.\\

Carstensen presents two models. One for the case where the replicates, and a second for when they are linked.\\

Carstensen's model does not take into account either between-item or within-item covariance between methods.\\


In the presented example, it is shown that Roy's LoAs are lower than those of Carstensen.
Carstensen makes some interesting remarks in this regard.

\begin{quote}
	The only slightly non-standard (meaning "not often used") feature is the differing residual variances between methods.
\end{quote}





It is also desirable to measure the influence of the case deletions on the covariance matrix of $\hat{\beta}$.



%===================================================================%

\begin{itemize}
	\item \textit{
		The previous Section (Section 4) is a literary review of residual diagnostics and influence procedures
		for Linear Mixed Effects Models, drawing heavily on Schabenberger and Zewotir.}
	
	\item \textit{	Section 4 begins with an introduction to key topics in residual diagnostics, such as influence, leverage, outliers
		and Cook's distance. Other concepts such as DFFITS and DFBETAs will be introduced briefly, mostly to explain why the are not particularly useful for
		the Method Comparison context, and therefore are not elaborated upon.}
	
	\item \textit{	In brief, Variable Selection is not applicable to Method Comparison Studies, in the 
		commonly used used context. 
		Testing a rather simplisticy specificied model against one with more random effects terms is tractable, but this research question is of secondary importance.}
\end{itemize}



\subsection{Matrix Notation for Case Deletion} %1.14

%\subsection{Case deletion notation} %1.14.1

For notational simplicity, $\boldsymbol{A}(i)$ denotes an $n \times m$ matrix $\boldsymbol{A}$ with the $i$-th row
removed, $a_i$ denotes the $i$-th row of $\boldsymbol{A}$, and $a_{ij}$ denotes the $(i, j)-$th element of $\boldsymbol{A}$.
%
%\subsection{Partitioning Matrices} %1.14.2
%Without loss of generality, matrices can be partitioned as if the $i-$th omitted observation is the first row; i.e. $i=1$.
















	\section{Remarks on the Multivariate Normal Distribution}
	
	Diligence is required when considering the models. Carstensen specifies his models in terms of the univariate normal distribution. ARoy2009's model is specified using the bivariate normal distribution.
	This gives rises to a key difference between the two model, in that a bivariate model accounts for covariance between the variables of interest.
	The multivariate normal distribution of a $k$-dimensional random vector $X = [X_1, X_2, \ldots, X_k]$
	can be written in the following notation:
	\[
	X\ \sim\ \mathcal{N}(\mu,\, \Sigma),
	\]
	or to make it explicitly known that $X$ is $k$-dimensional,
	\[
	X\ \sim\ \mathcal{N}_k(\mu,\, \Sigma).
	\]
	with $k$-dimensional mean vector
	\[ \mu = [ \operatorname{E}[X_1], \operatorname{E}[X_2], \ldots, \operatorname{E}[X_k]] \]
	and $k \times k$ covariance matrix
	\[ \Sigma = [\operatorname{Cov}[X_i, X_j]], \; i=1,2,\ldots,k; \; j=1,2,\ldots,k \]
	
	\bigskip
	
	\begin{enumerate}
		\item Univariate Normal Distribution
		
		\[
		X\ \sim\ \mathcal{N}(\mu,\, \sigma^2),
		\]
		
		\item Bivariate Normal Distribution
		
		\begin{itemize}
			\item[(a)] \[  X\ \sim\ \mathcal{N}_2(\mu,\, \Sigma), \vspace{1cm}\]
			\item[(b)] \[    \mu = \begin{pmatrix} \mu_x \\ \mu_y \end{pmatrix}, \quad
			\Sigma = \begin{pmatrix} \sigma_x^2 & \rho \sigma_x \sigma_y \\
			\rho \sigma_x \sigma_y  & \sigma_y^2 \end{pmatrix}.\]
		\end{itemize}
	\end{enumerate}
	
	
	



\addcontentsline{toc}{section}{Bibliography}

%--------------------------------------------------------------------------------------%

\bibliographystyle{chicago}
\bibliography{DB-txfrbib}


\end{document}


