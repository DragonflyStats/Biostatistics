\documentclass[Main.tex]{subfiles}
\begin{document}

\chapter{Residual Analysis}
%=========================================%
\section{Introduction to Residual Analysis}
Residual analysis is a widely used model validation technique. A residual is simply the difference between an observed value and the corresponding fitted value, as predicted by the model. The rationale is that, if the model is properly fitted to the model, then the residuals would approximate the random errors that one should expect.
that is to say, if the residuals behave randomly, with no discernible trend, the model has fitted the data well. If some sort of non-random trend is evident in the model, then the model can be considered to be poorly fitted.

	\newpage
	\section{Residual diagnostics} %1.3
	For classical linear models, residual diagnostics are typically implemented as a plot of the observed residuals and the predicted values. A visual inspection for the presence of trends inform the analyst on the validity of distributional assumptions, and to detect outliers and influential observations.
	
	
	
	%--Marginal and Conditional Residuals
	
%========================================================================================================= %
%\subsection{Introduction}
%A statistical model, whether of the fixed-effects or mixed-effects variety, represents how you think your data were generated. 
%Following model specification and estimation, it is of interest to explore the model-data
%agreement by raising questions such as

Statistical software environments, such as the \texttt{R} Programming language, provides a suite of tests and graphical procedure sfor appraising a fitted linear model, with several 
of these procedures analysing the model residuals.

\bigskip 

%=========================================================================%
\section{ Fundamentals of Residuals}


A residual is the difference between an observed quantity and its
estimated or predicted value. In LME models, there are two types
of residuals, marginal residuals and conditional residuals. A
marginal residual is the difference between the observed data and
the estimated marginal mean. A conditional residual is the
difference between the observed data and the predicted value of
the observation. In a model without random effects, both sets of
residuals coincide.

%----------------------------- %
\subsection{Residual}
A residual (or fitting error), on the other hand, is an observable estimate of the unobservable statistical error. Consider the previous example with men's heights and suppose we have a random sample of n people. The sample mean could serve as a good estimator of the population mean. Then we have:

The difference between the height of each man in the sample and the unobservable population mean is a statistical error, whereas
The difference between the height of each man in the sample and the observable sample mean is a residual.
Note that the sum of the residuals within a random sample is necessarily zero, and thus the residuals are necessarily not independent. The statistical errors on the other hand are independent, and their sum within the random sample is almost surely not zero.


\section{Residual diagnostics} %1.3
For classical linear models, residual diagnostics are typically implemented as a plot of the observed residuals and the predicted values. A visual inspection for the presence of trends inform the analyst on the validity of distributional assumptions, and to detect outliers and influential observations.

In linear mixed effects models, diagnostic techniques may consider `conditional' residuals. A conditional residual is the difference between an observed value $y_{i}$ and the conditional predicted value $\hat{y}_{i} $.

\[ \hat{epsilon}_{i} = y_{i} - \hat{y}_{i} = y_{i} - ( X_{i}\hat{beta} + Z_{i}\hat{b}_{i}) \]

However, using conditional residuals for diagnostics presents difficulties, as they tend to be correlatedand their variances may be different for different subgroups, which can lead to erroneous conclusions.



\subsection{Residuals}

The computation of internally studentized residuals relies on the diagonal entries of
$\boldsymbol{V} (\hat{\theta})$ - $\boldsymbol{Q} (\hat{\theta})$, where $\boldsymbol{Q} (\hat{\theta})$ is computed as

\[ \boldsymbol{Q} (\hat{\theta}) = \boldsymbol{X} ( \boldsymbol{X}^{\prime}\boldsymbol{Q} (\hat{\theta})^{-1}\boldsymbol{X})\boldsymbol{X}^{-1} \]

Externally \index{studentized residual} studentized residual require iterative influence analysis or a profiled residuals variance.




\section{Framework for Model Validation using Residual Diagnostics}
In statistical modelling, the process of model validation is a critical step, but also a step that is too often overlooked. A very simple procedure is to examine commonly encountered
metrics, such as the $R^2$ value. However, using a small handful of simple measures and methods is insufficient to properly assess the quality of a fitted model. To do so properly, a full and comprehensive
analysis that tests of all of the assumptions, as far as possible, must be carried out. A statistical model, whether of the fixed-effects or mixed-effects variety, represents how you think your data
were generated. Following model specification and estimation, it is of interest to explore the model-data
agreement by raising questions such as
\begin{itemize}
	\item Does the model-data agreement support the model assumptions?
	\item Should model components be refined, and if so, which components? For example, should regressors
	be added or removed, and is the covariation of the observations modeled properly?
	\item Are the results sensitive to model and/or data? Are individual data points or groups of cases particularly
	influential on the analysis?
\end{itemize}


\subsection{Residual Analysis}

In classical linear models, an examination of model-data agreement has traditionally revolved around

The second part of the chapter looks at diagnostics techniques for LME models, firsly covering the theory, then proceeding to a discussion on 
implementing these using \texttt{R} code.

While a substantial body of work has been developed in this area, there is still areas worth exploring. 
In particular the development of graphical techniques pertinent to LME models should be looked at.

	

\subsection{Introduction}
In statistics and optimization, statistical errors and residuals are two closely related and easily confused measures of the deviation of an observed value of an element of a statistical sample from its "theoretical value". The error (or disturbance) of an observed value is the deviation of the observed value from the (unobservable) true function value, while the residual of an observed value is the difference between the observed value and the estimated function value.

The distinction is most important in regression analysis, where it leads to the concept of studentized residuals.

%--Marginal and Conditional Residuals
%---------------------------------------------------------------------------%
\newpage

\section{Standardized and studentized residuals} %1.6
To alleviate the problem caused by inconstant variance, the residuals are scaled (i.e. divided) by their standard deviations. This results in a \index{standardized residual}`standardized residual'. Because true standard deviations are frequently unknown, one can instead divide a residual by the estimated standard deviation to obtain the \index{studentized residual}`studentized residual. 

Another possible scaled residual is the \index{Pearson residual} `Pearson residual' whereby a residual is divided by the standard deviation of the dependent variable. The Pearson residual can be used when the variability of $\hat{\beta}$ is disregarded in the underlying assumptions.

\subsection{Studentization} %1.6.1

A random variable is said to be standardized if the difference from its mean is scaled by its standard
deviation. The residuals above have mean zero but their variance is unknown, it depends on the true values
of $\theta$. Standardization is thus not possible in practice.
Instead, you can compute studentized residuals
by dividing a residual by an estimate of its standard deviation. If that estimate is independent of the ith
observation, the process is termed external studentization.
This is usually accomplished by excluding the
$i-$th observation when computing the estimate of its standard error. If the observation contributes to the
standard error computation, the residual is said to be \textbf{\emph{internally studentized}}.
%---------------------------------------------------------------------------%
\newpage
\subsection{Residuals diagnostics in mixed models}

%schabenberger
The marginal and conditional means in the linear mixed model are
$E[\boldsymbol{Y}] = \boldsymbol{X}\boldsymbol{\beta}$ and
$E[\boldsymbol{Y|\boldsymbol{u}}] = \boldsymbol{X}\boldsymbol{\beta} + \boldsymbol{Z}\boldsymbol{u}$, respectively.

A residual is the difference between an observed quantity and its estimated or predicted value. In the mixed
model you can distinguish marginal residuals $r_m$ and conditional residuals $r_c$. 


\subsection{Residual}
Residual (or error) represents unexplained (or residual) variation after fitting a regression model. It is the difference (or left over) between the observed value of the variable and the value suggested by the regression model.


%------------------------- %

The difference between the observed value of the dependent variable (y) and the predicted value (ŷ) is called the residual (e). Each data point has one residual.

Residual = Observed value - Predicted value 
\[e = y - \hat{y} \]

Both the sum and the mean of the residuals are equal to zero. That is, Σ e = 0 and e = 0.

%--------------------------------- %





%------------------------------- %

Other uses of the word "error" in statistics: 

The use of the term "error" as discussed in the sections above is in the sense of a deviation of a value from a hypothetical unobserved value. At least two other uses also occur in statistics, both referring to observable prediction errors:

\begin{itemize}
	\item Mean square error or mean squared error (abbreviated MSE) and root mean square error (RMSE) refer to the amount by which the values predicted by an estimator differ from the quantities being estimated (typically outside the sample from which the model was estimated).
	
	\item 
	Sum of squared errors, typically abbreviated SSE or SSe, refers to the residual sum of squares (the sum of squared residuals) of a regression; this is the sum of the squares of the deviations of the actual values from the predicted values, within the sample used for estimation. Likewise, the sum of absolute errors (SAE) refers to the sum of the absolute values of the residuals, which is minimized in the least absolute deviations approach to regression.
	
\end{itemize}



	\subsection{Standardization} %1.4.1
	
	A random variable is said to be standardized if the difference from its mean is scaled by its standard deviation. The residuals above have mean zero but their variance is unknown, it depends on the true values of $\theta$. Standardization is thus not possible in practice.
	
\subsection{Studentization}
In statistics, a studentized residual is the quotient resulting from the division of a residual by an estimate of its standard deviation. Typically the standard deviations of residuals in a sample vary greatly from one data point to another even when the errors all have the same standard deviation, particularly in regression analysis; thus it does not make sense to compare residuals at different data points without first studentizing. It is a form of a Student's t-statistic, with the estimate of error varying between points.

This is an important technique in the detection of outliers. It is named in honor of William Sealey Gosset, who wrote under the pseudonym Student, and dividing by an estimate of scale is called studentizing, in analogy with standardizing and normalizing: see Studentization.

	

	\subsection{Computation}%1.4.4
	
	The computation of internally studentized residuals relies on the diagonal entries of $\boldsymbol{V} (\hat{\theta})$ - $\boldsymbol{Q} (\hat{\theta})$, where $\boldsymbol{Q} (\hat{\theta})$ is computed as
	
	\[ \boldsymbol{Q} (\hat{\theta}) = \boldsymbol{X} ( \boldsymbol{X}^{\prime}\boldsymbol{Q} (\hat{\theta})^{-1}\boldsymbol{X})\boldsymbol{X}^{-1} \]
	
	%---------------------------------------------------------------------------%
	\newpage
\subsection{Standardized and studentized residuals} %1.4

Externally \index{studentized residual} studentized residual require iterative influence analysis or a profiled residuals variance.




	\subsection{Standardized and studentized residuals} %1.4
	%--Studentized and Standardized Residuals
	
	To alleviate the problem caused by inconstant variance, the residuals are scaled (i.e. divided) by their standard deviations. This results in a \index{standardized residual}`standardized residual'. Because true standard deviations are frequently unknown, one can instead divide a residual by the estimated standard deviation to obtain the \index{studentized residual}`studentized residual. 
	
	
	\subsection{Internal and External Studentization} %1.4.3
	If that estimate is independent of the $i-$th observation, the process is termed \index{external studentization}`external studentization'. This is usually accomplished by excluding the $i-$th observation when computing the estimate of its standard error. If the observation contributes to the
	standard error computation, the residual is said to be \index{internally studentization}internally studentized.
	
	Externally \index{studentized residual} studentized residual require iterative influence analysis or a profiled residuals variance.
	

	\subsection{Computation}%1.4.4
	
	The computation of internally studentized residuals relies on the diagonal entries of $\boldsymbol{V} (\hat{\theta})$ - $\boldsymbol{Q} (\hat{\theta})$, where $\boldsymbol{Q} (\hat{\theta})$ is computed as
	
	\[ \boldsymbol{Q} (\hat{\theta}) = \boldsymbol{X} ( \boldsymbol{X}^{\prime}\boldsymbol{Q} (\hat{\theta})^{-1}\boldsymbol{X})\boldsymbol{X}^{-1} \]
	
\subsection{Pearson Residual}%1.4.5

Another possible scaled residual is the \index{Pearson residual} `Pearson residual', whereby a residual is divided by the standard deviation of the dependent variable. The Pearson residual can be used when the variability of $\hat{\beta}$ is disregarded in the underlying assumptions.


\subsection{Computation}%1.4.4

The computation of internally studentized residuals relies on the diagonal entries of $\boldsymbol{V} (\hat{\theta})$ - $\boldsymbol{Q} (\hat{\theta})$, where $\boldsymbol{Q} (\hat{\theta})$ is computed as

\[ \boldsymbol{Q} (\hat{\theta}) = \boldsymbol{X} ( \boldsymbol{X}^{\prime}\boldsymbol{Q} (\hat{\theta})^{-1}\boldsymbol{X})\boldsymbol{X}^{-1} \]



\section{Covariance Parameters} %1.5
The unknown variance elements are referred to as the covariance parameters and collected in the vector $\theta$.
% - where is this coming from?
% - where is it used again?
% - Has this got anything to do with CovTrace etc?
%---------------------------------------------------------------------------%


\subsection{Standardized and studentized residuals} %1.4
%--Studentized and Standardized Residuals

To alleviate the problem caused by inconstant variance, the residuals are scaled (i.e. divided) by their standard deviations. This results in a \index{standardized residual}`standardized residual'. Because true standard deviations are frequently unknown, one can instead divide a residual by the estimated standard deviation to obtain the \index{studentized residual}`studentized residual. 


\chapter{Residuals for LME Models}
%-------------------------------------------------------------- %

\section{Residual Analysis for LME Models}

In classical linear models model diagnostics have been become a required part of any statistical analysis, and the methods are commonly available in statistical packages and standard textbooks on applied regression. However it has been noted by several papers that model diagnostics do not often accompany LME model analyses.

\textbf{Cite:Zewotir} lists several established methods of analyzing influence in LME models. These methods include \begin{itemize}
	\item Cook's distance for LME models,
	\item \index{likelihood distance} likelihood distance,
	\item the variance (information) ration,
	\item the \index{Cook-Weisberg statistic} Cook-Weisberg statistic,
	\item the \index{Andrews-Prebigon statistic} Andrews-Prebigon statistic.
\end{itemize}




%--------------------------------------------------------------%
\subsection{LME REsiduals}	
Cox and Snell (1968, JRSS-B): general definition of residuals for
models with single source of variability
Hilden-Minton (1995, PhD thesis UCLA), Verbeke and Lesaffre
(1997, CSDA) or Pinheiro and Bates (2000, Springer): extension to
define three types of residuals that accommodate the extra source of
variability present in linear mixed models, namely:

i) Marginal residuals, 
%bξ = y − X\hat{\beta} = \hat{M}^{-1}\hat{Q}y ,

predictors of marginal errors, 

%ξ = y − E[y] = y − X\beta = Zb + e

ii) Conditional residuals, 
\[be = y − X\hat{\beta} − Zbb = \hat{\sigma}Q\hat{y}\] , predictors of
conditional errors 
\[e = y − E[y|b] = y − X\beta − Zb\]

iii) BLUP, Zbb, predictors of random effects,
\[ Zb = E[y|b] − E[y]\]


%----------------------------- %
	\subsection{Residual Diagnostics in LME models}
	\begin{itemize}
		\item A \textbf{residual} is the difference between the observed quantity and the predicted value. In LME models a distinction is made between marginal residuals and conditional residuals.
		
		\item A \textbf{Marginal residual} is the difference between the observed data and the estimated marginal mean (Schabenberger  pg3)
		The computation of case deletion diagnostics in the classical model is made simple by the fact that important estimates can be computed without refitting the model. 
		
		\item Such update formulae are available in the mixed model only if you assume that the covariance parameters are not affect by the removal of the observation in question. Schabenberger remarks that this is not a reasonable assumption.
		
	\end{itemize}
	
	
	Basic procedure for quantifying influence is simple
	
	\begin{enumerate}
		\item  	Fit the model to the data
		\item   	Remove one or more data points from the analysis and compute updated estimates of model parameters
		\item  	Based on the full and reduced data estimates, contrast quantities of interest to determine how the absence of the observations changed the analysis.
	\end{enumerate}
	The likelihood distance is a global summary measure expressing the joint influence of the observations in the set U on all parameters in $\Psi$ that were subject to updating.
	
	
\subsection{Residuals diagnostics in mixed models}

A residual is the difference between an observed quantity and its estimated or predicted value. In the mixed
model you can distinguish marginal residuals $rm$ and conditional residuals $rc$. A marginal residual is the
difference between the observed data and the estimated (marginal) mean.

\citet{cook86} introduces powerful tools for local-influence
assessment and examining perturbations in the assumptions of a
model. In particular the effect of local perturbations of
parameters or observations are examined.


\section{Residual Analysis for LME Models}

In classical linear models model diagnostics have been become a required part of any statistical analysis, and the methods are commonly available in statistical packages and standard textbooks on applied regression. However it has been noted by several papers that model diagnostics do not often accompany LME model analyses.

\textbf{Cite:Zewotir} lists several established methods of analyzing influence in LME models. These methods include \begin{itemize}
	\item Cook's distance for LME models,
	\item \index{likelihood distance} likelihood distance,
	\item the variance (information) ration,
	\item the \index{Cook-Weisberg statistic} Cook-Weisberg statistic,
	\item the \index{Andrews-Prebigon statistic} Andrews-Prebigon statistic.
\end{itemize}




%--------------------------------------------------------------%
\subsection{LME REsiduals}	
Cox and Snell (1968, JRSS-B): general definition of residuals for
models with single source of variability
Hilden-Minton (1995, PhD thesis UCLA), Verbeke and Lesaffre
(1997, CSDA) or Pinheiro and Bates (2000, Springer): extension to
define three types of residuals that accommodate the extra source of
variability present in linear mixed models, namely:

i) Marginal residuals, 
%bξ = y − X\hat{\beta} = \hat{M}^{-1}\hat{Q}y ,

predictors of marginal errors, 

%ξ = y − E[y] = y − X\beta = Zb + e

ii) Conditional residuals, 
\[be = y − X\hat{\beta} − Zbb = \hat{\sigma}Q\hat{y}\] , predictors of
conditional errors 
\[e = y − E[y|b] = y − X\beta − Zb\]

iii) BLUP, Zbb, predictors of random effects,
\[ Zb = E[y|b] − E[y]\]


%----------------------------- %
\section{Marginal and Conditional Residuals} %1.5
%http://support.sas.com/documentation/cdl/en/statug/63033/HTML/default/viewer.htm#statug_mixed_sect024.htm
The marginal and conditional means in the linear mixed model are
$E[\boldsymbol{Y}] = \boldsymbol{X}\boldsymbol{\beta}$ and
$E[\boldsymbol{Y|\boldsymbol{u}}] = \boldsymbol{X}\boldsymbol{\beta} + \boldsymbol{Z}\boldsymbol{u}$, respectively.

%schabenberger
A residual is the difference between an observed quantity and its estimated or predicted value. In the mixed
model you can distinguish marginal residuals $r_m$ and conditional residuals $r_c$. A marginal residual is the
difference between the observed data and the estimated (marginal) mean, $r_{mi} = y_i - x_0^{\prime} \hat{b}$
A conditional residual is the difference between the observed data and the predicted value of the observation,
$r_{ci} = y_i - x_i^{\prime} \hat{b} - z_i^{\prime} \hat{\gamma}$

\subsection{Marginal and Conditional Residuals}

\begin{equation}
r_{mi}=x^{T}_{i}\hat{\beta}
\end{equation}

\subsection{Marginal Residuals}
\begin{eqnarray}
\hat{\beta} &=& (X^{T}R^{-1}X)^{-1}X^{T}R^{-1}Y \nonumber \\
&=& BY \nonumber
\end{eqnarray}

%---------------------------------------------------------------------------%

\subsection{Marginal and Conditional Residuals}

A marginal residual is the difference between the observed data and the estimated (marginal) mean, $r_{mi} = y_i - x_0^{\prime} \hat{b}$
A conditional residual is the difference between the observed data and the predicted value of the observation,
$r_{ci} = y_i - x_i^{\prime} \hat{b} - z_i^{\prime} \hat{\gamma}$

In linear mixed effects models, diagnostic techniques may consider `conditional' residuals. A conditional residual is the difference between an observed value $y_{i}$ and the conditional predicted value $\hat{y}_{i} $.

\[ \hat{epsilon}_{i} = y_{i} - \hat{y}_{i} = y_{i} - ( X_{i}\hat{beta} + Z_{i}\hat{b}_{i}) \]

However, using conditional residuals for diagnostics presents difficulties, as they tend to be correlated and their variances may be different for different subgroups, which can lead to erroneous conclusions.

%1.5
%http://support.sas.com/documentation/cdl/en/statug/63033/HTML/default/viewer.htm#statug_mixed_sect024.htm






\begin{equation}
r_{mi}=x^{T}_{i}\hat{\beta}
\end{equation}

\subsection{Marginal Residuals}
\begin{eqnarray}
\hat{\beta} &=& (X^{T}R^{-1}X)^{-1}X^{T}R^{-1}Y \nonumber \\
&=& BY \nonumber
\end{eqnarray}

%-------------------------------------------------------------- %


	\subsection{Marginal and Conditional Residuals}
	
	A marginal residual is the difference between the observed data and the estimated (marginal) mean, $r_{mi} = y_i - x_0^{\prime} \hat{b}$
	A conditional residual is the difference between the observed data and the predicted value of the observation,
	$r_{ci} = y_i - x_i^{\prime} \hat{b} - z_i^{\prime} \hat{\gamma}$
	
	In linear mixed effects models, diagnostic techniques may consider `conditional' residuals. A conditional residual is the difference between an observed value $y_{i}$ and the conditional predicted value $\hat{y}_{i} $.
	
	\[ \hat{epsilon}_{i} = y_{i} - \hat{y}_{i} = y_{i} - ( X_{i}\hat{beta} + Z_{i}\hat{b}_{i}) \]
	
	However, using conditional residuals for diagnostics presents difficulties, as they tend to be correlated and their variances may be different for different subgroups, which can lead to erroneous conclusions.
	
	
	
	
	
	
	
	\begin{equation}
	r_{mi}=x^{T}_{i}\hat{\beta}
	\end{equation}
	
	
	\subsection{Marginal and Conditional Residuals}
	
	
	A marginal residual is the difference between the observed data and the estimated (marginal) mean, $r_{mi} = y_i - x_0^{\prime} \hat{b}$
	A conditional residual is the difference between the observed data and the predicted value of the observation,
	$r_{ci} = y_i - x_i^{\prime} \hat{b} - z_i^{\prime} \hat{\gamma}$
	
	
	In linear mixed effects models, diagnostic techniques may consider `conditional' residuals. A conditional residual is the difference between an observed value $y_{i}$ and the conditional predicted value $\hat{y}_{i} $.
	
	
	\[ \hat{epsilon}_{i} = y_{i} - \hat{y}_{i} = y_{i} - ( X_{i}\hat{beta} + Z_{i}\hat{b}_{i}) \]
	
	
	However, using conditional residuals for diagnostics presents difficulties, as they tend to be correlated and their variances may be different for different subgroups, which can lead to erroneous conclusions.
	
	
	%1.5
	%http://support.sas.com/documentation/cdl/en/statug/63033/HTML/default/viewer.htm#statug_mixed_sect024.htm
	
	
	
	\subsection{Residuals diagnostics in LME Models}
	
	%schabenberger
	The marginal and conditional means in the linear mixed model are
	$E[\boldsymbol{Y}] = \boldsymbol{X}\boldsymbol{\beta}$ and
	$E[\boldsymbol{Y|\boldsymbol{u}}] = \boldsymbol{X}\boldsymbol{\beta} + \boldsymbol{Z}\boldsymbol{u}$, respectively.
	
	
	
	
	
	
	
	
	
	
	\begin{equation}
	r_{mi}=x^{T}_{i}\hat{\beta}
	\end{equation}
	
	
	\subsection{Marginal Residuals}
	\begin{eqnarray}
	\hat{\beta} &=& (X^{T}R^{-1}X)^{-1}X^{T}R^{-1}Y \nonumber \\
	&=& BY \nonumber
	\end{eqnarray}
	
	
	
	
	
	%------------------------------------------------------------%
	\section*{Residuals}
	
	Residuals are used to examine model assumptions and to detect outliers and potentially influential data
	point. The raw residuals $r_{mi}$ and $r_{ci}$ are usually not well suited for these purposes.
	
	\begin{itemize}
		\item Conditional Residuals $r_{ci}$
		\item Marginal Residuals $r_{mi}$
		\item 
	\end{itemize}
	
	
	\section{Residual diagnostics} %1.3
	For classical linear models, residual diagnostics are typically implemented as a plot of the observed residuals and the predicted values. A visual inspection for the presence of trends inform the analyst on the validity of distributional assumptions, and to detect outliers and influential observations.
	
	
	
	%--Marginal and Conditional Residuals
	
	
	\subsection{Marginal Residuals}
	\begin{eqnarray}
	\hat{\beta} &=& (X^{T}R^{-1}X)^{-1}X^{T}R^{-1}Y \nonumber \\
	&=& BY \nonumber
	\end{eqnarray}
	
	
	
	
	\subsection{Confounded Residuals}
	Hilden-Minton (1995, PhD thesis, UCLA): residual is pure for a
	specific type of error if it depends only on the fixed components and
	on the error that it is supposed to predict
	Residuals that depend on other types of errors are called \textit{\textbf{confounded
			residuals}}
	\section{Conditional and Marginal Residuals}
	Conditional residuals include contributions from both fixed and random effects, whereas marginal residuals include contribution from only fixed effects.
	
	Suppose the linear mixed-effects model lme has an $n \times p$ fixed-effects design matrix $\boldsymbol{X}$ and an $n \times q$ random-effects design matrix $\boldsymbol{Z}$. 
	
	Also, suppose the p-by-1 estimated fixed-effects vector is $\hat{\beta}$ , and the q-by-1 estimated best linear unbiased predictor (BLUP) 
	vector of random effects is $\hat{b}$ . The fitted conditional response is
	
	\[ \hat{y}_{Cond} = X \hat{\beta} + Z \hat{b} \]
	
	and the fitted marginal response is
	
	
	\[ \hat{y}_{Mar} = X \hat{\beta} \]
	
	residuals can return three types of residuals:
	\begin{itemize} 
		\item raw, 
		\item Pearson, and 
		\item standardized.\end{itemize} For any type, you can compute the conditional or the marginal residuals. For example, the conditional raw residual is
	
	
	\[ r_{Cond} = y - X \hat{\beta} - Z \hat{b} \]
	
	and the marginal raw residual is
	
	
	
	\[ r_{Mar} = y - X \hat{\beta} \]
	
	\newpage
	%=================================================== %
	% http://www.ime.usp.br/~jmsinger/MAE0610/Mixedmodelresiduals.pdf
	
	Cox and Snell (1968, JRSS-B): general definition of residuals for
	models with single source of variability
	Hilden-Minton (1995, PhD thesis UCLA), Verbeke and Lesaffre
	(1997, CSDA) or Pinheiro and Bates (2000, Springer): extension to
	define three types of residuals that accommodate the extra source of
	variability present in linear mixed models, namely:
	
	i) Marginal residuals, 
	%bξ = y − X\hat{\beta} = \hat{M}^{-1}\hat{Q}y ,
	
	predictors of marginal errors, 
	
	%ξ = y − E[y] = y − X\beta = Zb + e
	
	ii) Conditional residuals, 
	\[be = y − X\hat{\beta} − Zbb = \hat{\sigma}Q\hat{y}\] , predictors of
	conditional errors 
	\[e = y − E[y|b] = y − X\beta − Zb\]
	
	iii) BLUP, Zbb, predictors of random effects,
	\[ Zb = E[y|b] − E[y]\]
	
	
	%------------------------------------------------------------------%
	\newpage
	
	\subsection*{Marginal residuals}
	
	\[y - X\beta = Z \eta +\epsilon \]
	\begin{itemize}
		\item
		Should be mean 0, but may show grouping structure
		\item
		May not be homoskedastic.
		\item
		Good for checking fixed effects, just like linear regr.
	\end{itemize}
	%----------------------------------------------------%
	\subsection*{Conditional residuals}
	\[y - X\beta - Z \eta = \epsilon \]
	\begin{itemize}
		\item
		Should be mean zero with no grouping structure
		\item
		Should be homoscedastic.
		\item
		Good for checking normality of outliers
	\end{itemize}
	
	%-----------------------------------------------------%
	\subsection*{Random effects}
	\[y - X\beta -\epsilon = Z \eta \]
	\begin{itemize}
		\item
		Should be mean zero with no grouping structure
		\item
		May not be be homoscedastic.
	\end{itemize}
	
	\subsection*{Marginal Residuals}
	
	%------------------------------------------------------------%
	
	Distinction From Linear Models
	\begin{itemize}
		\item The differences between perturbation and residual analysis in the linear model and the linear mixed model
		are connected to the important facts that b and b
		depend on the estimates of the covariance parameters,
		that b has the form of an (estimated) generalized least squares (GLS) estimator, and that 
		is a random
		vector.
		\item In a mixed model, you can consider the data in a conditional and an unconditional sense. If you imagine
		a particular realization of the random effects, then you are considering the conditional distribution
		Y|
		\item If you are interested in quantities averaged over all possible values of the random effects, then
		you are interested in Y; this is called the marginal formulation. In a clinical trial, for example, you
		may be interested in drug efficacy for a particular patient. If random effects vary by patient, that is a
		conditional problem. If you are interested in the drug efficacy in the population of all patients, you are
		using a marginal formulation. Correspondingly, there will be conditional and marginal residuals, for
		example.
		\item The estimates of the fixed effects  depend on the estimates of the covariance parameters. If you are
		interested in determining the influence of an observation on the analysis, you must determine whether
		this is influence on the fixed effects for a given value of the covariance parameters, influence on the
		covariance parameters, or influence on both.
		\item Mixed models are often used to analyze repeated measures and longitudinal data. The natural experimental
		or sampling unit in those studies is the entity that is repeatedly observed, rather than each
		individual repeated observation. For example, you may be analyzing monthly purchase records by
		customer. 
		\item An influential “data point” is then not necessarily a single purchase. You are probably more
		interested in determining the influential customer. This requires that you can measure the influence
		of sets of observations on the analysis, not just influence of individual observations.
		\item The computation of case deletion diagnostics in the classical model is made simple by the fact that
		%estimates of  and 2, which exclude the ith observation, can be %computed without re-fitting the
		model. Such update formulas are available in the mixed model only if you assume that the covariance
		parameters are not affected by the removal of the observation in question. This is rarely a reasonable
		assumption.
		\item The application of well-known concepts in model-data diagnostics to the mixed model can produce results
		that are at first counter-intuitive, since our understanding is steeped in the ordinary least squares
		(OLS) framework. As a consequence, we need to revisit these important concepts, ask whether they
		are “portable” to the mixed model, and gain new appreciation for their changed properties. An important
		example is the ostensibly simple concept of leverage. 
		\item The definition of leverage adopted by
		the MIXED procedure can, in some instances, produce negative values, which are mathematically
		impossible in OLS. Other measures that have been proposed may be non-negative, but trade other
		advantages. Another example are properties of residuals. While OLS residuals necessarily sum to
		zero in any model (with intercept), this not true of the residuals in many mixed models.
	\end{itemize}
	\newpage
	%---------------------------------------------------------------------------%
	
	\subsection{Residuals diagnostics in mixed models}
	
	%schabenberger
	The marginal and conditional means in the linear mixed model are
	$E[\boldsymbol{Y}] = \boldsymbol{X}\boldsymbol{\beta}$ and
	$E[\boldsymbol{Y|\boldsymbol{u}}] = \boldsymbol{X}\boldsymbol{\beta} + \boldsymbol{Z}\boldsymbol{u}$, respectively.
	
	A residual is the difference between an observed quantity and its estimated or predicted value. In the mixed
	model you can distinguish marginal residuals $r_m$ and conditional residuals $r_c$. 
	
	
	
	
	
	
	
	
	\begin{equation}
	r_{mi}=x^{T}_{i}\hat{\beta}
	\end{equation}
	
	%======================================================================%
	
	\subsection{Marginal Residuals}
	\begin{eqnarray}
	\hat{\beta} &=& (X^{T}R^{-1}X)^{-1}X^{T}R^{-1}Y \nonumber \\
	&=& BY \nonumber
	\end{eqnarray}
	
	
	
\section{Diagnostic Tools for the nlme package}


With the nlme package, the generic function \texttt{lme()} fits a linear mixed-effects model in the formulation described in Laird and Ware (1982) but allowing for nested random effects. 

The within-group errors are allowed to be correlated and/or have unequal variances, which is very important in fitting the models for Roy's Tests

The nlme package has a limited set of diagnostic tools that can be used to assess the model fit. A review of the package manual is sufficient to get a sense of the package's capability in that regard.




\section{Computation and Notation } %2.3
with $\boldsymbol{V}$ unknown, a standard practice for estimating $\boldsymbol{X \beta}$ is the estime the variance components $\sigma^2_j$,
compute an estimate for $\boldsymbol{V}$ and then compute the projector matrix $A$, $\boldsymbol{X \hat{\beta}}  = \boldsymbol{AY}$.


%\citet{zewotir} remarks that $\boldsymbol{D}$ is a block diagonal with the $i-$th block being $u \boldsymbol{I}$


\bibliography{DB-txfrbib}

\end{document}