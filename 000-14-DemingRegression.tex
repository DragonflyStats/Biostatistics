\newpage
\section*{Deming Regression}

Performance of Deming regression analysis in case of misspecified analytical error ratio in method comparison studies

%-----------------------------------------------------------------------------------------------%
Application of Deming regression analysis to interpret method comparison data presupposes specification of the 
squared analytical error ratio ($\lambda$, but in cases involving only single measurements by each method, this 
ratio may be unknown and is often assigned a default value of one. 

On the basis of simulations, this practice was evaluated in situations with real error ratios deviating from one. 
Comparisons of two electrolyte methods and two glucose methods were simulated. 

In the first case, misspecification of $\lambda$ produced a bias that amounted to two-thirds of the maximum bias of the 
ordinary least-squares regression method. Standard errors and the results of hypothesis-testing also became misleading. 
In the second situation, a misspecified error ratio resulted only in a negligible bias. 

Thus, given a short range of values in relation to the measurement errors, it is important that $\lambda$ is correctly 
estimated either from duplicate sets of measurements or, in the case of single measurement sets, specified from 
quality-control data. However, even with a misspecified error ratio, Deming regression analysis is likely to perform 
better than least-squares regression analysis.


\newpage
Performance of Deming regression analysis in case of misspecified analytical error ratio in method comparison studies

%-----------------------------------------------------------------------------------------------%
Application of Deming regression analysis to interpret method comparison data presupposes specification of the 
squared analytical error ratio ($\lambda$, but in cases involving only single measurements by each method, this 
ratio may be unknown and is often assigned a default value of one. 

On the basis of simulations, this practice was evaluated in situations with real error ratios deviating from one. 
Comparisons of two electrolyte methods and two glucose methods were simulated. 

In the first case, misspecification of $\lambda$ produced a bias that amounted to two-thirds of the maximum bias of the 
ordinary least-squares regression method. Standard errors and the results of hypothesis-testing also became misleading. 
In the second situation, a misspecified error ratio resulted only in a negligible bias. 

Thus, given a short range of values in relation to the measurement errors, it is important that $\lambda$ is correctly 
estimated either from duplicate sets of measurements or, in the case of single measurement sets, specified from 
quality-control data. However, even with a misspecified error ratio, Deming regression analysis is likely to perform 
better than least-squares regression analysis.

%---------------------------------------------------------------------------------------------------%

\section*{Deming regression}
The Deming regression line is estimated by minimizing the sums of squared deviations in both the x and y directions at an angle determined by the ratio of the analytical standard deviations for the two methods.
This ratio can be estimated if multiple measurements were taken with each method, but if only one measurement was taken with each method, it can be assumed to be equal to one.

%---------------------------------------------------------------------------------------------------%

\section*{Koning}
http://www.springerlink.com/content/r1063462u618q483/

Use of deming regression in method comparison studies.
Henk Konings

Accuracy is closeness to the true value, or alternatively, having a low measurement error.

The determination of a true value for a biological specimen is difficult and sometimes impossible.

Precision is expressed in terms of standard deviation, coefficient of variance or variance.

In Deming regression, the errors between methods are assigned to both methods in proportion to the variances of the methods.

%---------------------------------------------------------------------------------------------------%

\section*{Linnet}
The statistical procedures are described in:
Linnet K. Necessary sample size for method comparison studies based on regression analysis. Clin Chem 1999; 45: 882-94.
Linnet K. Performance of Deming regression analysis in case of misspecified analytical error ratio in method comparison studies. Clin Chem 1998; 44: 1024-1031.
Linnet K. Evaluation of regression procedures for methods comparison studies. Clin Chem 1993; 39. 424-432.
Linnet K. Estimation of the linear relationship between measurements of two methods with proportional errors. Stat Med 1990; 9: 1463-1473.

