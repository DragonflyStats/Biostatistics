\documentclass[12pt, a4paper]{report}
\usepackage{natbib}
\usepackage{vmargin}
\usepackage{graphicx}
\usepackage{epsfig}
\usepackage{subfigure}
%\usepackage{amscd}
\usepackage{amssymb}
\usepackage{amsbsy}
\usepackage{amsthm, amsmath}
%\usepackage[dvips]{graphicx}
\bibliographystyle{chicago}
\renewcommand{\baselinestretch}{1.8}

% left top textwidth textheight headheight % headsep footheight footskip
\setmargins{3.0cm}{2.5cm}{15.5 cm}{23.5cm}{0.5cm}{0cm}{1cm}{1cm}

\pagenumbering{arabic}


\begin{document}

\section*{Mountain Plot}
%% http://www.clinchem.org/content/43/11/2039.long

Krouwer and Monti (29) presented a graphical method for evaluation of laboratory assays (a mountain plot). 
They computed the percentile for each ranked difference between the two methods, and by “turning” at the 50th percentile 
produced a histogram-like function (the mountain). 
This method is relevant for detecting large infrequent errors (differences) but lacks the aspect of concentration relationship. 
These investigators, therefore, recommend use of their plot together with difference plots. 
Introduction of analytical quality specifications in the mountain plots may be useful in method evaluations.

\newpage
\subsection*{Bartko’s Ellipse}

Bartko’s proposes a graphical approach that complements the Bland-Altman approach.
[Bartko 1998]

Mountain Plot
A mountain plot (or "folded empirical cumulative distribution plot") is created by computing a percentile for each ranked difference between two methods of measurement.

To get a folded plot, the following transformation is performed for all percentiles above 50: percentile = 100 - percentile. These percentiles are then plotted against the differences between the two methods [Krouwer & Monti, 1995].

The mountain plot is a useful complementary plot to the Bland & Altman plot. 
In particular, the mountain plot offers the following advantages:
\begin{itemize}
	\item It is easier to find the central 95% of the data, even when the data are not Normally distributed.
	\item Different distributions can be compared more easily.
	
\end{itemize}
\addcontentsline{toc}{section}{Bibliography}

\bibliography{transferbib}
\end{document}
