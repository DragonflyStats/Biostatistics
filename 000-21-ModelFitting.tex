\documentclass[12pt, a4paper]{article}
\usepackage{natbib}
\usepackage{vmargin}
\usepackage{graphicx}
\usepackage{epsfig}
\usepackage{subfigure}
%\usepackage{amscd}
\usepackage{amssymb}
\usepackage{subfiles}
\usepackage{subfigure}
\usepackage{framed}
\usepackage{subfiles}
\usepackage{amsbsy}
\usepackage{amsthm, amsmath}
%\usepackage[dvips]{graphicx}
\bibliographystyle{chicago}
\renewcommand{\baselinestretch}{1.1}

% left top textwidth textheight headheight % headsep footheight footskip
\setmargins{3.0cm}{2.5cm}{15.5 cm}{23.5cm}{0.25cm}{0cm}{0.5cm}{0.5cm}

\pagenumbering{arabic}

\begin{document}

\section{Off-Diagonal Components in Roy's Model}

The Within-item variability is specified as follows, where $x$ and $y$ are the methods of measurement in question.
\[ \left(
\begin{array}{cc}
\sigma^2_x & \sigma_{xy} \\
\sigma_{xy} & \sigma^2_y \\
\end{array}
\right)
\]

$\sigma^2_x$ and $\sigma^2_y$ describe the level of measurement error associated with each of the measurement methods for a given item. Attention must be given to the off-diagonal elements of the matrix.

It is intuitive to consider the measurement error of the two methods as independent of each other.

\section{Formal Testing}
A formal test can be performed to test the hypothesis that the off-diagonal terms are zero.
\[ \left(
\begin{array}{cc}
\sigma^2_x & \sigma_xy \\
\sigma_xy & \sigma^2_y \\
\end{array}
\right) vs \left(
\begin{array}{cc}
\sigma^2_x & 0 \\
0 & \sigma^2_y \\
\end{array}
\right)
\]


%================================================================================%
\section{Basic Models Fits}
Further to \citet{PB}, several simple LME models are constructed
for the blood pressure data. This data set is the subject of a
method comparison study in \citet{BA99}.

\subsection{Implementing the Mixed Models Fits}
They are implemented using the following {\tt{R}} code, utilising the
`nlme' package. An analysis of variance is used to compare the model fits.

The {\tt{R}} script:
\begin{verbatim}
fit1 = lme( BP ~ method, data = dat, random = ~1 | subject )
fit2 = update(fit1, random = ~1 | subject/method )
fit3 = update(fit1, random = ~method - 1 | subject )
#analysis of variance
anova(fit1,fit2,fit3)
\end{verbatim}


\begin{enumerate}
	
	
	\item Simplest workable model, allows differences between methods
	and incorporates a random intercept for each subject. For subject
	1 we have
	\[
	\boldsymbol{X}_i =
	\left(%
	\begin{array}{cc}
	1 & 0 \\
	1 & 0 \\
	1 & 0 \\
	1 & 1 \\
	1 & 1 \\
	1 & 1 \\
	\end{array}%
	\right),\quad
	\boldsymbol{\beta} =
	\left(%
	\begin{array}{c}
	\beta_0 \\
	\beta_1 \\
	\end{array}%
	\right), \quad
	\boldsymbol{Z}_i =
	\left(%
	\begin{array}{c}
	1 \\
	1 \\
	1 \\
	1 \\
	1 \\
	1 \\
	\end{array}%
	\right), \quad \boldsymbol{b}_i = b
	\]
	where $\mathrm{E}(b)=0$ and $\mathrm{var}(b)=\psi.$
	
	\item
	\[
	\boldsymbol{Z}_i =
	\left(%
	\begin{array}{c c}
	1 & 0 \\
	1 & 0 \\
	1 & 0 \\
	0 & 1 \\
	0 & 1 \\
	0 & 1 \\
	\end{array}%
	\right)
	\quad \boldsymbol{b}_i =
	\left(%
	\begin{array}{c c}
	b_1 & 0  \\
	0 & b_2  \\
	\end{array}%
	\right)
	\]
	
	where $\mathrm{E}(b_i)=0$ and $\mathrm{var}(\boldsymbol{b})=
	\boldsymbol{\Psi}$.
	
	The variance of error terms is a $6 \times 6$ matrix.
	
\end{enumerate}

%============================================================================%

\subsection{Model Fit 1}

This is a simple model with no interactions. There is a fixed effect for each method and a random effect for each subject.
\begin{equation*}
y_{ijk} = \beta_{j}  + b_{i} + \epsilon_{ijk}, \qquad i=1,\dots,2, j=1,\dots,85, k=1,\dots,3
\end{equation*}

\begin{eqnarray*}
	b_{i} \sim \mathcal{N}(0,\sigma^2_{b}), \qquad \epsilon_{i} \sim \mathcal{N}(0,\sigma^2)
\end{eqnarray*}

\begin{verbatim}
Linear mixed-effects model fit by REML
Data: dat
Log-restricted-likelihood: -2155.853
Fixed: BP ~ method
(Intercept)     methodS
127.40784    15.61961

Random effects:
Formula: ~1 | subject
(Intercept) Residual
StdDev:    29.39085 12.44454

Number of Observations: 510
Number of Groups: 85
\end{verbatim}


\subsection{Model Fit 2}

This is a simple model, this time with an interaction effect.
There is a fixed effect for each method. This model has random effects at two levels $b_{i}$ for the subject, and
another, $b_{ij}$, for the respective method within each subject.
\begin{equation*}
y_{ijk} = \beta_{j}  + b_{i} + b_{ij} + \epsilon_{ijk}, \qquad i=1,\dots,2, j=1,\dots,85, k=1,\dots,3
\end{equation*}
\begin{eqnarray*}
	b_{i} \sim \mathcal{N}(0,\sigma^2_{1}), \qquad b_{ij} \sim \mathcal{N}(0,\sigma^2_{2}), \qquad \epsilon_{i} \sim \mathcal{N}(0,\sigma^2)
\end{eqnarray*}

In this model, the random interaction terms all have the same variance $\sigma^2_{2}$. These terms are assumed to be independent of each other, even
within the same subject.

\begin{verbatim}
Linear mixed-effects model fit by REML
Data: dat
Log-restricted-likelihood: -2047.714
Fixed: BP ~ method
(Intercept)     methodS
127.40784    15.61961

Random effects:
Formula: ~1 | subject
(Intercept)
StdDev:    28.28452

Formula: ~1 | method %in% subject
(Intercept) Residual
StdDev:    12.61562 7.763666

Number of Observations: 510
Number of Groups:
subject method %in% subject
85                 170
\end{verbatim}



\subsection{Model Fit 3}

This model is a more general model, compared to 'model fit 2'. This model treats the random interactions for each subject as a vector and
allows the variance-covariance matrix for that vector to be estimated from the set of all positive-definite matrices.
$\boldsymbol{y_{i}}$ is the entire response vector for the $i$th subject.
$\boldsymbol{X_{i}}$ and $\boldsymbol{Z_{i}}$  are the fixed- and random-effects design matrices respectively.
\begin{equation*}
\boldsymbol{y_{i}} = \boldsymbol{X_{i}\beta}  + \boldsymbol{Z_{i}b_{i}} + \boldsymbol{\epsilon_{i}}, \qquad i=1,\dots,85
\end{equation*}
\begin{eqnarray*}
	\boldsymbol{Z_{i}} \sim \mathcal{N}(\boldsymbol{0,\Psi}),\qquad
	\boldsymbol{\epsilon_{i}} \sim \mathcal{N}(\boldsymbol{0,\sigma^2\Lambda})
\end{eqnarray*}

For the first subject the response vector, $\boldsymbol{y_{1}}$, is:
\begin{table}[ht]
	\begin{center}
		\begin{tabular}{rrllr}
			\hline
			observation & BP & subject & method & replicate \\
			\hline
			1 & 100.00 & 1 & J &   1 \\
			86 & 106.00 & 1 & J &   2 \\
			171 & 107.00 & 1 & J &   3 \\
			511 & 122.00 & 1 & S &   1 \\
			596 & 128.00 & 1 & S &   2 \\
			681 & 124.00 & 1 & S &   3 \\
			\hline
		\end{tabular}
	\end{center}
\end{table}
%===============================================================================================%
The fixed effects design matrix $\boldsymbol{X_{i}}$ is given by:
\begin{table}[ht]
	\begin{center}
		\begin{tabular}{r|r}
			\hline
			(Intercept) & method S \\
			\hline
			1 & 0 \\
			1 & 0 \\
			1 & 0 \\
			1 & 1 \\
			1 & 1 \\
			1 & 1 \\
			\hline
		\end{tabular}
	\end{center}
\end{table}

The random effects design matrix $\boldsymbol{Z_{i}}$ is given by:
\begin{table}[ht]
	\begin{center}
		\begin{tabular}{r|r}
			\hline
			method J & method S \\
			\hline
			1 & 0 \\
			1 & 0 \\
			1 & 0 \\
			0 & 1 \\
			0 & 1 \\
			0 & 1 \\
			\hline
		\end{tabular}
	\end{center}
\end{table}
%========================================================================================%
The following output was obtained.
\begin{verbatim}
Linear mixed-effects model fit by REML
Data: dat
Log-restricted-likelihood: -2047.582
Fixed: BP ~ method
(Intercept)     methodS
127.40784    15.61961

Random effects:
Formula: ~method - 1 | subject
Structure: General positive-definite, Log-Cholesky parametrization
StdDev    Corr
methodJ  30.455093 methdJ
methodS  31.477237 0.835
Residual  7.763666

Number of Observations: 510
Number of Groups: 85

\end{verbatim}





\subsection{Extended LME model}
% Pinheiro Bates Page 202
The extended single level LME model relaxes the independence assumption, allowing heteroscedastic and correlated within group errors.


\begin{equation}
\epsilon_{i} = \mathcal{N}(0, \sigma^2 \Lambda_{i})
\end{equation}

$\Lambda_{i}$ are positive definite matrices. $\sigma^2$ is factored out of the matrix for computational reasons.


\section{Variance functions}

Variance functions are applied to LME models through the `weights' argument. $R$ supports several variance functions.

`varIdent' cosntructs a model with different variances per stratum.

\subsection{Diagnostic plots}
% Pinheiro Bates Page 391
Diagnostic plots for identifying within-group heteroscedascity and assessing the adequacy of a variance function can also be used with `nlme' objects.













\bibliographystyle{chicago}
\bibliography{DB-txfrbib}
\end{document}
