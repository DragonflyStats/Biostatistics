% !TEX TS-program = pdflatex
% !TEX encoding = UTF-8 Unicode

% This is a simple template for a LaTeX document using the "article" class.
% See "book", "report", "letter" for other types of document.

\documentclass[11pt]{article} % use larger type; default would be 10pt

\usepackage[utf8]{inputenc} % set input encoding (not needed with XeLaTeX)
\usepackage{framed}
%%% Examples of Article customizations
% These packages are optional, depending whether you want the features they provide.
% See the LaTeX Companion or other references for full information.

%%% PAGE DIMENSIONS
\usepackage{geometry} % to change the page dimensions
\geometry{a4paper} % or letterpaper (US) or a5paper or....
% \geometry{margin=2in} % for example, change the margins to 2 inches all round
% \geometry{landscape} % set up the page for landscape
%   read geometry.pdf for detailed page layout information

\usepackage{graphicx} % support the \includegraphics command and options

% \usepackage[parfill]{parskip} % Activate to begin paragraphs with an empty line rather than an indent

%%% PACKAGES
\usepackage{booktabs} % for much better looking tables
\usepackage{array} % for better arrays (eg matrices) in maths
\usepackage{paralist} % very flexible & customisable lists (eg. enumerate/itemize, etc.)
\usepackage{verbatim} % adds environment for commenting out blocks of text & for better verbatim
\usepackage{subfig} % make it possible to include more than one captioned figure/table in a single float
% These packages are all incorporated in the memoir class to one degree or another...

%%% HEADERS & FOOTERS
\usepackage{fancyhdr} % This should be set AFTER setting up the page geometry
\pagestyle{fancy} % options: empty , plain , fancy
\renewcommand{\headrulewidth}{0pt} % customise the layout...
\lhead{}\chead{}\rhead{}
\lfoot{}\cfoot{\thepage}\rfoot{}

%%% SECTION TITLE APPEARANCE
\usepackage{sectsty}
\allsectionsfont{\sffamily\mdseries\upshape} % (See the fntguide.pdf for font help)
% (This matches ConTeXt defaults)

%%% ToC (table of contents) APPEARANCE
\usepackage[nottoc,notlof,notlot]{tocbibind} % Put the bibliography in the ToC
\usepackage[titles,subfigure]{tocloft} % Alter the style of the Table of Contents
\renewcommand{\cftsecfont}{\rmfamily\mdseries\upshape}
\renewcommand{\cftsecpagefont}{\rmfamily\mdseries\upshape} % No bold!

%%% END Article customizations

%%% The "real" document content comes below...

\title{Brief Article}
\author{The Author}
%\date{} % Activate to display a given date or no date (if empty),
         % otherwise the current date is printed 

\begin{document}
\maketitle


\section{Part 1a}

Accuracy, Precision

Agreement and Repeatability

Difficulty in Obtaining Clinical Measurements (proxy measurements)

%-----------------------------------------%

\section{Part 1b}

Method Comparison Studies

Calibration and Gold Standard Methods

Exchangability

%-----------------------------------------%

\section{Part 1c : Bland-Altman Methods}

\begin{itemize}
\item Difference Plots
\item Limits of Agreement
\item Coefficient of Repeatability
\item Mountain Plots and Bartko's Ellipse
\item Survival Plots (Luiz et Al)
\end{itemize}

%-----------------------------------------%

\section{Regression Techniques for MCS}

The \textbf{\textit{mcr}} packages provides a set of regression techniques to quantify the relation between two measurement methods.

In particular, it address regression problems with errors in both variables, but without repeated measurements.
The \textbf{\textit{mcr}} package follows the CLSI EP09-A3 recommendations for analytical
method comparison and estimation of bias using patient samples.


\textit{Methods featured in the \textbf{mcr} package}

\begin{itemize}
\item Deming Regression
\item Weighted Deming Regression
\item Passing-Bablock Regression
\end{itemize}

The \textit{creatinine} gives the blood and serum preoperative creatinine measurements in 110 heart surgery patients.

\begin{framed}
\begin{verbatim}
library("mcr")
data("creatinine", package="mcr")
tail(creatinine)


fit.lr <- mcreg(as.matrix(creatinine), method.reg="LinReg", na.rm=TRUE)
fit.wlr <- mcreg(as.matrix(creatinine), method.reg="WLinReg", na.rm=TRUE)
compareFit( fit.lr, fit.wlr )
\end{verbatim}
\end{framed}


\subsection{Bootstap Techniques}
Use of Bootstap Techniques to obtain Confidence Interval estimates

%----------------------------------------%
\newpage
\section{Bayesian Approaches}

%----------------------------------------%
\newpage
\section{References}
Carpenter, J., Bithell, J. (2000) Bootstrap confidence intervals: when, which, what? A practical
guide for medical statisticians. Stat Med, 19 (9), 1141–1164.

\end{document}
