	\section{Intervals}
	
	\subsection{Purpose of Limits of Agreement} It must be established
	clearly the specific purpose of the limits of agreement.
	\citet*{BA95} comment that the limits of agreement \emph{how far
		apart measurements by the two methods were likely to be for most
		individuals.}, a definition echoed in their 1999 paper:
	\begin{quote} We can then say that nearly all pairs
		of measurements by the two methods will be closer together than
		these extreme values, which we call 95\% limits of agreement.
		These values define the range within which most differences
		between measurements by the two methods will lie\citep{BA99}.
	\end{quote}
	\citet{BXC} offers an alternative, more specific,  definition of
	the limits of agreement \emph{"a prediction interval for the
		difference between future measurements with the two methods on a
		new individual."} \citet{luiz} describes them as tolerance limits.
	
	Importantly they have the same construction as Shewhart Control
	limits.
	
	
	