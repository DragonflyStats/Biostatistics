	\section{Intervals}
	
	\subsection{Purpose of Limits of Agreement} It must be established
	clearly the specific purpose of the limits of agreement.
	\citet*{BA95} comment that the limits of agreement \emph{how far
		apart measurements by the two methods were likely to be for most
		individuals.}, a definition echoed in their 1999 paper:
	\begin{quote} We can then say that nearly all pairs
		of measurements by the two methods will be closer together than
		these extreme values, which we call 95\% limits of agreement.
		These values define the range within which most differences
		between measurements by the two methods will lie\citep{BA99}.
	\end{quote}
	\citet{BXC} offers an alternative, more specific,  definition of
	the limits of agreement \emph{"a prediction interval for the
		difference between future measurements with the two methods on a
		new individual."} \citet{luiz} describes them as tolerance limits.
	
	Importantly they have the same construction as Shewhart Control
	limits.
	
	
	\subsection*{What are Tolerance Intervals?}
	A tolerance interval is a statistical interval within which, with some confidence level, a specified proportion of a population falls.
	The Engineering Statistics Handbook describes the difference: Confidence limits are limits within which we expect a given population parameter, such as the mean, to lie. Statistical tolerance limits are limits within which we expect a stated proportion of the population to lie.
	
	It is useful to make the distinction between tolerance intervals and confidence intervals clear. The confidence interval describes a single-valued population parameter, commonly the mean, with a specified confidence level. The tolerance interval, on the other hand, describes the range of data values that includes a specific proportion of the population.
	
	As discussed in Vardeman (1992), the tolerance interval is not as widely used as the confidence interval and prediction interval, largely because of the emphasis placed on these in undergraduate teaching. Furthermore, Vardeman(1992) argues this lack of awareness can lead to misuse of confidence intervals where other types of intervals are more appropriate.
	Curiously Carstensen et al (2008) describe the Limits of agreement as a prediction interval, although stating that it is formulated in correctly for that purpose.
	
	\subsection*{Why Tolerance Intervals are appropriate?}
	It is clear from the definition of Tolerance intervals that they function precisely as Bland-Altman intend.
	Total Deviation Index and Coverage Probability
	
	The Coverage Probability describes the proportion captured within a pre-specified boundary of the absolute paired-measurement differences from two methods of measurement, i.e., the value of k such that P(|D| <k) = pk.
	