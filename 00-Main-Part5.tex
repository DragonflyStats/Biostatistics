\documentclass[12pt, a4paper]{article}
\usepackage{natbib}
\usepackage{vmargin}
\usepackage{graphicx}
\usepackage{epsfig}
\usepackage{subfigure}
%\usepackage{amscd}
\usepackage{amssymb}
\usepackage{subfiles}
\usepackage{subfigure}
\usepackage{framed}
\usepackage{subfiles}
\usepackage{amsbsy}
\usepackage{amsthm, amsmath}
%\usepackage[dvips]{graphicx}
\bibliographystyle{chicago}
\renewcommand{\baselinestretch}{1.1}

% left top textwidth textheight headheight % headsep footheight footskip
\setmargins{3.0cm}{2.5cm}{15.5 cm}{23.5cm}{0.25cm}{0cm}{0.5cm}{0.5cm}

\pagenumbering{arabic}
%-------------------------------------------------------------------Simplifying GLS by KH -%


\begin{document}

% \chapter{Augmented GLMs}



\chapter{General Linear model}
\section{General Linear model} Mixed Effects Models are seen as
especially robust in the analysis of unbalanced data when compared
to similar analyses done under the General Linear Model framework
(Pinheiro and Bates, 2000).

A Mixed Effects Model is an extension of the General Linear Model
that can specify additional random effects terms




\subsection{Equivalence of LME model}
Henderson's mixed model equations are presented on page 147 of
Youngjo et al. Youngjo et al demonstrate that this formulation is
equivalent to an augmented general linear model.

Youngjo et al show that the linear mixed effects model can be
shown to be the augmented classical linear model involving fixed
effects parameters only.
\section{The LME model as a general linear model}
Henderson's equations in (\ref{Henderson:Equations}) can be rewritten $( T^\prime W^{-1} T ) \delta = T^\prime W^{-1} y_{a} $ using
\[
\delta = \left(\begin{array}{c}\beta \\ b \end{array}\right),
\ y_{a} = \left(\begin{array}{c}
y \cr \psi
\end{array}\right),
\ T = \left(\begin{array}{cc}
X & Z  \\
0 & I
\end{array}\right),
\ \textrm{and} \ W = \left(\begin{array}{cc}
\Sigma & 0  \cr
0 &  D \end{array}\right),
\]
where \cite{Lee:Neld:Pawi:2006} describe $\psi = 0$ as quasi-data with mean $\mathrm{E}(\psi) = b.$ Their formulation suggests that the joint estimation of the coefficients $\beta$ and $b$ of the linear mixed effects model can be derived via a classical augmented general linear model $y_{a} = T\delta + \varepsilon$ where $\mathrm{E}(\varepsilon) = 0$ and $\mathrm{var}(\varepsilon) = W,$ with \emph{both} $\beta$ and $b$ appearing as fixed parameters. The usefulness of this reformulation of an LME as a general linear model will be revisited.


\section{The LME model as a general linear model} %3.2
Henderson's equations in %(\ref{Henderson:Equations})
can be rewritten $( T^\prime W^{-1} T ) \delta = T^\prime W^{-1} y_{a} $ using
\begin{verbatim}
\[
\delta = \pmatrix{\beta \cr b},
\ y_{a} = \pmatrix{
y \cr \psi
},
\ T = \pmatrix{
X & Z  \cr
0 & I
},
\ \textrm{and} \ W = \pmatrix{
\Sigma & 0  \cr
0 &  D },
\]
\end{verbatim}
where \textbf{cite[Lee:Neld:Pawi:2006]} describe $\psi = 0$ as quasi-data with mean $\mathrm{E}(\psi) = b.$ Their formulation suggests that the joint estimation of the coefficients $\beta$ and $b$ of the linear mixed effects model can be derived via a classical augmented general linear model $y_{a} = T\delta + \varepsilon$ where $\mathrm{E}(\varepsilon) = 0$ and $\mathrm{var}(\varepsilon) = W,$ with \emph{both} $\beta$ and $b$ appearing as fixed parameters. The usefulness of this reformulation of an LME as a general linear model will be revisited.





%---------------------------------------------------------------------------%
% - 3. Augmented GLMS
%---------------------------------------------------------------------------%



\section{Augmented GLMs}
%Augmented General linear models.
% Youngjo et al page 154
\subsection{Augmented linear model}
The subscript $M$ is a label referring to the mean model.
\begin{equation}
\left(%
\begin{array}{c}
Y \\
\psi_{M} \\
\end{array}%
\right) = \left(
\begin{array}{cc}
% after \\: \hline or \cline{col1-col2} \cline{col3-col4} ...
X & Z \\
0 & I \\
\end{array}\right) \left(%
\begin{array}{c}
\beta \\
\nu \\
\end{array}%
\right)+ e^{*}
\end{equation}




The error term $e^{*}$ is normal with mean zero. The variance
matrix of the error term is given by
\begin{equation}
\Sigma_{a} = \left(%
\begin{array}{cc}
\Sigma & 0 \\
0 & D \\
\end{array}%
\right).
\end{equation}

\begin{equation}
X = \left(%
\begin{array}{cc}
T & Z \\
0 & I \\
\end{array}%
\right)
\delta = \left(%
\begin{array}{c}
\beta  \\
\nu  \\
\end{array}%
\right)
\end{equation}



\begin{equation}
y_{a} = T \delta + e^{*}
\end{equation}

Weighted least squares equation


% Youngjo et al page 154
\newpage


Generalized linear models are a generalization of classical linear  models.

\section{Simplifying GLS (K Hayes)}

\subsection{Introduction}

Hayes and Haslett (1998) present an approach to the problem of \textbf{general least squares} estimation of the general linear model in terms of constrained optimization, which is in turn solved via Lagrange multipliers. The crux of the proposed approach is that one system of equations is sufficiently versatile, and provides for \begin{itemize} \item the estimation of new observations, \item estimation of fixed parameters in regression \item estimation of fixed and random effects in mixed models,\item the diagnostics associated with conditional and marginal residuals \item and of subset deletion. \end{itemize}

\subsection{Overview}
Hayes and Haslett (1998) have demonstrated how the problem of best linear unbiased estimation can be posed in terms of Lagrange multipliers. Both BLUE and BLUP can be treated as distinct estimation problems from the following equation.

\begin{equation}
\left(  \begin{array}{cc} V & X \\    X^t & 0 \\
\end{array}\right)\left(  \begin{array}{c}    \boldsymbol{\lambda}_{z}\\   \boldsymbol{\gamma}_z \\  \end{array}
\right)=\left(  \begin{array}{c}    \mbox{cov}(Y,Z)\\   A^{t} \\  \end{array}\right)\end{equation}


Hence BLUE and BLUP can be considered as the estimation of two different variables from $Y$. This equation has a natural role in the derivation of \emph{leave- k-out} residuals and diagnostic measures, and replaces the traditional approach of using a variety of clumsy updating formulas. Note that this approach may be used to determine the impact of deletion on any quantity computed from $Y$.




\section{Generalized Least Squares}


generalized least squares (GLS) is a technique for estimating the unknown parameters in a linear regression model. 
The GLS is applied when the variances of the observations are unequal (heteroscedasticity), or when there is a certain degree of correlation between the observations. 
In these cases ordinary least squares can be statistically inefficient, or even give misleading inferences.



\[ Y = X\beta + \varepsilon, \qquad \mathrm{E}[\varepsilon|X]=0,\ \operatorname{Var}[\varepsilon|X]=\Omega.\]



\subsection{Introduction to Generalized Least Squares}
\begin{equation}
\boldsymbol{y}_i = \boldsymbol{X}_i\boldsymbol{\beta} + \boldsymbol{\epsilon}_i
\end{equation}



Estimation under this model has been studied extensively in the linear regression model.

%-------------------------------------------------------------------Simplifying GLS by KH -%




\section{Hierarchical likelihood} %3.3
Inferential method was developed for the mixed linear model via Lee and Nelder's (1996) hierarchical-likelihood (h-likelihood).

\section{Importance-Weighted Least-Squares (IWLS)}  %3.4




% Generalized linear models are a generalization of classical linear  models.

\section{Augmented GLMs} %3.1

With the use of h-likihood, a random effected model of the form can be viewed as an `augmented GLM' with the response varaibkes $(y^t, \phi^t_m)^t$, (with $\mu = E(y)$,$ u = E(\phi)$, $var(y) = \theta V (\mu)$.
The augmented linear predictor is \[\eta_{ma}  = (\eta^t, \eta^t_m)^t) = T\omega. \].

%---------------------------------------------------------------------------%

%Augmented Generalized linear models.
% Youngjo et al page 154

The subscript $M$ is a label referring to the mean model.
\begin{equation}
\left(%
\begin{array}{c}
  Y \\
  \psi_{M} \\
\end{array}%
\right) = \left(
\begin{array}{cc}
  X & Z \\
  0 & I \\
\end{array}\right) \left(%
\begin{array}{c}
  \beta \\
  \nu \\
\end{array}%
\right)+ e^{*}
\end{equation}


%Augmented Generalized linear models.


The error term $e^{*}$ is normal with mean zero. The variance matrix of the error term is given by
\begin{equation}
\Sigma_{a} = \left(%
\begin{array}{cc}
  \Sigma & 0 \\
  0 & D \\
\end{array}%
\right).
\end{equation}
\begin{equation}
y_{a} = T \delta + e^{*}
\end{equation}

Weighted least squares equation


% Youngjo et al page 154



\section{Augmented linear model}
The subscript $M$ is a label referring to the mean model.
\begin{equation}
\left(%
\begin{array}{c}
Y \\
\psi_{M} \\
\end{array}%
\right) = \left(
\begin{array}{cc}
% after \\: \hline or \cline{col1-col2} \cline{col3-col4} ...
X & Z \\
0 & I \\
\end{array}\right) \left(%
\begin{array}{c}
\beta \\
\nu \\
\end{array}%
\right)+ e^{*}
\end{equation}




The error term $e^{*}$ is normal with mean zero. The variance
matrix of the error term is given by
\begin{equation}
\Sigma_{a} = \left(%
\begin{array}{cc}
\Sigma & 0 \\
0 & D \\
\end{array}%
\right).
\end{equation}

\begin{equation}
X = \left(%
\begin{array}{cc}
T & Z \\
0 & I \\
\end{array}%
\right)
\delta = \left(%
\begin{array}{c}
\beta  \\
\nu  \\
\end{array}%
\right)
\end{equation}



\begin{equation}
y_{a} = T \delta + e^{*}
\end{equation}

Weighted least squares equation


% Youngjo et al page 154
%=======================================================================================================================%

\section{Augmented GLMs} %3.1


With the use of h-likelihood, a random effected model of the form can be viewed as an `augmented GLM' with the response variables $(y^t, \phi^t_m)^t$, (with $\mu = E(y)$,$ u = E(\phi)$, $var(y) = \theta V (\mu)$.
The augmented linear predictor is \[\eta_{ma}  = (\eta^t, \eta^t_m)^t) = T\omega. \].


%Augmented Generalized linear models.
% Youngjo et al page 154


The subscript $M$ is a label referring to the mean model.
\begin{equation}
\left(%
\begin{array}{c}
  Y \\
  \psi_{M} \\
\end{array}%
\right) = \left(
\begin{array}{cc}
  X & Z \\
  0 & I \\
\end{array}\right) \left(%
\begin{array}{c}
  \beta \\
  \nu \\
\end{array}%
\right)+ e^{*}
\end{equation}




%Augmented Generalized linear models.




The error term $e^{*}$ is normal with mean zero. The variance matrix of the error term is given by
\begin{equation}
\Sigma_{a} = \left(%
\begin{array}{cc}
  \Sigma & 0 \\
  0 & D \\
\end{array}%
\right).
\end{equation}


$y_{a} = T \delta + e^{*}$


Weighted least squares equation




% Youngjo et al page 154




Generalized linear models are a generalization of classical linear  models.

\section{The Augmented Model Matrix}  %3.1.2
\begin{equation}
X = \left(%
\begin{array}{cc}
  T & Z \\
  0 & I \\
\end{array}%
\right)
\delta = \left(%
\begin{array}{c}
  \beta  \\
  \nu  \\
\end{array}%
\right)
\end{equation}





\section{Augmented GLMs} %3.1

With the use of h-likihood, a random effected model of the form can be viewed as an `augmented GLM' with the response varaibkes $(y^t, \phi^t_m)^t$, (with $\mu = E(y)$,$ u = E(\phi)$, $var(y) = \theta V (\mu)$.
The augmented linear predictor is \[\eta_{ma}  = (\eta^t, \eta^t_m)^t) = T\omega. \].



%Augmented Generalized linear models.
% Youngjo et al page 154

The subscript $M$ is a label referring to the mean model.
\begin{equation}
\left(%
\begin{array}{c}
Y \\
\psi_{M} \\
\end{array}%
\right) = \left(
\begin{array}{cc}
X & Z \\
0 & I \\
\end{array}\right) \left(%
\begin{array}{c}
\beta \\
\nu \\
\end{array}%
\right)+ e^{*}
\end{equation}


%Augmented Generalized linear models.


The error term $e^{*}$ is normal with mean zero. The variance matrix of the error term is given by
\begin{equation}
\Sigma_{a} = \left(%
\begin{array}{cc}
\Sigma & 0 \\
0 & D \\
\end{array}%
\right).
\end{equation}


%y_{a} = T \delta + e^{*}
%\end{equation}

Weighted least squares equation






%\chapter{Application to Method Comparison Studies} % Chapter 4


%---------------------------------------------------------------------------%
% - 1. Application to MCS
% - 2. Grubbs' Data
% - 3. R implementation
% - 4. Influence measures using R


\section{Grubbs' Data} %4.2

For the Grubbs data the $\hat{\beta}$ estimated are
$\hat{\beta}_{0}$ and $\hat{\beta}_{1}$ respectively. Leaving the
fourth case out, i.e. $k=4$ the corresponding estimates are
$\hat{\beta}_{0}^{-4}$ and $\hat{\beta}_{1}^{-4}$


\begin{equation}
Y^{-Q} = \hat{\beta}^{-Q}X^{-Q}
\end{equation}

When considering the regression of case-wise differences and averages, we write $D^{-Q} = \hat{\beta}^{-Q}A^{-Q}$


\begin{table}[ht]
	\begin{center}
		\begin{tabular}{rrrrr}
			\hline
			& F & C & D & A \\
			\hline
			1 & 793.80 & 794.60 & -0.80 & 794.20 \\
			2 & 793.10 & 793.90 & -0.80 & 793.50 \\
			3 & 792.40 & 793.20 & -0.80 & 792.80 \\
			4 & 794.00 & 794.00 & 0.00 & 794.00 \\
			5 & 791.40 & 792.20 & -0.80 & 791.80 \\
			6 & 792.40 & 793.10 & -0.70 & 792.75 \\
			7 & 791.70 & 792.40 & -0.70 & 792.05 \\
			8 & 792.30 & 792.80 & -0.50 & 792.55 \\
			9 & 789.60 & 790.20 & -0.60 & 789.90 \\
			10 & 794.40 & 795.00 & -0.60 & 794.70 \\
			11 & 790.90 & 791.60 & -0.70 & 791.25 \\
			12 & 793.50 & 793.80 & -0.30 & 793.65 \\
			\hline
		\end{tabular}
	\end{center}
\end{table}

%===================================%
\begin{equation}
Y^{(k)} = \hat{\beta}^{(k)}X^{(k)}
\end{equation}

Consider two sets of measurements , in this case F and C , with the vectors of case-wise averages $A$ and case-wise differences $D$ respectively. A regression model of differences on averages can be fitted with the view to exploring some characteristics of the data.

When considering the regression of case-wise differences and averages, we write

\begin{equation}
D^{-Q} = \hat{\beta}^{-Q}A^{-Q}
\end{equation}
Let $\hat{\beta}$ denote the least square estimate of $\beta$ based upon the full set of observations, and let $\hat{\beta}^{(k)}$ denoted the estimate with the $k^{th}$ case excluded.

For the Grubbs data the $\hat{\beta}$ estimated are $\hat{\beta}_{0}$ and $\hat{\beta}_{1}$ respectively. Leaving the
fourth case out, i.e. $k=4$ the corresponding estimates are $\hat{\beta}_{0}^{-4}$ and $\hat{\beta}_{1}^{-4}$

\begin{equation}
Y^{(k)} = \hat{\beta}^{(k)}X^{(k)}
\end{equation}

Consider two sets of measurements , in this case F and C , with the vectors of case-wise averages $A$ and case-wise differences $D$ respectively. A regression model of differences on averages can be fitted with the view to exploring some characteristics of the data.

\begin{verbatim}
Call: lm(formula = D ~ A)

Coefficients: (Intercept)            A
-37.51896      0.04656

\end{verbatim}




When considering the regression of case-wise differences and averages, we write

\begin{equation}
D^{-Q} = \hat{\beta}^{-Q}A^{-Q}
\end{equation}

%===============================================================================================%

\section{Influence measures using R} %4.4
\texttt{R} provides the following influence measures of each observation.

%Influence measures: This suite of functions can be used to compute
%some of the regression (leave-one-out deletion) diagnostics for
%linear and generalized linear models discussed in Belsley, Kuh and
% Welsch (1980), Cook and Weisberg (1982)



\begin{table}[ht]
	\begin{center}
		\begin{tabular}{|c|c|c|c|c|c|c|}
			\hline
			& dfb.1\_ & dfb.A & dffit & cov.r & cook.d & hat \\
			\hline
			1 & 0.42 & -0.42 & -0.56 & 1.13 & 0.15 & 0.18 \\
			2 & 0.17 & -0.17 & -0.34 & 1.14 & 0.06 & 0.11 \\
			3 & 0.01 & -0.01 & -0.24 & 1.17 & 0.03 & 0.08 \\
			4 & -1.08 & 1.08 & 1.57 & 0.24 & 0.56 & 0.16 \\
			5 & -0.14 & 0.14 & -0.24 & 1.30 & 0.03 & 0.13 \\
			6 & -0.00 & 0.00 & -0.11 & 1.31 & 0.01 & 0.08 \\
			7 & -0.04 & 0.04 & -0.08 & 1.37 & 0.00 & 0.11 \\
			8 & 0.02 & -0.02 & 0.15 & 1.28 & 0.01 & 0.09 \\
			9 & 0.69 & -0.68 & 0.75 & 2.08 & 0.29 & 0.48 \\
			10 & 0.18 & -0.18 & -0.22 & 1.63 & 0.03 & 0.27 \\
			11 & -0.03 & 0.03 & -0.04 & 1.53 & 0.00 & 0.19 \\
			12 & -0.25 & 0.25 & 0.44 & 1.05 & 0.09 & 0.12 \\
			\hline
		\end{tabular}
	\end{center}
\end{table}




\section{Augmented GLMs} 


%---------------------------------------------------------------------------%
% - 3. Augmented GLMS
%---------------------------------------------------------------------------%


Generalized linear models are a generalization of classical linear  models.

With the use of h-likihood, a random effected model of the form can be viewed as an `augmented GLM' with the response varaibkes $(y^t, \phi^t_m)^t$, (with $\mu = E(y)$,$ u = E(\phi)$, $var(y) = \theta V (\mu)$.
The augmented linear predictor is \[\eta_{ma}  = (\eta^t, \eta^t_m)^t) = T\omega. \].



%Augmented Generalized linear models.
% Youngjo et al page 154

The subscript $M$ is a label referring to the mean model.
\begin{equation}
\left(%
\begin{array}{c}
Y \\
\psi_{M} \\
\end{array}%
\right) = \left(
\begin{array}{cc}
X & Z \\
0 & I \\
\end{array}\right) \left(%
\begin{array}{c}
\beta \\
\nu \\
\end{array}%
\right)+ e^{*}
\end{equation}


%Augmented Generalized linear models.


The error term $e^{*}$ is normal with mean zero. The variance matrix of the error term is given by
\begin{equation}
\Sigma_{a} = \left(%
\begin{array}{cc}
\Sigma & 0 \\
0 & D \\
\end{array}%
\right).
\end{equation}


%y_{a} = T \delta + e^{
%\end{equation}

Weighted least squares equation


% Youngjo et al page 154




%=================================================================================%


\section{Grubbs' Data} %4.2

For the Grubbs data the $\hat{\beta}$ estimated are
$\hat{\beta}_{0}$ and $\hat{\beta}_{1}$ respectively. Leaving the
fourth case out, i.e. $k=4$ the corresponding estimates are
$\hat{\beta}_{0}^{-4}$ and $\hat{\beta}_{1}^{-4}$


\begin{equation}
Y^{-Q} = \hat{\beta}^{-Q}X^{-Q}
\end{equation}

When considering the regression of case-wise differences and averages, we write $D^{-Q} = \hat{\beta}^{-Q}A^{-Q}$


\begin{table}[ht]
	\begin{center}
		\begin{tabular}{rrrrr}
			\hline
			& F & C & D & A \\
			\hline
			1 & 793.80 & 794.60 & -0.80 & 794.20 \\
			2 & 793.10 & 793.90 & -0.80 & 793.50 \\
			3 & 792.40 & 793.20 & -0.80 & 792.80 \\
			4 & 794.00 & 794.00 & 0.00 & 794.00 \\
			5 & 791.40 & 792.20 & -0.80 & 791.80 \\
			6 & 792.40 & 793.10 & -0.70 & 792.75 \\
			7 & 791.70 & 792.40 & -0.70 & 792.05 \\
			8 & 792.30 & 792.80 & -0.50 & 792.55 \\
			9 & 789.60 & 790.20 & -0.60 & 789.90 \\
			10 & 794.40 & 795.00 & -0.60 & 794.70 \\
			11 & 790.90 & 791.60 & -0.70 & 791.25 \\
			12 & 793.50 & 793.80 & -0.30 & 793.65 \\
			\hline
		\end{tabular}
	\end{center}
\end{table}


%=====================================================================================================%

\begin{equation}
Y^{(k)} = \hat{\beta}^{(k)}X^{(k)}
\end{equation}

Consider two sets of measurements , in this case F and C , with the vectors of case-wise averages $A$ and case-wise differences $D$ respectively. A regression model of differences on averages can be fitted with the view to exploring some characteristics of the data.

When considering the regression of case-wise differences and averages, we write

\begin{equation}
D^{-Q} = \hat{\beta}^{-Q}A^{-Q}
\end{equation}
Let $\hat{\beta}$ denote the least square estimate of $\beta$ based upon the full set of observations, and let $\hat{\beta}^{(k)}$ denoted the estimate with the $k^{th}$ case excluded.

For the Grubbs data the $\hat{\beta}$ estimated are $\hat{\beta}_{0}$ and $\hat{\beta}_{1}$ respectively. Leaving the
fourth case out, i.e. $k=4$ the corresponding estimates are $\hat{\beta}_{0}^{-4}$ and $\hat{\beta}_{1}^{-4}$

\begin{equation}
Y^{(k)} = \hat{\beta}^{(k)}X^{(k)}
\end{equation}

Consider two sets of measurements , in this case F and C , with the vectors of case-wise averages $A$ and case-wise differences $D$ respectively. A regression model of differences on averages can be fitted with the view to exploring some characteristics of the data.

\begin{verbatim}
Call: lm(formula = D ~ A)

Coefficients: (Intercept)            A
-37.51896      0.04656

\end{verbatim}




When considering the regression of case-wise differences and averages, we write

\begin{equation}
D^{-Q} = \hat{\beta}^{-Q}A^{-Q}
\end{equation}


\section{Application to MCS} %4.1
	
Let $\hat{\beta}$ denote the least square estimate of $\beta$ based upon the full set of observations, and let
$\hat{\beta}^{(k)}$ denoted the estimate with the $k^{th}$ case	excluded.

\end{document}
%-----------------------------------------------------------------------------------%
