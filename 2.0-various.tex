\documentclass[Main.tex]{subfiles}
\begin{document}
	%------------------------------------------------------------%

\begin{equation}
r_{mi}=x^{T}_{i}\hat{\beta}
\end{equation}

\subsection{Marginal Residuals}
\begin{eqnarray}
\hat{\beta} &=& (X^{T}R^{-1}X)^{-1}X^{T}R^{-1}Y \nonumber \\
&=& BY \nonumber
\end{eqnarray}

\newpage
\section{Covariance Parameters} %1.5
The unknown variance elements are referred to as the covariance parameters and collected in the vector $\theta$.
% - where is this coming from?
% - where is it used again?
% - Has this got anything to do with CovTrace etc?
%---------------------------------------------------------------------------%

\subsection{Methods and Measures}
The key to making deletion diagnostics useable is the development of efficient computational formulas, allowing one to obtain the \index{case deletion diagnostics} case deletion diagnostics by making use of basic building blocks, computed only once for the full model.

\citet{Zewotir} lists several established methods of analyzing influence in LME models. These methods include \begin{itemize}
\item Cook's distance for LME models,
\item \index{likelihood distance} likelihood distance,
\item the variance (information) ration,
\item the \index{Cook-Weisberg statistic} Cook-Weisberg statistic,
\item the \index{Andrews-Prebigon statistic} Andrews-Prebigon statistic.
\end{itemize}


%---------------------------------------------------------------------------%
\newpage








\subsection{Influence measures using R} %4.4
\texttt{R} provides the following influence measures of each observation.


%Influence measures: This suite of functions can be used to compute
%some of the regression (leave-one-out deletion) diagnostics for
%linear and generalized linear models discussed in Belsley, Kuh and
% Welsch (1980), Cook and Weisberg (1982)






\begin{table}[ht]
	\begin{center}
		\begin{tabular}{|c|c|c|c|c|c|c|}
			\hline
			& dfb.1\_ & dfb.A & dffit & cov.r & cook.d & hat \\
			\hline
			1 & 0.42 & -0.42 & -0.56 & 1.13 & 0.15 & 0.18 \\
			2 & 0.17 & -0.17 & -0.34 & 1.14 & 0.06 & 0.11 \\
			3 & 0.01 & -0.01 & -0.24 & 1.17 & 0.03 & 0.08 \\
			4 & -1.08 & 1.08 & 1.57 & 0.24 & 0.56 & 0.16 \\
			5 & -0.14 & 0.14 & -0.24 & 1.30 & 0.03 & 0.13 \\
			6 & -0.00 & 0.00 & -0.11 & 1.31 & 0.01 & 0.08 \\
			7 & -0.04 & 0.04 & -0.08 & 1.37 & 0.00 & 0.11 \\
			8 & 0.02 & -0.02 & 0.15 & 1.28 & 0.01 & 0.09 \\
			9 & 0.69 & -0.68 & 0.75 & 2.08 & 0.29 & 0.48 \\
			10 & 0.18 & -0.18 & -0.22 & 1.63 & 0.03 & 0.27 \\
			11 & -0.03 & 0.03 & -0.04 & 1.53 & 0.00 & 0.19 \\
			12 & -0.25 & 0.25 & 0.44 & 1.05 & 0.09 & 0.12 \\
			\hline
		\end{tabular}
	\end{center}
\end{table}


\end{document}
