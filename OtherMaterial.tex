
%-------------------------------------------------------------------------------%
\section{ICC, Reproducibility Index and Passing-Bablok }


\subsection{Intraclass Correlation Coefficient} This measure of agreement is estimated using variance components from appropriate analysis of variance models. Measures of agreement are variance dependent, and so the ICC can be misleading. The ICC takes a value between $0$ and $1$, and is based on Analysis of Variance
methodologies.
\\
The ICC is a measure of reliability.
\\
\\\citet{bartko} considers the ICC as just another measure of agreement.

%-------------------------------------------------------------------------------%

\subsection{Passing and Bablok ( 1983) }
Passing \& Bablok have described a linear regression model that
are without the usual assumptions regarding the distribution of
the samples and the measurement errors. The result does not depend
on the assignment of the methods (or instruments) to X and Y. The
slope and intercept  are calculated with their 95\% confidence
interval.Hypothesis tests on the slope and intercept maybe then
carried out.
\\
If the hypothesis of the intercept is rejected, then it is
concluded that it is significant different from $0$ and both
raters differ at least by a constant amount.
\\
If the hypothesis of the slope is rejected, then it is concluded
that the slope is significant different from $1$ and there is at
least a proportional difference between the two raters.

\subsection{Lin's Reproducibility Index} Lin proposes the use of a reproducibility index, called the Concordance Correlation Coefficent (CCC).While it is not strictly a measure of agreement as such, we have included it.
%-------------------------------------------------------------------------------%






\addcontentsline{toc}{section}{Bibliography}
