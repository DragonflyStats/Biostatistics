\section{Quiroz Burdick}
%-------------------------------------------------------------------------%
Assessmet of individual agreements with repeated measurements based on Generalized Confidence intervals.

%KEYWORDS

Bootstrap confidence intervals.
Coverage probability (CP)
Equivalence Studies
Individual agreements
Generalized Confidence intervals (GCI)
Total deviation index (TDI)
Variance components

%--------------------------------------------------------------------------%
%SUMMARY
Proposing an equivalence test for assessing individual agreement based on TDI and CP.
The bounds used in the tests are constucted using a bootstrap approach and generalized confidence intervals (GCI).

%--------------------------------------------------------------------------%
%PAGE 345
%Exchangability

Equivalence testing is an approach commonly used to determine the acceptability of a new method 
against a reference method.

%--------------------------------------------------------------------------%
%PAGE 346 TDI and CP

Both the TDI and CP are attractive criteria as they are easy to interpret.

Bootstrap approach was later applied to mixed models with repeated measurements by Choudhary (2007)

%--------------------------------------------------------------------------%
%PAGE 346 - EXPERIMENTAL DESIGN
$T$ for test measurement, $R$ for reference measurement

%--------------------------------------------------------------------------%
%PAGE 347

$\otimes$ is the Kroneckor Product operator.

\[  \Sigma_{MS} =  \left[     \begin{array}{cc} \sigma^2_{TS} & 0 \\ 0 & \sigma^2_{RS} \end{array}\right] \]

%--------------------------------------------------------------------------%
% PAPERS
\section{Quiroz Burdick}
%-------------------------------------------------------------------------%
Assessmet of individual agreements with repeated measurements based on Generalized Confidence intervals.

%KEYWORDS

Bootstrap confidence intervals.
Coverage probability (CP)
Equivalence Studies
Individual agreements
Generalized Confidence intervals (GCI)
Total deviation index (TDI)
Variance components

%--------------------------------------------------------------------------%
%SUMMARY
Proposing an equivalence test for assessing individual agreement based on TDI and CP.
The bounds used in the tests are constucted using a bootstrap approach and generalized confidence intervals (GCI).

%--------------------------------------------------------------------------%
%PAGE 345
%Exchangability

Equivalence testing is an approach commonly used to determine the acceptability of a new method 
against a reference method.

%--------------------------------------------------------------------------%
%PAGE 346 TDI and CP

Both the TDI and CP are attractive criteria as they are easy to interpret.

Bootstrap approach was later applied to mixed models with repeated measurements by Choudhary (2007)

%--------------------------------------------------------------------------%
%PAGE 346 - EXPERIMENTAL DESIGN
$T$ for test measurement, $R$ for reference measurement

%--------------------------------------------------------------------------%
%PAGE 347

$\otimes$ is the Kroneckor Product operator.

\[  \Sigma_{MS} =  \left[     \begin{array}{cc} \sigma^2_{TS} & 0 \\ 0 & \sigma^2_{RS} \end{array}\right] \]

%--------------------------------------------------------------------------%
% PAPERS


\section{Quiroz-Burdick PEFR Example}

The data consist of two paired measurements on the same subject made with the large Wright peak flow meter and a mini
Wright meter.

Paired differences of less than 101/min are considered of no practical clinical significance. That is to say, it would have no bearing on
any decision related to a clinical matter.

A serious error would be declare that the mini-meter is as effective as the large meter when in fact it is not.

%---------------------------------------------%
% PAGE 355

\[ H_0 : \kappa_{0.90} \geq 10 \]
\[ H_A : \kappa_{0.90} < 10 \]


%Paper on GCIs - http://www.jstor.org/discover/10.2307/2290779?uid=3738232&uid=2&uid=4&sid=21103546837943

\newpage

Assessing equivalence of two assays using sensitivity and specificity.
Quiroz J1, Burdick RK.

% http://www.ncbi.nlm.nih.gov/pubmed/17479392
%----------------------------------------------------------------------------%

The equivalence of two assays is determined using the sensitivity and specificity relative to a gold standard.
The equivalence-testing criterion is based on a misclassification rate proposed by Burdick et al. (2005) and
the intersection-union test (IUT) method proposed by Berger (1982). 

Using a variance components model and IUT methods, we construct bounds for the sensitivity and specificity 
relative to the gold standard assay based on generalized confidence intervals. We conduct a simulation study 
to assess whether the bounds maintain the stated test size. 

We present a computational example to demonstrate the method described in the paper.

%----------------------------------------------------------------------------%
Assessing equivalence of two assays using sensitivity and specificity.
Quiroz J1, Burdick RK.

% http://www.ncbi.nlm.nih.gov/pubmed/17479392
%----------------------------------------------------------------------------%

The equivalence of two assays is determined using the sensitivity and specificity relative to a gold standard.
The equivalence-testing criterion is based on a misclassification rate proposed by Burdick et al. (2005) and
the intersection-union test (IUT) method proposed by Berger (1982). 

Using a variance components model and IUT methods, we construct bounds for the sensitivity and specificity 
relative to the gold standard assay based on generalized confidence intervals. We conduct a simulation study 
to assess whether the bounds maintain the stated test size. 

We present a computational example to demonstrate the method described in the paper.

%----------------------------------------------------------------------------%


\section{Quiroz-Burdick PEFR Example}

The data consist of two paired measurements on the same subject made with the large Wright peak flow meter and a mini
Wright meter.

Paired differences of less than 101/min are considered of no practical clinical significance. That is to say, it would have no bearing on
any decision related to a clinical matter.

A serious error would be declare that the mini-meter is as effective as the large meter when in fact it is not.

%---------------------------------------------%
% PAGE 355

\[ H_0 : \kappa_{0.90} \geq 10 \]
\[ H_A : \kappa_{0.90} < 10 \]


%Paper on GCIs - http://www.jstor.org/discover/10.2307/2290779?uid=3738232&uid=2&uid=4&sid=21103546837943

\newpage

Assessing equivalence of two assays using sensitivity and specificity.
Quiroz J1, Burdick RK.

% http://www.ncbi.nlm.nih.gov/pubmed/17479392
%----------------------------------------------------------------------------%

The equivalence of two assays is determined using the sensitivity and specificity relative to a gold standard.
The equivalence-testing criterion is based on a misclassification rate proposed by Burdick et al. (2005) and
the intersection-union test (IUT) method proposed by Berger (1982). 

Using a variance components model and IUT methods, we construct bounds for the sensitivity and specificity 
relative to the gold standard assay based on generalized confidence intervals. We conduct a simulation study 
to assess whether the bounds maintain the stated test size. 

We present a computational example to demonstrate the method described in the paper.

%----------------------------------------------------------------------------%
Assessing equivalence of two assays using sensitivity and specificity.
Quiroz J1, Burdick RK.

% http://www.ncbi.nlm.nih.gov/pubmed/17479392
%----------------------------------------------------------------------------%

The equivalence of two assays is determined using the sensitivity and specificity relative to a gold standard.
The equivalence-testing criterion is based on a misclassification rate proposed by Burdick et al. (2005) and
the intersection-union test (IUT) method proposed by Berger (1982). 

Using a variance components model and IUT methods, we construct bounds for the sensitivity and specificity 
relative to the gold standard assay based on generalized confidence intervals. We conduct a simulation study 
to assess whether the bounds maintain the stated test size. 

We present a computational example to demonstrate the method described in the paper.

%----------------------------------------------------------------------------%

%---------------------------------------------%



\chapter{Quiroz J1, Burdick RK.}
Assessing equivalence of two assays using sensitivity and specificity.
Quiroz J1, Burdick RK.

% http://www.ncbi.nlm.nih.gov/pubmed/17479392
%----------------------------------------------------------------------------%

The equivalence of two assays is determined using the sensitivity and specificity relative to a gold standard.
The equivalence-testing criterion is based on a misclassification rate proposed by Burdick et al. (2005) and
the intersection-union test (IUT) method proposed by Berger (1982). 

Using a variance components model and IUT methods, we construct bounds for the sensitivity and specificity 
relative to the gold standard assay based on generalized confidence intervals. We conduct a simulation study 
to assess whether the bounds maintain the stated test size. 

We present a computational example to demonstrate the method described in the paper.
