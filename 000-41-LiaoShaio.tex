
\section{Lai Shiao}
\citet{LaiShiao} use mixed models to determine the factors that
affect the difference of two methods of measurement using the
conventional formulation of linear mixed effects models.

If the parameter \textbf{b}, and the variance components are not
significantly different from zero, the conclusion that there is no
inter-method bias can be drawn. If the fixed effects component
contains only the intercept, and a simple correlation coefficient
is used, then the estimate of the intercept in the model is the
inter-method bias. Conversely the estimates for the fixed effects
factors can advise the respective influences each factor has on
the differences. The Proc Mixed package allows users to specify
different correlation structures of the variance components
\textbf{G} and \textbf{R}.


Oxygen saturation is one of the most frequently measured variables
in clinical nursing studies. `Fractional saturation' ($HbO_{2}$)
is considered to be the gold standard method of measurement, with
`functional saturation' ($SO_{2}$) being an alternative method.
The method of examining the causes of differences between these
two methods is applied to a clinical study conducted by
\citet{Shiao}. This experiment was conducted by 8 lab
practitioners on blood samples, with varying levels of
haemoglobin, from two donors. The samples have been in storage for
varying periods ( described by the variable `Bloodage') and are
categorized according to haemoglobin percentages(i.e
$0\%$,$20\%$,$40\%$,$60\%$,$80\%$,$100\%$). There are 625
observations in all.

\citet{LaiShiao} fits two models on this data, with the lab
technicians and the replicate measurements as the random effects
in both models. The first model uses haemoglobin level as a fixed
effects component. For the second model, blood age is added as a
second fixed factor.

\subsubsection{Single fixed effect} The first model fitted by \citet{LaiShiao} takes the
blood level as the sole fixed effect to be analyzed. The following
coefficient estimates are estimated by `Proc Mixed';
\begin{eqnarray}
\mbox{fixed effects :   } 2.5056 - 0.0263\mbox{Fhbperct}_{ijtl} \\
(\mbox{p-values :   } = 0.0054, <0.0001, <0.0001)\nonumber\\\nonumber\\
\mbox{random effects :   } u(\sigma^{2}=3.1826) + e_{ijtl}
(\sigma^{2}_{e}=0.1525, \rho= 0.6978) \nonumber\\
(\mbox{p-values :   } = 0.8113, <0.0001, <0.0001)\nonumber
\end{eqnarray}

With the intercept estimate being both non-zero and statistically
significant ($p=0.0054$), this models supports the presence
inter-method bias is $2.5\%$ in favour of $SO_{2}$. Also, the
negative value of the haemoglobin level coefficient indicate that
differences will decrease by $0.0263\%$ for every percentage
increase in the haemoglobin .

In the random effects estimates, the variance due to the
practitioners is $3.1826$, indicating that there is a significant
variation due to technicians ($p=0.0311$) affecting the
differences. The variance for the estimates is given as $0.1525$,
($p<0.0001$).

\subsubsection{Two fixed effects}
Blood age is added as a second fixed factor to the model,
whereupon new estimates are calculated;
\begin{eqnarray}
\mbox{fixed effects :   } -0.2866 + 0.1072 \mbox{Bloodage}_{ijtl}
- 0.0264\mbox{Fhbperct}_{ijtl}\nonumber\\
( \mbox{p-values :   } = 0.8113, <0.0001, <0.0001)\nonumber\\\nonumber\\
\mbox{random effects :   } u(\sigma^{2}=10.2346) + e_{ijtl}
(\sigma^{2}_{e}=0.0920, \rho= 0.5577) \nonumber\\
(\mbox{p-values :   } = 0.0446, <0.0001, <0.0001)
\end{eqnarray}


With this extra fixed effect added to the model, the intercept
term is no longer statistically significant. Therefore, with the
presence of the second fixed factor, the model is no longer
supporting the presence of inter-method bias. Furthermore, the
second coefficient indicates that the blood age of the observation
has a significant bearing on the size of the difference between
both methods ($p <0.0001$). Longer storage times for blood will
lead to higher levels of particular blood factors such as MetHb
and HbCO (due to the breakdown and oxidisation of the
haemoglobin). Increased levels of MetHb and HbCO are concluded to
be the cause of the differences. The coefficient for the
haemoglobin level doesn't differ greatly from the single fixed
factor model, and has a much smaller effect on the differences.
The random effects estimates also indicate significant variation
for the various technicians; $10.2346$ with $p=0.0446$.

\citet{LaiShiao} demonstrates how that linear mixed effects models
can be used to provide greater insight into the cause of the
differences. Naturally the addition of further factors to the
model provides for more insight into the behavior of the data.



\newpage
\subsection{ Liao Shaio}

Lai et Shiao is interesting in that it extends the usual method comparison study question. It correctly identifies LME models as a methodoloy that can used to make such questions tractable.
The Data Set used in their examples is unavailable for independent use. Therefore, for the sake of consistency, a data set will be simulated based on the Blood Data that will allow for extra variables.





	\section{Lai Shiao}
	
	\citet{LaiShiao} advocates the use of LME models to study method comparison problems. The authors analyse a data set typical of method comparison studies using SAS software, with particular use of the \emph{`Proc Mixed'} package. The stated goal of this study is to determine which factor from a specified group of factors is the key contributor to the difference in the two methods.
	
	The study relates to oxygen saturation, the most investigated variable in clinical nursing studies \citep{LaiShiao}. The two method compared are functional saturation (SO2, percent functional oxy-hemoglobin) and fractional saturation (HbO2, percent fractional oxy-hemoglobin), which is considered to be the `gold standard' method of measurement.
	
	\citet{LaiShiao} establishes an LME model for analysing the differences $D_{ijtl}$, where $D_{ijtl}$ is the differences of the measurements (i.e = $SO2_{ijtl}$ - $HbO2_{ijtl}$) for the ith donor at the $j$th level of foetal haemoglobin percent (Fhbperct) and the $t$th repeated measurement by the $l$th practitioner of the experiment.
	
	
	(\citet{BXC2004} also advocates the use of LME models in comparing methods, but with a different emphasis.)
	\citet{LaiShiao} use mixed models to determine the factors that
	affect the difference of two methods of measurement using the
	conventional formulation of linear mixed effects models.
	
	If the parameter \textbf{b}, and the variance components are not
	significantly different from zero, the conclusion that there is no
	inter-method bias can be drawn. If the fixed effects component
	contains only the intercept, and a simple correlation coefficient
	is used, then the estimate of the intercept in the model is the
	inter-method bias. Conversely the estimates for the fixed effects
	factors can advise the respective influences each factor has on
	the differences. It is possible to pre-specify different
	correlation structures of the variance components \textbf{G} and
	\textbf{R}.
	
	
	Oxygen saturation is one of the most frequently measured variables
	in clinical nursing studies. `Fractional saturation' ($HbO_{2}$)
	is considered to be the gold standard method of measurement, with
	`functional saturation' ($SO_{2}$) being an alternative method.
	The method of examining the causes of differences between these
	two methods is applied to a clinical study conducted by
	\citet{Shiao}. This experiment was conducted by 8 lab
	practitioners on blood samples, with varying levels of
	haemoglobin, from two donors. The samples have been in storage for
	varying periods ( described by the variable `Bloodage') and are
	categorized according to haemoglobin percentages(i.e
	$0\%$,$20\%$,$40\%$,$60\%$,$80\%$,$100\%$). There are 625
	observations in all.
	
	\citet{LaiShiao} fits two models on this data, with the lab
	technicians and the replicate measurements as the random effects
	in both models. The first model uses haemoglobin level as a fixed
	effects component. For the second model, blood age is added as a
	second fixed factor.
	
	\subsubsection{Single fixed effect} The first model fitted by \citet{LaiShiao} takes the
	blood level as the sole fixed effect to be analyzed. The following
	coefficient estimates are estimated by `Proc Mixed';
	\begin{eqnarray}
	\mbox{fixed effects :   } 2.5056 - 0.0263\mbox{Fhbperct}_{ijtl} \\
	(\mbox{p-values :   } = 0.0054, <0.0001, <0.0001)\nonumber\\\nonumber\\
	\mbox{random effects :   } u(\sigma^{2}=3.1826) + e_{ijtl}
	(\sigma^{2}_{e}=0.1525, \rho= 0.6978) \nonumber\\
	(\mbox{p-values :   } = 0.8113, <0.0001, <0.0001)\nonumber
	\end{eqnarray}
	
	With the intercept estimate being both non-zero and statistically
	significant ($p=0.0054$), this models supports the presence
	inter-method bias is $2.5\%$ in favour of $SO_{2}$. Also, the
	negative value of the haemoglobin level coefficient indicate that
	differences will decrease by $0.0263\%$ for every percentage
	increase in the haemoglobin .
	
	In the random effects estimates, the variance due to the
	practitioners is $3.1826$, indicating that there is a significant
	variation due to technicians ($p=0.0311$) affecting the
	differences. The variance for the estimates is given as $0.1525$,
	($p<0.0001$).
	
	\subsubsection{Two fixed effects}
	Blood age is added as a second fixed factor to the model,
	whereupon new estimates are calculated;
	\begin{eqnarray}
	\mbox{fixed effects :   } -0.2866 + 0.1072 \mbox{Bloodage}_{ijtl}
	- 0.0264\mbox{Fhbperct}_{ijtl}\nonumber\\
	( \mbox{p-values :   } = 0.8113, <0.0001, <0.0001)\nonumber\\\nonumber\\
	\mbox{random effects :   } u(\sigma^{2}=10.2346) + e_{ijtl}
	(\sigma^{2}_{e}=0.0920, \rho= 0.5577) \nonumber\\
	(\mbox{p-values :   } = 0.0446, <0.0001, <0.0001)
	\end{eqnarray}
	
	
	With this extra fixed effect added to the model, the intercept
	term is no longer statistically significant. Therefore, with the
	presence of the second fixed factor, the model is no longer
	supporting the presence of inter-method bias. Furthermore, the
	second coefficient indicates that the blood age of the observation
	has a significant bearing on the size of the difference between
	both methods ($p <0.0001$). Longer storage times for blood will
	lead to higher levels of particular blood factors such as MetHb
	and HbCO (due to the breakdown and oxidisation of the
	haemoglobin). Increased levels of MetHb and HbCO are concluded to
	be the cause of the differences. The coefficient for the
	haemoglobin level doesn't differ greatly from the single fixed
	factor model, and has a much smaller effect on the differences.
	The random effects estimates also indicate significant variation
	for the various technicians; $10.2346$ with $p=0.0446$.
	
	\citet{LaiShiao} demonstrates how that linear mixed effects models
	can be used to provide greater insight into the cause of the
	differences. Naturally the addition of further factors to the
	model provides for more insight into the behavior of the data.
	
\section{Lai Shiao}
\citet{LaiShiao} use mixed models to determine the factors that
affect the difference of two methods of measurement using the
conventional formulation of linear mixed effects models.

If the parameter \textbf{b}, and the variance components are not
significantly different from zero, the conclusion that there is no
inter-method bias can be drawn. If the fixed effects component
contains only the intercept, and a simple correlation coefficient
is used, then the estimate of the intercept in the model is the
inter-method bias. Conversely the estimates for the fixed effects
factors can advise the respective influences each factor has on
the differences. It is possible to pre-specify different
correlation structures of the variance components \textbf{G} and
\textbf{R}.


Oxygen saturation is one of the most frequently measured variables
in clinical nursing studies. `Fractional saturation' ($HbO_{2}$)
is considered to be the gold standard method of measurement, with
`functional saturation' ($SO_{2}$) being an alternative method.
The method of examining the causes of differences between these
two methods is applied to a clinical study conducted by
\citet{Shiao}. This experiment was conducted by 8 lab
practitioners on blood samples, with varying levels of
haemoglobin, from two donors. The samples have been in storage for
varying periods ( described by the variable `Bloodage') and are
categorized according to haemoglobin percentages(i.e
$0\%$,$20\%$,$40\%$,$60\%$,$80\%$,$100\%$). There are 625
observations in all.

\citet{LaiShiao} fits two models on this data, with the lab
technicians and the replicate measurements as the random effects
in both models. The first model uses haemoglobin level as a fixed
effects component. For the second model, blood age is added as a
second fixed factor.

\subsubsection{Single fixed effect} The first model fitted by \citet{LaiShiao} takes the
blood level as the sole fixed effect to be analyzed. The following
coefficient estimates are estimated by `Proc Mixed';
\begin{framed}\begin{eqnarray}
	\mbox{fixed effects :   } 2.5056 - 0.0263\mbox{Fhbperct}_{ijtl} \\
	(\mbox{p-values :   } = 0.0054, <0.0001, <0.0001)\nonumber\\\nonumber\\
	\mbox{random effects :   } u(\sigma^{2}=3.1826) + e_{ijtl}
	(\sigma^{2}_{e}=0.1525, \rho= 0.6978) \nonumber\\
	(\mbox{p-values :   } = 0.8113, <0.0001, <0.0001)\nonumber
	\end{eqnarray}
\end{framed}
With the intercept estimate being both non-zero and statistically
significant ($p=0.0054$), this models supports the presence
inter-method bias is $2.5\%$ in favour of $SO_{2}$. Also, the
negative value of the haemoglobin level coefficient indicate that
differences will decrease by $0.0263\%$ for every percentage
increase in the haemoglobin .

In the random effects estimates, the variance due to the
practitioners is $3.1826$, indicating that there is a significant
variation due to technicians ($p=0.0311$) affecting the
differences. The variance for the estimates is given as $0.1525$,
($p<0.0001$).

\subsubsection{Two fixed effects}
Blood age is added as a second fixed factor to the model,
whereupon new estimates are calculated;
\begin{framed}
	\begin{eqnarray}
	\mbox{fixed effects :   } -0.2866 + 0.1072 \mbox{Bloodage}_{ijtl}
	- 0.0264\mbox{Fhbperct}_{ijtl}\nonumber\\
	( \mbox{p-values :   } = 0.8113, <0.0001, <0.0001)\nonumber\\\nonumber\\
	\mbox{random effects :   } u(\sigma^{2}=10.2346) + e_{ijtl}
	(\sigma^{2}_{e}=0.0920, \rho= 0.5577) \nonumber\\
	(\mbox{p-values :   } = 0.0446, <0.0001, <0.0001)
	\end{eqnarray}
\end{framed}


With this extra fixed effect added to the model, the intercept
term is no longer statistically significant. Therefore, with the
presence of the second fixed factor, the model is no longer
supporting the presence of inter-method bias. Furthermore, the
second coefficient indicates that the blood age of the observation
has a significant bearing on the size of the difference between
both methods ($p <0.0001$). Longer storage times for blood will
lead to higher levels of particular blood factors such as MetHb
and HbCO (due to the breakdown and oxidisation of the
haemoglobin). Increased levels of MetHb and HbCO are concluded to
be the cause of the differences. The coefficient for the
haemoglobin level doesn't differ greatly from the single fixed
factor model, and has a much smaller effect on the differences.
The random effects estimates also indicate significant variation
for the various technicians; $10.2346$ with $p=0.0446$.

\citet{LaiShiao} demonstrates how that linear mixed effects models
can be used to provide greater insight into the cause of the
differences. Naturally the addition of further factors to the
model provides for more insight into the behavior of the data.








	