
\documentclass[12pt, a4paper]{article}
\usepackage{natbib}
\usepackage{vmargin}
\usepackage{graphicx}
\usepackage{epsfig}
\usepackage{subfigure}
%\usepackage{amscd}
\usepackage{amssymb}
\usepackage{subfigure}
\usepackage{amsbsy}
\usepackage{amsthm, amsmath}
%\usepackage[dvips]{graphicx}
\bibliographystyle{chicago}
\renewcommand{\baselinestretch}{1.4}

% left top textwidth textheight headheight % headsep footheight footskip
\setmargins{3.0cm}{2.5cm}{15.5 cm}{23.5cm}{0.5cm}{0cm}{1cm}{1cm}

\pagenumbering{arabic}


\begin{document}
\author{Kevin O'Brien}
\title{Mixed Models for Method Comparison Studies}
\tableofcontents

	%%%%%%%%%%%%%%%%%%%%%%%%%%%%%%%%%%%%%%%%%%%%%%%%%%%%%%%%%%%%%%%%%%%%%%%%%
	%%%%%%%  Blackwood Bradley Model         %%%%%%%%%%%%%%%%%%%%%%%%%%%%%%%%%
	%%%%%%%%%%%%%%%%%%%%%%%%%%%%%%%%%%%%%%%%%%%%%%%%%%%%%%%%%%%%%%%%%%%%%%%%%
	\section{Outlier Testing} 
	A new outlier identification test for method comparison studies based on robust regression.
	
	The identification of outliers in method comparison studies (MCS) is an important part of data analysis, as outliers can indicate serious errors in the measurement process. Common outlier tests proposed in the literature usually require a homogeneous sample distribution and homoscedastic random error variances. However, datasets in MCS usually do not meet these assumptions. In this work, a new outlier test based on robust linear regression is proposed to overcome these special problems. The LORELIA (local reliability) residual test is based on a local, robust residual variance estimator, given as a weighted sum of the observed residuals. The new test is compared to a standard test proposed in the literature by a Monte Carlo simulation. Its performance is illustrated in examples.
	
\section{Lorelia}


Method comparison studies are performed in order to prove equivalence between two measurement methods or instruments. The identification of outliers is an important part of data analysis as outliers can indicate serious errors in the measurement process. Common outlier tests 
proposed in the literature require a homogeneous sample distribution and homoscedastic random error variances. However, datasets in method comparison studies usually do not meet these assumptions. To overcome this problem, different data transformation methods are proposed in the literature. However, they will only be applicable if the random errors can be described by simple additive or multiplicative models. In this work, a new outlier test based on robust linear regression is proposed which provides a general solution to the above problem. The LORELIA (LOcal RELIAbility) residual test is based on a local, robust residual variance estimator, given as a weighted sum of the observed residuals. Outlier limits are estimated from the actual data situation without making assumptions on the underlying error variance model. The performance of the new test is demonstrated in examples and simulations.

\section{Note on Roy's paper}
\begin{enumerate}


\item Basic model:
\begin{center}
$ \boldsymbol{y_{i}} = \boldsymbol{X_{i}\beta}
+ \boldsymbol{Z_{i}b_{i}} + \boldsymbol{\epsilon_{i}}, \qquad i=1,\dots,n$ \\
$\boldsymbol{Z_{i}} \sim \mathcal{N}(\boldsymbol{0,\Sigma}),\quad
\boldsymbol{\epsilon_{i}} \sim \mathcal{N}(\boldsymbol{0, \sigma^2
\boldsymbol{I} })$
\end{center}

Assumptions are made about homoskedasticity.

\item General model:
\begin{center}
$ \boldsymbol{y_{i}} = \boldsymbol{X_{i}\beta}
+ \boldsymbol{Z_{i}b_{i}} + \boldsymbol{\epsilon_{i}}, \qquad i=1,\dots,n$ \\
$\boldsymbol{Z_{i}} \sim \mathcal{N}(\boldsymbol{0,\Psi}),\quad
\boldsymbol{\epsilon_{i}} \sim \mathcal{N}(\boldsymbol{0,\sigma^2 \boldsymbol{\Lambda} })$
\end{center}

Assumptions about homoskedasticity are relaxed \cite[pg.202]{pb}.





\item $\sigma^2 \boldsymbol{\Lambda}$ is the general form for the VC structure for residuals.

\item The response vector $\boldsymbol{y}_{i}$ comprises the observations of
the subject, as measured by two methods, taking three measurements each.
Hence a $6 \times 1$ random vector corresponding to the $i$th subject.
\begin{equation}
\boldsymbol{y}_{i} = (y_{i}^{j1},y_{i}^{Jj2},y_{i}^{j3},y_{i}^{s1},y_{i}^{s2},y_{i}^{s3}) \prime
\end{equation}

\item The number of replicates is $p$. A subject will have up to
 $2p$ measurements, for the two instrument case, i.e. $Max(n_{i}) = 2p$.
(Let $k$ denote number of instruments, which is assumed to be $2$
unless stated otherwise.) For the blood pressure data $p=3$.


\end{enumerate}








\addcontentsline{toc}{section}{Bibliography}

\bibliography{transferbib}
\end{document}
