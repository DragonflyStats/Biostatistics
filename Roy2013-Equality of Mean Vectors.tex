Roy2013

http://business.utsa.edu/wps/MSS/0017MSS-253-2013.pdf


Testing the Equality of Mean Vectors for Paired Doubly Multivariate Observations 


Example 2. (Mineral Data): This data set is taken from Johnson and Wichern (2007, p. 43).
An investigator measured the mineral content of bones (radius, humerus and ulna) by photon
absorptiometry to examine whether dietary supplements would slow bone loss in 25 older women.
Measurements were recorded for three bones on the dominant and nondominant sides. Thus,
the data is doubly multivariate and clearly u = 2 and q = 3.
The bone mineral contents for the first 24 women one year after their participation in an
experimental program is given in Johnson and Wichern (2007, p. 353). Thus, for our analysis
we take only first 24 women in the first data set. We test whether there has been a bone loss
considering the data as doubly multivariate and has BCS structure. We rearrange the variables
in the data set by grouping together the mineral content of the dominant sides of radius, humerus
and ulna as the first three variables, that is, the variables in the first location (u = 1) and then
the mineral contents for the non-dominant side of the same bones (u = 2)
