Bayesian Bland Altman Approaches
%================================%
A multivariate hierarchical Bayesian approach to measuring agreement in repeated
measurement method comparison studies

*http://www.biomedcentral.com/1471-2288/9/6*

Philip J Schluter

\subsection*{Background}
Assessing agreement in method comparison studies depends on two fundamentally important components; validity (the between method agreement) and reproducibility (the within method agreement). 

The Bland-Altman limits of agreement technique is one of the favoured approaches in medical literature for assessing between method validity. However, few researchers have adopted this approach for the assessment of both validity and reproducibility. 

This may be partly due to a lack of a flexible, easily implemented and readily available statistical machinery to analyse repeated measurement method comparison data.

### Methods
Adopting the Bland-Altman framework, but using Bayesian methods, we present this statistical machinery. Two multivariate hierarchical Bayesian models are advocated, one which assumes that the underlying values for subjects remain static (exchangeable replicates) and one which assumes that the underlying values can change between repeated measurements (non-exchangeable replicates).

### Results
We illustrate the salient advantages of these models using two separate datasets that have been previously analysed and presented; 
(i) assuming static underlying values analysed using both multivariate hierarchical Bayesian models,  
(ii) assuming each subject's underlying value is continually changing quantity and analysed using the non-exchangeable replicate multivariate hierarchical Bayesian model.  

### Conclusion
These easily implemented models allow for full parameter uncertainty, simultaneous method comparison, handle unbalanced or missing data, and provide estimates and credible regions for all the parameters of interest. Computer code for the analyses in also presented, provided in the freely available and currently cost free software package WinBUGS.
<hr>
