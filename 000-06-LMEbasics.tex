\documentclass[12pt, a4paper]{article}
\usepackage{natbib}
\usepackage{vmargin}
\usepackage{graphicx}
\usepackage{epsfig}
\usepackage{subfigure}
%\usepackage{amscd}
\usepackage{amssymb}
\usepackage{subfiles}
\usepackage{subfigure}
\usepackage{framed}
\usepackage{subfiles}
\usepackage{amsbsy}
\usepackage{amsthm, amsmath}
%\usepackage[dvips]{graphicx}
\bibliographystyle{chicago}
\renewcommand{\baselinestretch}{1.1}

% left top textwidth textheight headheight % headsep footheight footskip
\setmargins{3.0cm}{2.5cm}{15.5 cm}{23.5cm}{0.25cm}{0cm}{0.5cm}{0.5cm}

\pagenumbering{arabic}

\begin{document}

\subsection{Laird Ware Formulation}
\begin{equation*}
\boldsymbol{y_{i}} = \boldsymbol{X_{i}\beta}  + \boldsymbol{Z_{i}b_{i}} + \boldsymbol{\epsilon_{i}}, \qquad i=1,\dots,85
\end{equation*}
\begin{eqnarray*}
	\boldsymbol{Z_{i}} \sim \mathcal{N}(\boldsymbol{0,\Psi}),\qquad
	\boldsymbol{\epsilon_{i}} \sim \mathcal{N}(\boldsymbol{0,\sigma^2\Lambda})
\end{eqnarray*}





\subsection{Stating the LME Model}
The general linear mixed
model is
\[
Y = X\beta + Zu + \varepsilon\]
where Y is a $(n\times1)$ vector of observed data, X is an $(n\times p)$ fixed-effects design or regressor matrix of rank
k, Z is a $(n \times g)$ random-effects design or regressor matrix, $u$ is a $(g \times 1)$ vector of random effects, and $\varepsilon$ is
an $(n\times1)$ vector of model errors (also random effects). The distributional assumptions made by the MIXED
procedure are as follows: γ is normal with mean 0 and variance G; $\varepsilon$ is normal with mean 0 and variance
R; the random components $u$ and $\varepsilon$ are independent. Parameters of this model are the fixed-effects β and
all unknowns in the variance matrices G and R. The unknown variance elements are referred to as the
covariance parameters and collected in the vector $theta$.
%===========================================================================%

The concept of critiquing the model-data agreement applies in mixed models in the same way as in linear
fixed-effects models. In fact, because of the more complex model structure, you can argue that model and
data diagnostics are even more important. For example, you are not only concerned with capturing the
important variables in the model. You are also concerned with “distributing” them correctly between the
fixed and random components of the model. The mixed model structure presents unique and interesting
challenges that prompt us to reexamine the traditional ideas of influence and residual analysis.
%==========================================================================%
This paper presents the extension of traditional tools and statistical measures for influence and residual
analysis to the linear mixed model and demonstrates their implementation in the MIXED procedure (experimental
features in SAS 9.1). The remainder of this paper is organized as follows. The “Background” section
briefly discusses some mixed model estimation theory and the challenges to model diagnosis that result
from it.

















\bibliographystyle{chicago}
\bibliography{DB-txfrbib}
\end{document}
