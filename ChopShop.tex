\documentclass[12pt, a4paper]{article}
\usepackage{epsfig}
\usepackage{subfigure}
%\usepackage{amscd}
\usepackage{amssymb}
\usepackage{amsbsy}
\usepackage{amsthm, amsmath}
%\usepackage[dvips]{graphicx}
\usepackage{natbib}
\bibliographystyle{chicago}
\usepackage{vmargin}
% left top textwidth textheight headheight
% headsep footheight footskip
\setmargins{3.0cm}{2.5cm}{15.5 cm}{22cm}{0.5cm}{0cm}{1cm}{1cm}
\renewcommand{\baselinestretch}{1.5}
\pagenumbering{arabic}
\theoremstyle{plain}
\newtheorem{theorem}{Theorem}[section]
\newtheorem{corollary}[theorem]{Corollary}
\newtheorem{ill}[theorem]{Example}
\newtheorem{lemma}[theorem]{Lemma}
\newtheorem{proposition}[theorem]{Proposition}
\newtheorem{conjecture}[theorem]{Conjecture}
\newtheorem{axiom}{Axiom}
\theoremstyle{definition}
\newtheorem{definition}{Definition}[section]
\newtheorem{notation}{Notation}
\theoremstyle{remark}
\newtheorem{remark}{Remark}[section]
\newtheorem{example}{Example}[section]
\renewcommand{\thenotation}{}
\renewcommand{\thetable}{\thesection.\arabic{table}}
\renewcommand{\thefigure}{\thesection.\arabic{figure}}
\title{Research notes: linear mixed effects models}
\author{ } \date{ }


\begin{document}
\author{Kevin O'Brien}
\title{Updating techniques for LME models}

\addcontentsline{toc}{section}{Bibliography}

%----------------------------------------------------------------------------------------%
\newpage
%\chapter{Limits of Agreement}
\newpage
\section{Lambda Structure}

\begin{equation}
\boldsymbol{\epsilon} \sim \mathcal{N}(\boldsymbol{0},\sigma^2 \boldsymbol{\Lambda})
\end{equation}
\begin{enumerate}
\item A simple assumption is to assumes that residuals are independent and homoscedastic, i.e. $\boldsymbol{Lambda = I}$.

\item For the Bland Altman blood pressure data,$\boldsymbol{\Lambda}$ has kronecker product structure
and has dimensions $6 \times 6$.
\end{enumerate}


\section{Basic Models Fits}
Further to \citet{PB}, several simple LME models are constructed
for the blood pressure data. This data set is the subject of a
method comparison study in \citet{BA99}.

\subsection{Implementing the Mixed Models Fits}
They are implemented using the following {\tt{R}} code, utilising the
`nlme' package. An analysis of variance is used to compare the model fits.

The {\tt{R}} script:
\begin{verbatim}
fit1 = lme( BP ~ method, data = dat, random = ~1 | subject )
fit2 = update(fit1, random = ~1 | subject/method )
fit3 = update(fit1, random = ~method - 1 | subject )
#analysis of variance
anova(fit1,fit2,fit3)
\end{verbatim}


\begin{enumerate}


\item Simplest workable model, allows differences between methods
and incorporates a random intercept for each subject. For subject
1 we have
\[
\boldsymbol{X}_i =
\left(%
\begin{array}{cc}
  1 & 0 \\
  1 & 0 \\
  1 & 0 \\
  1 & 1 \\
  1 & 1 \\
  1 & 1 \\
\end{array}%
\right),\quad
\boldsymbol{\beta} =
\left(%
\begin{array}{c}
  \beta_0 \\
  \beta_1 \\
\end{array}%
\right), \quad
\boldsymbol{Z}_i =
\left(%
\begin{array}{c}
  1 \\
  1 \\
  1 \\
  1 \\
  1 \\
  1 \\
\end{array}%
\right), \quad \boldsymbol{b}_i = b
\]
where $\mathrm{E}(b)=0$ and $\mathrm{var}(b)=\psi.$

\item
\[
\boldsymbol{Z}_i =
\left(%
\begin{array}{c c}
  1 & 0 \\
  1 & 0 \\
  1 & 0 \\
  0 & 1 \\
  0 & 1 \\
  0 & 1 \\
\end{array}%
\right)
\quad \boldsymbol{b}_i =
\left(%
\begin{array}{c c}
  b_1 & 0  \\
  0 & b_2  \\
\end{array}%
\right)
\]

where $\mathrm{E}(b_i)=0$ and $\mathrm{var}(\boldsymbol{b})=
\boldsymbol{\Psi}$.

The variance of error terms is a $6 \times 6$ matrix.

\end{enumerate}












\newpage
\subsection{Model Fit 1}

This is a simple model with no interactions. There is a fixed effect for each method and a random effect for each subject.
\begin{equation*}
y_{ijk} = \beta_{j}  + b_{i} + \epsilon_{ijk}, \qquad i=1,\dots,2, j=1,\dots,85, k=1,\dots,3
\end{equation*}

\begin{eqnarray*}
b_{i} \sim \mathcal{N}(0,\sigma^2_{b}), \qquad \epsilon_{i} \sim \mathcal{N}(0,\sigma^2)
\end{eqnarray*}

\begin{verbatim}
Linear mixed-effects model fit by REML
  Data: dat
  Log-restricted-likelihood: -2155.853
  Fixed: BP ~ method
(Intercept)     methodS
  127.40784    15.61961

Random effects:
 Formula: ~1 | subject
        (Intercept) Residual
StdDev:    29.39085 12.44454

Number of Observations: 510
Number of Groups: 85
\end{verbatim}

The following output was obtained.
\begin{verbatim}
Linear mixed-effects model fit by REML
  Data: dat
  Log-restricted-likelihood: -2047.582
  Fixed: BP ~ method
(Intercept)     methodS
  127.40784    15.61961

Random effects:
 Formula: ~method - 1 | subject
 Structure: General positive-definite, Log-Cholesky parametrization
         StdDev    Corr
methodJ  30.455093 methdJ
methodS  31.477237 0.835
Residual  7.763666

Number of Observations: 510
Number of Groups: 85

\end{verbatim}



\subsection{Featured approaches}

\citet{bxc2008} computes the limits of agreement to the case with repeated measurements by using LME models.

\citet{Roy} formulates a very powerful method of assessing whether two methods of measurement, with replicate measurements, also using LME models. Roy's approach is based on the construction of variance-covariance matrices.
Importantly, Roy's approach does not address the issue of limits of agreement (though another related analysis , the coefficient of repeatability, is mentioned).

This paper seeks to use Roy's approach to estimate the limits of agreement. These estimates will be compared to estimates computed under Carstensen's formulation.

In computing limits of agreement, it is first necessary to have an estimate for the variance of differences. When the agreement of two methods is analyzed using LME models, a clear method of how to compute the variance is required. As the estimate for inter-method bias and the quantile would be the same for both methodologies, the focus hereon is solely on the variance of differences.

\newpage

\section{Note on Roy's paper}

\newpage
\subsection{Model Fit 2}

This is a simple model, this time with an interaction effect.
There is a fixed effect for each method. This model has random effects at two levels $b_{i}$ for the subject, and
another, $b_{ij}$, for the respective method within each subject.
\begin{equation*}
y_{ijk} = \beta_{j}  + b_{i} + b_{ij} + \epsilon_{ijk}, \qquad i=1,\dots,2, j=1,\dots,85, k=1,\dots,3
\end{equation*}
\begin{eqnarray*}
b_{i} \sim \mathcal{N}(0,\sigma^2_{1}), \qquad b_{ij} \sim \mathcal{N}(0,\sigma^2_{2}), \qquad \epsilon_{i} \sim \mathcal{N}(0,\sigma^2)
\end{eqnarray*}

In this model, the random interaction terms all have the same variance $\sigma^2_{2}$. These terms are assumed to be independent of each other, even
within the same subject.

\begin{verbatim}
Linear mixed-effects model fit by REML
  Data: dat
  Log-restricted-likelihood: -2047.714
  Fixed: BP ~ method
(Intercept)     methodS
  127.40784    15.61961

Random effects:
 Formula: ~1 | subject
        (Intercept)
StdDev:    28.28452

 Formula: ~1 | method %in% subject
        (Intercept) Residual
StdDev:    12.61562 7.763666

Number of Observations: 510
Number of Groups:
            subject method %in% subject
                 85                 170
\end{verbatim}
\newpage
\subsection{Model Fit 3}

This model is a more general model, compared to 'model fit 2'. This model treats the random interactions for each subject as a vector and
allows the variance-covariance matrix for that vector to be estimated from the set of all positive-definite matrices.
$\boldsymbol{y_{i}}$ is the entire response vector for the $i$th subject.
$\boldsymbol{X_{i}}$ and $\boldsymbol{Z_{i}}$  are the fixed- and random-effects design matrices respectively.
\begin{equation*}
\boldsymbol{y_{i}} = \boldsymbol{X_{i}\beta}  + \boldsymbol{Z_{i}b_{i}} + \boldsymbol{\epsilon_{i}}, \qquad i=1,\dots,85
\end{equation*}
\begin{eqnarray*}
\boldsymbol{Z_{i}} \sim \mathcal{N}(\boldsymbol{0,\Psi}),\qquad
\boldsymbol{\epsilon_{i}} \sim \mathcal{N}(\boldsymbol{0,\sigma^2\Lambda})
\end{eqnarray*}

For the first subject the response vector, $\boldsymbol{y_{1}}$, is:
\begin{table}[ht]
\begin{center}
\begin{tabular}{rrllr}
  \hline
observation & BP & subject & method & replicate \\
  \hline
1 & 100.00 & 1 & J &   1 \\
  86 & 106.00 & 1 & J &   2 \\
  171 & 107.00 & 1 & J &   3 \\
  511 & 122.00 & 1 & S &   1 \\
  596 & 128.00 & 1 & S &   2 \\
  681 & 124.00 & 1 & S &   3 \\
   \hline
\end{tabular}
\end{center}
\end{table}
\newpage
The fixed effects design matrix $\boldsymbol{X_{i}}$ is given by:
\begin{table}[ht]
\begin{center}
\begin{tabular}{r|r}
  \hline
  (Intercept) & method S \\
  \hline
 1 & 0 \\
 1 & 0 \\
 1 & 0 \\
 1 & 1 \\
 1 & 1 \\
 1 & 1 \\
   \hline
\end{tabular}
\end{center}
\end{table}

\newpage




The random effects design matrix $\boldsymbol{Z_{i}}$ is given by:
\begin{table}[ht]
\begin{center}
\begin{tabular}{r|r}
  \hline
 method J & method S \\
  \hline
 1 & 0 \\
 1 & 0 \\
 1 & 0 \\
 0 & 1 \\
 0 & 1 \\
 0 & 1 \\
   \hline
\end{tabular}
\end{center}
\end{table}
\newpage


\section{Classical model for single measurements}
In the first instance, we require a simple model to describe a measurement by method $m$. We use the term $item$ to denote an individual, subject or sample, to be measured, being randomly sampled from a population. Let $y_{mi}$ be the measurement for item $i$ made by method $m$.

\[ y_{mi} = \alpha_{m} + \mu_{i} + e_{mi}  \]

\begin{itemize}
\item $\alpha_m$ is the fixed effect associated with method $m$,
\item $\mu_i$ is the true value for subject $i$ (fixed effect),
\item $e_{mi}$ is a
random effect term for errors with $e_{mi}  \sim \mathcal{N}(0,\sigma^2_m)$. \end{itemize}.

This model implies that the difference between the paired measurements can be expressed as

\[ d_{i} = y_{1i} - y_{2i} \sim \mathcal{N} (\alpha_{1} - \alpha_{2}, \sigma^2_{1} - \sigma^2_{2}). \]

Importantly, this is independent of the item levels $\mu_i$. As the case-wise differences are of interest, the parameters of interest are the fixed effects for methods $\alpha_{m}$.

\[ y_{mi} =  \alpha_{m}  + \mu_{i} + e_{mi}  \]

\newpage



\newpage



Importantly these variance covariance structures are central to Roy methodology.
          \right) \]

\citet{Roy} proposes a series of hypothesis tests based on these matrices as part of her methodology. These tests shall be reverted to in due course.

The standard deviation of the differences of variables $a$ and $b$ is computed as
\[
\mbox{var}(a - b) = \mbox{var} ( a )  + \mbox{var} ( b ) - 2\mbox{cov} ( a ,b )
\]

Hence the variance of the difference of two methods, that allows for the calculation of the limits of agreement, can be calculated as

\[
\mbox{var}(d) = \omega^2_1  + \omega^2_2 - 2 \times \omega_12
\]




%----------------------------------------------------------------------------%



\newpage
\subsection{Difference Variance further to Carstensen}

\citet{bxc2008} states a model where the variation between items
for method $m$ is captured by $\tau_m$ (our notation $d^2_m$) and the within-item
variation by $\sigma_m$.

\emph{The formulation of this model is general and refers to comparison
of any number of methods � however, if only two methods are
compared, separate values of $\tau^2_1$ and $\tau^2_2$ cannot be
estimated, only their average value $\tau$, so in the case of only
two methods we are forced to assume that $\tau_1 = \tau_2 = \tau$}\citep{bxc2008}.

Another important point is that there is no covariance terms, so
further to  \citet{bxc2008} the variance covariance matrices for
between-item and within-item variability are respectively.

\[\boldsymbol{D} = \left(
            \begin{array}{cc}
              d^1_2  & 0 \\
              0 & d^2_2 \\
            \end{array}
          \right) \]
and  $\boldsymbol{\Sigma}$ is constructed as follows:
\[\boldsymbol{\Sigma} = \left(
            \begin{array}{cc}
              \sigma^1_2  & 0 \\
              0 & \sigma^2_2 \\
            \end{array}
          \right) \]


Under this model the limits of agreement should be computed based
on the standard deviation of the difference between a pair of
measurements by the two methods on a new individual, j, say:

\[ \mbox{var}(y_{1j} - y_{2j}) = 2d^2 + \sigma^2_1 + \sigma^2_2  \]

Further to his model, Carstensen computes the limits of agreement
as

\[
\hat{\alpha}_1 - \hat{\alpha}_2 \pm \sqrt{2 \hat{d}^2 +
\hat{\sigma}^2_1 + \hat{\sigma}^2_2}
\]


\subsection{Relevance of Roy's Methodology}

The relevance of Roy's methodology is that estimates for the
between-item variances for both methods $\hat{d}^2_m$ are
computed. Also the VC matrices are constructed with covariance
terms and, so the difference variance must be formulated
accordingly.


\[
\hat{\alpha}_1 - \hat{\alpha}_2 \pm \sqrt{ \hat{d}^2_1  +
\hat{d}^2_1 + \hat{\sigma}^2_1 + \hat{\sigma}^2_2 - 2 \hat{d}_{12}
- 2 \hat{\sigma}_12}
\]

\newpage



\bibliography{DB-txfrbib}
\end{document}
