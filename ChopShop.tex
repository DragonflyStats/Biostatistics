\documentclass[12pt, a4paper]{article}
\usepackage{epsfig}
\usepackage{subfigure}
%\usepackage{amscd}
\usepackage{amssymb}
\usepackage{amsbsy}
\usepackage{amsthm, amsmath}
%\usepackage[dvips]{graphicx}
\usepackage{natbib}
\bibliographystyle{chicago}
\usepackage{vmargin}
% left top textwidth textheight headheight
% headsep footheight footskip
\setmargins{3.0cm}{2.5cm}{15.5 cm}{22cm}{0.5cm}{0cm}{1cm}{1cm}
\renewcommand{\baselinestretch}{1.5}
\pagenumbering{arabic}
\theoremstyle{plain}
\newtheorem{theorem}{Theorem}[section]
\newtheorem{corollary}[theorem]{Corollary}
\newtheorem{ill}[theorem]{Example}
\newtheorem{lemma}[theorem]{Lemma}
\newtheorem{proposition}[theorem]{Proposition}
\newtheorem{conjecture}[theorem]{Conjecture}
\newtheorem{axiom}{Axiom}
\theoremstyle{definition}
\newtheorem{definition}{Definition}[section]
\newtheorem{notation}{Notation}
\theoremstyle{remark}
\newtheorem{remark}{Remark}[section]
\newtheorem{example}{Example}[section]
\renewcommand{\thenotation}{}
\renewcommand{\thetable}{\thesection.\arabic{table}}
\renewcommand{\thefigure}{\thesection.\arabic{figure}}
\title{Research notes: linear mixed effects models}
\author{ } \date{ }


\begin{document}
\author{Kevin O'Brien}
\title{Updating techniques for LME models}

\addcontentsline{toc}{section}{Bibliography}

%----------------------------------------------------------------------------------------%
\newpage
%\chapter{Limits of Agreement}
\newpage
\section{Lambda Structure}

\begin{equation}
\boldsymbol{\epsilon} \sim \mathcal{N}(\boldsymbol{0},\sigma^2 \boldsymbol{\Lambda})
\end{equation}
\begin{enumerate}
\item A simple assumption is to assumes that residuals are independent and homoscedastic, i.e. $\boldsymbol{Lambda = I}$.

\item For the Bland Altman blood pressure data,$\boldsymbol{\Lambda}$ has kronecker product structure
and has dimensions $6 \times 6$.
\end{enumerate}











\subsection{Featured approaches}

\citet{bxc2008} computes the limits of agreement to the case with repeated measurements by using LME models.

\citet{Roy} formulates a very powerful method of assessing whether two methods of measurement, with replicate measurements, also using LME models. Roy's approach is based on the construction of variance-covariance matrices.
Importantly, Roy's approach does not address the issue of limits of agreement (though another related analysis , the coefficient of repeatability, is mentioned).

This paper seeks to use Roy's approach to estimate the limits of agreement. These estimates will be compared to estimates computed under Carstensen's formulation.

In computing limits of agreement, it is first necessary to have an estimate for the variance of differences. When the agreement of two methods is analyzed using LME models, a clear method of how to compute the variance is required. As the estimate for inter-method bias and the quantile would be the same for both methodologies, the focus hereon is solely on the variance of differences.

\newpage

\section{Note on Roy's paper}



\section{Classical model for single measurements}
In the first instance, we require a simple model to describe a measurement by method $m$. We use the term $item$ to denote an individual, subject or sample, to be measured, being randomly sampled from a population. Let $y_{mi}$ be the measurement for item $i$ made by method $m$.

\[ y_{mi} = \alpha_{m} + \mu_{i} + e_{mi}  \]

\begin{itemize}
\item $\alpha_m$ is the fixed effect associated with method $m$,
\item $\mu_i$ is the true value for subject $i$ (fixed effect),
\item $e_{mi}$ is a
random effect term for errors with $e_{mi}  \sim \mathcal{N}(0,\sigma^2_m)$. \end{itemize}.

This model implies that the difference between the paired measurements can be expressed as

\[ d_{i} = y_{1i} - y_{2i} \sim \mathcal{N} (\alpha_{1} - \alpha_{2}, \sigma^2_{1} - \sigma^2_{2}). \]

Importantly, this is independent of the item levels $\mu_i$. As the case-wise differences are of interest, the parameters of interest are the fixed effects for methods $\alpha_{m}$.

\[ y_{mi} =  \alpha_{m}  + \mu_{i} + e_{mi}  \]

\newpage



\newpage



Importantly these variance covariance structures are central to Roy methodology.
%          \right) \]

\citet{Roy} proposes a series of hypothesis tests based on these matrices as part of her methodology. These tests shall be reverted to in due course.

The standard deviation of the differences of variables $a$ and $b$ is computed as
\[
\mbox{var}(a - b) = \mbox{var} ( a )  + \mbox{var} ( b ) - 2\mbox{cov} ( a ,b )
\]

Hence the variance of the difference of two methods, that allows for the calculation of the limits of agreement, can be calculated as

\[
\mbox{var}(d) = \omega^2_1  + \omega^2_2 - 2 \times \omega_12
\]





\bibliography{DB-txfrbib}
\end{document}

