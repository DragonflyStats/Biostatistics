	Linear Mixed Effects Models can be implemented by using one of the following R packages.
	lme4
	nlme
	
	The first package to be introducted was nlme, developled by Jose Pinheiro and Douglas Bates ( Authors of the the companion textbook, NAME)
	
	As this package has been under ongoing development for quite a long time, it is now allows for a lot of complex LME implementations. 
	Furthermore, nlme is one of the base R packages.  That is to say, when one downloads and installs R, nlme is automatically installed also, and can be called immediately.
	
	Having said that, the authors have pointed to several limitations of the overall methodology thrugh R.
	The original developers have both left the project, but other statisticians have taken over the development, and indeed a new version of nlme was released.
	
	LME4 is a more recent package. at a glance, the syntax is easier, but the development is less advanced. There are several functionalities that can not be implemented with lme4 yet. 
	As an example - CHAP5 in PB - has no equivalent in LME4. Indeed no textbook exists to co-incide with LME4.
	
	The main author, Douglas Bates, has turned his attention to development of LME models in the Julia programming language.
	
	The nlmeU package is described by its authors as an extesntion of the nlme package, and indeed provides for additionally functionality. The package is also useful as it serves as a companion piece to the 
	book by Galecki and Burzwhatski.
	
	The nlme package also allows for the specification of GLS models.
	
	%-----------------------------------------------------------------------------%