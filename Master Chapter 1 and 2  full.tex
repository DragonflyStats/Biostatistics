\documentclass[Main.tex]{subfiles}
\begin{document}
	
	

	\chapter{Method Comparison Studies}
	
	\section{Introduction}
	The problem of assessing the agreement between two or more methods
	of measurement is ubiquitous in scientific research, and is
	commonly referred to as a `method comparison study'. Published
	examples of method comparison studies can be found in disciplines
	as diverse as pharmacology \citep{ludbrook97}, anaesthesia
	\citep{Myles}, and cardiac imaging methods \citep{Krumm}.
	\smallskip
	
	To illustrate the characteristics of a typical method comparison
	study consider the data in Table I \citep{Grubbs73}. In each of
	twelve experimental trials, a single round of ammunition was fired
	from a 155mm gun and its velocity was measured simultaneously (and
	independently) by three chronographs devices, identified here by
	the labels `Fotobalk', `Counter' and `Terma'.
	\smallskip
	
	
	\newpage
	
	\begin{table}[ht]
		\begin{center}
			\begin{tabular}{rrrr}
				\hline
				Round& Fotobalk [F] & Counter [C]& Terma [T]\\
				\hline
				1 & 793.8 & 794.6 & 793.2 \\
				2 & 793.1 & 793.9 & 793.3 \\
				3 & 792.4 & 793.2 & 792.6 \\
				4 & 794.0 & 794.0 & 793.8 \\
				5 & 791.4 & 792.2 & 791.6 \\
				6 & 792.4 & 793.1 & 791.6 \\
				7 & 791.7 & 792.4 & 791.6 \\
				8 & 792.3 & 792.8 & 792.4 \\
				9 & 789.6 & 790.2 & 788.5 \\
				10 & 794.4 & 795.0 & 794.7 \\
				11 & 790.9 & 791.6 & 791.3 \\
				12 & 793.5 & 793.8 & 793.5 \\
				\hline
			\end{tabular}
			\caption{Velocity measurement from the three chronographs (Grubbs
				1973).}
		\end{center}
	\end{table}
	
	An important aspect of the these data is that all three methods of
	measurement are assumed to have an attended measurement error, and
	the velocities reported in Table 1.1 can not be assumed to be
	`true values' in any absolute sense.
	
	%While lack of
	%agreement between two methods is inevitable, the question , as
	%posed by \citet{BA83}, is 'do the two methods of measurement agree
	%sufficiently closely?'
	
	A method of measurement should ideally be both accurate and
	precise. \citet{Barnhart} describes agreement as being a broader
	term that contains both of those qualities. An accurate
	measurement method will give results close to the unknown `true
	value'. The precision of a method is indicated by how tightly
	measurements obtained under identical conditions are distributed
	around their mean measurement value. A precise and accurate method
	will yield results consistently close to the true value. Of course
	a method may be accurate, but not precise, if the average of its
	measurements is close to the true value, but those measurements
	are highly dispersed. Conversely a method that is not accurate may
	be quite precise, as it consistently indicates the same level of
	inaccuracy. The tendency of a method of measurement to
	consistently give results above or below the true value is a
	source of systematic bias. The smaller the systematic bias, the
	greater the accuracy of the method.
	
	% The FDA define precision as the closeness of agreement (degree of
	% scatter) between a series of measurements obtained from multiple
	% sampling of the same homogeneous sample under prescribed
	% conditions. \citet{Barnhart} describes precision as being further
	% subdivided as either within-run, intra-batch precision or
	% repeatability (which assesses precision during a single analytical
	% run), or between-run, inter-batch precision or repeatability
	%(which measures precision over time).
	
	In the context of the agreement of two methods, there is also a
	tendency of one measurement method to consistently give results
	above or below the other method. Lack of agreement is a
	consequence of the existence of `inter-method bias'. For two
	methods to be considered in good agreement, the inter-method bias
	should be in the region of zero. A simple estimate of the
	inter-method bias is given by the differences between pairs of measurements, for example,  Table~\ref{FCTdata} is a good example of
	possible inter-method bias; the `Fotobalk' consistently recording
	smaller velocities than the `Counter' method. A cursory inspection of the table will indicate a systematic tendency for the Counter method to result in higher measurements than the Fotobalk method. % Consequently one would conclude that there is lack of agreement % between the two methods.
	
	The absence of inter-method bias is, by itself, not sufficient to
	establish that two measurement methods agree. The two methods
	must also have equivalent levels of precision. Should one method
	yield results considerably more variable than those of the other,
	they can not be considered to be in agreement. Hence, method comparison studies are required to take account of both inter-method bias and difference in precision of measurements.
	\newpage
	% latex table generated in R 2.6.0 by xtable 1.5-5 package
	% Wed Aug 26 15:22:41 2009
	\begin{table}[h!]
		
		\begin{center}
			
			\begin{tabular}{rrrr}
				\hline
				Round& Fotobalk (F) & Counter (C) & Difference (F-C) \\
				\hline
				1 & 793.8& 794.6 & -0.8 \\
				2 & 793.1 & 793.9 & -0.8 \\
				3 & 792.4 & 793.2 & -0.8 \\
				4 & 794.0 & 794.0 & 0.0 \\
				5 & 791.4 & 792.2 & -0.8 \\
				6 & 792.4 & 793.1 & -0.7 \\
				7 & 791.7 & 792.4 & -0.7 \\
				8 & 792.3 & 792.8 & -0.5 \\
				9 & 789.6 & 790.2 & -0.6 \\
				10 & 794.4 & 795.0 & -0.6 \\
				11 & 790.9 & 791.6 & -0.7 \\
				12 & 793.5 & 793.8 & -0.3 \\
				\hline
			\end{tabular}
			\caption{Difference between Fotobalk and Counter measurements.}
			\label{FCTdata}\end{center}
	\end{table}
	
	
	
	
\chapter{Introduction to Method Comparison Studies}


\newpage
% latex table generated in R 2.6.0 by xtable 1.5-5 package
% Wed Aug 26 15:22:41 2009
\begin{table}[h!]
	
	\begin{center}
		
		\begin{tabular}{rrrr}
			\hline
			Round& Fotobalk (F) & Counter (C) & F-C \\
			\hline
			1 & 793.8& 794.6 & -0.8 \\
			2 & 793.1 & 793.9 & -0.8 \\
			3 & 792.4 & 793.2 & -0.8 \\
			4 & 794.0 & 794.0 & 0.0 \\
			5 & 791.4 & 792.2 & -0.8 \\
			6 & 792.4 & 793.1 & -0.7 \\
			7 & 791.7 & 792.4 & -0.7 \\
			8 & 792.3 & 792.8 & -0.5 \\
			9 & 789.6 & 790.2 & -0.6 \\
			10 & 794.4 & 795.0 & -0.6 \\
			11 & 790.9 & 791.6 & -0.7 \\
			12 & 793.5 & 793.8 & -0.3 \\
			\hline
		\end{tabular}
		\caption{Difference between Fotobalk and Counter measurements.}
	\end{center}
\end{table}

\bigskip

\newpage


	\section{Replicate Measurements}
	
	Thus far, the formulation for comparison of two measurement
	methods is one where one measurement by each method is taken on
	each subject. Should there be two or more measurements by each
	methods, these measurement are known as `replicate measurements'.
	\citet{BXC2008} recommends the use of replicate measurements, but
	acknowledges the additional computational complexity.
	
	\citet*{BA86} address this problem by offering two different
	approaches. The premise of the first approach is that replicate
	measurements can be treated as independent measurements. The
	second approach is based upon using the mean of the each group of
	replicates as a representative value of that group. Using either
	of these approaches will allow an analyst to estimate the inter
	method bias.
	
	%\subsubsection{Mean of Replicates Limits of Agreement}
	
	However, because of the removal of the effects of the replicate
	measurements error, this would cause the estimation of the
	standard deviation of the differences to be unduly small.
	\citet*{BA86} propose a correction for this.
	
	\citet{BXC2008} takes issue with the limits of agreement based on
	mean values of replicate measurements, in that they can only be interpreted as prediction
	limits for difference between means of repeated measurements by
	both methods, as opposed to the difference of all measurements.
	Incorrect conclusions would be caused by such a misinterpretation.
	\citet{BXC2008} demonstrates how the limits of agreement
	calculated using the mean of replicates are `much too narrow as
	prediction limits for differences between future single
	measurements'. This paper also comments that, while treating the
	replicate measurements as independent will cause a downward bias
	on the limits of agreement calculation, this method is preferable
	to the `mean of replicates' approach.
	


	\chapter{Extending Current Methodologies}
	\section{Extension of Roy's methodology}
	Roy's methodology is constructed to compare two methods in the presence of replicate measurements. Necessarily it is worth examining whether this methodology can be adapted for different circumstances.
	
	An implementation of Roy's methodology, whereby three or more methods are used, is not feasible due to computational restrictions. Specifically there is a failure to reach convergence before the iteration limit is reached. This may be due to the presence of additional variables, causing the problem of non-identifiability. In the case of two variables, it is required to estimate two variance terms and four correlation terms, six in all. For the case of three variabilities, three variance terms must be estimated as well as nine correlation terms, twelve in all. In general for $n$ methods has $2 \times T_{n}$ variance terms, where $T_n$ is the triangular number for $n$, i.e. the addition analogue of the factorial. Hence the computational complexity quite increases substantially for every increase in $n$.
	
	Should an implementation be feasible, further difficulty arises when interpreting the results. The fundamental question is whether two methods have close agreement so as to be interchangeable. When three methods are present in the model, the null hypothesis is that all three methods have the same variability relevant to the respective tests. The outcome of the analysis will either be that all three are interchangeable or that all three are not interchangeable.
	
	The tests would not be informative as to whether any two of those three were interchangeable, or equivalently if one method in particular disagreed with the other two. Indeed it is easier to perform three pair-wise comparisons separately and then to combine the results.
	
	Roy's methodology is not suitable for the case of single measurements because it follows from the decomposition for the covariance matrix of the response vector $y_{i}$, as presented in \citet{hamlett}. The decomposition depends on the estimation of correlation terms, which would be absent in the single measurement case. Indeed there can be no within-subject variability if there are no repeated terms for it to describe. There would only be the covariance matrix of the measurements by both methods, which doesn't require the use of LME models. To conclude, simpler existing methodologies, such as Deming regression, would be the correct approach where there only one measurements by each method.
	
	\section{Conclusion}
	\citet{BXC2008} and \citet{roy} highlight the need for method comparison methodologies suitable for use in the presence of replicate measurements. \citet{roy} presents a comprehensive methodology for assessing the agreement of two methods, for replicate measurements. This methodology has the added benefit of overcoming the problems of unbalanced data and unequal numbers of replicates. Implementation of the methodology, and interpretation of the results, is relatively easy for practitioners who have only basic statistical training. Furthermore, it can be shown that widely used existing methodologies, such as the limits of agreement, can be incorporated into Roy's methodology.
	
	
	\newpage
	\section{Outline of Thesis}
	In the first chapter the study of method comparison is introduced, while the second chapter provides a review of current methodologies. The intention of this thesis is to progress the
	study of method comparison studies, using a statistical method known as Linear mixed effects models.
	Chapter three shall describes linear mixed effects models, and how the use of the linear mixed
	effects models have so far extended to method comparison studies. Implementations of important existing work shall be presented, using the \texttt{R} programming language.
	
	Model diagnostics are an integral component of a complete statistical analysis.
	In chapter three model diagnostics shall be described in depth, with particular
	emphasis on linear mixed effects models, further to chapter two.
	
	For the fourth chapter, important linear mixed effects model diagnostic methods shall be extended to method comparison studies, and proposed methods shall be demonstrated on data sets that have become well known in literature on method comparison. The purpose is to both calibrate these methods and to demonstrate applications for them.
	The last chapter shall focus on robust measures of important parameters such as agreement.
	\addcontentsline{toc}{section}{Bibliography}
	
	\bibliography{DB-txfrbib}
\end{document}
