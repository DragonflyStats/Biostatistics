\documentclass[12pt, a4paper]{report}
\usepackage{epsfig}
\usepackage{subfigure}
%\usepackage{amscd}
\usepackage{amssymb}
\usepackage{graphicx}
%\usepackage{amscd}
\usepackage{amssymb}
\usepackage{subfiles}
\usepackage{framed}
\usepackage{amsthm, amsmath}
\usepackage{amsbsy}
\usepackage[usenames]{color}
\usepackage{listings}
\lstset{% general command to set parameter(s)
	basicstyle=\small, % print whole listing small
	keywordstyle=\color{red}\itshape,
	% underlined bold black keywords
	commentstyle=\color{blue}, % white comments
	stringstyle=\ttfamily, % typewriter type for strings
	showstringspaces=false,
	numbers=left, numberstyle=\tiny, stepnumber=1, numbersep=5pt, %
	frame=shadowbox,
	rulesepcolor=\color{black},
	,columns=fullflexible
} %
%\usepackage[dvips]{graphicx}
\usepackage{natbib}
\bibliographystyle{chicago}
\usepackage{vmargin}
% left top textwidth textheight headheight
% headsep footheight footskip
\setmargins{3.0cm}{2.5cm}{15.5 cm}{22cm}{0.5cm}{0cm}{1cm}{1cm}
\renewcommand{\baselinestretch}{1.5}
\pagenumbering{arabic}
\theoremstyle{plain}
\newtheorem{theorem}{Theorem}[section]
\newtheorem{corollary}[theorem]{Corollary}
\newtheorem{ill}[theorem]{Example}
\newtheorem{lemma}[theorem]{Lemma}
\newtheorem{proposition}[theorem]{Proposition}
\newtheorem{conjecture}[theorem]{Conjecture}
\newtheorem{axiom}{Axiom}
\theoremstyle{definition}
\newtheorem{definition}{Definition}[section]
\newtheorem{notation}{Notation}
\theoremstyle{remark}
\newtheorem{remark}{Remark}[section]
\newtheorem{example}{Example}[section]
\renewcommand{\thenotation}{}
\renewcommand{\thetable}{\thesection.\arabic{table}}
\renewcommand{\thefigure}{\thesection.\arabic{figure}}
\title{Research notes: linear mixed effects models}
\author{ } \date{ }


\begin{document}
	\author{Kevin O'Brien}
	\title{Mixed Models for Method Comparison Studies}
	\tableofcontents
	
	%----------------------------------------------------------------------------------------%
	\newpage
	\chapter{Method Comparison Studies}
	
	\section{Introduction}
	The problem of assessing the agreement between two or more methods
	of measurement is ubiquitous in scientific research, and is
	commonly referred to as a `method comparison study'. Published
	examples of method comparison studies can be found in disciplines
	as diverse as pharmacology \citep{ludbrook97}, anaesthesia
	\citep{Myles}, and cardiac imaging methods \citep{Krumm}.
	\smallskip
	
	To illustrate the characteristics of a typical method comparison
	study consider the data in Table I \citep{Grubbs73}. In each of
	twelve experimental trials, a single round of ammunition was fired
	from a 155mm gun and its velocity was measured simultaneously (and
	independently) by three chronographs devices, identified here by
	the labels `Fotobalk', `Counter' and `Terma'.
	\smallskip
	
	
	\newpage
	
	\begin{table}[ht]
		\begin{center}
			\begin{tabular}{rrrr}
				\hline
				Round& Fotobalk [F] & Counter [C]& Terma [T]\\
				\hline
				1 & 793.8 & 794.6 & 793.2 \\
				2 & 793.1 & 793.9 & 793.3 \\
				3 & 792.4 & 793.2 & 792.6 \\
				4 & 794.0 & 794.0 & 793.8 \\
				5 & 791.4 & 792.2 & 791.6 \\
				6 & 792.4 & 793.1 & 791.6 \\
				7 & 791.7 & 792.4 & 791.6 \\
				8 & 792.3 & 792.8 & 792.4 \\
				9 & 789.6 & 790.2 & 788.5 \\
				10 & 794.4 & 795.0 & 794.7 \\
				11 & 790.9 & 791.6 & 791.3 \\
				12 & 793.5 & 793.8 & 793.5 \\
				\hline
			\end{tabular}
			\caption{Velocity measurement from the three chronographs (Grubbs
				1973).}
		\end{center}
	\end{table}
	
	An important aspect of the these data is that all three methods of
	measurement are assumed to have an attended measurement error, and
	the velocities reported in Table 1.1 can not be assumed to be
	`true values' in any absolute sense.
	
	%While lack of
	%agreement between two methods is inevitable, the question , as
	%posed by \citet{BA83}, is 'do the two methods of measurement agree
	%sufficiently closely?'
	
	A method of measurement should ideally be both accurate and
	precise. \citet{Barnhart} describes agreement as being a broader
	term that contains both of those qualities. An accurate
	measurement method will give results close to the unknown `true
	value'. The precision of a method is indicated by how tightly
	measurements obtained under identical conditions are distributed
	around their mean measurement value. A precise and accurate method
	will yield results consistently close to the true value. Of course
	a method may be accurate, but not precise, if the average of its
	measurements is close to the true value, but those measurements
	are highly dispersed. Conversely a method that is not accurate may
	be quite precise, as it consistently indicates the same level of
	inaccuracy. The tendency of a method of measurement to
	consistently give results above or below the true value is a
	source of systematic bias. The smaller the systematic bias, the
	greater the accuracy of the method.
	
	% The FDA define precision as the closeness of agreement (degree of
	% scatter) between a series of measurements obtained from multiple
	% sampling of the same homogeneous sample under prescribed
	% conditions. \citet{Barnhart} describes precision as being further
	% subdivided as either within-run, intra-batch precision or
	% repeatability (which assesses precision during a single analytical
	% run), or between-run, inter-batch precision or repeatability
	%(which measures precision over time).
	
	In the context of the agreement of two methods, there is also a
	tendency of one measurement method to consistently give results
	above or below the other method. Lack of agreement is a
	consequence of the existence of `inter-method bias'. For two
	methods to be considered in good agreement, the inter-method bias
	should be in the region of zero. A simple estimate of the
	inter-method bias is given by the differences between pairs of measurements, for example,  Table~\ref{FCTdata} is a good example of
	possible inter-method bias; the `Fotobalk' consistently recording
	smaller velocities than the `Counter' method. A cursory inspection of the table will indicate a systematic tendency for the Counter method to result in higher measurements than the Fotobalk method. % Consequently one would conclude that there is lack of agreement % between the two methods.
	
	The absence of inter-method bias is, by itself, not sufficient to
	establish that two measurement methods agree. The two methods
	must also have equivalent levels of precision. Should one method
	yield results considerably more variable than those of the other,
	they can not be considered to be in agreement. Hence, method comparison studies are required to take account of both inter-method bias and difference in precision of measurements.
	\newpage
	% latex table generated in R 2.6.0 by xtable 1.5-5 package
	% Wed Aug 26 15:22:41 2009
	\begin{table}[h!]
		
		\begin{center}
			
			\begin{tabular}{rrrr}
				\hline
				Round& Fotobalk (F) & Counter (C) & Difference (F-C) \\
				\hline
				1 & 793.8& 794.6 & -0.8 \\
				2 & 793.1 & 793.9 & -0.8 \\
				3 & 792.4 & 793.2 & -0.8 \\
				4 & 794.0 & 794.0 & 0.0 \\
				5 & 791.4 & 792.2 & -0.8 \\
				6 & 792.4 & 793.1 & -0.7 \\
				7 & 791.7 & 792.4 & -0.7 \\
				8 & 792.3 & 792.8 & -0.5 \\
				9 & 789.6 & 790.2 & -0.6 \\
				10 & 794.4 & 795.0 & -0.6 \\
				11 & 790.9 & 791.6 & -0.7 \\
				12 & 793.5 & 793.8 & -0.3 \\
				\hline
			\end{tabular}
			\caption{Difference between Fotobalk and Counter measurements.}
			\label{FCTdata}\end{center}
	\end{table}
	
	
	
	
	\chapter{Review of Current Methodologies}
	\section{Bland-Altman Approach}
	The issue of whether two measurement methods comparable to the
	extent that they can be used interchangeably with sufficient
	accuracy is encountered frequently in scientific research.
	Historically, comparison of two methods of measurement was carried
	out by use of paired sample $t-$test, correlation coefficients or
	simple linear regression. However, simple linear regression is unsuitable for method comparison studies due to the assumption that one variable is measured without error. In comparing two methods, both methods are assume to have attendant random error.
	
	\citet{BA83} highlighted the inadequacies of these approaches for comparing two methods of measurement, and proposed methodologies with this specific application in mind. Although the authors also acknowledge the opportunity to apply other, more complex, approaches, but argue that simpler approaches is preferable, especially when the
	results must be `explained to non-statisticians'.
	
	Notwithstanding previous remarks about linear regression, the first step recommended, which the authors argue should be mandatory, is the construction of a scatter plot of the data. Scatterplots can facilitate an initial judgement and
	helping to identify potential outliers, with the addition of the line of equality. In the case of good agreement, the observations would be distributed closely along this line. However, they are not useful for a thorough examination of the data. \citet{BritHypSoc} notes that
	data points will tend to cluster around the line of equality, obscuring interpretation.
	
	
	A scatter plot of the Grubbs data is shown in Figure 1.1. Visual inspection confirms the previous conclusion that inter-method bias is present, i.e. the Fotobalk device has a tendency to record a lower velocity.
	
	\begin{figure}[h!]
		\begin{center}
			\includegraphics[width=125mm]{images/GrubbsScatter.jpeg}
			\caption{Scatter plot for Fotobalk and Counter methods.}\label{GrubbsScatter}
		\end{center}
	\end{figure}
	
	\citet{Dewitte} notes that scatter plots were very seldom
	presented in the Annals of Clinical Biochemistry. This apparently
	results from the fact that the `Instructions for Authors' dissuade
	the use of regression analysis, which conventionally is
	accompanied by a scatter plot.
	
	\newpage
	\subsection{Bland-Altman plots}
	
	In light of shortcomings associated with scatterplots,
	\citet*{BA83} recommend a further analysis of the data. Firstly
	case-wise differences of measurements of two methods $d_{i} =
	y_{1i}-y_{2i}, \mbox{ for }i=1,2,\dots,n$, on the same subject
	should be calculated, and then the average of those measurements, 
	($a_{i} = (y_{1i} + y_{2i})/2 \mbox{ for }i=1,2,\dots, n$.
	
	\citet{BA83} proposed that $a_i$ should be plotted against $d_i$, a plot now widely known as the Bland-Altman plot, and motivated this plot as follows:
	\begin{quote}
		``From this type of plot it is much easier to assess the magnitude
		of disagreement (both error and bias), spot outliers, and see
		whether there is any trend, for example an increase in (difference) for high values. This way of plotting the data is a very powerful way of displaying the results of a method comparison study."
	\end{quote}
	
	The case wise-averages capture several aspects of the data, such as expressing the range over which the values were taken, and assessing whether the assumptions of constant variance holds.
	Case-wise averages also allow the case-wise differences to be presented on a two-dimensional plot, with better data visualization qualities than a one dimensional plot. \citet{BA86}
	cautions that it would be the difference against either measurement value instead of their average, as the difference relates to both value. This approach has proved very popular, and the Bland-Altman plots is widely regarded as powerful graphical tool for making a visual assessment of the data.
	
	The magnitude of the inter-method bias between the two methods is simply the average of the differences $\bar{d}$. This inter-method bias is represented with a line on the Bland-Altman plot. As the objective of the Bland-Altman plot is to advise on the agreement of two methods, the individual case-wise differences are also particularly relevant. The variances around this bias is estimated by the standard deviation of these differences $S_{d}$.
	
	\subsection{Bland-Altman plots for the Grubbs data}
	
	In the case of the Grubbs data the inter-method bias is $-0.61$ metres per second, and is indicated by the dashed line on Figure 1.2. By inspection of the plot, it is also possible to compare the precision of each method. Noticeably the differences tend to increase as the averages increase.
	
	
	The Bland-Altman plot for comparing the `Fotobalk' and `Counter'
	methods, which shall henceforth be referred to as the `F vs C'
	comparison,  is depicted in Figure 1.2, using data from Table 1.3.
	The presence and magnitude of the inter-method bias is indicated
	by the dashed line.
	\newpage
	
	%Later it will be shown that case-wise differences are the sole
	%component of the next part of the methodology, the limits of
	%agreement.
	
	
	\begin{table}[h!]
		\renewcommand\arraystretch{0.7}%
		\begin{center}
			\begin{tabular}{|c||c|c||c|c|}
				\hline
				Round & Fotobalk  & Counter  & Differences  & Averages  \\
				&  [F] & [C] & [F-C] &  [(F+C)/2] \\
				\hline
				1 & 793.8 & 794.6 & -0.8 & 794.2 \\
				2 & 793.1 & 793.9 & -0.8 & 793.5 \\
				3 & 792.4 & 793.2 & -0.8 & 792.8 \\
				4 & 794.0 & 794.0 & 0.0 & 794.0 \\
				5 & 791.4 & 792.2 & -0.8 & 791.8 \\
				6 & 792.4 & 793.1 & -0.7 & 792.8 \\
				7 & 791.7 & 792.4 & -0.7 & 792.0 \\
				8 & 792.3 & 792.8 & -0.5 & 792.5 \\
				9 & 789.6 & 790.2 & -0.6 & 789.9 \\
				10 & 794.4 & 795.0 & -0.6 & 794.7 \\
				11 & 790.9 & 791.6 & -0.7 & 791.2 \\
				12 & 793.5 & 793.8 & -0.3 & 793.6 \\
				\hline
			\end{tabular}
			\caption{Fotobalk and Counter methods: differences and averages.}
		\end{center}
	\end{table}
	
	\begin{table}[h!]
		\renewcommand\arraystretch{0.7}%
		\begin{center}
			\begin{tabular}{|c||c|c||c|c|}
				\hline
				Round & Fotobalk  & Terma  & Differences  & Averages  \\
				&  [F] & [T] & [F-T] &  [(F+T)/2] \\
				\hline
				1 & 793.8 & 793.2 & 0.6 & 793.5 \\
				2 & 793.1 & 793.3 & -0.2 & 793.2 \\
				3 & 792.4 & 792.6 & -0.2 & 792.5 \\
				4 & 794.0 & 793.8 & 0.2 & 793.9 \\
				5 & 791.4 & 791.6 & -0.2 & 791.5 \\
				6 & 792.4& 791.6 & 0.8 & 792.0 \\
				7 & 791.7 & 791.6 & 0.1 & 791.6 \\
				8 & 792.3 & 792.4 & -0.1 & 792.3 \\
				9 & 789.6 & 788.5 & 1.1 & 789.0 \\
				10 & 794.4 & 794.7 & -0.3 & 794.5 \\
				11 & 790.9 & 791.3 & -0.4 & 791.1 \\
				12 & 793.5 & 793.5 & 0.0 & 793.5 \\
				
				\hline
			\end{tabular}
			\caption{Fotobalk and Terma methods: differences and averages.}
		\end{center}
	\end{table}
	
	\newpage
	
	\begin{figure}[h!]
		\begin{center}
			\includegraphics[width=120mm]{images/GrubbsBAplot-noLOA.jpeg}
			\caption{Bland-Altman plot For Fotobalk and Counter methods.}\label{GrubbsBA-noLOA}
		\end{center}
	\end{figure}
	
	
	
	In Figure 1.3 Bland-Altman plots for the `F vs C' and `F vs T'
	comparisons are shown, where `F vs T' refers to the comparison of
	the `Fotobalk' and `Terma' methods. Usage of the Bland-Altman plot
	can be demonstrate in the contrast between these comparisons. By inspection, there exists a larger inter-method bias in the `F vs C' comparison than in the `F vs T' comparison. Conversely there
	appears to be less precision in `F vs T' comparison, as indicated
	by the greater dispersion of covariates.
	
	\begin{figure}[h!]
		\begin{center}
			\includegraphics[height=90mm]{images/GrubbsDataTwoBAplots.jpeg}
			\caption{Bland-Altman plots for Grubbs' F vs C and F vs T comparisons.}\label{GrubbsDataTwoBAplots}
		\end{center}
	\end{figure}
	
	\newpage
	
	\subsection{Prevalence of the Bland-Altman plot}
	\citet*{BA86}, which further develops the Bland-Altman approach,
	was found to be the sixth most cited paper of all time by \citet{BAcite}. \cite{Dewitte} describes the rate at which
	prevalence of the Bland-Altman plot has developed in scientific
	literature, by examining all articles in the journal `Clinical Chemistry'
	between 1995 and 2001. This study concluded that use of the
	Bland�Altman plot increased over the years, from 8\% in 1995 to
	14\% in 1996, and 31�36\% in 2002.
	
	The Bland-Altman Plot has since become expected, and
	often obligatory, approach for presenting method comparison
	studies in many scientific journals \citep{hollis}. Furthermore
	\citet{BritHypSoc} recommend its use in papers pertaining to
	method comparison studies for the journal of the British
	Hypertension Society.
	
	\subsection{Adverse features}
	
	Estimates for inter-method bias and variance of differences are only meaningful if there is uniform inter-bias and variability throughout the range of measurements. Fulfilment of these assumptions can be checked by visual inspection of the plot.The prototype Bland-Altman plots depicted in Figures 1.4, 1.5 and 1.6 are derived from simulated data, for the purpose of demonstrating how the plot would inform an analyst of features that would adversely affect use of the recommended approach.
	
	Figure 1.4 demonstrates how the Bland-Altman plot would indicate
	increasing variance of differences over the measurement range.
	Fitted regression lines, for both the upper and lower half of the
	plot, has been added to indicate the trend. Figure 1.5 is an
	example of cases where the inter-method bias changes over the
	measurement range. This is known as proportional bias, and is
	defined by \citet{ludbrook97} as meaning that `one method gives
	values that are higher (or lower) than those from the other by an
	amount that is proportional to the level of the measured
	variable'. In both Figures 1.4 and 1.5, the assumptions necessary
	for further analysis using the limits of agreement are violated.
	
	Application of regression techniques to the Bland-Altman plot, and
	subsequent formal testing for the constant variability of
	differences is informative. The data set may be divided into two
	subsets, containing the observations wherein the difference values
	are less than and greater than the inter-method bias respectively.
	For both of these fits, hypothesis tests for the respective slopes
	can be performed. While both tests could be considered separately,
	multiple comparison procedures, such as the Benjamini-Hochberg
	\citep{BH} test, are advisable.
	
	\begin{figure}[h!]
		\begin{center}
			\includegraphics[height=90mm]{images/BAFanEffect.jpeg}
			\caption{Bland-Altman plot demonstrating the increase of variance over the range.}\label{BAFanEffect}
		\end{center}
	\end{figure}
	
	\begin{figure}[h!]
		\begin{center}
			\includegraphics[height=90mm]{images/PropBias.jpeg}
			\caption{Bland-Altman plot indicating the presence of proportional bias.}\label{PropBias}
		\end{center}
	\end{figure}
	
	\begin{figure}[h!]
		\begin{center}
			\includegraphics[width=125mm]{images/BAOutliers.jpeg}
			\caption{Bland-Altman plot indicating the presence of potential outliers.}\label{Outliers}
		\end{center}
	\end{figure}
	
	\newpage
	
	
	The Bland-Altman plot also can be used to identify outliers. An
	outlier is an observation that is conspicuously different from the
	rest of the data that it arouses suspicion that it occurs due to a
	mechanism, or conditions, different to that of the rest of the
	observations. \citet*{BA99} do not recommend excluding outliers from analyses,
	but remark that recalculation of the inter-method bias estimate,
	and further calculations based upon that estimate, are useful for
	assessing the influence of outliers. The authors remark that `we
	usually find that this method of analysis is not too sensitive to
	one or two large outlying differences'. Figure 1.6 demonstrates how the Bland-Altman
	plot can be used to visually inspect the presence of potential
	outliers.
	
	As a complement to the Bland-Altman plot, \citet{Bartko} proposes
	the use of a bivariate confidence ellipse, constructed for a
	predetermined level. \citet{AltmanEllipse} provides the relevant calculations for the
	ellipse. This ellipse is intended as a visual
	guidelines for the scatter plot, for detecting outliers and to
	assess the within- and between-subject variances.
	
	The minor axis relates to the between subject variability, whereas
	the major axis relates to the error mean square, with the ellipse
	depicting the size of both relative to each other.
	Consequently Bartko's ellipse provides a visual aid to determining the
	relationship between variances. If $\mbox{var}(a)$ is greater than $\mbox{var}(d)$, the orientation of the ellipse is horizontal. Conversely if $\mbox{var}(a)$ is less than $\mbox{var}(d)$, the orientation of the ellipse is vertical.
	
	
	%(Furthermore \citet{Bartko}
	%proposes formal testing procedures, that shall be discussed in due
	%course.)
	
	The Bland-Altman plot for the Grubbs data, complemented by Bartko's ellipse, is depicted in Figure 1.7.
	The fourth observation is shown to be outside the bounds of the ellipse, indicating that it is a potential outlier.
	
	
	\begin{figure}[h!]
		% Requires \usepackage{graphicx}
		\includegraphics[width=130mm]{images/GrubbsBartko.jpeg}
		\caption{Bartko's Ellipse for Grubbs' data.}\label{GrubbsBartko}
	\end{figure}
	
	The limitations of using bivariate approaches to outlier detection
	in the Bland-Altman plot can demonstrated using Bartko's ellipse.
	A covariate is added to the `F vs C' comparison that has a
	difference value equal to the inter-method bias, and an average
	value that markedly deviates from the rest of the average values
	in the comparison, i.e. 786. Table 1.8 depicts a $95\%$ confidence
	ellipse for this manipulated data set. By inspection of the
	confidence interval, we would conclude that this extra
	covariate is an outlier, in spite of the fact that this
	observation is very close to the inter-method bias as determined by this approach.
	
	\begin{figure}[h!]
		% Requires \usepackage{graphicx}
		\includegraphics[width=130mm]{images/GrubbsBartko2.jpeg}
		\caption{Bartko's Ellipse for Grubbs' data, with an extra covariate.}\label{GrubbsBartko2}
	\end{figure}
	
	
	Importantly, outlier classification must be informed by the logic of the
	mechanism that produces the data. In the Bland-Altman plot, the horizontal displacement (i.e. the average) of any
	observation is supported by two separate measurements. Any
	observation should not be considered an outlier on the basis of a
	noticeable horizontal displacement from the main cluster, as in
	the case with the extra covariate. Conversely, the fourth
	observation, from the original data set, should be considered an
	outlier, as it has a noticeable vertical displacement from the
	rest of the observations.
	
	%Grubbs' test is a statistical test used for detecting outliers in a
	%univariate data set that is assumed to be normally distributed.
	
	%\citet{Grubbs} defined an outlier as a co-variate that appears to
	%deviate markedly from other members of the sample in which it
	%occurs.
	
	In classifying whether a observation from a univariate data set is
	an outlier, many formal tests are available, such as the Grubbs test for outliers. In assessing
	whether a covariate in a Bland-Altman plot is an outlier, this
	test is useful when applied to the case-wise difference values treated as a
	univariate data set. The null hypothesis of the Grubbs test procedure is the absence
	of any outliers in the data set. Conversely, the alternative hypotheses is that there is at least one outlier
	present.
	
	The test statistic for the Grubbs test ($G$) is the largest
	absolute deviation from the sample mean divided by the standard
	deviation of the differences,
	\[
	G =  \displaystyle\max_{i=1,\ldots, n}\frac{\left \vert d_i -
		\bar{d}\right\vert}{S_{d}}.
	\]
	
	For the `F vs C' comparison it is the fourth observation gives
	rise to the test statistic, $G = 3.64$. The critical value is
	calculated using Student's $t$ distribution and the sample size,
	\[
	U = \frac{n-1}{\sqrt{n}} \sqrt{\frac{t_{\alpha/(2n),n-2}^2}{n - 2
			+ t_{\alpha/(2n),n-2}^2}}.
	\]
	For this test $U = 0.75$. The conclusion of this test is that the fourth observation in the `F vs C' comparison is an outlier, with $p-$value = 0.003, in accordance with the previous result of Bartko's ellipse.
	
	
	\section{Limits of Agreement}
	% introduces
	A third element of the Bland-Altman approach, an interval known
	as `limits of agreement' is introduced in \citet*{BA86}
	(sometimes referred to in literature as 95\% limits of agreement).
	Limits of agreement are used to assess whether the two methods of
	measurement can be used interchangeably. \citet{BA86} refer to
	this as the `equivalence' of two measurement methods. The specific question to which limits of
	agreement are intended as the answer to must be
	established clearly. \citet*{BA95} comment that the limits of agreement show `how
	far apart measurements by the two methods were likely to be for
	most individuals', a definition echoed in their 1999 paper:
	
	\begin{quote}``We can then say that nearly all pairs
		of measurements by the two methods will be closer together than
		these extreme values, which we call 95\% limits of agreement.
		These values define the range within which most differences
		between measurements by the two methods will lie."
	\end{quote}
	
	The limits of agreement (LoA) are computed by the following
	formula:
	\[
	LoA = \bar{d} \pm 1.96 s_{d}
	\]
	with $\bar{d}$ as the estimate of the inter method bias, $s_{d}$
	as the standard deviation of the differences and 1.96 (sometimes rounded to 2) is the 95\%
	quantile for the standard normal distribution. The limits of agreement methodology assumes a constant level of bias throughout the range of measurements. Importantly the authors recommend prior determination of what would constitute acceptable
	agreement, and that sample sizes should be predetermined to give an accurate conclusion. However \citet{mantha} highlight inadequacies in the correct application of limits of agreement, resulting in contradictory estimates of limits of agreement in various papers.
	
	%\begin{quote}
	%``How far apart measurements can be without causing difficulties
	%will be a question of judgment. Ideally, it should be defined in
	%advance to help in the interpretation of the method comparison and
	%to choose the sample size \citep{BA86}".
	%\end{quote}
	
	
	For the Grubbs `F vs C' comparison, these limits
	of agreement are calculated as -0.132 for the upper bound, and
	-1.08 for the lower bound. Figure 1.9 shows the resultant
	Bland-Altman plot, with the limits of agreement shown in dashed
	lines.
	
	
	\begin{figure}[h!]
		\begin{center}
			\includegraphics[width=125mm]{images/GrubbsBAplot-LOA.jpeg}
			\caption{Bland-Altman plot with limits of agreement}\label{GrubbsBAplot-noLOA}
		\end{center}
	\end{figure}
	
	%But as \citet*{BA86} point out this may not be the case. Variants of the limits of agreement that overcome this
	% problem shall be introduced in due course.
	
	\subsection{Inferences on Bland-Altman estimates}
	\citet*{BA99} advises on how to calculate confidence intervals for the inter-method bias and limits of agreement.
	For the inter-method bias, the confidence interval is a simply that of a mean: $\bar{d} \pm t_{(\alpha/2,n-1)} S_{d}/\sqrt{n}$.
	The confidence
	intervals and standard error for the limits of agreement follow from the variance of the limits of agreement, which is shown to be
	
	\[
	\mbox{Var}(LoA) = (\frac{1}{n}+\frac{1.96^{2}}{2(n-1)})s_{d}^{2}.
	\]
	
	If $n$ is sufficiently large this can be following approximation
	can be used
	\[
	\mbox{Var}(LoA) \approx 1.71^{2}\frac{s_{d}^{2}}{n}.
	\]
	Consequently the standard errors of both limits can be
	approximated as $1.71$ times the standard error of the
	differences.
	
	A $95\%$ confidence interval can be determined, by means of the
	\emph{t} distribution with $n-1$ degrees of freedom. However, \citet*{BA99} comment that such calculations  may be `somewhat optimistic' on account of the associated assumptions not being realized.
	
	%\subsubsection{Small Sample Sizes} The limits of agreement are
	%estimates derived from the sample studied, and will differ from
	%values relevant to the whole population, hence the importance of a
	%suitably large sample size. A different sample would give
	%different limits of agreement. Student's t-distribution is a well
	%known probability distribution used in statistical inference for
	%normally distributed populations when the sample size is small
	%\citep{student,Fisher3}. Consequently, using 't' quantiles , as
	%opposed to standard normal quantiles, may give a more appropriate
	%calculation for limits of agreement when the sample size is small.
	%For sample size $n=12$ the `t' quantile is 2.2 and the limits of
	%agreement are (-0.074,-1.143).
	
	
	\subsection{Formal definition of limits of agreement}
	\citet{BA99} note the similarity of limits of agreement to
	confidence intervals, but are clear that they are not the same
	thing. Interestingly, they describe the limits as `being like a
	reference interval'.
	
	Limits of agreement have very similar construction to Shewhart
	control limits. The Shewhart chart is a well known graphical
	methodology used in statistical process control. Consequently
	there is potential for misinterpreting the limits of agreement as
	they were Shewhart control limits. 
	%Importantly the
	%parameters used to determine the Shewhart limits are time ordered, based on the process's historical values, a key difference with Bland-Altman limits of agreement.
	
	\citet{BXC2008} regards the limits of agreement as a prediction
	interval for the difference between future measurements with the
	two methods on a new individual, but states that it does not fit
	the formal definition of a prediction interval, since the
	definition does not consider the errors in estimation of the
	parameters. Prediction intervals, which are often used in
	regression analysis, are estimates of an interval in which future
	observations will fall, with a certain probability, given what has
	already been observed. \citet{BXC2008} offers an alternative
	formulation, a $95\%$ prediction interval for the difference
	\[
	\bar{d} \pm t_{(0.025, n-1)}s_{d} \sqrt{1+\frac{1}{n}}
	\]
	
	\noindent where $n$ is the number of subjects. Carstensen is
	careful to consider the effect of the sample size on the interval
	width, adding that only for 61 or more subjects is the
	quantile less than 2.
	
	\citet{luiz} offers an alternative description of limits of
	agreement, this time as tolerance limits. A tolerance interval for
	a measured quantity is the interval in which a specified fraction
	of the population's values lie, with a specified level of
	confidence. \citet{Barnhart} describes them as a probability
	interval, and offers a clear description of how they should be
	used; `if the absolute limit is less than an acceptable difference
	$d_{0}$, then the agreement between the two methods is deemed
	satisfactory'.
	
	The prevalence of contradictory definitions of what limits of agreement strictly are will inevitably attenuate the poor standard of reporting using limits of agreement, as mentioned by \citet{mantha}.
	
	%At least 100 historical
	%values must be used to determine the acceptable value (i.e the
	%process mean) and the process standard deviation. The principle
	%that the mean and variance of a large sample of a homogeneous
	%population is a close approximation of the population's mean and
	%variance justifies this.
	
	%\begin{figure}[h!]
	%\begin{center}
	%  \includegraphics[width=125mm]{images/GrubbsLOAwCIs.jpeg}
	%  \caption{Limits of agreement with confidence intervals}\label{LOAwCIs}
	%\end{center}
	%\end{figure}
	
	%\newpage
	%\section{Agreement Indices}
	%\citet{Barnhart} provided an overview of several agreement
	%indices, including the limits of agreement. Other approaches, such
	%as mean squared deviation, the tolerance deviation index and
	%coverage probability are also discussed.
	
	\subsection{Alternative Agreement Indices}
	As an alternative to limits of agreement, \citet{lin2002} proposes the use of
	the mean square deviation in assessing agreement. The mean square
	deviation is defined as the expectation of the squared differences
	of two readings. The MSD is usually used for the case of two
	measurement methods $X$ and $Y$, each making one measurement for
	the same subject, and is given by
	\[
	MSDxy = E[(x - y)^2]  = (\mu_{x} - \mu_{y})^2 + (\sigma_{x} -
	\sigma_{y})^2 + 2\sigma_{x}\sigma_{y}(1-\rho_{xy}).
	\]
	
	
	\citet{Barnhart} advises the use of a predetermined upper limit
	for the MSD value, $MSD_{ul}$, to define satisfactory agreement.
	However, a satisfactory upper limit may not be easily
	determinable, thus creating a drawback to this methodology.
	
	
	Alternative indices, proposed by \citet{Barnhart}, are the square root of the MSD and the expected absolute difference (EAD). 
	\[
	EAD = E(|x - y|) = \frac{\sum |x_{i}- y_{i}|}{n}
	\]
	
	
	Both of these indices can be interpreted intuitively, since their units are the same as that of the original
	measurements. Also they can be compared to the maximum acceptable
	absolute difference between two methods of measurement $d_{0}$. 
	
	The EAD can be used to supplement the inter-method bias in an
	initial comparison study, as the EAD is informative as a measure
	of dispersion, is easy to calculate and requires no distributional
	assumptions. A consequence of using absolute differences is that high variances would result in a higher EAD value. 
	
	% latex table generated in R 3.1.1 by xtable 1.7-4 package
	% Mon Feb 23 21:12:33 2015
	% latex table generated in R 3.1.1 by xtable 1.7-4 package
	% Mon Feb 23 21:13:45 2015
	% latex table generated in R 3.1.1 by xtable 1.7-4 package
	% Mon Feb 23 22:10:26 2015
	\begin{table}[ht]
		\centering
		\begin{tabular}{rrrrr}
			\hline
			& X & Y & U & V \\ 
			\hline
			1 & 101.83 & 102.52 & 98.05 & 99.53 \\ 
			2 & 101.68 & 102.69 & 99.17 & 96.53 \\ 
			3 & 97.89 & 99.01 & 100.31 & 97.55 \\ 
			4 & 98.15 & 99.57 & 100.35 & 96.03 \\ 
			5 & 99.94 & 100.85 & 99.51 & 99.00 \\ 
			6 & 98.85 & 98.86 & 98.50 & 100.76 \\ 
			7 & 99.86 & 97.85 & 100.66 & 99.37 \\ 
			8 & 101.57 & 100.21 & 99.66 & 108.87 \\ 
			9 & 100.12 & 99.85 & 99.70 & 105.16 \\ 
			10 & 99.49 & 98.77 & 101.55 & 94.31 \\ 
			\hline
		\end{tabular}
	\end{table}
	
	
	\begin{verbatim}
	
	Differences  2.5% limit 97.5% limit    SD(diff) 
	-0.08078844 -2.39471014  2.23313327  1.15696085 
	\end{verbatim}
	
	\begin{table}[ht]
		\centering
		\begin{tabular}{|c|c|c|c|c|}
			\hline
			& X & Y & $X-Y$ & $|X-Y|$ \\ 
			\hline
			1 & 98.05 & 99.53 & -1.49 & 1.49 \\ 
			2 & 99.17 & 96.53 & 2.64 & 2.64 \\ 
			3 & 100.31 & 97.55 & 2.75 & 2.75 \\ 
			4 & 100.35 & 96.03 & 4.32 & 4.32 \\ 
			5 & 99.51 & 99.00 & 0.51 & 0.51 \\ 
			6 & 98.50 & 100.76 & -2.26 & 2.26 \\ 
			7 & 100.66 & 99.37 & 1.29 & 1.29 \\ 
			8 & 99.66 & 108.87 & -9.21 & 9.21 \\ 
			9 & 99.70 & 105.16 & -5.45 & 5.45 \\ 
			10 & 101.55 & 94.31 & 7.24 & 7.24 \\ 
			\hline
		\end{tabular}
		\caption{Example data set}
		\label{EADdata}
	\end{table}
	
	To illustrate the use of EAD, consider table ~\ref{EADdata}. The inter-method bias is 0.03, which is quite close to zero, and conducive to agreement between methods. However, an identity plot would indicate very poor agreement, as the points are noticeably distant from the line of equality.
	\begin{figure}
		\centering
		\includegraphics[width=0.7\linewidth]{images/EADidentity}
		\caption{Identity Plot for example data}
		\label{fig:EADidentity}
	\end{figure}
	
	The limits of agreement are $[-9.61, 9.68]$, a wide interval for this data. As with the identity plot, this would indicate lack of agreement. The EAD is 3.71.
	
	
	The Bland-Altman plot remains a useful part of the analysis. In \ref{fig:EAD1}, it is clear there is a systematic decrease in differences across the range of measurements.
	\begin{figure}
		\centering
		\includegraphics[width=0.7\linewidth]{images/EAD1}
		\caption{Expected Absolute Difference}
		\label{fig:EAD1}
	\end{figure}
	
	\citet{Barnhart} remarks that a comparison of EAD and MSD , using
	simulation studies, would be interesting, while further adding
	that `\textit{It will be of interest to investigate the benefits of these
		possible new unscaled agreement indices}'. For the Grubbs' `F vs C' and `F vs T' comparisons, the inter-method bias, difference variances, limits of agreement and EADs are shown
	in Table 1.5. The corresponding Bland-Altman plots for `F vs C' and `F vs T' comparisons were depicted previously on Figure 1.3. While the inter-method bias for the `F vs T' comparison is smaller, the EAD penalizes the comparison for having a greater variance of differences. Hence the EAD values for both comparisons are much closer.
	\begin{table}[ht]
		\begin{center}
			\begin{tabular}{|c|c|c|}
				\hline
				& F vs C & F vs T  \\
				\hline
				Inter-method bias & -0.61 & 0.12 \\
				Difference variance & 0.06 & 0.22  \\
				Limits of agreement & (-1.08,	-0.13) & (-0.81,1.04) \\
				EAD & 0.61 & 0.35  \\
				\hline
			\end{tabular}
			\caption{Agreement indices for Grubbs' data comparisons.}
		\end{center}
	\end{table}
	
	Further to  \citet{lin2000} and \citet{lin2002}, individual agreement between two measurement methods may be
	assessed using the the coverage probability (CP) criteria or the total deviation index (TDI). If $d_{0}$ is predetermined as the maximum acceptable absolute difference between two methods of measurement, the probability that the absolute difference of two measures being less than $d_{0}$ can be computed. This is known as the coverage probability (CP).
	
	\begin{equation}
	CP = P(|x_{i} - y_{i}| \leq d_{0})
	\end{equation}
	
	If $\pi_{0}$ is set as the predetermined coverage probability, the
	boundary under which the proportion of absolute differences is
	$\pi_{0}$ may be determined. This boundary is known as the `total
	deviation index' (TDI). Hence the TDI is the $100\pi_{0}$
	percentile of the absolute difference of paired observations.
	
	\section{Variations and Alternative Graphical Methods}
	In this section, we will look at some variations and enhancements of the Bland-Altman plot, as well as some alternative graphcial techniques. Strictly speaking, the Identity Plot is advised by Bland and Altman as a prior analysis to the Bland-Alman plot, and therefore is neither a variant nor an alternative approach. However it is worth mentioning, as it is a simple, powerful and elegant technique that is often overlooked in method comparison studies. The identity plot is a simple scatter-plot approach of measurements for both methods on either axis, with the line of equality (the $X=Y$ line, i.e. the 45 degree line through the origin). This plot can gives the analyst a cursory examination of how well the measurement methods agree. In the case of good agreement, the covariates of the plot accord closely with the line of equality.
	
	\subsection{Variants of the Bland-Altman Plot}
	In light of some potential pitfalls associated with the conventional difference plot, a series of alternative formulations for the Bland-Altman approach have been proposed.
	
	Referring to the assumption that bias and variability are constant across the range
	of measurements, \citet{BA99} address the case where there is an increase in variability as the magnitude increases. They remark 	that it is possible to ignore the issue altogether, but the limits of agreement would be wider apart than necessary when just lower magnitude measurements are considered. Conversely the limits would be too narrow should only higher magnitude measurements be used.	To address the issue, they propose the logarithmic transformation of the data. The plot is then formulated as the difference of paired log values against their mean. Bland and Altman acknowledge that this is not easy to interpret, and may not be suitable in all cases.
	
	\subsubsection*{Bland and Altman's Percentage and Ratio Plots}
	%------------------------------------------------------------- %
	\citet{BA99} offer two variations of the Bland-Altman plot intended to overcome situations where the conventional plot is inappropriate. The first variation is a plot of casewise differences as percentage of averages, and is appropriate when the variability of the differences increase as the
	magnitude increases. 
	
	%------------------------------------------------------------- %
	% % RATIO / EKSBORG
	The second variation is a plot of casewise ratios as percentage of averages. This will remove the need for
	logarithmic transformation. This approach is useful when there is an increase in variability of the differences as the magnitude of the measurement increases. \citet{Eksborg} proposed such a ratio plot,
	independently of Bland and Altman. \citet{Dewitte} commented on
	the reception of this article by saying `\textit{Strange to say, this 
		report has been overlooked}'.
	
	
	%	%----------------------------------------------------------------%
	%	\section{Dewitte et al }
	%	\begin{quote}When the standard deviation increases with concentration, Bland and Altman recommend a logarithmic y scale, whereas others propose a percent y scale (Pollock et al, 2002). Although generally there is not much difference in effect between using percentages and using a log transformation of the data, we prefer the percent plot (except when data extend over several orders of magnitude) because numbers can be read directly from the plot without the need for back-transformation.
	%	\end{quote}
	%	
	%	\begin{verbatim}
	%	absolute - small range
	%	percentage - medium range
	%	log scale - large range
	%	\end{verbatim}
	%==================================================== %
	\newpage
	%%%%%%%%%%%%%%%%%%%%%%%%%%%%%%%%%%%%%%%%%%%%%%%%%%%%%%%%%%%%%%%%%%%%%%%%%%%%%%%%%%%%%%%%%%%%%%%%%%%%%%%%%%%%%%%%%%%%%%%%
	
	\subsubsection{Bartko's Ellipse}
	
	As an enhancement on the Bland Altman Plot, \citet{Bartko} has
	expounded a confidence ellipse for the covariates. \citet{Bartko} proposes
	a bivariate confidence ellipse as a boundary for dispersion. The stated purpose is to `amplify dispersion', which presumably is for  the purposes of outlier detection. The orientation of the the ellipse is key to interpreting the results. The minor axis is related to the between-item variability whereas the major axis is related to the mean squared error (referred to here as Error Mean Square).The ellipse illustrates the size of both relative to each
	other. 
	
	
	Consequently Bartko's ellipse provides a visual aid to determining the
	relationship between variances. 
	Furthermore, the ellipse provides a visual aid to determining the relationship
	between the variance of the means $Var(a_{i})$ and the variance of the differences $Var(d_{i})$. If $\mbox{var}(a)$ is greater than $\mbox{var}(d)$, the orientation of the ellipse is horizontal. Conversely if $\mbox{var}(a)$ is less than $\mbox{var}(d)$, the orientation of the ellipse is vertical. The more horizontal the ellipse, the greater the degree of agreement between the two methods being tested.
	
	
	%(Furthermore \citet{Bartko}
	%proposes formal testing procedures, that shall be discussed in due
	%course.)
	Bartko states that the ellipse can, inter alia, be used to detect the presence of outliers (furthermore
	\citet{Bartko} proposes formal testing procedures, that shall be discussed in due course). 
	The Bland-Altman plot for the Grubbs data, complemented by Bartko's ellipse, is depicted in Figure ~\ref{GrubbsBartko1}.
	The fourth observation is shown to be outside the bounds of the ellipse, indicating that it is a potential outlier.
	
	
	\begin{centering}
		\begin{figure}[h!]
			% Requires \usepackage{graphicx}
			\includegraphics[width=130mm]{images/GrubbsBartko.jpeg}
			\caption{Bartko's Ellipse For Grubbs' Data.}
			\label{GrubbsBartko1}
		\end{figure}
	\end{centering}
	
	The limitations of using bivariate approaches to outlier detection
	in the Bland-Altman plot can demonstrated using Bartko's ellipse.
	A covariate is added to the `F vs C' comparison that has a
	difference value equal to the inter-method bias, and an average
	value that markedly deviates from the rest of the average values
	in the comparison, i.e. 786. Table 1.8 depicts a $95\%$ confidence
	ellipse for this manipulated data set. By inspection of the
	confidence interval, a conclusion would be reached that this extra
	covariate is an outlier, in spite of the fact that this
	observation is wholly consistent with the conclusion of the
	Bland-Altman plot.
	
	%\begin{figure}[h!]
	%  % Requires \usepackage{graphicx}
	%  \includegraphics[width=130mm]{images/GrubbsBartko2.jpeg}
	%  \caption{Bartko's Ellipse For Grubbs' Data, with an extra covariate.}\label{GrubbsBartko2}
	%\end{figure}
	
	
	Importantly, outlier classification must be informed by the logic of the
	data's formulation. In the Bland-Altman plot, the horizontal displacement of any
	observation is supported by two independent measurements. Any
	observation should not be considered an outlier on the basis of a
	noticeable horizontal displacement from the main cluster, as in
	the case with the extra covariate. Conversely, the fourth
	observation, from the original data set, should be considered an
	outlier, as it has a noticeable vertical displacement from the
	rest of the observations.
	\newpage
	
	\begin{figure}[h!]
		% Requires \usepackage{graphicx}
		\includegraphics[width=130mm]{images/GrubbsBartko2.jpeg}
		\caption{Bartko's Ellipse For Grubbs' Data, with an extra covariate.}\label{GrubbsBartko2}
	\end{figure}
	
	In the Bland-Altman plot, the horizontal displacement of any point on the plot is supported by two independent measurements. Any point should not be considered an outlier on the basis of a noticeable horizontal displacement from the main cluster, as in the case with the extra co-variate. Conversely, the fourth point, from the original data set, should be considered an
	outlier, as it has a noticeable vertical displacement from the rest of the observations.
	\newpage
	
	
	\subsubsection{Survival-Agreement Plot}
	A graphical technique for method comparison studies, that is entirely different to the Bland-Altman plot, was proposed by \citet{luiz}. This approach, known as the survival-agreement plot, is used to determine the degree of agreement using the Kaplan-Meier method, a well known graphical technique in the area of Survival Analysis. Furthermore \citet{luiz} propose that commonly used survival analysis techniques should complement this method,\textit{ providing a new analytical insight
		for agreement}. Two survival?agreement plots are used to detect the bias between to measurements of the same variable. The presence of inter-method bias is tested with the log-rank test, and its magnitude with Cox regression.
	
	%% TOLERANCE - REWRITE THIS
	
	The degree of agreement (or disagreement) of a measure is expressed as a function of several limits of tolerance, using the Kaplan-Meier method, where the failures occur exactly at absolute values of the differences between the two methods of measurement. 
	
	According to Luiz et al, the survival-agreement plot is a step function of a typical survival analysis without censored data, where the Y axis represents the proportion of discordant cases. This is equivalent to a step function where the X axis represents the absolute  observed differences and the Y axis is the proportion of the cases with at least the observed 
	difference ($x_i$). 
	
	% % PREVALENCE
	% % Implementation
	
	
	%============================================================================================================ %
	
	
	
	
	% MCS Mountain Plot Notebook
	
	\subsubsection{Mountain Plot} Krouwer and Monti have proposed a folded empirical cumulative distribution plot, otherwise known as a Mountain plot.
	
	They argue that it is suitable for detecting large, infrequent errors. This is a non-parametric method that can be used as a complement with the Bland Altman plot.  Mountain plots are created by computing a percentile
	for each ranked difference between a new method and a reference method. (Folded plots are so called because of the following transformation is performed for all percentiles above 50: percentile = 100 - percentile.) These percentiles are then plotted against the differences between the two methods.
	
	Krouwer and Monti argue that the mountain plot offers some following advantages. It is easier to find the central $95\%$ of the data, even when the data are not normally distributed. Also, comparison on different distributions can be performed with ease.
	
	
	
	\subsection{Replicate Measurements}
	
	Thus far, the formulation for comparison of two measurement
	methods is based on one measurement by each method per subject. Should there be two or more measurements by each
	method, these measurements are known as `replicate measurements'.
	\citet{BXC2008} recommends the use of replicate measurements, but
	acknowledges the additional computational complexity.
	
	\citet*{BA86} address this situation via two different
	approaches. The premise of the first approach is that replicate
	measurements can be treated as independent measurements. The
	second approach is based upon using the mean of the each group of
	replicates as one single representative value. 
	
	%\subsubsection{Mean of Replicates Limits of Agreement}
	
	Although either approach may be used to estimate the inter-method bias, removal of the effects of replicate
	measurements error leads to the underestimation of the
	standard deviation of the differences.
	\citet*{BA86} propose a correction for this.
	% % STATE WHAT THIS CORRECTION IS
	
	\citet{BXC2008} take issue with the limits of agreement based on
	mean values of replicate measurements, since these must be interpreted as prediction
	limits for the difference between means of repeated measurements by
	both methods, rather than the difference of individual measurements.
	\citet{BXC2008} demonstrates how the limits of agreement
	calculated using the mean of replicates are `much too narrow as
	prediction limits for differences between future single
	measurements'. This paper also comments that, while treating the
	replicate measurements as independent will cause a downward bias
	on the limits of agreement calculation, this method is preferable
	to the `mean of replicates' approach.
	
	
	
	%%%%%%%%%%%%%%%%%%%%%%%%%%%%%%%%%%%%%%%%%%%%%%%%%%%%%%%%%%%%%%%%%%%%%%%%%%%%%%%%%%%%%%%
	\newpage
	
	\section{Blackwood Bradley Model} 
	
	\citet{BB89} have developed a regression based procedure for
	assessing the agreement. This approach performs a simultaneous test for the equivalence of
	means and variances of the respective methods. Using simple linear
	regression of the differences of each pair against the sums, a
	line is fitted to the model, with estimates for intercept and
	slope ($\hat{\beta}_{0}$ and $\hat{\beta}_{1}$).
	%We have identified
	%this approach  to be examined to see if it can be used as a %foundation for a test perform a test on
	%means and variances individually.
	\begin{equation}
	D = (X_{1}-X_{2})
	\end{equation}
	\begin{equation}
	M = (X_{1} + X_{2}) /2
	\end{equation}
	The Bradley Blackwood Procedure fits D on M as follows:\\
	\begin{equation}
	D = \beta_{0} + \beta_{1}M
	\end{equation}
	This technique offers a formal simultaneous hypothesis test for the
	mean and variance of two paired data sets.  The null
	hypothesis of this test is that the mean ($\mu$) and variance
	($\sigma^{2}$) of both data sets are equal if the slope and
	intercept estimates are equal to zero(i.e $\sigma^{2}_{1} =
	\sigma^{2}_{2}$ and $\mu_{1}=\mu_{2}$ if and only if $\beta_{0}=
	\beta_{1}=0$ )
	
	A test statistic is then calculated from the regression analysis
	of variance values \citep{BB89} and is distributed as `$F$' random
	variable. The degrees of freedom are $\nu_{1}=2$ and $\nu_{1}=n-2$
	(where $n$ is the number of pairs). The critical value is chosen
	for $\alpha\%$ significance with those same degrees of freedom.
	\citet{Bartko} amends this approach for use in method
	comparison studies, using the averages of the pairs, as opposed to
	the sums, and their differences. This approach can facilitate
	simultaneous usage of test with the Bland-Altman approach.
	Bartko's test statistic take the form:
	\[ F.test = \frac{(\Sigma d^{2})-SSReg}{2MSReg}
	\]
	% latex table generated in R 2.6.0 by xtable 1.5-5 package
	% Mon Aug 31 15:53:51 2009
	\begin{table}[h!]
		\begin{center}
			\begin{tabular}{lrrrrr}
				\hline
				& Df & Sum Sq & Mean Sq & F value & Pr($>$F) \\
				\hline
				Averages & 1 & 0.04 & 0.04 & 0.74 & 0.4097 \\
				Residuals & 10 & 0.60 & 0.06 &  &  \\
				\hline
			\end{tabular}
			\caption{Regression ANOVA of case-wise differences and averages
				for Grubbs Data}
		\end{center}
	\end{table}
	%(calculate using R code $qf(0.95,2,10)$).
	
	For the Grubbs data, $\Sigma d^{2}=5.09 $, $SSReg = 0.60$ and
	$MSreg=0.06$ Therefore the test statistic is $37.42$, with a
	critical value of $4.10$. Hence the means and variance of the
	Fotobalk and Counter chronometers are assumed to be simultaneously
	equal.
	
	Importantly, this approach determines whether there is both
	inter-method bias and precision present, or alternatively if there
	is neither present. It has previously been demonstrated that there
	is a inter-method bias present, but as this procedure does not
	allow for separate testing, no conclusion can be drawn on the
	comparative precision of both methods.
	
	\subsection{Bland-Altman correlation test}
	
	The approach proposed by \citet{BA83} is a formal test on the
	Pearson correlation coefficient of case-wise differences and means ($\rho_{AD}$). According to the authors, this test is equivalent
	to the `Pitman Morgan Test'. For the Grubbs data, the correlation coefficient estimate ($r_{AD}$) is 0.2625, with a 95\% confidence
	interval of (-0.366, 0.726) estimated by Fishers `$r$ to $z$' transformation \citep*{Cohen}. The null hypothesis ($\rho_{AD}$ =0)
	fail to be rejected. Consequently the null hypothesis of equal variances of each method would also fail to be rejected. There has
	no been no further mention of this particular test in \citet{BA86}, although \citet{BA99} refers to Spearman's rank
	correlation coefficient. \citet{BA99} state that they ` do not see a place for methods of analysis based on hypothesis testing'.
	\citet{BA99} also states that consider structural equation models to be inappropriate.
	
	\subsection{Identifiability}
	\citet{DunnSEME} highlights an important issue regarding using models such as structural equation modelling, which is the identifiability problem. This comes as a
	result of there being too many parameters to be estimated. Therefore assumptions about some parameters, or estimators used, must be made so that others can be estimated. For example, in the literature, the variance ratio $\lambda=\frac{\sigma^{2}_{1}}{\sigma^{2}_{2}}$
	must often be assumed to be equal to $1$ \citep{linnet98}. \citet{DunnSEME} considers approaches based on two methods with single measurements on each subject as inadequate for a serious
	study on the measurement characteristics of the methods. This is because there would not be enough data to allow for a meaningful
	analysis. There is, however, a counter-argument that in many practical settings it is very difficult to get replicate observations when, for example, the measurement method requires invasive medical
	procedure.
	
	%%%%%%%%%%%%%%%%%%%%%%%%%%%%%%%%%%%%%%%%%%%%%%%%%%%%%%%%%%%%%%%%%%%%%%%%%%%%%%%Bartko's BB
	\citet{BB89} offer a formal simultaneous hypothesis test for the mean and variance of paired data sets. This approach is based upon regressing the differences of each pair on the sum of each pair, a
	line is fitted to the model, with estimates for intercept and
	slope ($\hat{\beta}_{0}$ and $\hat{\beta}_{1}$). The null
	hypothesis of this test is that the mean ($\mu$) and variance
	($\sigma^{2}$) of both data sets are equal if the slope and
	intercept estimates are equal to zero (i.e $\sigma^{2}_{1} =
	\sigma^{2}_{2}$ and $\mu_{1}=\mu_{2}$ if and only if $\beta_{0}=
	\beta_{1}=0$ )
	
	A test statistic is then calculated from the regression analysis
	of variance values \citep{BB89} and is distributed as `$F$' random
	variable. The degrees of freedom are $\nu_{1}=2$ and $\nu_{2}=n-2$
	(where $n$ is the number of pairs). 
	\citet{Bartko} amends this approach for use in method
	comparison studies, using the averages of the pairs, as opposed to
	the sums, and their differences. This approach can facilitate
	simultaneous usage of test with the Bland-Altman approach.
	Bartko's test statistic take the form:
	\[ F.test = \frac{(\Sigma d^{2})-SSReg}{2MSReg}
	\]
	% latex table generated in R 2.6.0 by xtable 1.5-5 package
	% Mon Aug 31 15:53:51 2009
	\begin{table}[ht]
		\begin{center}
			\begin{tabular}{lrrrrr}
				\hline
				& Df & Sum Sq & Mean Sq & F value & Pr($>$F) \\
				\hline
				Averages & 1 & 0.04 & 0.04 & 0.74 & 0.4097 \\
				Residuals & 10 & 0.60 & 0.06 &  &  \\
				\hline
			\end{tabular}
			\caption{Regression ANOVA of case-wise differences and averages
				for Grubbs Data}
		\end{center}
	\end{table}
	%(calculate using R code $qf(0.95,2,10)$).
	
	For the Grubbs data, $\Sigma d^{2}=5.09 $, $SSReg = 0.60$ and $MSreg=0.06$. Therefore the test statistic is $3.742$, with a critical value of $4.10$. Hence the means and variance of the
	Fotobalk and Counter chronometers are assumed to be simultaneously equal.
	
	Importantly, this methodology determines whether there is both inter-method bias and precision present, or alternatively if there
	is neither present. It has previously been demonstrated that there is a inter-method bias present, but as this procedure does not allow for separate testing, no conclusion can be drawn on the comparative precision of both methods.
	
	
	
	%This application of the
	%Grubbs method presumes the existence of this condition, and necessitates
	%replication of observations by means external to and independent of the first
	%means. The Grubbs estimators method is based on the laws of propagation of
	%error. By making three independent simultaneous measurements on the same
	%physical material, it is possible by appropriate mathematical manipulation of
	%the sums and differences of the associated variances to obtain a valid
	%estimate of the precision of the primary means. Application of the Grubbs
	%estimators procedure to estimation of the precision of an apparatus uses
	%the results of a physical test conducted in such a way as to obtain a series
	%of sets of three independent observations.
	
	
	\section{Regression Methods for Method Comparison}
	Conventional regression models are estimated using the ordinary
	least squares (OLS) technique, and are referred to as `Model I
	regression' \citep{CornCoch,ludbrook97}. A key feature of Model I
	models is that the independent variable is assumed to be measured
	without error. However this assumption invalidates simple linear
	regression for use in method comparison studies, as both methods
	must be assumed to be measured with error \citep{BA83,ludbrook97}.
	
	The use of regression models that assumes the presence of error in both variables $X$ and $Y$ have been proposed for use instead
	\citep{CornCoch,ludbrook97}. These methodologies are collectively known as `Model II regression'. They differ in the method used to
	estimate the parameters of the regression.
	
	Regression estimates depend on formulation of the model. A formulation with one method considered as the $X$ variable will yield different estimates for a formulation where it is the $Y$
	variable. With Model I regression, the models fitted in both cases will entirely different and inconsistent. However with Model II
	regression, they will be consistent and complementary.
	
	Regression approaches are useful for a making a detailed examination of the biases across the range of measurements, allowing bias to be decomposed into fixed bias and proportional bias.
	Fixed bias describes the case where one method gives values that are consistently different to the other across the whole range. Proportional
	bias describes the difference in measurements getting progressively greater, or smaller, across the range of measurements. A measurement method may have either an attendant fixed bias or proportional bias, or both. \citep{ludbrook97}. Determination of these biases shall be discussed in due course.
	
	\section{Regression Methods (duplication)}
	Conventional regression models are estimated using the ordinary least squares (OLS) technique, and are referred to as `Model I regression' \citep{CornCoch,ludbrook97}. A key feature of Model I
	models is that the independent variable is assumed to be measured without error. As often pointed out in several papers
	\citep{BA83,ludbrook97}, this assumption invalidates simple linear regression for use in method comparison studies, as both methods
	must be assumed to be measured with error.
	
	The use of regression models that assumes the presence of error in both variables $X$ and $Y$ have been proposed for use instead
	\citep{CornCoch,ludbrook97}. These approaches are collectively known as `Model II regression'. They differ in the method used to
	estimate the parameters of the regression.
	
	Regression estimates depend on formulation of the model. A formulation with one method considered as the $X$ variable will
	yield different estimates for a formulation where it is the $Y$ variable. With Model I regression, the models fitted in both cases
	will entirely different and inconsistent. However with Model II regression, they will be consistent and complementary.
	
	Regression approaches are useful for a making a detailed examination of the biases across the range of measurements, allowing bias to be decomposed into fixed bias and proportional bias.
	Fixed bias describes the case where one method gives values that are consistently different
	to the other across the whole range. Proportional bias describes the difference in measurements getting progressively greater, or smaller, across the range of measurements. A measurement method may have either an attendant fixed bias or proportional bias, or both. \citep{ludbrook}. Determination of these biases shall be discussed in due course.
	\newpage
	
	%================================================================================================= %
	\subsection{Deming Regression}
	
	As stated previously, the fundamental flaw of simple linear regression is that it allows for measurement error in one variable only. This causes a downward biased slope estimate.
	
	Deming regression is a regression fitting approach that assumes error in both variables. Deming regression is recommended by \citet*{CornCoch} as the
	preferred Model II regression for use in method comparison studies.
	The sum of squared distances from measured sets of values to the regression line is minimized at an angles specified by the ratio $\lambda$ of the residual variance of both variables. I
	When $\lambda$ is one, the angle is 45 degrees. In ordinary linear regression, the distances are minimized in the vertical directions \citep{linnet99}.
	In cases involving only single measurements by each method, $\lambda$ may be unknown and is therefore assumes a value of one. While this will produce biased estimates, they are less biased than ordinary linear regression.
	
	The Bland Altman Plot is uninformative about the comparative influence of proportional bias and fixed bias. Model II approaches, such as Deming regression,  can provide independent tests for
	both types of bias.
	
	For a given $\lambda$, \citet{Kummel} derived the following estimate that would later be used for the Deming regression slope
	parameter. The intercept estimate $\alpha$ is simply estimated in the same way as in conventional linear
	regression, by using the identity $\bar{Y}-\hat{\beta}\bar{X}$;
	\begin{equation}
	\hat{\beta} =\quad \frac{S_{yy} - \lambda S_{xx}+[(S_{yy} -
		\lambda S_{xx})^{2}+ 4\lambda S^{2}_{xy}]^{1/2}}{2S_{xy}}
	\end{equation},
	with $\lambda$ as the variance ratio. As stated previously $\lambda$ is often unknown, and therefore must be assumed to equal one. \citet{CarollRupert} states that Deming
	regression is acceptable only when the precision ratio ($\lambda$,in their paper as $\eta$) is correctly specified, but in practice this is often not the case, with the $\lambda$ being underestimated. Several candidate models, with varying variance ratios may be fitted, and estimates of the slope and intercept are produced. However no model selection information is available to determine the best fitting model.
	
	As with conventional regression methodologies, Deming regression calculates an estimate for both the slope and intercept for the
	fitted line, and standard errors thereof. Therefore there is sufficient information to carry out hypothesis tests on both
	estimates, that are informative about presence of fixed and proportional bias.
	
	A $95\%$ confidence interval for the intercept estimate can be used to test the intercept, and hence fixed bias, is equal to
	zero. This hypothesis is accepted if the confidence interval for the estimate contains the value $0$ in its range. Should this be,
	it can be concluded that fixed bias is not present. Conversely, if the hypothesis is rejected, then it is concluded that the
	intercept is non zero, and that fixed bias is present.
	
	Testing for proportional bias is a very similar procedure. The
	$95\%$ confidence interval for the slope estimate can be used to
	test the hypothesis that the slope is equal to $1$. This
	hypothesis is accepted if the confidence interval for the estimate
	contains the value $1$ in its range. If the hypothesis is
	rejected, then it is concluded that the slope is significant
	different from $1$ and that a proportional bias exists.
	
	For convenience, a new data set shall be introduced to demonstrate
	Deming regression. Measurements of transmitral volumetric flow
	(MF) by doppler echocardiography, and left ventricular stroke
	volume (SV) by cross sectional echocardiography in 21 patients
	with aortic valve disease are tabulated in \citet{zhang}. This
	data set features in the discussion of method comparison studies
	in \citet[p.398]{AltmanBook} .
	
	
	% latex table generated in R 2.6.0 by xtable 1.5-5 package
	% Tue Sep 01 13:31:17 2009
	\begin{table}[h!]
		\begin{center}
			\begin{tabular}{|c|c|c||c|c|c||c|c|c|}
				\hline
				Patient & MF  & SV  & Patient & MF  & SV  & Patient & MF  & SV \\
				&($cm^{3}$)&  ($cm^{3}$) & &($cm^{3}$)&  ($cm^{3}$) & &($cm^{3}$)&  ($cm^{3}$)
				\\
				\hline
				1 & 47 & 43 &  8 & 75 & 72 &  15 & 90 & 82 \\
				2 & 66 & 70 & 9 & 79 & 92 &  16 & 100 & 100 \\
				3 & 68 & 72 & 10 & 81 & 76 & 17 & 104 & 94 \\
				4 & 69 & 81 & 11 & 85 & 85 &  18 & 105 & 98 \\
				5 & 70 & 60 & 12 & 87 & 82 & 19 & 112 & 108 \\
				6 & 70 & 67 & 13 & 87 & 90 & 20 & 120 & 131 \\
				7 & 73 & 72 & 14 & 87 & 96 &  21 & 132 & 131 \\
				
				\hline
			\end{tabular}
			\caption{Transmitral volumetric flow(MF) and left ventricular
				stroke volume (SV) in 21 patients. (Zhang et al 1986)}
		\end{center}
	\end{table}
	
	\begin{figure}[h!]
		% Requires \usepackage{graphicx}
		\includegraphics[width=130mm]{images/ZhangDeming.jpeg}
		\caption{Deming Regression For Zhang's Data}\label{ZhangDeming}
	\end{figure}
	
	
	\citet{CarollRupert} states that Deming's
	regression is acceptable only when the precision ratio ($\lambda$,
	in their paper as $\eta$) is correctly specified, but in practice
	this is often not the case, with the $\lambda$ being
	underestimated.
	\newpage
	\section{Other Types of Studies}
	\citet{lewis} categorize method comparison studies into three
	different types.  The key difference between the first two is
	whether or not a `gold standard' method is used. In situations
	where one instrument or method is known to be `accurate and
	precise', it is considered as the`gold standard' \citep{lewis}. A
	method that is not considered to be a gold standard is referred to
	as an `approximate method'. In calibration studies they are
	referred to a criterion methods and test methods respectively.
	
	
	\textbf{1. Calibration problems}. The purpose is to establish a
	relationship between methods, one of which is an approximate
	method, the other a gold standard. The results of the approximate
	method can be mapped to a known probability distribution of the
	results of the gold standard \citep{lewis}. (In such studies, the
	gold standard method and corresponding approximate method are
	generally referred to a criterion method and test method
	respectively.) \citet*{BA83} make clear that their methodology is
	not intended for calibration problems.
	
	\bigskip \textbf{2. Comparison problems}. When two approximate
	methods, that use the same units of measurement, are to be
	compared. This is the case which the Bland-Altman methodology is
	specfically intended for, and therefore it is the most relevant of
	the three.
	
	\bigskip \textbf{3. Conversion problems}. When two approximate
	methods, that use different units of measurement, are to be
	compared. This situation would arise when the measurement methods
	use 'different proxies', i.e different mechanisms of measurement.
	\citet{lewis} deals specifically with this issue. In the context
	of this study, it is the least relevant of the three.
	
	\citet[p.47]{DunnSEME} cautions that`gold standards' should not be
	assumed to be error free. `It is of necessity a subjective
	decision when we come to decide that a particular method or
	instrument can be treated as if it was a gold standard'. The
	clinician gold standard , the sphygmomanometer, is used as an
	example thereof.  The sphygmomanometer `leaves considerable room
	for improvement' \citep{DunnSEME}. \citet{pizzi} similarly
	addresses the issue of glod standards, `well-established gold
	standard may itself be imprecise or even unreliable'.
	
	
	The NIST F1 Caesium fountain atomic clock is considered to be the
	gold standard when measuring time, and is the primary time and
	frequency standard for the United States. The NIST F1 is accurate
	to within one second per 60 million years \citep{NIST}.
	
	Measurements of the interior of the human body are, by definition,
	invasive medical procedures. The design of method must balance the
	need for accuracy of measurement with the well-being of the
	patient. This will inevitably lead to the measurement error as
	described by \citet{DunnSEME}. The magnetic resonance angiogram,
	used to measure internal anatomy,  is considered to the gold
	standard for measuring aortic dissection. Medical test based upon
	the angiogram is reported to have a false positive reporting rate
	of 5\% and a false negative reporting rate of 8\%. This is
	reported as sensitivity of 95\% and a specificity of 92\%
	\citep{ACR}.
	
	In literature they are, perhaps more accurately, referred to as
	`fuzzy gold standards' \citep{phelps}. Consequently when one of the methods is
	essentially a fuzzy gold standard, as opposed to a `true' gold
	standard, the comparison of the criterion and test methods should
	be consider in the context of a comparison study, as well as of a
	calibration study.
	
	\newpage
	
	
	%%%%%%%%%%%%%%%%%%%%%%%%%%%%%%%%%%%%%%%%%%%%%%%%%%%%%%%%%%%%%%%%%%%%%%%%%%%%%%%%%%%%%%%%%%%%%%%%%%%%%%%%%%%%%%%%%%%%%
	%%%%%%%%%%%%%%%%%%%%%%%%%%%%%%%%%%%%%%%%%%%%%%%%%%%%%%%%%%%%%%%%%%%%%%%%%%%%%%%%%%%%%%%%%%%%%%%%%%%%%%%%%%%%%%%%%%%%%
	
	

	\chapter{Extending Current Methodologies}
	\section{Extension of Roy's methodology}
	Roy's methodology is constructed to compare two methods in the presence of replicate measurements. Necessarily it is worth examining whether this methodology can be adapted for different circumstances.
	
	An implementation of Roy's methodology, whereby three or more methods are used, is not feasible due to computational restrictions. Specifically there is a failure to reach convergence before the iteration limit is reached. This may be due to the presence of additional variables, causing the problem of non-identifiability. In the case of two variables, it is required to estimate two variance terms and four correlation terms, six in all. For the case of three variabilities, three variance terms must be estimated as well as nine correlation terms, twelve in all. In general for $n$ methods has $2 \times T_{n}$ variance terms, where $T_n$ is the triangular number for $n$, i.e. the addition analogue of the factorial. Hence the computational complexity quite increases substantially for every increase in $n$.
	
	Should an implementation be feasible, further difficulty arises when interpreting the results. The fundamental question is whether two methods have close agreement so as to be interchangeable. When three methods are present in the model, the null hypothesis is that all three methods have the same variability relevant to the respective tests. The outcome of the analysis will either be that all three are interchangeable or that all three are not interchangeable.
	
	The tests would not be informative as to whether any two of those three were interchangeable, or equivalently if one method in particular disagreed with the other two. Indeed it is easier to perform three pair-wise comparisons separately and then to combine the results.
	
	Roy's methodology is not suitable for the case of single measurements because it follows from the decomposition for the covariance matrix of the response vector $y_{i}$, as presented in \citet{hamlett}. The decomposition depends on the estimation of correlation terms, which would be absent in the single measurement case. Indeed there can be no within-subject variability if there are no repeated terms for it to describe. There would only be the covariance matrix of the measurements by both methods, which doesn't require the use of LME models. To conclude, simpler existing methodologies, such as Deming regression, would be the correct approach where there only one measurements by each method.
	
	\section{Conclusion}
	\citet{BXC2008} and \citet{roy} highlight the need for method comparison methodologies suitable for use in the presence of replicate measurements. \citet{roy} presents a comprehensive methodology for assessing the agreement of two methods, for replicate measurements. This methodology has the added benefit of overcoming the problems of unbalanced data and unequal numbers of replicates. Implementation of the methodology, and interpretation of the results, is relatively easy for practitioners who have only basic statistical training. Furthermore, it can be shown that widely used existing methodologies, such as the limits of agreement, can be incorporated into Roy's methodology.
	
	
	\newpage
	\section{Outline of Thesis}
	In the first chapter the study of method comparison is introduced, while the second chapter provides a review of current methodologies. The intention of this thesis is to progress the
	study of method comparison studies, using a statistical method known as Linear mixed effects models.
	Chapter three shall describes linear mixed effects models, and how the use of the linear mixed
	effects models have so far extended to method comparison studies. Implementations of important existing work shall be presented, using the \texttt{R} programming language.
	
	Model diagnostics are an integral component of a complete statistical analysis.
	In chapter three model diagnostics shall be described in depth, with particular
	emphasis on linear mixed effects models, further to chapter two.
	
	For the fourth chapter, important linear mixed effects model diagnostic methods shall be extended to method comparison studies, and proposed methods shall be demonstrated on data sets that have become well known in literature on method comparison. The purpose is to both calibrate these methods and to demonstrate applications for them.
	The last chapter shall focus on robust measures of important parameters such as agreement.
	\addcontentsline{toc}{section}{Bibliography}
	
	\bibliography{DB-txfrbib}
\end{document}
