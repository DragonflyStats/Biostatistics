\documentclass[Main.tex]{subfiles}
\begin{document}
	
	

	\chapter{Method Comparison Studies}
	
	\section{Introduction}
	The problem of assessing the agreement between two or more methods
	of measurement is ubiquitous in scientific research, and is
	commonly referred to as a `method comparison study'. Published
	examples of method comparison studies can be found in disciplines
	as diverse as pharmacology \citep{ludbrook97}, anaesthesia
	\citep{Myles}, and cardiac imaging methods \citep{Krumm}.
	\smallskip
	
	To illustrate the characteristics of a typical method comparison
	study consider the data in Table I \citep{Grubbs73}. In each of
	twelve experimental trials, a single round of ammunition was fired
	from a 155mm gun and its velocity was measured simultaneously (and
	independently) by three chronographs devices, identified here by
	the labels `Fotobalk', `Counter' and `Terma'.
	\smallskip
	
	
	\newpage
	
	\begin{table}[ht]
		\begin{center}
			\begin{tabular}{rrrr}
				\hline
				Round& Fotobalk [F] & Counter [C]& Terma [T]\\
				\hline
				1 & 793.8 & 794.6 & 793.2 \\
				2 & 793.1 & 793.9 & 793.3 \\
				3 & 792.4 & 793.2 & 792.6 \\
				4 & 794.0 & 794.0 & 793.8 \\
				5 & 791.4 & 792.2 & 791.6 \\
				6 & 792.4 & 793.1 & 791.6 \\
				7 & 791.7 & 792.4 & 791.6 \\
				8 & 792.3 & 792.8 & 792.4 \\
				9 & 789.6 & 790.2 & 788.5 \\
				10 & 794.4 & 795.0 & 794.7 \\
				11 & 790.9 & 791.6 & 791.3 \\
				12 & 793.5 & 793.8 & 793.5 \\
				\hline
			\end{tabular}
			\caption{Velocity measurement from the three chronographs (Grubbs
				1973).}
		\end{center}
	\end{table}
	
	An important aspect of the these data is that all three methods of
	measurement are assumed to have an attended measurement error, and
	the velocities reported in Table 1.1 can not be assumed to be
	`true values' in any absolute sense.
	
	%While lack of
	%agreement between two methods is inevitable, the question , as
	%posed by \citet{BA83}, is 'do the two methods of measurement agree
	%sufficiently closely?'
	
	A method of measurement should ideally be both accurate and
	precise. \citet{Barnhart} describes agreement as being a broader
	term that contains both of those qualities. An accurate
	measurement method will give results close to the unknown `true
	value'. The precision of a method is indicated by how tightly
	measurements obtained under identical conditions are distributed
	around their mean measurement value. A precise and accurate method
	will yield results consistently close to the true value. Of course
	a method may be accurate, but not precise, if the average of its
	measurements is close to the true value, but those measurements
	are highly dispersed. Conversely a method that is not accurate may
	be quite precise, as it consistently indicates the same level of
	inaccuracy. The tendency of a method of measurement to
	consistently give results above or below the true value is a
	source of systematic bias. The smaller the systematic bias, the
	greater the accuracy of the method.
	
	% The FDA define precision as the closeness of agreement (degree of
	% scatter) between a series of measurements obtained from multiple
	% sampling of the same homogeneous sample under prescribed
	% conditions. \citet{Barnhart} describes precision as being further
	% subdivided as either within-run, intra-batch precision or
	% repeatability (which assesses precision during a single analytical
	% run), or between-run, inter-batch precision or repeatability
	%(which measures precision over time).
	
	In the context of the agreement of two methods, there is also a
	tendency of one measurement method to consistently give results
	above or below the other method. Lack of agreement is a
	consequence of the existence of `inter-method bias'. For two
	methods to be considered in good agreement, the inter-method bias
	should be in the region of zero. A simple estimate of the
	inter-method bias is given by the differences between pairs of measurements, for example,  Table~\ref{FCTdata} is a good example of
	possible inter-method bias; the `Fotobalk' consistently recording
	smaller velocities than the `Counter' method. A cursory inspection of the table will indicate a systematic tendency for the Counter method to result in higher measurements than the Fotobalk method. % Consequently one would conclude that there is lack of agreement % between the two methods.
	
	The absence of inter-method bias is, by itself, not sufficient to
	establish that two measurement methods agree. The two methods
	must also have equivalent levels of precision. Should one method
	yield results considerably more variable than those of the other,
	they can not be considered to be in agreement. Hence, method comparison studies are required to take account of both inter-method bias and difference in precision of measurements.
	\newpage
	% latex table generated in R 2.6.0 by xtable 1.5-5 package
	% Wed Aug 26 15:22:41 2009
	\begin{table}[h!]
		
		\begin{center}
			
			\begin{tabular}{rrrr}
				\hline
				Round& Fotobalk (F) & Counter (C) & Difference (F-C) \\
				\hline
				1 & 793.8& 794.6 & -0.8 \\
				2 & 793.1 & 793.9 & -0.8 \\
				3 & 792.4 & 793.2 & -0.8 \\
				4 & 794.0 & 794.0 & 0.0 \\
				5 & 791.4 & 792.2 & -0.8 \\
				6 & 792.4 & 793.1 & -0.7 \\
				7 & 791.7 & 792.4 & -0.7 \\
				8 & 792.3 & 792.8 & -0.5 \\
				9 & 789.6 & 790.2 & -0.6 \\
				10 & 794.4 & 795.0 & -0.6 \\
				11 & 790.9 & 791.6 & -0.7 \\
				12 & 793.5 & 793.8 & -0.3 \\
				\hline
			\end{tabular}
			\caption{Difference between Fotobalk and Counter measurements.}
			\label{FCTdata}\end{center}
	\end{table}
	
	
	
	
\chapter{Introduction to Method Comparison Studies}


\newpage
% latex table generated in R 2.6.0 by xtable 1.5-5 package
% Wed Aug 26 15:22:41 2009
\begin{table}[h!]
	
	\begin{center}
		
		\begin{tabular}{rrrr}
			\hline
			Round& Fotobalk (F) & Counter (C) & F-C \\
			\hline
			1 & 793.8& 794.6 & -0.8 \\
			2 & 793.1 & 793.9 & -0.8 \\
			3 & 792.4 & 793.2 & -0.8 \\
			4 & 794.0 & 794.0 & 0.0 \\
			5 & 791.4 & 792.2 & -0.8 \\
			6 & 792.4 & 793.1 & -0.7 \\
			7 & 791.7 & 792.4 & -0.7 \\
			8 & 792.3 & 792.8 & -0.5 \\
			9 & 789.6 & 790.2 & -0.6 \\
			10 & 794.4 & 795.0 & -0.6 \\
			11 & 790.9 & 791.6 & -0.7 \\
			12 & 793.5 & 793.8 & -0.3 \\
			\hline
		\end{tabular}
		\caption{Difference between Fotobalk and Counter measurements.}
	\end{center}
\end{table}

\bigskip

\newpage

\section{Bland-Altman methodology}
The issue of whether two measurement methods comparable to the
extent that they can be used interchangeably with sufficient
accuracy is encountered frequently in scientific research.
Historically comparison of two methods of measurement was carried
out by use of paired sample $t-$test, correlation coefficients or
simple linear regression. Simple linear regression is unsuitable for method comparison studies because of the required assumption that one variable is measured without error. In comparing two methods, both methods are assume to have attendant random error.

Statisticians Martin Bland and Douglas Altman recognized the inadequacies of these analyzes and
articulated quite thoroughly the basis on which of which they are unsuitable for comparing two methods of measurement \citep*{BA83}. Furthermore they proposed their simple methodology specifically
constructed for method comparison studies. They acknowledge the opportunity to apply other valid, but complex, methodologies, but argue that a simple approach is preferable, especially when the
results must be `explained to non-statisticians'.

Notwithstanding previous remarks about linear regression, the first step recommended, which the authors argue should be mandatory, is construction of a simple scatter plot of the data. The line of equality should also be shown, as it is necessary to give the correct interpretation of how both methods compare. In the case of good agreement, the observations would be distributed closely along the line of equality. A scatter plot of the Grubbs data is shown in Figure 1.1. Visual inspection confirms the previous conclusion that there is an inter-method bias present, i.e. Fotobalk device has a tendency to record a lower velocity.

%\begin{figure}[h!]
%\begin{center}
%  \includegraphics[width=125mm]{GrubbsScatter.jpeg}
%  \caption{Scatter plot For Fotobalk and Counter Methods.}\label{GrubbsScatter}
%\end{center}
%\end{figure}

\citet{Dewitte} notes that scatter plots were very seldom
presented in the Annals of Clinical Biochemistry. This apparently
results from the fact that the `Instructions for Authors' dissuade
the use of regression analysis, which conventionally is
accompanied by a scatter plot.

\newpage
\subsection{Bland-Altman plots}

In light of shortcomings associated with scatterplots,
\citet*{BA83} recommend a further analysis of the data. Firstly
case-wise differences of measurements of two methods $d_{i} =
y_{1i}-y_{2i} \mbox{ for }i=1,2,\dots,n$ on the same subject
should be calculated, and then the average of those measurements
($a_{i} = (y_{1i} + y_{2i})/2 \mbox{ for }i=1,2,\dots, n$).

\citet{BA83} proposes a scatterplot of the case-wise averages and differences of two methods of measurement. This scatterplot has since become widely known as the Bland-Altman plot. \citet*{BA83} express the
motivation for this plot thusly:
\begin{quote}
	``From this type of plot it is much easier to assess the magnitude
	of disagreement (both error and bias), spot outliers, and see
	whether there is any trend, for example an increase in (difference) for high values. This way of plotting the data is a very powerful way of displaying the results of a method comparison study."
\end{quote}

The case wise-averages capture several aspects of the data, such as expressing the range over which the values were taken, and assessing whether the assumptions of constant variance holds.
Case-wise averages also allow the case-wise differences to be presented on a two-dimensional plot, with better data visualization qualities than a one dimensional plot. \citet{BA86}
cautions that it would be the difference against either measurement value instead of their average, as the difference relates to both value. This methodology has proved very popular, and the Bland-Altman plots is widely regarded as powerful graphical methodology for making a visual assessment of the data.

The magnitude of the inter-method bias between the two methods is simply the average of the differences $\bar{d}$. This inter-method bias is represented with a line on the Bland-Altman plot. As the objective of the Bland-Altman plot is to advise on the agreement of two methods, it is the case-wise differences that are also particularly relevant. The variances around this bias is estimated by the standard deviation of these differences $S_{d}$.

\subsection{Bland-Altman plots for the Grubbs data}

In the case of the Grubbs data the inter-method bias is $-0.61$ metres per second, and is indicated by the dashed line on Figure 1.2. By inspection of the plot, it is also possible to compare the precision of each method. Noticeably the differences tend to increase as the averages increase.


The Bland-Altman plot for comparing the `Fotobalk' and `Counter'
methods, which shall henceforth be referred to as the `F vs C'
comparison,  is depicted in Figure 1.2, using data from Table 1.3.
The presence and magnitude of the inter-method bias is indicated
by the dashed line.
\newpage

%Later it will be shown that case-wise differences are the sole
%component of the next part of the methodology, the limits of
%agreement.


\begin{table}[h!]
	\renewcommand\arraystretch{0.7}%
	\begin{center}
		\begin{tabular}{|c||c|c||c|c|}
			\hline
			Round & Fotobalk  & Counter  & Differences  & Averages  \\
			&  [F] & [C] & [F-C] &  [(F+C)/2] \\
			\hline
			1 & 793.8 & 794.6 & -0.8 & 794.2 \\
			2 & 793.1 & 793.9 & -0.8 & 793.5 \\
			3 & 792.4 & 793.2 & -0.8 & 792.8 \\
			4 & 794.0 & 794.0 & 0.0 & 794.0 \\
			5 & 791.4 & 792.2 & -0.8 & 791.8 \\
			6 & 792.4 & 793.1 & -0.7 & 792.8 \\
			7 & 791.7 & 792.4 & -0.7 & 792.0 \\
			8 & 792.3 & 792.8 & -0.5 & 792.5 \\
			9 & 789.6 & 790.2 & -0.6 & 789.9 \\
			10 & 794.4 & 795.0 & -0.6 & 794.7 \\
			11 & 790.9 & 791.6 & -0.7 & 791.2 \\
			12 & 793.5 & 793.8 & -0.3 & 793.6 \\
			\hline
		\end{tabular}
		\caption{Fotobalk and Counter methods: differences and averages.}
	\end{center}
\end{table}

\begin{table}[h!]
	\renewcommand\arraystretch{0.7}%
	\begin{center}
		\begin{tabular}{|c||c|c||c|c|}
			\hline
			Round & Fotobalk  & Terma  & Differences  & Averages  \\
			&  [F] & [T] & [F-T] &  [(F+T)/2] \\
			\hline
			1 & 793.8 & 793.2 & 0.6 & 793.5 \\
			2 & 793.1 & 793.3 & -0.2 & 793.2 \\
			3 & 792.4 & 792.6 & -0.2 & 792.5 \\
			4 & 794.0 & 793.8 & 0.2 & 793.9 \\
			5 & 791.4 & 791.6 & -0.2 & 791.5 \\
			6 & 792.4& 791.6 & 0.8 & 792.0 \\
			7 & 791.7 & 791.6 & 0.1 & 791.6 \\
			8 & 792.3 & 792.4 & -0.1 & 792.3 \\
			9 & 789.6 & 788.5 & 1.1 & 789.0 \\
			10 & 794.4 & 794.7 & -0.3 & 794.5 \\
			11 & 790.9 & 791.3 & -0.4 & 791.1 \\
			12 & 793.5 & 793.5 & 0.0 & 793.5 \\
			
			\hline
		\end{tabular}
		\caption{Fotobalk and Terma methods: differences and averages.}
	\end{center}
\end{table}

\newpage

%\begin{figure}[h!]
%\begin{center}
%  \includegraphics[width=120mm]{GrubbsBAplot-noLOA.jpeg}
%  \caption{Bland-Altman plot For Fotobalk and Counter methods.}\label{GrubbsBA-noLOA}
%\end{center}
%\end{figure}



In Figure 1.3 Bland-Altman plots for the `F vs C' and `F vs T'
comparisons are shown, where `F vs T' refers to the comparison of
the `Fotobalk' and `Terma' methods. Usage of the Bland-Altman plot
can be demonstrate in the contrast between these comparisons. By inspection, there exists a larger inter-method bias in the `F vs C' comparison than in the `F vs T' comparison. Conversely there
appears to be less precision in `F vs T' comparison, as indicated
by the greater dispersion of covariates.

%\begin{figure}[h!]
%\begin{center}
%  \includegraphics[height=90mm]{GrubbsDataTwoBAplots.jpeg}
%  \caption{Bland-Altman plots for Grubbs' F vs C and F vs T comparisons.}\label{GrubbsDataTwoBAplots}
%\end{center}
%\end{figure}

\newpage

\subsection{Prevalence of the Bland-Altman plot}
\citet*{BA86}, which further develops the Bland-Altman methodology,
was found to be the sixth most cited paper of all time by the
\citet{BAcite}. \cite{Dewitte} describes the rate at which
prevalence of the Bland-Altman plot has developed in scientific
literature. \citet{Dewitte} reviewed the use of Bland-Altman plots
by examining all articles in the journal `Clinical Chemistry'
between 1995 and 2001. This study concluded that use of the
Bland?Altman plot increased over the years, from 8\% in 1995 to
14\% in 1996, and 31?36\% in 2002.

The Bland-Altman Plot has since become expected, and
often obligatory, approach for presenting method comparison
studies in many scientific journals \citep{hollis}. Furthermore
\citet{BritHypSoc} recommend its use in papers pertaining to
method comparison studies for the journal of the British
Hypertension Society.

\subsection{Adverse features}

Estimates for inter-method bias and variance of differences are only meaningful if there is uniform inter-bias and variability throughout the range of measurements. Fulfilment of these assumptions can be checked by visual inspection of the plot.The prototype Bland-Altman plots depicted in Figures 1.4, 1.5 and 1.6 are derived from simulated data, for the purpose of demonstrating how the plot would inform an analyst of features that would adversely affect use of the recommended methodology.

Figure 1.4 demonstrates how the Bland-Altman plot would indicate
increasing variance of differences over the measurement range.
Fitted regression lines, for both the upper and lower half of the
plot, has been added to indicate the trend. Figure 1.5 is an
example of cases where the inter-method bias changes over the
measurement range. This is known as proportional bias, and is
defined by \citet{ludbrook97} as meaning that `one method gives
values that are higher (or lower) than those from the other by an
amount that is proportional to the level of the measured
variable'. In both Figures 1.4 and 1.5, the assumptions necessary
for further analysis using the limits of agreement are violated.

Application of regression techniques to the Bland-Altman plot, and
subsequent formal testing for the constant variability of
differences is informative. The data set may be divided into two
subsets, containing the observations wherein the difference values
are less than and greater than the inter-method bias respectively.
For both of these fits, hypothesis tests for the respective slopes
can be performed. While both tests can be considered separately,
multiple comparison procedures, such as the Benjamini-Hochberg
\citep{BH} test, should be also be used.

\begin{figure}[h!]
	\begin{center}
		\includegraphics[height=90mm]{BAFanEffect.jpeg}
		\caption{Bland-Altman plot demonstrating the increase of variance over the range.}\label{BAFanEffect}
	\end{center}
\end{figure}

\begin{figure}[h!]
	\begin{center}
		\includegraphics[height=90mm]{PropBias.jpeg}
		\caption{Bland-Altman plot indicating the presence of proportional bias.}\label{PropBias}
	\end{center}
\end{figure}

\begin{figure}[h!]
	\begin{center}
		\includegraphics[width=125mm]{BAOutliers.jpeg}
		\caption{Bland-Altman plot indicating the presence of potential outliers.}\label{Outliers}
	\end{center}
\end{figure}

\newpage


The Bland-Altman plot also can be used to identify outliers. An
outlier is an observation that is conspicuously different from the
rest of the data that it arouses suspicion that it occurs due to a
mechanism, or conditions, different to that of the rest of the
observations. \citet*{BA99} do not recommend excluding outliers from analyzes,
but remark that recalculation of the inter-method bias estimate,
and further calculations based upon that estimate, are useful for
assessing the influence of outliers. The authors remark that `we
usually find that this method of analysis is not too sensitive to
one or two large outlying differences'. Figure 1.6 demonstrates how the Bland-Altman
plot can be used to visually inspect the presence of potential
outliers.






	\section{Replicate Measurements}
	
	Thus far, the formulation for comparison of two measurement
	methods is one where one measurement by each method is taken on
	each subject. Should there be two or more measurements by each
	methods, these measurement are known as `replicate measurements'.
	\citet{BXC2008} recommends the use of replicate measurements, but
	acknowledges the additional computational complexity.
	
	\citet*{BA86} address this problem by offering two different
	approaches. The premise of the first approach is that replicate
	measurements can be treated as independent measurements. The
	second approach is based upon using the mean of the each group of
	replicates as a representative value of that group. Using either
	of these approaches will allow an analyst to estimate the inter
	method bias.
	
	%\subsubsection{Mean of Replicates Limits of Agreement}
	
	However, because of the removal of the effects of the replicate
	measurements error, this would cause the estimation of the
	standard deviation of the differences to be unduly small.
	\citet*{BA86} propose a correction for this.
	
	\citet{BXC2008} takes issue with the limits of agreement based on
	mean values of replicate measurements, in that they can only be interpreted as prediction
	limits for difference between means of repeated measurements by
	both methods, as opposed to the difference of all measurements.
	Incorrect conclusions would be caused by such a misinterpretation.
	\citet{BXC2008} demonstrates how the limits of agreement
	calculated using the mean of replicates are `much too narrow as
	prediction limits for differences between future single
	measurements'. This paper also comments that, while treating the
	replicate measurements as independent will cause a downward bias
	on the limits of agreement calculation, this method is preferable
	to the `mean of replicates' approach.
	

	\chapter{Review of Current Methodologies}
	\section{Bland-Altman Approach}
	The issue of whether two measurement methods comparable to the
	extent that they can be used interchangeably with sufficient
	accuracy is encountered frequently in scientific research.
	Historically, comparison of two methods of measurement was carried
	out by use of paired sample $t-$test, correlation coefficients or
	simple linear regression. However, simple linear regression is unsuitable for method comparison studies due to the assumption that one variable is measured without error. In comparing two methods, both methods are assume to have attendant random error.
	
	\citet{BA83} highlighted the inadequacies of these approaches for comparing two methods of measurement, and proposed methodologies with this specific application in mind. Although the authors also acknowledge the opportunity to apply other, more complex, approaches, but argue that simpler approaches is preferable, especially when the
	results must be `explained to non-statisticians'.
	
	Notwithstanding previous remarks about linear regression, the first step recommended, which the authors argue should be mandatory, is the construction of a scatter plot of the data. Scatterplots can facilitate an initial judgement and
	helping to identify potential outliers, with the addition of the line of equality. In the case of good agreement, the observations would be distributed closely along this line. However, they are not useful for a thorough examination of the data. \citet{BritHypSoc} notes that
	data points will tend to cluster around the line of equality, obscuring interpretation.
	
	
	A scatter plot of the Grubbs data is shown in Figure 1.1. Visual inspection confirms the previous conclusion that inter-method bias is present, i.e. the Fotobalk device has a tendency to record a lower velocity.
	
	\begin{figure}[h!]
		\begin{center}
			\includegraphics[width=125mm]{images/GrubbsScatter.jpeg}
			\caption{Scatter plot for Fotobalk and Counter methods.}\label{GrubbsScatter}
		\end{center}
	\end{figure}
	
	\citet{Dewitte} notes that scatter plots were very seldom
	presented in the Annals of Clinical Biochemistry. This apparently
	results from the fact that the `Instructions for Authors' dissuade
	the use of regression analysis, which conventionally is
	accompanied by a scatter plot.
	
	\newpage
	\subsection{Bland-Altman plots}
	
	In light of shortcomings associated with scatterplots,
	\citet*{BA83} recommend a further analysis of the data. Firstly
	case-wise differences of measurements of two methods $d_{i} =
	y_{1i}-y_{2i}, \mbox{ for }i=1,2,\dots,n$, on the same subject
	should be calculated, and then the average of those measurements, 
	($a_{i} = (y_{1i} + y_{2i})/2 \mbox{ for }i=1,2,\dots, n$.
	
	\citet{BA83} proposed that $a_i$ should be plotted against $d_i$, a plot now widely known as the Bland-Altman plot, and motivated this plot as follows:
	\begin{quote}
		``From this type of plot it is much easier to assess the magnitude
		of disagreement (both error and bias), spot outliers, and see
		whether there is any trend, for example an increase in (difference) for high values. This way of plotting the data is a very powerful way of displaying the results of a method comparison study."
	\end{quote}
	
	The case wise-averages capture several aspects of the data, such as expressing the range over which the values were taken, and assessing whether the assumptions of constant variance holds.
	Case-wise averages also allow the case-wise differences to be presented on a two-dimensional plot, with better data visualization qualities than a one dimensional plot. \citet{BA86}
	cautions that it would be the difference against either measurement value instead of their average, as the difference relates to both value. This approach has proved very popular, and the Bland-Altman plots is widely regarded as powerful graphical tool for making a visual assessment of the data.
	
	The magnitude of the inter-method bias between the two methods is simply the average of the differences $\bar{d}$. This inter-method bias is represented with a line on the Bland-Altman plot. As the objective of the Bland-Altman plot is to advise on the agreement of two methods, the individual case-wise differences are also particularly relevant. The variances around this bias is estimated by the standard deviation of these differences $S_{d}$.
	
	\subsection{Bland-Altman plots for the Grubbs data}
	
	In the case of the Grubbs data the inter-method bias is $-0.61$ metres per second, and is indicated by the dashed line on Figure 1.2. By inspection of the plot, it is also possible to compare the precision of each method. Noticeably the differences tend to increase as the averages increase.
	
	
	The Bland-Altman plot for comparing the `Fotobalk' and `Counter'
	methods, which shall henceforth be referred to as the `F vs C'
	comparison,  is depicted in Figure 1.2, using data from Table 1.3.
	The presence and magnitude of the inter-method bias is indicated
	by the dashed line.
	\newpage
	
	%Later it will be shown that case-wise differences are the sole
	%component of the next part of the methodology, the limits of
	%agreement.
	
	
	\begin{table}[h!]
		\renewcommand\arraystretch{0.7}%
		\begin{center}
			\begin{tabular}{|c||c|c||c|c|}
				\hline
				Round & Fotobalk  & Counter  & Differences  & Averages  \\
				&  [F] & [C] & [F-C] &  [(F+C)/2] \\
				\hline
				1 & 793.8 & 794.6 & -0.8 & 794.2 \\
				2 & 793.1 & 793.9 & -0.8 & 793.5 \\
				3 & 792.4 & 793.2 & -0.8 & 792.8 \\
				4 & 794.0 & 794.0 & 0.0 & 794.0 \\
				5 & 791.4 & 792.2 & -0.8 & 791.8 \\
				6 & 792.4 & 793.1 & -0.7 & 792.8 \\
				7 & 791.7 & 792.4 & -0.7 & 792.0 \\
				8 & 792.3 & 792.8 & -0.5 & 792.5 \\
				9 & 789.6 & 790.2 & -0.6 & 789.9 \\
				10 & 794.4 & 795.0 & -0.6 & 794.7 \\
				11 & 790.9 & 791.6 & -0.7 & 791.2 \\
				12 & 793.5 & 793.8 & -0.3 & 793.6 \\
				\hline
			\end{tabular}
			\caption{Fotobalk and Counter methods: differences and averages.}
		\end{center}
	\end{table}
	
	\begin{table}[h!]
		\renewcommand\arraystretch{0.7}%
		\begin{center}
			\begin{tabular}{|c||c|c||c|c|}
				\hline
				Round & Fotobalk  & Terma  & Differences  & Averages  \\
				&  [F] & [T] & [F-T] &  [(F+T)/2] \\
				\hline
				1 & 793.8 & 793.2 & 0.6 & 793.5 \\
				2 & 793.1 & 793.3 & -0.2 & 793.2 \\
				3 & 792.4 & 792.6 & -0.2 & 792.5 \\
				4 & 794.0 & 793.8 & 0.2 & 793.9 \\
				5 & 791.4 & 791.6 & -0.2 & 791.5 \\
				6 & 792.4& 791.6 & 0.8 & 792.0 \\
				7 & 791.7 & 791.6 & 0.1 & 791.6 \\
				8 & 792.3 & 792.4 & -0.1 & 792.3 \\
				9 & 789.6 & 788.5 & 1.1 & 789.0 \\
				10 & 794.4 & 794.7 & -0.3 & 794.5 \\
				11 & 790.9 & 791.3 & -0.4 & 791.1 \\
				12 & 793.5 & 793.5 & 0.0 & 793.5 \\
				
				\hline
			\end{tabular}
			\caption{Fotobalk and Terma methods: differences and averages.}
		\end{center}
	\end{table}
	
	\newpage
	
	\begin{figure}[h!]
		\begin{center}
			\includegraphics[width=120mm]{images/GrubbsBAplot-noLOA.jpeg}
			\caption{Bland-Altman plot For Fotobalk and Counter methods.}\label{GrubbsBA-noLOA}
		\end{center}
	\end{figure}
	
	
	
	In Figure 1.3 Bland-Altman plots for the `F vs C' and `F vs T'
	comparisons are shown, where `F vs T' refers to the comparison of
	the `Fotobalk' and `Terma' methods. Usage of the Bland-Altman plot
	can be demonstrate in the contrast between these comparisons. By inspection, there exists a larger inter-method bias in the `F vs C' comparison than in the `F vs T' comparison. Conversely there
	appears to be less precision in `F vs T' comparison, as indicated
	by the greater dispersion of covariates.
	
	\begin{figure}[h!]
		\begin{center}
			\includegraphics[height=90mm]{images/GrubbsDataTwoBAplots.jpeg}
			\caption{Bland-Altman plots for Grubbs' F vs C and F vs T comparisons.}\label{GrubbsDataTwoBAplots}
		\end{center}
	\end{figure}
	
	\newpage
	
	\subsection{Prevalence of the Bland-Altman plot}
	\citet*{BA86}, which further develops the Bland-Altman approach,
	was found to be the sixth most cited paper of all time by \citet{BAcite}. \cite{Dewitte} describes the rate at which
	prevalence of the Bland-Altman plot has developed in scientific
	literature, by examining all articles in the journal `Clinical Chemistry'
	between 1995 and 2001. This study concluded that use of the
	Bland�Altman plot increased over the years, from 8\% in 1995 to
	14\% in 1996, and 31�36\% in 2002.
	
	The Bland-Altman Plot has since become expected, and
	often obligatory, approach for presenting method comparison
	studies in many scientific journals \citep{hollis}. Furthermore
	\citet{BritHypSoc} recommend its use in papers pertaining to
	method comparison studies for the journal of the British
	Hypertension Society.
	
	\subsection{Adverse features}
	
	Estimates for inter-method bias and variance of differences are only meaningful if there is uniform inter-bias and variability throughout the range of measurements. Fulfilment of these assumptions can be checked by visual inspection of the plot.The prototype Bland-Altman plots depicted in Figures 1.4, 1.5 and 1.6 are derived from simulated data, for the purpose of demonstrating how the plot would inform an analyst of features that would adversely affect use of the recommended approach.
	
	Figure 1.4 demonstrates how the Bland-Altman plot would indicate
	increasing variance of differences over the measurement range.
	Fitted regression lines, for both the upper and lower half of the
	plot, has been added to indicate the trend. Figure 1.5 is an
	example of cases where the inter-method bias changes over the
	measurement range. This is known as proportional bias, and is
	defined by \citet{ludbrook97} as meaning that `one method gives
	values that are higher (or lower) than those from the other by an
	amount that is proportional to the level of the measured
	variable'. In both Figures 1.4 and 1.5, the assumptions necessary
	for further analysis using the limits of agreement are violated.
	
	Application of regression techniques to the Bland-Altman plot, and
	subsequent formal testing for the constant variability of
	differences is informative. The data set may be divided into two
	subsets, containing the observations wherein the difference values
	are less than and greater than the inter-method bias respectively.
	For both of these fits, hypothesis tests for the respective slopes
	can be performed. While both tests could be considered separately,
	multiple comparison procedures, such as the Benjamini-Hochberg
	\citep{BH} test, are advisable.
	
	\begin{figure}[h!]
		\begin{center}
			\includegraphics[height=90mm]{images/BAFanEffect.jpeg}
			\caption{Bland-Altman plot demonstrating the increase of variance over the range.}\label{BAFanEffect}
		\end{center}
	\end{figure}
	
	\begin{figure}[h!]
		\begin{center}
			\includegraphics[height=90mm]{images/PropBias.jpeg}
			\caption{Bland-Altman plot indicating the presence of proportional bias.}\label{PropBias}
		\end{center}
	\end{figure}
	
	\begin{figure}[h!]
		\begin{center}
			\includegraphics[width=125mm]{images/BAOutliers.jpeg}
			\caption{Bland-Altman plot indicating the presence of potential outliers.}\label{Outliers}
		\end{center}
	\end{figure}
	
	\newpage
	
	
	The Bland-Altman plot also can be used to identify outliers. An
	outlier is an observation that is conspicuously different from the
	rest of the data that it arouses suspicion that it occurs due to a
	mechanism, or conditions, different to that of the rest of the
	observations. \citet*{BA99} do not recommend excluding outliers from analyses,
	but remark that recalculation of the inter-method bias estimate,
	and further calculations based upon that estimate, are useful for
	assessing the influence of outliers. The authors remark that `we
	usually find that this method of analysis is not too sensitive to
	one or two large outlying differences'. Figure 1.6 demonstrates how the Bland-Altman
	plot can be used to visually inspect the presence of potential
	outliers.
	
	As a complement to the Bland-Altman plot, \citet{Bartko} proposes
	the use of a bivariate confidence ellipse, constructed for a
	predetermined level. \citet{AltmanEllipse} provides the relevant calculations for the
	ellipse. This ellipse is intended as a visual
	guidelines for the scatter plot, for detecting outliers and to
	assess the within- and between-subject variances.
	
	The minor axis relates to the between subject variability, whereas
	the major axis relates to the error mean square, with the ellipse
	depicting the size of both relative to each other.
	Consequently Bartko's ellipse provides a visual aid to determining the
	relationship between variances. If $\mbox{var}(a)$ is greater than $\mbox{var}(d)$, the orientation of the ellipse is horizontal. Conversely if $\mbox{var}(a)$ is less than $\mbox{var}(d)$, the orientation of the ellipse is vertical.
	
	
	%(Furthermore \citet{Bartko}
	%proposes formal testing procedures, that shall be discussed in due
	%course.)
	
	The Bland-Altman plot for the Grubbs data, complemented by Bartko's ellipse, is depicted in Figure 1.7.
	The fourth observation is shown to be outside the bounds of the ellipse, indicating that it is a potential outlier.
	
	
	\begin{figure}[h!]
		% Requires \usepackage{graphicx}
		\includegraphics[width=130mm]{images/GrubbsBartko.jpeg}
		\caption{Bartko's Ellipse for Grubbs' data.}\label{GrubbsBartko}
	\end{figure}
	
	The limitations of using bivariate approaches to outlier detection
	in the Bland-Altman plot can demonstrated using Bartko's ellipse.
	A covariate is added to the `F vs C' comparison that has a
	difference value equal to the inter-method bias, and an average
	value that markedly deviates from the rest of the average values
	in the comparison, i.e. 786. Table 1.8 depicts a $95\%$ confidence
	ellipse for this manipulated data set. By inspection of the
	confidence interval, we would conclude that this extra
	covariate is an outlier, in spite of the fact that this
	observation is very close to the inter-method bias as determined by this approach.
	
	\begin{figure}[h!]
		% Requires \usepackage{graphicx}
		\includegraphics[width=130mm]{images/GrubbsBartko2.jpeg}
		\caption{Bartko's Ellipse for Grubbs' data, with an extra covariate.}\label{GrubbsBartko2}
	\end{figure}
	
	
	Importantly, outlier classification must be informed by the logic of the
	mechanism that produces the data. In the Bland-Altman plot, the horizontal displacement (i.e. the average) of any
	observation is supported by two separate measurements. Any
	observation should not be considered an outlier on the basis of a
	noticeable horizontal displacement from the main cluster, as in
	the case with the extra covariate. Conversely, the fourth
	observation, from the original data set, should be considered an
	outlier, as it has a noticeable vertical displacement from the
	rest of the observations.
	
	%Grubbs' test is a statistical test used for detecting outliers in a
	%univariate data set that is assumed to be normally distributed.
	
	%\citet{Grubbs} defined an outlier as a co-variate that appears to
	%deviate markedly from other members of the sample in which it
	%occurs.
	
	In classifying whether a observation from a univariate data set is
	an outlier, many formal tests are available, such as the Grubbs test for outliers. In assessing
	whether a covariate in a Bland-Altman plot is an outlier, this
	test is useful when applied to the case-wise difference values treated as a
	univariate data set. The null hypothesis of the Grubbs test procedure is the absence
	of any outliers in the data set. Conversely, the alternative hypotheses is that there is at least one outlier
	present.
	
	The test statistic for the Grubbs test ($G$) is the largest
	absolute deviation from the sample mean divided by the standard
	deviation of the differences,
	\[
	G =  \displaystyle\max_{i=1,\ldots, n}\frac{\left \vert d_i -
		\bar{d}\right\vert}{S_{d}}.
	\]
	
	For the `F vs C' comparison it is the fourth observation gives
	rise to the test statistic, $G = 3.64$. The critical value is
	calculated using Student's $t$ distribution and the sample size,
	\[
	U = \frac{n-1}{\sqrt{n}} \sqrt{\frac{t_{\alpha/(2n),n-2}^2}{n - 2
			+ t_{\alpha/(2n),n-2}^2}}.
	\]
	For this test $U = 0.75$. The conclusion of this test is that the fourth observation in the `F vs C' comparison is an outlier, with $p-$value = 0.003, in accordance with the previous result of Bartko's ellipse.
	
	

	\section{Blackwood Bradley Model} 
	
	\citet{BB89} have developed a regression based procedure for
	assessing the agreement. This approach performs a simultaneous test for the equivalence of
	means and variances of the respective methods. Using simple linear
	regression of the differences of each pair against the sums, a
	line is fitted to the model, with estimates for intercept and
	slope ($\hat{\beta}_{0}$ and $\hat{\beta}_{1}$).
	%We have identified
	%this approach  to be examined to see if it can be used as a %foundation for a test perform a test on
	%means and variances individually.
	\begin{equation}
	D = (X_{1}-X_{2})
	\end{equation}
	\begin{equation}
	M = (X_{1} + X_{2}) /2
	\end{equation}
	The Bradley Blackwood Procedure fits D on M as follows:\\
	\begin{equation}
	D = \beta_{0} + \beta_{1}M
	\end{equation}
	This technique offers a formal simultaneous hypothesis test for the
	mean and variance of two paired data sets.  The null
	hypothesis of this test is that the mean ($\mu$) and variance
	($\sigma^{2}$) of both data sets are equal if the slope and
	intercept estimates are equal to zero(i.e $\sigma^{2}_{1} =
	\sigma^{2}_{2}$ and $\mu_{1}=\mu_{2}$ if and only if $\beta_{0}=
	\beta_{1}=0$ )
	
	A test statistic is then calculated from the regression analysis
	of variance values \citep{BB89} and is distributed as `$F$' random
	variable. The degrees of freedom are $\nu_{1}=2$ and $\nu_{1}=n-2$
	(where $n$ is the number of pairs). The critical value is chosen
	for $\alpha\%$ significance with those same degrees of freedom.
	\citet{Bartko} amends this approach for use in method
	comparison studies, using the averages of the pairs, as opposed to
	the sums, and their differences. This approach can facilitate
	simultaneous usage of test with the Bland-Altman approach.
	Bartko's test statistic take the form:
	\[ F.test = \frac{(\Sigma d^{2})-SSReg}{2MSReg}
	\]
	% latex table generated in R 2.6.0 by xtable 1.5-5 package
	% Mon Aug 31 15:53:51 2009
	\begin{table}[h!]
		\begin{center}
			\begin{tabular}{lrrrrr}
				\hline
				& Df & Sum Sq & Mean Sq & F value & Pr($>$F) \\
				\hline
				Averages & 1 & 0.04 & 0.04 & 0.74 & 0.4097 \\
				Residuals & 10 & 0.60 & 0.06 &  &  \\
				\hline
			\end{tabular}
			\caption{Regression ANOVA of case-wise differences and averages
				for Grubbs Data}
		\end{center}
	\end{table}
	%(calculate using R code $qf(0.95,2,10)$).
	
	For the Grubbs data, $\Sigma d^{2}=5.09 $, $SSReg = 0.60$ and
	$MSreg=0.06$ Therefore the test statistic is $37.42$, with a
	critical value of $4.10$. Hence the means and variance of the
	Fotobalk and Counter chronometers are assumed to be simultaneously
	equal.
	
	Importantly, this approach determines whether there is both
	inter-method bias and precision present, or alternatively if there
	is neither present. It has previously been demonstrated that there
	is a inter-method bias present, but as this procedure does not
	allow for separate testing, no conclusion can be drawn on the
	comparative precision of both methods.
	
	\subsection{Bland-Altman correlation test}
	
	The approach proposed by \citet{BA83} is a formal test on the
	Pearson correlation coefficient of case-wise differences and means ($\rho_{AD}$). According to the authors, this test is equivalent
	to the `Pitman Morgan Test'. For the Grubbs data, the correlation coefficient estimate ($r_{AD}$) is 0.2625, with a 95\% confidence
	interval of (-0.366, 0.726) estimated by Fishers `$r$ to $z$' transformation \citep*{Cohen}. The null hypothesis ($\rho_{AD}$ =0)
	fail to be rejected. Consequently the null hypothesis of equal variances of each method would also fail to be rejected. There has
	no been no further mention of this particular test in \citet{BA86}, although \citet{BA99} refers to Spearman's rank
	correlation coefficient. \citet{BA99} state that they ` do not see a place for methods of analysis based on hypothesis testing'.
	\citet{BA99} also states that consider structural equation models to be inappropriate.
	
	\subsection{Identifiability}
	\citet{DunnSEME} highlights an important issue regarding using models such as structural equation modelling, which is the identifiability problem. This comes as a
	result of there being too many parameters to be estimated. Therefore assumptions about some parameters, or estimators used, must be made so that others can be estimated. For example, in the literature, the variance ratio $\lambda=\frac{\sigma^{2}_{1}}{\sigma^{2}_{2}}$
	must often be assumed to be equal to $1$ \citep{linnet98}. \citet{DunnSEME} considers approaches based on two methods with single measurements on each subject as inadequate for a serious
	study on the measurement characteristics of the methods. This is because there would not be enough data to allow for a meaningful
	analysis. There is, however, a counter-argument that in many practical settings it is very difficult to get replicate observations when, for example, the measurement method requires invasive medical
	procedure.
	
	%%%%%%%%%%%%%%%%%%%%%%%%%%%%%%%%%%%%%%%%%%%%%%%%%%%%%%%%%%%%%%%%%%%%%%%%%%%%%%%Bartko's BB
	\citet{BB89} offer a formal simultaneous hypothesis test for the mean and variance of paired data sets. This approach is based upon regressing the differences of each pair on the sum of each pair, a
	line is fitted to the model, with estimates for intercept and
	slope ($\hat{\beta}_{0}$ and $\hat{\beta}_{1}$). The null
	hypothesis of this test is that the mean ($\mu$) and variance
	($\sigma^{2}$) of both data sets are equal if the slope and
	intercept estimates are equal to zero (i.e $\sigma^{2}_{1} =
	\sigma^{2}_{2}$ and $\mu_{1}=\mu_{2}$ if and only if $\beta_{0}=
	\beta_{1}=0$ )
	
	A test statistic is then calculated from the regression analysis
	of variance values \citep{BB89} and is distributed as `$F$' random
	variable. The degrees of freedom are $\nu_{1}=2$ and $\nu_{2}=n-2$
	(where $n$ is the number of pairs). 
	\citet{Bartko} amends this approach for use in method
	comparison studies, using the averages of the pairs, as opposed to
	the sums, and their differences. This approach can facilitate
	simultaneous usage of test with the Bland-Altman approach.
	Bartko's test statistic take the form:
	\[ F.test = \frac{(\Sigma d^{2})-SSReg}{2MSReg}
	\]
	% latex table generated in R 2.6.0 by xtable 1.5-5 package
	% Mon Aug 31 15:53:51 2009
	\begin{table}[ht]
		\begin{center}
			\begin{tabular}{lrrrrr}
				\hline
				& Df & Sum Sq & Mean Sq & F value & Pr($>$F) \\
				\hline
				Averages & 1 & 0.04 & 0.04 & 0.74 & 0.4097 \\
				Residuals & 10 & 0.60 & 0.06 &  &  \\
				\hline
			\end{tabular}
			\caption{Regression ANOVA of case-wise differences and averages
				for Grubbs Data}
		\end{center}
	\end{table}
	%(calculate using R code $qf(0.95,2,10)$).
	
	For the Grubbs data, $\Sigma d^{2}=5.09 $, $SSReg = 0.60$ and $MSreg=0.06$. Therefore the test statistic is $3.742$, with a critical value of $4.10$. Hence the means and variance of the
	Fotobalk and Counter chronometers are assumed to be simultaneously equal.
	
	Importantly, this methodology determines whether there is both inter-method bias and precision present, or alternatively if there
	is neither present. It has previously been demonstrated that there is a inter-method bias present, but as this procedure does not allow for separate testing, no conclusion can be drawn on the comparative precision of both methods.
	
	
	
	%This application of the
	%Grubbs method presumes the existence of this condition, and necessitates
	%replication of observations by means external to and independent of the first
	%means. The Grubbs estimators method is based on the laws of propagation of
	%error. By making three independent simultaneous measurements on the same
	%physical material, it is possible by appropriate mathematical manipulation of
	%the sums and differences of the associated variances to obtain a valid
	%estimate of the precision of the primary means. Application of the Grubbs
	%estimators procedure to estimation of the precision of an apparatus uses
	%the results of a physical test conducted in such a way as to obtain a series
	%of sets of three independent observations.
	
	
	\section{Regression Methods for Method Comparison}
	Conventional regression models are estimated using the ordinary
	least squares (OLS) technique, and are referred to as `Model I
	regression' \citep{CornCoch,ludbrook97}. A key feature of Model I
	models is that the independent variable is assumed to be measured
	without error. However this assumption invalidates simple linear
	regression for use in method comparison studies, as both methods
	must be assumed to be measured with error \citep{BA83,ludbrook97}.
	
	The use of regression models that assumes the presence of error in both variables $X$ and $Y$ have been proposed for use instead
	\citep{CornCoch,ludbrook97}. These methodologies are collectively known as `Model II regression'. They differ in the method used to
	estimate the parameters of the regression.
	
	Regression estimates depend on formulation of the model. A formulation with one method considered as the $X$ variable will yield different estimates for a formulation where it is the $Y$
	variable. With Model I regression, the models fitted in both cases will entirely different and inconsistent. However with Model II
	regression, they will be consistent and complementary.
	
	Regression approaches are useful for a making a detailed examination of the biases across the range of measurements, allowing bias to be decomposed into fixed bias and proportional bias.
	Fixed bias describes the case where one method gives values that are consistently different to the other across the whole range. Proportional
	bias describes the difference in measurements getting progressively greater, or smaller, across the range of measurements. A measurement method may have either an attendant fixed bias or proportional bias, or both. \citep{ludbrook97}. Determination of these biases shall be discussed in due course.
	


	\chapter{Extending Current Methodologies}
	\section{Extension of Roy's methodology}
	Roy's methodology is constructed to compare two methods in the presence of replicate measurements. Necessarily it is worth examining whether this methodology can be adapted for different circumstances.
	
	An implementation of Roy's methodology, whereby three or more methods are used, is not feasible due to computational restrictions. Specifically there is a failure to reach convergence before the iteration limit is reached. This may be due to the presence of additional variables, causing the problem of non-identifiability. In the case of two variables, it is required to estimate two variance terms and four correlation terms, six in all. For the case of three variabilities, three variance terms must be estimated as well as nine correlation terms, twelve in all. In general for $n$ methods has $2 \times T_{n}$ variance terms, where $T_n$ is the triangular number for $n$, i.e. the addition analogue of the factorial. Hence the computational complexity quite increases substantially for every increase in $n$.
	
	Should an implementation be feasible, further difficulty arises when interpreting the results. The fundamental question is whether two methods have close agreement so as to be interchangeable. When three methods are present in the model, the null hypothesis is that all three methods have the same variability relevant to the respective tests. The outcome of the analysis will either be that all three are interchangeable or that all three are not interchangeable.
	
	The tests would not be informative as to whether any two of those three were interchangeable, or equivalently if one method in particular disagreed with the other two. Indeed it is easier to perform three pair-wise comparisons separately and then to combine the results.
	
	Roy's methodology is not suitable for the case of single measurements because it follows from the decomposition for the covariance matrix of the response vector $y_{i}$, as presented in \citet{hamlett}. The decomposition depends on the estimation of correlation terms, which would be absent in the single measurement case. Indeed there can be no within-subject variability if there are no repeated terms for it to describe. There would only be the covariance matrix of the measurements by both methods, which doesn't require the use of LME models. To conclude, simpler existing methodologies, such as Deming regression, would be the correct approach where there only one measurements by each method.
	
	\section{Conclusion}
	\citet{BXC2008} and \citet{roy} highlight the need for method comparison methodologies suitable for use in the presence of replicate measurements. \citet{roy} presents a comprehensive methodology for assessing the agreement of two methods, for replicate measurements. This methodology has the added benefit of overcoming the problems of unbalanced data and unequal numbers of replicates. Implementation of the methodology, and interpretation of the results, is relatively easy for practitioners who have only basic statistical training. Furthermore, it can be shown that widely used existing methodologies, such as the limits of agreement, can be incorporated into Roy's methodology.
	
	
	\newpage
	\section{Outline of Thesis}
	In the first chapter the study of method comparison is introduced, while the second chapter provides a review of current methodologies. The intention of this thesis is to progress the
	study of method comparison studies, using a statistical method known as Linear mixed effects models.
	Chapter three shall describes linear mixed effects models, and how the use of the linear mixed
	effects models have so far extended to method comparison studies. Implementations of important existing work shall be presented, using the \texttt{R} programming language.
	
	Model diagnostics are an integral component of a complete statistical analysis.
	In chapter three model diagnostics shall be described in depth, with particular
	emphasis on linear mixed effects models, further to chapter two.
	
	For the fourth chapter, important linear mixed effects model diagnostic methods shall be extended to method comparison studies, and proposed methods shall be demonstrated on data sets that have become well known in literature on method comparison. The purpose is to both calibrate these methods and to demonstrate applications for them.
	The last chapter shall focus on robust measures of important parameters such as agreement.
	\addcontentsline{toc}{section}{Bibliography}
	
	\bibliography{DB-txfrbib}
\end{document}
