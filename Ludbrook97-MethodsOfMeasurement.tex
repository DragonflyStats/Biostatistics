Comparing methods of measurements.
Ludbrook J.
%-----------------------------------------------------------------------------------------------------%
1. The purpose of comparing two methods of measurement of a continuous biological variable is to uncover systematic 
differences not to point to similarities. 

2. There are two potential sources of systematic disagreement between methods of measurement: fixed and 
proportional bias. 

3. Fixed bias means that one method gives values that are higher (or lower) than those from the other by a constant amount. Proportional bias means that one method gives values that are higher (or lower) than those from the 
other by an amount that is proportional to the level of the measured variable. 

4. It must be assumed that measurements made by either method are attended by random error: in making measurements and from biological variation. 

5. Investigators often use the Pearson product-moment correlation coefficient (r) to compare methods of measurement. 
This cannot detect systematic biases, only random error. 

6. Investigators sometimes use \textbf{least squares (Model I)} regression analysis to calibrate one method of measurement against 
another. In this technique, the sum of the squares of the vertical deviations of y values from the line is minimized. 
This approach is invalid, because both y and x values are attended by random error. 

7. \textbf{Model II} regression analysis caters for cases in which random error is attached to both dependent and independent 
variables. Comparing methods of measurement is just such a case. 

8. \textbf{Least products} regression is the reviewer's preferred technique for analysing the Model II case. In this, the sum of 
the products of the vertical and horizontal deviations of the x,y values from the line is minimized. 

9. Least products regression analysis is suitable for calibrating one method against another. It is also a sensitive 
technique for detecting and distinguishing fixed and proportional bias between methods. 

10. An alternative approach is to examine the differences between methods in order to detect bias. 
This has been recommended to clinical scientists and has been adopted by many. 

11. It is the reviewer's opinion that the least products regression technique is to be preferred to that of examining differences, because the 
former distinguishes between fixed and proportional bias, whereas the latter does not.
%-----------------------------------------------------------------------------------------------------%

Sources of Random Errors
The goal of comparing methods of measurement
Correlation
