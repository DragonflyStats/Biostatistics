\subsection{Introduction}
\begin{itemize}
\item Comparing two methods of measurement is normally done by computing limits of agreement (LoA), i.e. prediction limits for
a future difference between measurements with the two methods. When the difference is not constant it is not clear what
this means, since the difference between the methods depends on the average; hence, unlike the case where the difference is
constant, LoA cannot directly be translated into a prediction interval for a measurement by one method given that of another.
\item The main point in the paper by Bland and Altman [1] is however different from the outlook in this paper; Bland and Altman
mainly discuss whether two methods of measurement can be used interchangeably and how to assess this with the help of
proper statistical methods to derive LoA, i.e. prediction limits for differences between two methods.
This paper takes as starting point that the classical LoA can be converted to a prediction interval for one method given a
measurement by the other (details in the next section). This sort of relationship can be shown in a plot as a line with slope 1
and prediction limits as lines also with slope 1; applicable for the prediction both from method 1 to method 2 and vice versa. In
the case of non-constant difference it would be desirable to be able to produce a similar plot, usable both ways. Thus, the aim
of this paper is to produce a conversion from one method to another that also applies in the case where the difference between
methods is not constant.
\item In this paper, I set up a proper model for data for method comparison studies which in the case of constant difference between
methods leads to the classical LoA, and in the case of linear bias gives a simple formula for the prediction. The paper only
addresses the situation where only one measurement by each method is available, although replicate measurements by each
method are desirable whenever possible [2]. Moreover, the situation with non-constant variance over the range of measurements
is not covered either.
\end{itemize}
\newpage
\subsection{Discussion}
%----------------------------------------------------------------------------------------------------------------%
I have here proposed a simple twist to the results from regression of the differences on the sums in the case of a linear relationship
between two methods of measurement. It is consistent with the obvious underlying model, and exploits the fact that although
the parameters of the model cannot be estimated, those functions of the parameters that are needed for creating predictions
can be estimated.
%----------------------------------------------------------------------------------------------------------------%
The prediction limits provided have the attractive property that if the prediction line with limits is drawn in a coordinate
system, the chart will apply in both ways; hence, both the line and the limits are symmetric. Precisely as the prediction intervals
derived from the classical LoA are in the case where the difference between methods is constant.
%----------------------------------------------------------------------------------------------------------------%
The drawback is that the regression of the differences on the means ignores that the averages are correlated with the residuals
(i.e. the error terms), and therefore gives biased estimates if the slope linking the two methods is far from 1 or the residual
variances are very different. However, both of these are rather uncommon in method comparison studies, so the method proposed
here is widely applicable.
%----------------------------------------------------------------------------------------------------------------%
When considering LoA, the only feasible transformation is the log-transform, which gives LoA for the ratio of measurements,
which is immediately understandable. If, for example, the measurements are fractions where some are close to either 0 or 1 a
logit transform may be adequate. 

LoA would then be for (log) odds-ratios, not very easily understood. For other more arbitrarily
chosen transformation the situation may be even worse. But if a plot with conversion lines and limits are constructed, then the
plot is readily back-transformed to the original scale for practical use.
%----------------------------------------------------------------------------------------------------------------%

\end{document}
