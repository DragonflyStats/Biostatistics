% !TEX TS-program = pdflatex
% !TEX encoding = UTF-8 Unicode

% This file is a template using the "beamer" package to create slides for a talk or presentation
% - Talk at a conference/colloquium.
% - Talk length is about 20min.
% - Style is ornate.

% MODIFIED by Jonathan Kew, 2008-07-06
% The header comments and encoding in this file were modified for inclusion with TeXworks.
% The content is otherwise unchanged from the original distributed with the beamer package.

\documentclass{beamer}


% Copyright 2004 by Till Tantau <tantau@users.sourceforge.net>.
%
% In principle, this file can be redistributed and/or modified under
% the terms of the GNU Public License, version 2.
%
% However, this file is supposed to be a template to be modified
% for your own needs. For this reason, if you use this file as a
% template and not specifically distribute it as part of a another
% package/program, I grant the extra permission to freely copy and
% modify this file as you see fit and even to delete this copyright
% notice. 


\mode<presentation>
{
  \usetheme{Warsaw}
  % or ...

  \setbeamercovered{transparent}
  % or whatever (possibly just delete it)
}


\usepackage[english]{babel}
% or whatever

\usepackage[utf8]{inputenc}
% or whatever

\usepackage{times}
\usepackage[T1]{fontenc}
\title[Method Comparison Studies with \texttt{R}] % (optional, use only with long paper titles)
%{Title As It Is In the Proceedings}

\subtitle
{Include Only If Paper Has a Subtitle}

\author[Author, Another] % (optional, use only with lots of authors)
{Kevin~O'Brien\inst{1} \and Dr. Kevin ~Hayes\inst{2}}
% - Give the names in the same order as the appear in the paper.
% - Use the \inst{?} command only if the authors have different
%   affiliation.

\institute[Universities of Limerick] % (optional, but mostly needed)
{
  \inst{1}%
  Department of Mathematics and Statistics\\
  University of Limerick
%  \and
%  \inst{2}%
%  Department of Mathematics and Statistics\\
%  University of  Universities of Limerick}


\date[Dublin R] % (optional, should be abbreviation of conference name)
Dublin R, 2014}
% - Either use conference name or its abbreviation.
% - Not really informative to the audience, more for people (including
%   yourself) who are reading the slides online

\subject{Method Comparison Studies with R}

% If you have a file called "university-logo-filename.xxx", where xxx
% is a graphic format that can be processed by latex or pdflatex,
% resp., then you can add a logo as follows:

% \pgfdeclareimage[height=0.5cm]{university-logo}{university-logo-filename}
% \logo{\pgfuseimage{university-logo}}



% Delete this, if you do not want the table of contents to pop up at
% the beginning of each subsection:
\AtBeginSubsection[]
{
  \begin{frame}<beamer>{Outline}
    \tableofcontents[currentsection,currentsubsection]
  \end{frame}
}


% If you wish to uncover everything in a step-wise fashion, uncomment
% the following command: 

%\beamerdefaultoverlayspecification{<+->}


\begin{document}

\begin{frame}
  \titlepage
\end{frame}

\begin{frame}{Outline}
  \tableofcontents
  % You might wish to add the option [pausesections]
\end{frame}


% Structuring a talk is a difficult task and the following structure
% may not be suitable. Here are some rules that apply for this
% solution: 

% - Exactly two or three sections (other than the summary).
% - At *most* three subsections per section.
% - Talk about 30s to 2min per frame. So there should be between about
%   15 and 30 frames, all told.

% - A conference audience is likely to know very little of what you
%   are going to talk about. So *simplify*!
% - In a 20min talk, getting the main ideas across is hard
%   enough. Leave out details, even if it means being less precise than
%   you think necessary.
% - If you omit details that are vital to the proof/implementation,
%   just say so once. Everybody will be happy with that.

\section{Overview of Talk}
%--------------------------------------------------------------------------%
\begin{frame}
\frametitle{Introduction To MCS with \texttt{R} }
\textbf{1. Basic Concepts and Definitions}

\begin{itemize}
\item Agreement
\item Consistency 
\item Repeatability
\end{itemize}

\textbf{2. Bland Altman Analysis: Limits of Agreement}
\begin{itemize}
\item R packages agRee \& MethComp
\item Deming Regression
\end{itemize}
\end{frame}
\end{document}
%--------------------------------------------------------------------------%
\begin{frame}
\frametitle{Introduction To MCS with \texttt{R} }
\textbf{3. LME models}
\begin{itemize}
\item Formulation of LME Models
\end{itemize}

\textbf{4. Roys Methods}
\begin{itemize}
\item Implementation with R
\item Likelihood Ratios Tests
\end{itemize}




\end{frame}

\end{document}
\section{Motivation}

\subsection{The Basic Problem That We Studied}

\begin{frame}{Make Titles Informative. Use Uppercase Letters.}{Subtitles are optional.}
  % - A title should summarize the slide in an understandable fashion
  %   for anyone how does not follow everything on the slide itself.

  \begin{itemize}
  \item
    Use \texttt{itemize} a lot.
  \item
    Use very short sentences or short phrases.
  \end{itemize}
\end{frame}
%----------------------------------------------------------------------------%
\begin{frame}{LME Models}

  You can create overlays\dots
  \begin{itemize}
  \item using the \texttt{pause} command:
    \begin{itemize}
    \item
      First item.
      \pause
    \item    
      Second item.
    \end{itemize}
  \item
    using overlay specifications:
    \begin{itemize}
    \item<3->
      First item.
    \item<4->
      Second item.
    \end{itemize}
  \item
    using the general \texttt{uncover} command:
    \begin{itemize}
      \uncover<5->{\item
        First item.}
      \uncover<6->{\item
        Second item.}
    \end{itemize}
  \end{itemize}
\end{frame}


\subsection{Previous Work}

\begin{frame}{Make Titles Informative.}
\end{frame}

\begin{frame}{Make Titles Informative.}
\end{frame}



\section{Our Results/Contribution}

\subsection{Main Results}

\begin{frame}{Make Titles Informative.}
\end{frame}

\begin{frame}{Make Titles Informative.}
\end{frame}

\begin{frame}{Make Titles Informative.}
\end{frame}


\subsection{Basic Ideas for Proofs/Implementation}

\begin{frame}{Make Titles Informative.}
\end{frame}

\begin{frame}{Make Titles Informative.}
\end{frame}

\begin{frame}{Make Titles Informative.}
\end{frame}



\section*{Summary}

\begin{frame}{Summary}

  % Keep the summary *very short*.
  \begin{itemize}
  \item
    The \alert{first main message} of your talk in one or two lines.
  \item
    The \alert{second main message} of your talk in one or two lines.
  \item
    Perhaps a \alert{third message}, but not more than that.
  \end{itemize}
  
  % The following outlook is optional.
  \vskip0pt plus.5fill
  \begin{itemize}
  \item
    Outlook
    \begin{itemize}
    \item
      Something you haven't solved.
    \item
      Something else you haven't solved.
    \end{itemize}
  \end{itemize}
\end{frame}




%----------------------------------------------------------- %
 %SLIDE 1

\begin{frame}{\bf \tcb{Carstensen's Mixed Models}}
\begin{itemize}
\item Carstensen \textit{et al} \cite{BXC2004} proposes linear mixed effects models for deriving
conversion calculations similar to Deming's regression, and for
estimating variance components for measurements by different
methods. 
%\item The following model (in the authors own notation) is
%formulated as follows, where $y_{mir}$ is the $r$th replicate
%measurement on subject $i$ with method $m$.
\end{itemize}
\end{frame}
%---------------------------------------------------------- %
 %SLIDE 2
\begin{frame}

\begin{itemize}
\item The following model (in the authors own notation) is
formulated as follows, where $y_{mir}$ is the $r$th replicate
measurement on subject $i$ with method $m$.
\end{itemize}
{
\LARGE
\begin{equation}
y_{mir}  = \alpha_{m} + \beta_{m}\mu_{i} + c_{mi} + e_{mir} 
\end{equation}
}
\vspace{0.3cm}
{
\normalsize
\[ e_{mi} \sim N(0,\sigma^{2}_{m}), c_{mi} \sim N(0,\tau^{2}_{m})\]
}
\end{frame}
%------------------------------------------------------- %
 %SLIDE 3
\begin{frame}[fragile]
\frametitle{Carstensen's Mixed Models}
\begin{itemize}
\item The intercept term $\alpha$ and the $\beta_{m}\mu_{i}$ term follow
from \textit{Dunn} \cite{DunnSEME}, expressing constant and proportional bias
respectively , in the presence of a real value $\mu_{i}.$
\item $c_{mi}$ is a interaction term to account for replicate, and
 $e_{mir}$ is the residual associated with each observation.
\item Since variances are specific to each method, this model can be
fitted separately for each method.
\end{itemize}

\end{frame}
%---------------------------------------------------------------- %
\begin{frame}
\frametitle{Carstensen's Mixed Models}
\begin{itemize}
\item The above formulation doesn't require the data set to be balanced.
However, it does require a sufficient large number of replicates
and measurements to overcome the problem of identifiability. 
\item The
import of which is that more than two methods of measurement may
be required to carry out the analysis. 
\end{itemize}

\end{frame}
%---------------------------------------------------------------- %
\begin{frame}
\frametitle{Carstensen's Mixed Models}
\begin{itemize}
\item There is also the
assumptions that observations of measurements by particular
methods are exchangeable within subjects. \item \textbf{\textit{Exchangeability}} means
that future samples from a population behaves like earlier
samples).
\end{itemize}
\end{frame}
%---------------------------------------------------------------- %


\subsection{Using LME models to create Prediction Intervals}

\begin{frame}
\Large
\begin{itemize}
\item Carstensen \textit{et al} \cite{BXC2004} also advocates the use of linear mixed models in
the study of method comparisons. The model is constructed to
describe the relationship between a value of measurement and its
real value.
\item  The non-replicate case is considered first, as it is
the context of the Bland Altman plots. This model assumes that
inter-method bias is the only difference between the two methods.
A measurement $y_{mi}$ by method $m$ on individual $i$ is
formulated as follows;
\end{itemize}
\begin{equation}
y_{mi}  = \alpha_{m} + \mu_{i} + e_{mi} \qquad ( e_{mi} \sim
N(0,\sigma^{2}_{m}))
\end{equation}

\end{frame}
\begin{frame}
\Large
\begin{itemize}
\item The differences are expressed as $d_{i} = y_{1i} - y_{2i}$ For the
replicate case, an interaction term $c$ is added to the model,
with an associated variance component. 
\item All the random effects are
assumed independent, and that all replicate measurements are
assumed to be exchangeable within each method.
\end{itemize}
\begin{eqnarray}
y_{mir}  = \alpha_{m} + \mu_{i} + c_{mi} + e_{mir} 
\end{eqnarray}

%\[  e_{mi} \sim N(0,\sigma^{2}_{m}) \c_{mi} \sim N(0,\tau^{2}_{m}) \]
%\end{eqnarray}
\end{frame}

\begin{frame}
\Large
\begin{itemize}
\item Carstensen \textit{et al} \cite{BXC2008} proposes a methodology to calculate prediction
intervals in the presence of replicate measurements, overcoming
problems associated with Bland-Altman methodology in this regard.
\item It is not possible to estimate the interaction variance components
$\tau^{2}_{1}$ and $\tau^{2}_{2}$ separately. Therefore it must be
assumed that they are equal. The variance of the difference can be
estimated as follows:
\begin{equation}
var(y_{1j}-y_{2j})
\end{equation}
\end{itemize}
\end{frame}


\section[References]{References}
\subsection{References}
\begin{frame}{\bf \tcb{References}}
\begin{thebibliography}{99}

\bibitem{BXC2004} Carstensen, Gurrin (2004): \emph{Generalized Linear Models},
Chapman and Hall/CRC.
\bibitem{BXC2008} Carstensen, Gurrin (2008): \emph{Generalized Linear Models},
Chapman and Hall/CRC.

\bibitem{DunnSEME} Dunn, P. and Nelder, J. (1989): \emph{SEME},
Chapman and Hall/CRC.
\bibitem{Roy2009} A Roy (2009): \emph{An application of linear mixed effects model to assess the agreement between two methods with replicated observations} Journal of Biopharmaceutical Statistics
\bibitem{BA86} Bland JM, Altman DG (1986) \emph{Statistical method for assessing agreement between two methods of clinical measurement}.
\bibitem{BA99} Bland JM, Altman DG (1999)  \emph{Measuring agreement in method comparison studies.} Statistical Methods in Medical Research
\bibitem{PB} Pinheiro JC, Bates DM (2000): \emph{Mixed-effects models in S and S-PLUS},
Springer.
%\bibitem{BXC2008} B Carstensen, J Simpson, and LC Gurrin (2008): \emph{ Statistical models for assessing agreement in method comparison studies with replicate measurements.} International Journal of Biostatistics.
\end{thebibliography}
\end{frame}


%------------------------------------------------------------------------%

\begin{frame}{\bf \tcb{Thanks}}
\begin{itemize}
\item Dr Kevin Hayes, University of Limerick
\item Mr Kevin Burke, university of Limerick
\end{itemize}
\end{frame}


\end{document} 


%---------------------------------------------------------------------- %

%\begin{frame}[allowframebreaks]
%        \frametitle{References}
%        \bibliographystyle{amsalpha}
%        \bibliography{DB-txfrbib.bib}
%\end{frame}



\end{document}


