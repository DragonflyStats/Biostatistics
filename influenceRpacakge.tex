%================================================================== %
\newpage
\section*{Influence Measures with \textit{influence.ME}}influence.ME calculates measures of influence for mixed effects models estimated with the lme4 R package. The
basic rationale behind measuring influential cases is that when iteratively single units are omitted

from the data, models based on these data should not produce substantially different estimates. 

Calculating measures of influential data for an LME model requires the re-estimation
of this model for each set of potentially influential data separately. The \texttt{estex(} function does this,
and returns the altered estimates resulting from each re-estimation. 

The main function in the influence.ME package is the \texttt{influence()}.

\begin{quote} Based on a priorly estimated
	mixed effects regression model (estimated using lme4), the \texttt{influence()} function iteratively modifies
	the mixed effects model to neutralize the effect a grouped set of data has on the parameters, and
	which returns returns the fixed parameters of these iteratively modified models. These are used to
	compute measures of influential data. ()
\end{quote}

\subsection*{Using the influence.ME package}
Influence Analysis can onnly be carried out with LME models fitted using the functions in the \textbf{lme4} package. Such models are known as \texttt{mer} objects.
Hence the \texttt{estex()} function only works on LME
models of class \texttt{mer}.
The package developers advise that it is required that the \texttt{mer} model was estimated using a factor variable to indicate group levels.
When using something similar to \texttt{+ (1 | as.factor(variable))}, the function is not able of
identifying the correct grouping factors, and returns an error.

Executing this procedure can be computationally highly demanding, because \texttt{estex()} entails the re-estimation of the provided mixed effects model for each level of the specified grouping factor (after alteration of the data).
%============================================================================= %
\subsection*{Functionality of the influence.ME pacakge}
To standardize the assessment of how influential an observation (or group of observations)is, several measures
of influence are used by influence.ME.


\begin{itemize}
	\item DFBETAS is a standardized measure of the absolute difference
	between the estimate with a particular case included and the estimate without that particular
	case. 
	\item Cook’s distance provides an overall measurement of the change in all parameter
	estimates, or a selection thereof.
\end{itemize}

The \texttt{estex()} command computes revised estimates can subsequently
be entered to the \texttt{cooks.distance} and \texttt{dfbetas} commands, to calculate Cook’s Distance
and the DFBETAS (standardized difference of the beta) measures.
%--------------------------------------------------------------%
\subsubsection*{The \texttt{pchange} command}

The \texttt{pchange} command computes the percentile change, as a measure of influential data. This unstandardized measure can
serve to help interpret the magnitude of the influence single or combined grouping levels exert on
mixed effects models. 

The percentage change in parameter estimates between an LME model based on a full set of data, and a model from which a (potentially influential)
subset of data is removed. A value of percentage change is calculated for each parameter in the
model separately, based on the information returned by the \texttt{estex()} function.

\subsection*{sigtest}

The \texttt{sigTest} function can test for changes in the level of statistical significance resulting from
the deletion of potentially influential observations

