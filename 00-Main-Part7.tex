\documentclass[12pt, a4paper]{article}
\usepackage{natbib}
\usepackage{vmargin}
\usepackage{graphicx}
\usepackage{epsfig}
\usepackage{subfigure}
%\usepackage{amscd}
\usepackage{amssymb}
\usepackage{subfiles}
\usepackage{subfigure}
\usepackage{framed}
\usepackage{subfiles}
\usepackage{amsbsy}
\usepackage{amsthm, amsmath}
%\usepackage[dvips]{graphicx}
\bibliographystyle{chicago}
\renewcommand{\baselinestretch}{1.1}

% left top textwidth textheight headheight % headsep footheight footskip
\setmargins{3.0cm}{2.5cm}{15.5 cm}{23.5cm}{0.25cm}{0cm}{0.5cm}{0.5cm}

\pagenumbering{arabic}
%-------------------------------------------------------------------Simplifying GLS by KH -%


\begin{document}

\chapter{Residual Analysis}
\section{Introduction to Residual Analysis}

Residual analysis is a widely used model validation technique. A residual is simply the difference between an observed value and the corresponding fitted value, as predicted by the model. The rationale is that, if the model is properly fitted to the model, then the residuals would approximate the random errors that one should expect.
that is to say, if the residuals behave randomly, with no discernible trend, the model has fitted the data well. If some sort of non-random trend is evident in the model, then the model can be considered to be poorly fitted.

%========================================================================================================= %
%\subsection{Introduction}
%A statistical model, whether of the fixed-effects or mixed-effects variety, represents how you think your data were generated. 
%Following model specification and estimation, it is of interest to explore the model-data
%agreement by raising questions such as

Statistical software environments, such as the \texttt{R} Programming language, provides a suite of tests and graphical procedure sfor appraising a fitted linear model, with several 
of these procedures analysing the model residuals.

\newpage
%=========================================================================%
\section{Framework for Model Validation using Residual Diagnostics}
In statistical modelling, the process of model validation is a critical step, but also a step that is too often overlooked. A very simple procedure is to examine commonly encountered
metrics, such as the $R^2$ value. However, using a small handful of simple measures and methods is insufficient to properly assess the quality of a fitted model. To do so properly, a full and comprehensive
analysis that tests of all of the assumptions, as far as possible, must be carried out. A statistical model, whether of the fixed-effects or mixed-effects variety, represents how you think your data
were generated. Following model specification and estimation, it is of interest to explore the model-data
agreement by raising questions such as
\begin{itemize}
	\item Does the model-data agreement support the model assumptions?
	\item Should model components be refined, and if so, which components? For example, should regressors
	be added or removed, and is the covariation of the observations modeled properly?
	\item Are the results sensitive to model and/or data? Are individual data points or groups of cases particularly
	influential on the analysis?
\end{itemize}


\subsection{Residual Analysis}
A residual is the difference between an observed quantity and its
estimated or predicted value. 
Residual analysis is a widely used model validation technique. A residual is simply the difference between an observed value and the corresponding fitted value, as predicted by the model. The rationale is that, if the model is properly fitted to the model, then the residuals would approximate the random errors that one should expect.
that is to say, if the residuals behave randomly, with no discernible trend, the model has fitted the data well. If some sort of non-random trend is evident in the model, then the model can be considered to be poorly fitted.
Statistical software environments, such as the \texttt{R} Programming language, provides a suite of tests and graphical procedure sfor appraising a fitted linear model, with several 
of these procedures analysing the model residuals.

In classical linear models, an examination of model-data agreement has traditionally revolved around

The second part of the chapter looks at diagnostics techniques for LME models, firsly covering the theory, then proceeding to a discussion on 
implementing these using \texttt{R} code.

While a substantial body of work has been developed in this area, there is still areas worth exploring. 
In particular the development of graphical techniques pertinent to LME models should be looked at.




%\section{Introduction (Page 1)}
%
%Linear models for uncorrelated data have well established measures to gauge the influence of one or more
%observations on the analysis. For such models, closed-form update expressions allow efficient computations
%without refitting the model. 
%
%
%When similar notions of statistical influence are applied to mixed models,
%things are more complicated. Removing data points affects fixed effects and covariance parameter estimates.
%Update formulas for “\textit{leave-one-out}” estimates typically fail to account for changes in covariance
%parameters. 
%
%Moreover, in repeated measures or longitudinal studies, one is often interested in multivariate
%influence, rather than the impact of isolated points. 

% This paper examines extensions of influence measures
% in linear mixed models and their implementation in the MIXED procedure.















%========================================================================================================= %
\subsection{Outliers and Leverage}



The question of whether or not a point should be considered an outlier must also be addressed. An outlier is an observation whose true value is unusual given its value on the predictor variables. The leverage of an observation is a further consideration. Leverage describes an observation with an extreme value on a predictor variable is a point with high leverage. High leverage points can have a great amount of effect on the estimate of regression coefficients.
% - Leverage is a measure of how far an independent variable deviates from its mean.

Influence can be thought of as the product of leverage and outlierness. An observation is said to be influential if removing the observation substantially changes the estimate of the regression coefficients. The \texttt{R} programming language has a variety of methods used to study each of the aspects for a linear model. While linear models and GLMS can be studied with a wide range of well-established diagnostic technqiues, the choice of methodology is much more restricted for the case of LMEs.

%---------------------------------------------------------------------------%
%\newpage
%\section{Residual diagnostics} %1.3
For classical linear models, residual diagnostics are typically conducted using a plot of the observed residuals and the predicted values. A visual inspection for the presence of trends inform the analyst on the validity of distributional assumptions, and to detect outliers and influential observations.

%\section{Case Deletion Diagnostics}
%
%
%Linear models for uncorrelated data have well established measures to gauge the influence of one or more
%observations on the analysis. For such models, closed-form update expressions allow efficient computations
%without refitting the model. 
%
%
%Since the pioneering work of Cook in 1977, deletion measures have been applied to many statistical models for identifying influential observations. Case-deletion diagnostics provide a useful tool for identifying influential observations and outliers.
%
%The key to making deletion diagnostics useable is the development of efficient computational formulas, allowing one to obtain the \index{case deletion diagnostics} case deletion diagnostics by making use of basic building blocks, computed only once for the full model.
%
%The computation of case deletion diagnostics in the classical model is made simple by the fact that estimates of $\beta$ and $\sigma^2$, which exclude the $i-$th observation, can be computed without re-fitting the model. %\subsection{Terminology for Case Deletion diagnostics} %1.8
%
%\citet{preisser} describes two type of diagnostics. When the set consists of only one observation, the type is called
%`\textit{observation-diagnostics}'. For multiple observations, Preisser describes the diagnostics as `\textit{cluster-deletion}' diagnostics. When applied to LME models, such update formulas are available only if one assumes that the covariance parameters are not affected by the removal of the observation in question. However, this is rarely a reasonable assumption.
%
%
%
%
%%---------------------------------------------------------------------------%
\subsection{Matrix Notation for Case Deletion} %1.14

%\subsection{Case deletion notation} %1.14.1

For notational simplicity, $\boldsymbol{A}(i)$ denotes an $n \times m$ matrix $\boldsymbol{A}$ with the $i$-th row
removed, $a_i$ denotes the $i$-th row of $\boldsymbol{A}$, and $a_{ij}$ denotes the $(i, j)-$th element of $\boldsymbol{A}$.
%
%\subsection{Partitioning Matrices} %1.14.2
%Without loss of generality, matrices can be partitioned as if the $i-$th omitted observation is the first row; i.e. $i=1$.



%-------------------------------------------------------------------------------------------------------------------------------------%
%--------------------------------------%
\subsection{Extension of Diagnostic Methods to LME models}

<<<<<<< HEAD

When similar notions of statistical influence are applied to mixed models,
things are more complicated. Removing data points affects fixed effects and covariance parameter estimates.
Update formulas for “\textit{leave-one-out}” estimates typically fail to account for changes in covariance
parameters. 
%
%
%In LME models, there are two types of residuals, marginal residuals and conditional residuals. A
%marginal residual is the difference between the observed data and the estimated marginal mean. A conditional residual is the
%difference between the observed data and the predicted value of the observation. In a model without random effects, both sets of residuals coincide \citep{schab}.

\citet{Christiansen} noted the case deletion diagnostics techniques have not been applied to linear mixed effects models and seeks to develop methodologies in that respect. \citet{Christiansen} develops these techniques in the context of REML.
=======
\citet{CPJ} noted the case deletion diagnostics techniques had not been applied to linear mixed effects models and seeks to develop methodologies in that respect. \citet{CPJ} develops these techniques in the context of REML.
>>>>>>> origin/master

%\citet{CPJ} develops \index{case deletion diagnostics} case deletion diagnostics, in particular the equivalent of \index{Cook's distance} Cook's distance, a well-known metric, for diagnosing influential observations when estimating the fixed effect parameters and variance components. Deletion diagnostics provide a means of assessing the influence of an observation (or groups of observations) on inference on the estimated parameters of LME models. We shall provide a fuller discussion of Cook's distance in due course.


\citet{Demi} extends several regression diagnostic techniques commonly used in linear regression, such as leverage, infinitesimal influence, case deletion diagnostics, Cook's distance, and local influence to the linear mixed-effects model. In each case, the proposed new measure has a direct interpretation in terms of the effects on a parameter of interest, and reduces to the familiar linear regression measure when there are no random effects. 

The new measures that are proposed by \citet{Demi} are explicitly defined functions and do not require re-estimation of the model, especially for cluster deletion diagnostics. The basis for both the cluster deletion diagnostics and Cook's distance is a generalization of Miller's simple update formula for case deletion for linear models. Furthermore \citet{Demi} shows how Pregibon's infinitesimal case deletion diagnostics is adapted to the linear mixed-effects model. 
%A simple compact matrix formula is derived to assess the local influence of the fixed-effects regression coefficients. 


%
%
%\section{Case Deletion Diagnostics for LME models} %1.6
%
%Data from single individuals, or a small group of subjects may influence non-linear mixed effects model selection. Diagnostics routinely applied in model building may identify such individuals, but these methods are not specifically designed for that purpose and are, therefore, not optimal. 

\citet{Demi} proposes two likelihood-based diagnostics for identifying individuals that can influence the choice between two competing models.


\newpage




\section{Model Validation Framework}
%\section{Model Validation using Residual Diagnostics}
In statistical modelling, the process of model validation is a critical step of model fitting process, but also a step that is too often overlooked. A very simple procedure is to examine commonly-used
metrics, such as the $R^2$ value. However, using a small handful of simple measures and methods is insufficient to properly assess the quality of a fitted model. To do so properly, a full and comprehensive
analysis that tests of all of the assumptions, as far as possible, must be carried out.

%=========================================================================%
%\subsection{Model Validation Framework}
%In classical linear models, this examination of model-data agreement has traditionally revolved around
\citet{schab} describes the model validatin framework as comprised of the following tasks

\begin{itemize}
	\item  overall measures of goodness-of-fit
	\item the informal, graphical examination of estimates of model errors to assess the quality of distributional
	assumptions: residual analysis
	
	
	\item the quantitative assessment of the inter-relationship of model components; for example, collinearity 	diagnostics
	\item the qualitative and quantitative assessment of influence of cases on the analysis, i.e. influence analysis.
\end{itemize}

The sensitivity of a model is studied through measures that express its stability under perturbations. You
are not interested in a model that is either overly stable or overly sensitive. Changes in the data or model
components should produce commensurate changes in the model output. The difficulty is to determine
when the changes are substantive enough to warrant further investigation, possibly leading to a reformulation
of the model or changes in the data (such as dropping outliers). This paper is primarily concerned
with stability of linear mixed models to perturbations of the data; that is, with influence analysis. 
%========================================================================================================= %
\newpage
\subsection{Residual Analysis}

Residual analysis is a widely used model validation technique. A residual is simply the difference between an observed value and the corresponding fitted value, as predicted by the model. The rationale is that, if the model is properly fitted to the model, then the residuals would approximate the random errors that one should expect.
that is to say, if the residuals behave randomly, with no discernible trend, the model has fitted the data well. If some sort of non-random trend is evident in the model, then the model can be considered to be poorly fitted.

%========================================================================================================= %
%\subsection{Introduction}
%A statistical model, whether of the fixed-effects or mixed-effects variety, represents how you think your data were generated. 
%Following model specification and estimation, it is of interest to explore the model-data
%agreement by raising questions such as

Statistical software environments, such as the \texttt{R} Programming language, provides a suite of tests and graphical procedure sfor appraising a fitted linear model, with several 
of these procedures analysing the model residuals.

%========================================================================================================= %
\subsection{Outliers and Leverage}



The question of whether or not a point should be considered an outlier must also be addressed. An outlier is an observation whose true value is unusual given its value on the predictor variables. The leverage of an observation is a further consideration. Leverage describes an observation with an extreme value on a predictor variable is a point with high leverage. High leverage points can have a great amount of effect on the estimate of regression coefficients.
% - Leverage is a measure of how far an independent variable deviates from its mean.

Influence can be thought of as the product of leverage and outlierness. An observation is said to be influential if removing the observation substantially changes the estimate of the regression coefficients. The \texttt{R} programming language has a variety of methods used to study each of the aspects for a linear model. While linear models and GLMS can be studied with a wide range of well-established diagnostic technqiues, the choice of methodology is much more restricted for the case of LMEs.



\section{Regression Of Differences On Averages}
Further to Carstensen, we can formulate the two measurements
$y_{1}$ and $y_{2}$ as follows:
\\
$y_{1} = \alpha + \beta\mu + \epsilon_{1}$
\\
$y_{2} = \alpha + \beta\mu + \epsilon_{2}$









\newpage
\subsection{Remarks on the Multivariate Normal Distribution}

Diligence is required when considering the models. Carstensen specifies his models in terms of the univariate normal distribution. Roy's model is specified using the bivariate normal distribution.
This gives rises to a key difference between the two model, in that a bivariate model accounts for covariance between the variables of interest.
The multivariate normal distribution of a $k$-dimensional random vector $X = [X_1, X_2, \ldots, X_k]$
can be written in the following notation:
\[
X\ \sim\ \mathcal{N}(\mu,\, \Sigma),
\]
or to make it explicitly known that $X$ is $k$-dimensional,
\[
X\ \sim\ \mathcal{N}_k(\mu,\, \Sigma).
\]
with $k$-dimensional mean vector
\[ \mu = [ \operatorname{E}[X_1], \operatorname{E}[X_2], \ldots, \operatorname{E}[X_k]] \]
and $k \times k$ covariance matrix
\[ \Sigma = [\operatorname{Cov}[X_i, X_j]], \; i=1,2,\ldots,k; \; j=1,2,\ldots,k \]

\bigskip

\begin{enumerate}
	\item Univariate Normal Distribution
	
	\[
	X\ \sim\ \mathcal{N}(\mu,\, \sigma^2),
	\]
	
	\item Bivariate Normal Distribution
	
	\begin{itemize}
		\item[(a)] \[  X\ \sim\ \mathcal{N}_2(\mu,\, \Sigma), \vspace{1cm}\]
		\item[(b)] \[    \mu = \begin{pmatrix} \mu_x \\ \mu_y \end{pmatrix}, \quad
		\Sigma = \begin{pmatrix} \sigma_x^2 & \rho \sigma_x \sigma_y \\
		\rho \sigma_x \sigma_y  & \sigma_y^2 \end{pmatrix}.\]
	\end{itemize}
\end{enumerate}
\newpage

\subsection{Note 1: Coefficient of Repeatability}
The coefficient of repeatability is a measure of how well a
measurement method agrees with itself over replicate measurements
\citep{BA99}. Once the within-item variability is known, the
computation of the coefficients of repeatability for both methods
is straightforward.



\subsection{Note 2: Carstensen model in the single measurement case}
\citet{BXC2004} presents a model to describe the relationship between a value of measurement and its real value.
The non-replicate case is considered first, as it is the context of the Bland-Altman plots.
This model assumes that inter-method bias is the only difference between the two methods.


\begin{equation}
y_{mi}  = \alpha_{m} + \mu_{i} + e_{mi} \qquad  e_{mi} \sim \mathcal{N}(0,\sigma^{2}_{m})
\end{equation}

The differences are expressed as $d_{i} = y_{1i} - y_{2i}$.

For the replicate case, an interaction term $c$ is added to the model, with an associated variance component.




\subsection{Note 3: Model terms}
It is important to note the following characteristics of this model.
\begin{itemize}
	\item Let the number of replicate measurements on each item $i$ for both methods be $n_i$, hence $2 \times n_i$ responses. However, it is assumed that there may be a different number of replicates made for different items. Let the maximum number of replicates be $p$. An item will have up to $2p$ measurements, i.e. $\max(n_{i}) = 2p$.
	
	% \item $\boldsymbol{y}_i$ is the $2n_i \times 1$ response vector for measurements on the $i-$th item.
	% \item $\boldsymbol{X}_i$ is the $2n_i \times  3$ model matrix for the fixed effects for observations on item $i$.
	% \item $\boldsymbol{\beta}$ is the $3 \times  1$ vector of fixed-effect coefficients, one for the true value for item $i$, and one effect each for both methods.
	
	\item Later on $\boldsymbol{X}_i$ will be reduced to a $2 \times 1$ matrix, to allow estimation of terms. This is due to a shortage of rank. The fixed effects vector can be modified accordingly.
	\item $\boldsymbol{Z}_i$ is the $2n_i \times  2$ model matrix for the random effects for measurement methods on item $i$.
	\item $\boldsymbol{b}_i$ is the $2 \times  1$ vector of random-effect coefficients on item $i$, one for each method.
	\item $\boldsymbol{\epsilon}$  is the $2n_i \times  1$ vector of residuals for measurements on item $i$.
	\item $\boldsymbol{G}$ is the $2 \times  2$ covariance matrix for the random effects.
	\item $\boldsymbol{R}_i$ is the $2n_i \times  2n_i$ covariance matrix for the residuals on item $i$.
	\item The expected value is given as $\mbox{E}(\boldsymbol{y}_i) = \boldsymbol{X}_i\boldsymbol{\beta}.$ \citep{hamlett}
	\item The variance of the response vector is given by $\mbox{Var}(\boldsymbol{y}_i)  = \boldsymbol{Z}_i \boldsymbol{G} \boldsymbol{Z}_i^{\prime} + \boldsymbol{R}_i$ \citep{hamlett}.
\end{itemize}
\newpage

%\chapter{Limits of Agreement}

%\section{Modelling Agreement with LME Models}

% Carstensen pages 22-23


Roys uses and LME model approach to provide a set of formal tests for method comparison studies.\\

Four candidates models are fitted to the data.\\

These models are similar to one another, but for the imposition of equality constraints.\\

These tests are the pairwise comparison of candidate models, one formulated without constraints, the other with a constraint.\\


Roy's model uses fixed effects $\beta_0 + \beta_1$ and $\beta_0 + \beta_1$ to specify the mean of all observationsby \\ methods 1 and 2 respectuively.





Roy adheres to Random Effect ideas in ANOVA

Roy treats items as a sample from a population.\\

Allocation of fixed effects and random effects are very different in each model\\

Carstensen's interest lies in the difference between the population from which they were drawn.\\

Carstensen's model is a mixed effects ANOVA.\\

\[
Y_{mir}  =  \alpha_m + \mu_i + c_{mi} + e_{mir}, \qquad c_{mi} \sim \mathcal{\tau^2_m}, \qquad e_{mir} \sim \mathcal{\sigma^2_m},
\]

This model includes a method by item iteration term.\\

Carstensen presents two models. One for the case where the replicates, and a second for when they are linked.\\

Carstensen's model does not take into account either between-item or within-item covariance between methods.\\


In the presented example, it is shown that Roy's LoAs are lower than those of Carstensen.
Carstensen makes some interesting remarks in this regard.

\begin{quote}
	The only slightly non-standard (meaning "not often used") feature is the differing residual variances between methods.
\end{quote}
\newpage


\subsection{Note 3: Model terms}
It is important to note the following characteristics of this model.
\begin{itemize}
	\item Let the number of replicate measurements on each item $i$ for both methods be $n_i$, hence $2 \times n_i$ responses. However, it is assumed that there may be a different number of replicates made for different items. Let the maximum number of replicates be $p$. An item will have up to $2p$ measurements, i.e. $\max(n_{i}) = 2p$.
	
	% \item $\boldsymbol{y}_i$ is the $2n_i \times 1$ response vector for measurements on the $i-$th item.
	% \item $\boldsymbol{X}_i$ is the $2n_i \times  3$ model matrix for the fixed effects for observations on item $i$.
	% \item $\boldsymbol{\beta}$ is the $3 \times  1$ vector of fixed-effect coefficients, one for the true value for item $i$, and one effect each for both methods.
	
	\item Later on $\boldsymbol{X}_i$ will be reduced to a $2 \times 1$ matrix, to allow estimation of terms. This is due to a shortage of rank. The fixed effects vector can be modified accordingly.
	\item $\boldsymbol{Z}_i$ is the $2n_i \times  2$ model matrix for the random effects for measurement methods on item $i$.
	\item $\boldsymbol{b}_i$ is the $2 \times  1$ vector of random-effect coefficients on item $i$, one for each method.
	\item $\boldsymbol{\epsilon}$  is the $2n_i \times  1$ vector of residuals for measurements on item $i$.
	\item $\boldsymbol{G}$ is the $2 \times  2$ covariance matrix for the random effects.
	\item $\boldsymbol{R}_i$ is the $2n_i \times  2n_i$ covariance matrix for the residuals on item $i$.
	\item The expected value is given as $\mbox{E}(\boldsymbol{y}_i) = \boldsymbol{X}_i\boldsymbol{\beta}.$ \citep{hamlett}
	\item The variance of the response vector is given by $\mbox{Var}(\boldsymbol{y}_i)  = \boldsymbol{Z}_i \boldsymbol{G} \boldsymbol{Z}_i^{\prime} + \boldsymbol{R}_i$ \citep{hamlett}.
\end{itemize}
\newpage
\newpage
\section{Regression Of Differences On Averages}
Further to Carstensen, we can formulate the two measurements
$y_{1}$ and $y_{2}$ as follows:
\\
$y_{1} = \alpha + \beta\mu + \epsilon_{1}$
\\
$y_{2} = \alpha + \beta\mu + \epsilon_{2}$




\bibliographystyle{chicago}
\bibliography{DB-txfrbib}
\end{document}

