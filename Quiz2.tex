Question 1
What is the delta method asymptotic standard error of p^√ where p^ is X/n where X∼Binomial(n,p)?

Your Answer		Score	Explanation
1−p2pn−−−√			
p(1−p)/n−−−−−−−−√	Inorrect	0.00	
(p(1−p)n)1/4			
(1−p)4n−−−−√			
Total		0.00 / 1.00	
Question Explanation

f(x)=x½ then f′(x)=x−½/2. Then Var(p^)=p(1−p)/n and thus the asymptotic standard error is
12p½p(1−p)n−−−−−−−√=(1−p)4n−−−−−−√
Question 2
You are a bookie taking $1 bets on the NFL Ravens game. The odds you give betters determines their payout. If you give 4 to 1 odds against the Ravens winning, then if a person bets on the Ravens winning and they win, you owe them their original dollar, plus an additional $4 (a total of $5). If a person bets on the Ravens losing and they lose, then you owe them their original dollar back, plus $0.25 (a total of $1.25).

Suppose you collect a > 0 one dollar bets on the Ravens winning and b > 0 one dollar bets on the Ravens losing. What should you set the odds so that, regardless of the outcome of the game, you neither win nor lose money? Note, in this case, the betters place their bets and learn the odds later. Note also, the odds are something that you set in this case, not a direct measure of probability or randomness.

Your Answer		Score	Explanation
(b+1)/(a+1)			
b/a			
a/b			
(a+1)/(b+1)	Inorrect	0.00	
Total		0.00 / 1.00	
Question Explanation

Suppose that you give odds against the Ravens of H. If the Ravens win, you owe
Ha+a−(a+b)
where Ha+a is the payout for those that bet on the Ravens winning, plus their original dollars back, and (a+b) is the amount of money that you collected.

If the Ravens lose, you owe
b/H+b−(a+b)
where b/H+b is the payout for those betting on the Ravens losing, plus their original dollars back and (a+b) is the amount that you collected.

Solving either equation for 0 yeilds H=b/a.

Note, how do you make money as a bookie? You charge a convenience fee for accepting the bet, say of $0.10. This way you are not gambling, only making money on the number of bets made. Note also, even though the odds you set are not a direct measure of probability, odds collected this way tend to have good frequency properties.
Question 3
In a randomly sampled survey of self-reported stress levels from two occupations, the following data were obtained

High Stress	Low Stress
Professor	70
Lion Tamer	15
What is the P-value for a Z score test of equality of the proportions of high stress?

(Note the notation 1e-5, for example, is 1×10−5).
Your Answer		Score	Explanation
4e-15			
1e-14	Inorrect	0.00	
1e-5			
1e-3			
Total		0.00 / 1.00	
Question Explanation


    2-sample test for equality of proportions without continuity
    correction

data:  matrix(c(70, 15, 30, 85), 2) 
X-squared = 61.89, df = 1, p-value = 3.627e-15
alternative hypothesis: two.sided 
95 percent confidence interval:
 0.4361 0.6639 
sample estimates:
prop 1 prop 2 
  0.70   0.15 
Question 4
Consider the following data recording case status relative to an environmental exposure

Case	Control
Exposed	45	21
Unexposed	15	52
What would be the estimated asymptotic standard error for the log relative risk for this data? Consider case status as the outcome. That is, we're interested in evaluating the ratio of the proportion of cases comparing exposed to unexposed groups.

Your Answer		Score	Explanation
Around 1.25			
Around 1.05	Inorrect	0.00	
Around 0.05			
Around 0.25			
Total		0.00 / 1.00	
Question Explanation

n1 <- 45 + 21
n2 <- 15 + 52
p1 <- 45/n1
p2 <- 15/n2
sqrt((1 - p1)/n1/p1 + (1 - p2)/n2/p2)
[1] 0.2425
Question 5
If x1∼Binomial(n1,p1) and x2∼Binomial(n2,p2) and independent Beta(2,2) priors are placed on p1 and p2, what is the posterior mean for p1−p2?

Your Answer		Score	Explanation
x1n1−x2n2			
x1+2n1+4−x2+2n2+4	Correct	1.00	
x1+1n1+1−x2+1n2+1			
x1+1n1+2−x2+1n2+2			
Total		1.00 / 1.00	
Question Explanation

Consider the posterior for p1
px11(1−p1)n1−x1×p1(1−p1)=px1+2−11(1−p1)n1−x1+2−1
Thus p1 given the data is Beta(x1+2,n1−x1+2). This has mean x1+2n1+4. Thus
E[p1−p2|x1,x2]=x1+2n1+4−x2+2n2+4
Question 6
Researchers conducted a blind taste test of Coke versus Pepsi. Each of four people was asked which of two blinded drinks given in random order that they preferred. The data was such that 3 of the 4 people chose Coke. Assuming that this sample is representative, report a P-value for a test of the hypothesis that Coke is preferred to Pepsi using a **two sided** exact test.

Your Answer		Score	Explanation
Around 0.6	Correct	1.00	
Around 0.5			
Around 0.4			
Around 0.3			
Total		1.00 / 1.00	
Question Explanation

Let p be the proportion of people who prefer Coke. Then, we want to test H0:p=.5 versus Ha:p≠0.5. Let X be the number out of 4 that prefer Coke; assume X∼Binomial(p,.5). Pvalue=2∗P(X≥3)=2choose(4,3)0.530.51+2choose(4,4)0.540.50
2 * pbinom(2, size = 4, prob = 0.5, lower.tail = FALSE)
[1] 0.625
2 * (choose(4, 3) * 0.5^4 + choose(4, 4) * 0.5^4)
[1] 0.625
