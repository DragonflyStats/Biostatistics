
\documentclass[12pt, a4paper]{report}
\usepackage{natbib}
\usepackage{vmargin}
\usepackage{graphicx}
\usepackage{epsfig}
\usepackage{subfigure}
%\usepackage{amscd}
\usepackage{amssymb}
\usepackage{subfigure}
\usepackage{amsbsy}
\usepackage{amsthm, amsmath}
%\usepackage[dvips]{graphicx}
\bibliographystyle{chicago}
\renewcommand{\baselinestretch}{1.8}

% left top textwidth textheight headheight % headsep footheight footskip
\setmargins{3.0cm}{2.5cm}{15.5 cm}{23.5cm}{0.5cm}{0cm}{1cm}{1cm}

\pagenumbering{arabic}


\begin{document}
\author{Kevin O'Brien}
\title{Transfer Report}
\date{\today}
\maketitle

\tableofcontents \setcounter{tocdepth}{2}

\newpage
\chapter{Introduction}

\section{Introduction}

\chapter{Proposal of Research}

\section{The Linear Mixed Effects Model}
The linear mixed effects model is given by
\begin{equation}
Y = X\beta + Zu + \epsilon
\end{equation}


\textbf{Y} is the vector of $n$ observations, with dimension $n
\times 1$. \textbf{b} is a vector of fixed $p$ effects, and has
dimension $p \times 1$. It is composed of coefficients, with the
first element being the population mean.  \textbf{X} is known as
the design `matrix', model matrix for fixed effects, and comprises
$0$s or $1$s, depending on whether the relevant fixed effects have
any effect on the observation is question. \textbf{X} has
dimension $n \times p$. \textbf{e} is the vector of residuals with
dimension $n \times 1$.

The random effects models can be specified similarly. \textbf{Z}
is known as the `model matrix for random effects', and also
comprises $0$s or $1$s. It has dimension $n \times q$. \textbf{u
}is a vector of random $q$ effects, and has dimension $q \times
1$.


% \subsection{Formulation of the Variance Matrix V}
\textbf{V} , the variance matrix of \textbf{Y}, can be expressed
as follows;
\begin{eqnarray}
\textbf{V}= var ( \textbf{Xb} + \textbf{Zu} + \textbf{e})\\
\textbf{V}= var ( \textbf{Xb} ) + var (\textbf{Zu}) +
var(\textbf{e}))
\end{eqnarray}

$\mbox{var}(\textbf{Xb})$ is known to be zero. The variance of the
random effects $\mbox{var}(\textbf{Zu})$ can be written as
$Z\mbox{var}(\textbf{u})Z^{T}$.

By letting var$(u) = G$ (i.e $\textbf{u} ~ N(0,\textbf{G})$), this
becomes $ZGZ^{T}$. This specifies the covariance due to random
effects. The residual covariance matrix $var(e)$ is denoted as
$R$, ($\textbf{e} ~ N(0,\textbf{R})$). Residual are uncorrelated,
hence \textbf{R} is equivalent to $\sigma^{2}$\textbf{I}, where
\textbf{I} is the identity matrix. The variance matrix \textbf{V}
can therefore be written as;

\begin{equation}
\textbf{V}  = ZGZ^{T} + \textbf{R}
\end{equation}

%\subsection{Estimators and Predictors}

The best linear unbiased predictor (BLUP) is used to estimating
random effects, i.e to derive \textbf{u}. The best linear unbiased
estimator (BLUE) is used to estimate the fixed effects,
\textbf{b}. They were formulated in a paper by \cite{Henderson59},
which provides the derivations of both. Inferences about fixed
effects have come to be called `estimates', whereas inferences
about random effects have come be called `predictions`. hence the
naming of BLUP is to reinforce distinction between the two , but
it is essentially the same principal involved in both cases
\citep{Robinson}. The BLUE of \textbf{b}, and the BLUP of
\textbf{u} can be shown to be;

\begin{equation}
\hat{b} = (X^{T}V^{-1}X)^{-1}X^{T}V^{-1}y
\end{equation}
\begin{equation}
\hat{u} = GZ^{T}V^{-1}(y-X\hat{b})
\end{equation}

The practical application of both expressions requires that the
variance components be known. An estimate for the variance
components must be derived to  either maximum likelihood (ML) or
more commonly restricted maximum likelihood (REML).

Importantly calculations based on the above formulae require the
calculation of the inverse of \textbf{V}. In simple examples
$V^{-1}$ is a straightforward calculation, but with higher
dimensions it becomes a very complex calculation.
\newpage
\section{Lai Shiao}
\citet{LaiShiao} use mixed models to determine the factors that
affect the difference of two methods of measurement using the
conventional formulation of linear mixed effects models.

If the parameter \textbf{b}, and the variance components are not
significantly different from zero, the conclusion that there is no
inter-method bias can be drawn. If the fixed effects component
contains only the intercept, and a simple correlation coefficient
is used, then the estimate of the intercept in the model is the
inter-method bias. Conversely the estimates for the fixed effects
factors can advise the respective influences each factor has on
the differences. It is possible to pre-specify different
correlation structures of the variance components \textbf{G} and
\textbf{R}.


Oxygen saturation is one of the most frequently measured variables
in clinical nursing studies. `Fractional saturation' ($HbO_{2}$)
is considered to be the gold standard method of measurement, with
`functional saturation' ($SO_{2}$) being an alternative method.
The method of examining the causes of differences between these
two methods is applied to a clinical study conducted by
\citet{Shiao}. This experiment was conducted by 8 lab
practitioners on blood samples, with varying levels of
haemoglobin, from two donors. The samples have been in storage for
varying periods ( described by the variable `Bloodage') and are
categorized according to haemoglobin percentages(i.e
$0\%$,$20\%$,$40\%$,$60\%$,$80\%$,$100\%$). There are 625
observations in all.

\citet{LaiShiao} fits two models on this data, with the lab
technicians and the replicate measurements as the random effects
in both models. The first model uses haemoglobin level as a fixed
effects component. For the second model, blood age is added as a
second fixed factor.

\subsubsection{Single fixed effect} The first model fitted by \citet{LaiShiao} takes the
blood level as the sole fixed effect to be analyzed. The following
coefficient estimates are estimated by `Proc Mixed';
\begin{eqnarray}
\mbox{fixed effects :   } 2.5056 - 0.0263\mbox{Fhbperct}_{ijtl} \\
(\mbox{p-values :   } = 0.0054, <0.0001, <0.0001)\nonumber\\\nonumber\\
\mbox{random effects :   } u(\sigma^{2}=3.1826) + e_{ijtl}
(\sigma^{2}_{e}=0.1525, \rho= 0.6978) \nonumber\\
(\mbox{p-values :   } = 0.8113, <0.0001, <0.0001)\nonumber
\end{eqnarray}

With the intercept estimate being both non-zero and statistically
significant ($p=0.0054$), this models supports the presence
inter-method bias is $2.5\%$ in favour of $SO_{2}$. Also, the
negative value of the haemoglobin level coefficient indicate that
differences will decrease by $0.0263\%$ for every percentage
increase in the haemoglobin .

In the random effects estimates, the variance due to the
practitioners is $3.1826$, indicating that there is a significant
variation due to technicians ($p=0.0311$) affecting the
differences. The variance for the estimates is given as $0.1525$,
($p<0.0001$).

\subsubsection{Two fixed effects}
Blood age is added as a second fixed factor to the model,
whereupon new estimates are calculated;
\begin{eqnarray}
\mbox{fixed effects :   } -0.2866 + 0.1072 \mbox{Bloodage}_{ijtl}
- 0.0264\mbox{Fhbperct}_{ijtl}\nonumber\\
( \mbox{p-values :   } = 0.8113, <0.0001, <0.0001)\nonumber\\\nonumber\\
\mbox{random effects :   } u(\sigma^{2}=10.2346) + e_{ijtl}
(\sigma^{2}_{e}=0.0920, \rho= 0.5577) \nonumber\\
(\mbox{p-values :   } = 0.0446, <0.0001, <0.0001)
\end{eqnarray}


With this extra fixed effect added to the model, the intercept
term is no longer statistically significant. Therefore, with the
presence of the second fixed factor, the model is no longer
supporting the presence of inter-method bias. Furthermore, the
second coefficient indicates that the blood age of the observation
has a significant bearing on the size of the difference between
both methods ($p <0.0001$). Longer storage times for blood will
lead to higher levels of particular blood factors such as MetHb
and HbCO (due to the breakdown and oxidisation of the
haemoglobin). Increased levels of MetHb and HbCO are concluded to
be the cause of the differences. The coefficient for the
haemoglobin level doesn't differ greatly from the single fixed
factor model, and has a much smaller effect on the differences.
The random effects estimates also indicate significant variation
for the various technicians; $10.2346$ with $p=0.0446$.

\citet{LaiShiao} demonstrates how that linear mixed effects models
can be used to provide greater insight into the cause of the
differences. Naturally the addition of further factors to the
model provides for more insight into the behavior of the data.



\newpage
\section{Carstensen's Mixed Models}

\citet{BXC2004} proposes linear mixed effects models for deriving
conversion calculations similar to Deming's regression, and for
estimating variance components for measurements by different
methods. The following model ( in the authors own notation) is
formulated as follows, where $y_{mir}$ is the $r$th replicate
measurement on subject $i$ with method $m$.

\begin{equation}
y_{mir}  = \alpha_{m} + \beta_{m}\mu_{i} + c_{mi} + e_{mir} \qquad
( e_{mi} \sim N(0,\sigma^{2}_{m}), c_{mi} \sim N(0,\tau^{2}_{m}))
\end{equation}
The intercept term $\alpha$ and the $\beta_{m}\mu_{i}$ term follow
from \citet{DunnSEME}, expressing constant and proportional bias
respectively , in the presence of a real value $\mu_{i}.$
 $c_{mi}$ is a interaction term to account for replicate, and
 $e_{mir}$ is the residual associated with each observation.
Since variances are specific to each method, this model can be
fitted separately for each method.

The above formulation doesn't require the data set to be balanced.
However, it does require a sufficient large number of replicates
and measurements to overcome the problem of identifiability. The
import of which is that more than two methods of measurement may
be required to carry out the analysis. There is also the
assumptions that observations of measurements by particular
methods are exchangeable within subjects. (Exchangeability means
that future samples from a population behaves like earlier
samples).

%\citet{BXC2004} describes the above model as a `functional model',
%similar to models described by \citet{Kimura}, but without any
%assumptions on variance ratios. A functional model is . An
%alternative to functional models is structural modelling

\citet{BXC2004} uses the above formula to predict observations for
a specific individual $i$ by method $m$;

\begin{equation}BLUP_{mir} = \hat{\alpha_{m}} + \hat{\beta_{m}}\mu_{i} +
c_{mi} \end{equation}. Under the assumption that the $\mu$s are
the true item values, this would be sufficient to estimate
parameters. When that assumption doesn't hold, regression
techniques (known as updating techniques) can be used additionally
to determine the estimates. The assumption of exchangeability can
be unrealistic in certain situations. \citet{BXC2004} provides an
amended formulation which includes an extra interaction term ($
d_{mr} \sim N(0,\omega^{2}_{m}$)to account for this.


\newpage
\citet{BXC2008} sets out a methodology of computing the limits of
agreement based upon variance component estimates derived using
linear mixed effects models. Measures of repeatability, a
characteristic of individual methods of measurements, are also
derived using this method.

\citet{BXC2004} also advocates the use of linear mixed models in
the study of method comparisons. The model is constructed to
describe the relationship between a value of measurement and its
real value. The non-replicate case is considered first, as it is
the context of the Bland-Altman plots. This model assumes that
inter-method bias is the only difference between the two methods.
A measurement $y_{mi}$ by method $m$ on individual $i$ is
formulated as follows;
\begin{equation}
y_{mi}  = \alpha_{m} + \mu_{i} + e_{mi} \qquad ( e_{mi} \sim
N(0,\sigma^{2}_{m}))
\end{equation}
The differences are expressed as $d_{i} = y_{1i} - y_{2i}$ For the
replicate case, an interaction term $c$ is added to the model,
with an associated variance component. All the random effects are
assumed independent, and that all replicate measurements are
assumed to be exchangeable within each method.

\begin{equation}
y_{mir}  = \alpha_{m} + \mu_{i} + c_{mi} + e_{mir} \qquad ( e_{mi}
\sim N(0,\sigma^{2}_{m}), c_{mi} \sim N(0,\tau^{2}_{m}))
\end{equation}

\citet{BXC2008} proposes a methodology to calculate prediction
intervals in the presence of replicate measurements, overcoming
problems associated with Bland-Altman methodology in this regard.
It is not possible to estimate the interaction variance components
$\tau^{2}_{1}$ and $\tau^{2}_{2}$ separately. Therefore it must be
assumed that they are equal. The variance of the difference can be
estimated as follows:
\begin{equation}
var(y_{1j}-y_{2j})
\end{equation}

\addcontentsline{toc}{section}{Bibliography}

\bibliography{transferbib}
\end{document}
