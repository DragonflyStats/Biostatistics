

\section{Linear mixed effects models}

% http://www.artifex.org/~meiercl/R_statistics_guide.pdf
These models are used when there are both fixed and random effects that need to be incorporated into a model.

Fixed effects usually correspond to experimental treatments for which one has data for the entire population of samples corresponding to that treatment.

Random effects,on the other hand, are assigned in the case where we have measurements on a group of samples, and those
samples are taken from some larger sample pool, and are presumed to be representative.

As such, linear mixed effects models treat the error for fixed effects differently than the error for random effects.


\subsection{Stating the LME Model}
The general linear mixed
model is
\[
Y = X\beta + Zu + \varepsilon\]
where Y is a $(n\times1)$ vector of observed data, X is an $(n\times p)$ fixed-effects design or regressor matrix of rank
k, Z is a $(n \times g)$ random-effects design or regressor matrix, $u$ is a $(g \times 1)$ vector of random effects, and $\varepsilon$ is
an $(n\times1)$ vector of model errors (also random effects). The distributional assumptions made by the MIXED
procedure are as follows: γ is normal with mean 0 and variance G; $\varepsilon$ is normal with mean 0 and variance
R; the random components $u$ and $\varepsilon$ are independent. Parameters of this model are the fixed-effects β and
all unknowns in the variance matrices G and R. The unknown variance elements are referred to as the
covariance parameters and collected in the vector $theta$.
%===========================================================================%

The concept of critiquing the model-data agreement applies in mixed models in the same way as in linear
fixed-effects models. In fact, because of the more complex model structure, you can argue that model and
data diagnostics are even more important. For example, you are not only concerned with capturing the
important variables in the model. You are also concerned with “distributing” them correctly between the
fixed and random components of the model. The mixed model structure presents unique and interesting
challenges that prompt us to reexamine the traditional ideas of influence and residual analysis.
%==========================================================================%
This paper presents the extension of traditional tools and statistical measures for influence and residual
analysis to the linear mixed model and demonstrates their implementation in the MIXED procedure (experimental
features in SAS 9.1). The remainder of this paper is organized as follows. The “Background” section
briefly discusses some mixed model estimation theory and the challenges to model diagnosis that result
from it.

%	 The diagnostics implemented in the MIXED procedure are discussed in the “Residual Diagnostics
%	in the MIXED Procedure” section (page 3) and the “Influence Diagnostics in the MIXED Procedure” section
%	(page 5). The syntax options and suboptions you use to request the various diagnostics are briefly sketched
%	in the “Syntax” section (page 9). The presentation concludes with an example.
%	
%	
%====================================================================================================================%