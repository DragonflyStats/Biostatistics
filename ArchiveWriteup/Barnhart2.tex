Coverage Probability and Total Deviation Index
% Barnhart

As elaborated by Lin and colleagues (Lin, 2000; Lin et al., 2002), an intuitive measure of
agreement is a measure that captures a large proportion of data within a boundary for allowed
observers’ differences. The proportion and boundary are two quantities that correspond to
each other. If we set d0 as the predetermined boundary; i.e., the maximum acceptable
absolute difference between two observers’ readings, we can compute the probability of absolute
difference between any two observers’ readings less than d0. 

This probability is called
coverage probability (CP). On the other hand, if we set 0 as the predetermined coverage
probability, we can find the boundary so that the probability of absolute difference less than
this boundary is 0. This boundary is called total deviation index (TDI) and is the 1000
percentile of the absolute difference of paired observations. A satisfactory agreement may
require a large CP or, equivalently, a small TDI.

For J = 2 observers, let Yi1 and Yi2 be the
readings of these two observers, the CP and TDI are defined as
\[
CPd0 = Prob(|Yi1 − Yi2| < d0), TDI0 = f−1(0)
\]
where f−1(0) is the solution of d by setting $f(d) = Prob(|Yi1 − Yi2| < d) = 0$.
Estimation and inference on CPd0 and TDI0 often requires a normality assumption on
Di = Yi1 − Yi2.Assume that Di is normally distributed with mean μD and variance 2
D .
