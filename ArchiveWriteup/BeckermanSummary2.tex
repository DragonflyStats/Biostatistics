Beckerman - Conclusions
=========================================================

***MOVE TO CHAPTER 5B***

- Diagnostic methods are useful for assessing the
adequacy of assumptions underlying a modeling process
and for identifying unexpected characteristics of
the data that may seriously influence conclusions or
require special attention. Such methods may also
serve as the initial step in the determination of the
robustness of a sample.
- In recent years, deleting observations has become
a popular and extremely useful basis for studying
sensitivity in statistical problems. The deletion of observations,
however, is just one possible way in which
a postulated model might be perturbed meaningfully.
- The methodology discussed in this article is designed
to allow an assessment of the sensitivity of a mixedmodel
analysis to perturbations in many of the standard
assumptions. Results from sensitivity analyses
must be treated with caution, and the sensitivity itself
must become part of the conclusions. 
- When perturbing error variances, the cause of the sensitivity
can often be traced via hmax to a few influential observations,
and in such cases the proposed methodology
agrees well with case deletion. 
- The usefulness of case
deletion, however, seems limited to providing an understanding
only of certain aspects of the error component
of a mixed model. For example, case deletion
will not generally lead to the identification of influential
groups associated with a variance component. 
- As
briefly described in Section 3, the proposed methodology
can be adapted easily to handle such concerns.
