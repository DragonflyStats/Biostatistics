\section{Computing LoAs from LME models}

Computing limits of agreement features prominently in many method comparison studies, further to \citet{BA86,BA99}.
\citet{BA99} addresses the issue of computing LoAs in the presence of replicate measurements, suggesting several computationally simple approaches. When repeated measures data are available, it is desirable to use
all the data to compare the two methods.

However, the original Bland–Altman method was developed for two sets of measurements done on one occasion (i.e. independent data), and so this approach is not suitable for replicate measures data. However, as a naive analysis, it may be used to explore the data because of the simplicity of the method.

%-----------------------------------------------------------------------------------%
\subsection{Featured approaches}

\citet{bxc2008} computes the limits of agreement to the case with repeated measurements by using LME models.

\citet{Roy} formulates a very powerful method of assessing whether two methods of measurement, with replicate measurements, also using LME models. Roy's approach is based on the construction of variance-covariance matrices.
Importantly, Roy's approach does not address the issue of limits of agreement (though another related analysis , the coefficient of repeatability, is mentioned).

This paper seeks to use Roy's approach to estimate the limits of agreement. These estimates will be compared to estimates computed under Carstensen's formulation.

In computing limits of agreement, it is first necessary to have an estimate for the variance of differences. When the agreement of two methods is analyzed using LME models, a clear method of how to compute the variance is required. As the estimate for inter-method bias and the quantile would be the same for both methodologies, the focus hereon is solely on the variance of differences.

\newpage



%-----------------------------------------------------------------------------------%

\subsection{Calculation of limits of agreement }


Further to \citet{BA86}, the computation of the limits of agreement follows from the intermethod bias, and the variance of the difference of measurements. The computation of the inter-method bias is a straightforward subtraction calculation. The variance of differences is easily computable from the variance estimates in the ${\mbox{Block - }\boldsymbol \Omega_{i}}$ matrix, i.e.
\[
\mathrm{Var}(y_1 - y_2) = \sqrt{ \omega^2_1 + \omega^2_2 - 2\omega_{12}}.
\]

\citet{BXC2008} also presents a methodology to compute the limits of agreement based on LME models. In many cases the limits of agreement derived from this method accord with those to Roy's model. However, in other cases dissimilarities emerge. An explanation for this differences can be found by considering how the respective models account for covariance in the observations. Specifying the relevant terms using a bivariate normal distribution, Roy's model allows for both between-method and within-method covariance. \citet{BXC2008} formulate a model whereby random effects have univariate normal distribution, and no allowance is made for correlation between observations.

A consequence of this is that the between-method and within-method covariance are zero. In cases where there is negligible covariance between methods, both sets of limits of agreement are very similar to each other. In cases where there is a substantial level of covariance present between the two methods, the limits of agreement computed using models will differ.
%-----------------------------------------------------------------------------------%


\newpage
\section{Carstensen's Limits of agreement}
\citet{bxc2008} presents a methodology to compute the limits of
agreement based on LME models. Importantly, Carstensen's underlying model differs from Roy's model in some key respects, and therefore a prior discussion of Carstensen's model is required.

\subsection{Carstensen's Model}

\citet{BXC2004} presents a model to describe the relationship between a value of measurement and its
real value. The non-replicate case is considered first, as it is the context of the Bland Altman plots. This model assumes that inter-method bias is the only difference between the two methods.

A measurement $y_{mi}$ by method $m$ on individual $i$ is formulated as follows;
\begin{equation}
	y_{mi}  = \alpha_{m} + \mu_{i} + e_{mi} \qquad  e_{mi} \sim
	\mathcal{N}(0,\sigma^{2}_{m})
\end{equation}
The differences are expressed as $d_{i} = y_{1i} - y_{2i}$. For the replicate case, an interaction term $c$ is added to the model, with an associated variance component. All the random effects are assumed independent, and that all replicate measurements are assumed to be exchangeable within each method.

\begin{equation}
	y_{mir}  = \alpha_{m} + \mu_{i} + c_{mi} + e_{mir}, \qquad  e_{mi}
	\sim \mathcal{N}(0,\sigma^{2}_{m}), \quad c_{mi} \sim \mathcal{N}(0,\tau^{2}_{m}).
\end{equation}
%----

Of particular importance is terms of the model, a true value for item $i$ ($\mu_{i}$).  The fixed effect of Roy's model comprise of an intercept term and fixed effect terms for both methods, with no reference to the true value of any individual item. A distinction can be made between the two models: Roy's model is a standard LME model, whereas Carstensen's model is a more complex additive model.


\subsection{Assumptions on Variability}

Aside from the fixed effects, another important difference is that Carstensen's model requires that particular assumptions be applied, specifically that the off-diagonal elements of the between-item
and within-item variability matrices are zero. By extension the
overall variability off diagonal elements are also zero.

Also, implementation requires that the between-item variances are
estimated as the same value: $g^2_1 = g^2_2 = g^2$. Necessarily
Carstensen's method does not allow for a formal test of the
between-item variability.

\[\left(\begin{array}{cc}
\omega^1_2  & 0 \\
0 & \omega^2_2 \\
\end{array}  \right)
=  \left(
\begin{array}{cc}
g^2  & 0 \\
0 & g^2 \\
\end{array} \right)+
\left(
\begin{array}{cc}
\sigma^2_1  & 0 \\
0 & \sigma^2_2 \\
\end{array}\right)
\]

In cases where the off-diagonal terms in the overall variability
matrix are close to zero, the limits of agreement due to
\citet{bxc2008} are very similar to the limits of agreement that
follow from the general model.

\newpage

%--------------------------------------------------------------------------------------%
\section{Carstensen's Model}
\citet{BXC2008} uses an approach based on linear mixed effects (LME) models for the purpose of computing the limits of agreement for two methods of measurement, where replicate measurements are taken on items. As the emphasis of this methodology lies on the inter-method bias and the limits of agreement, the two key elements of the Bland-Altman methodology, other formal tests are not described.

Using Carstensen's notation, a measurement $y_{mi}$ by method $m$ on individual $i$ the measurement $y_{mir} $ is the $r$th replicate measurement on the $i$th item by the $m$th method, where $m=1,2,$ $i=1,\ldots,N,$ and $r = 1,\ldots,n_i$ is formulated as follows;

\begin{equation}
y_{mir}  = \alpha_{m} + \mu_{i} + c_{mi} + \epsilon_{mir}, \qquad  e_{mi}
\sim \mathcal{N}(0,\sigma^{2}_{m}), \quad c_{mi} \sim \mathcal{N}(0,\tau^{2}_{m}).
\end{equation}

Here the terms $\alpha_{m}$ and $\mu_{i}$ represent the fixed effect for method $m$ and a true value for item $i$ respectively. The random effect terms comprise an interaction term $c_{mi}$ and the residuals $\epsilon_{mir}$.
%---Key difference 1---The True Value
%---Colollary -- Difference in model types

The presence of the true value term $\mu_i$ gives rise to an important difference between Carstensen's and Roy's models. The fixed effect of Roy's model comprise of an intercept term and fixed effect terms for both methods, with no reference to the true value of any individual item. In other words, Roy considers the group of items being measured as a sample taken from a population. A distinction can be made between the two models: Roy's model is a standard LME model, whereas Carstensen's model is a more complex additive model.

%---Key Difference 2 --- Univariate normal distribution


\newpage



As the difference between methods is of interest, the item term can be disregarded.

We assume that that the variance of the measurements is different for both methods, but it does not mean that the separate variances can be estimated with the data available.\\
% Carstensen also uses a LME model for examining MCS with replicates.\\


% Carstensen allocates a fixed, but unknown, mean for each individual. [Grubbs(1948) model.]\\

% His interest lies in calculating the LoA as opposed to formalized testing.

\citet{BXC2004} presents a model to describe the relationship between a value of measurement and its real value.
The non-replicate case is considered first, as it is the context of the Bland-Altman plots.
This model assumes that inter-method bias is the only difference between the two methods.


\begin{equation}
y_{mi}  = \alpha_{m} + \mu_{i} + e_{mi} \qquad  e_{mi} \sim \mathcal{N}(0,\sigma^{2}_{m})
\end{equation}

The differences are expressed as $d_{i} = y_{1i} - y_{2i}$.
For the replicate case, an interaction term $c$ is added to the model, with an associated variance component.
All the random effects are assumed independent, and that all replicate measurements are assumed to be exchangeable within each method.




\subsection{Computation of limits of agreement }

%---Carstensen's limits of agreement
%---The between item variances are not individually computed. An estimate for their sum is used.
%---The within item variances are indivdually specified.
%---Carstensen remarks upon this in his book (page 61), saying that it is "not often used".
%---The Carstensen model does not include covariance terms for either VC matrices.
%---Some of Carstensens estimates are presented, but not extractable, from R code, so calculations have to be done by %---hand.
%--Importantly, estimates required to calculate the limits of agreement are not extractable, and therefore the calculation must be done by hand.
%---All of Roys stimates are  extractable from R code, so automatic compuation can be implemented
%---When there is negligible covariance between the two methods, Roys LoA and Carstensen's LoA are roughly the same.
%---When there is covariance between the two methods, Roy's LoA and Carstensen's LoA differ, Roys usually narrower.

The computation thereof require that the variance of the difference of measurements. This variance is easily computable from the  variance estimates in the ${\mbox{Block - }\boldsymbol \Omega_{i}}$ matrix, i.e.
\[
% Check this
\operatorname{Var}(y_1 - y_2) = \sqrt{ \omega^2_1 + \omega^2_2 - 2\omega_{12}}.
\]

\citet{BXC2008} also presents a methodology to compute the limits of agreement based on LME models. The method of computation is similar Roy's model, but for absence of the covariance estimates. In cases where there is negligible covariance between methods, the limits of agreement computed using Roy's model accord with those computed using model described by (\ref{BXC-model}). In cases where some degree of covariance is present between the two methods, the limits of agreement computed using models will differ. In the presented example, it is shown that Roy's LOAs are lower than those of (\ref{BXC-model}), when covariance between methods is present.

%==================================================================================================================== %
\section{Carstensen's Model}
\citet{BXC2008} uses an approach based on linear mixed effects (LME) models for the purpose of computing the limits of agreement for two methods of measurement, where replicate measurements are taken on items. As the emphasis of this methodology lies on the inter-method bias and the limits of agreement, the two key elements of the Bland-Altman methodology, other formal tests are not described.

Using Carstensen's notation, a measurement $y_{mi}$ by method $m$ on individual $i$ the measurement $y_{mir} $ is the $r$th replicate measurement on the $i$th item by the $m$th method, where $m=1,2,$ $i=1,\ldots,N,$ and $r = 1,\ldots,n_i$ is formulated as follows;

\begin{equation}
	y_{mir}  = \alpha_{m} + \mu_{i} + c_{mi} + \epsilon_{mir}, \qquad  e_{mi}
	\sim \mathcal{N}(0,\sigma^{2}_{m}), \quad c_{mi} \sim \mathcal{N}(0,\tau^{2}_{m}).
\end{equation}

Here the terms $\alpha_{m}$ and $\mu_{i}$ represent the fixed effect for method $m$ and a true value for item $i$ respectively. The random effect terms comprise an interaction term $c_{mi}$ and the residuals $\epsilon_{mir}$.
The $c_{mi}$ term represent random effect parameters corresponding to the two methods, having $\mathrm{E}(c_{mi})=0$ with $\mathrm{Var}(c_{mi})=\tau^2_m$. Carstensen specifies the variance of the interaction terms as being univariate normally distributed. As such, $\mathrm{Cov}(c_{mi}, c_{m^\prime i})= 0.$ All the random effects are assumed independent, and that all replicate measurements are assumed to be exchangeable within each method.

With regards to specifying the variance terms, Carstensen remarks that using his approach is common, remarking that \emph{
	The only slightly non-standard (meaning "not often used") feature is the differing residual variances between methods }\citep{bxc2010}.

In contrast to Roy's model, Carstensen's model requires that commonly used assumptions be applied, specifically that the off-diagonal elements of the between-item and within-item variability matrices are zero. By
extension the overall variability off-diagonal elements are also zero. Also, implementation requires that the between-item variances are estimated as the same value: $g^2_1 = g^2_2 = g^2$.
As a consequence, Carstensen's method does not allow for a formal test of the between-item variability.

\[\left(\begin{array}{cc}
\omega^1_2  & 0 \\
0 & \omega^2_2 \\
\end{array}  \right)
=  \left(
\begin{array}{cc}
\tau^2  & 0 \\
0 & \tau^2 \\
\end{array} \right)+
\left(
\begin{array}{cc}
\sigma^2_1  & 0 \\
0 & \sigma^2_2 \\
\end{array}\right)
\]


%---Key difference 1---The True Value
%---Colollary -- Difference in model types
The presence of the true value term $\mu_i$ gives rise to an important difference between Carstensen's and Roy's models. The fixed effect of Roy's model comprise of an intercept term and fixed effect terms for both methods, with no reference to the true value of any individual item. In other words, Roy considers the group of items being measured as a sample taken from a population. Therefore a distinction can be made between the two models: Roy's model is a standard LME model, whereas Carstensen's model is a more complex additive model.

%---Carstensen's limits of agreement
%---The between item variances are not individually computed. An estimate for their sum is used.
%---The within item variances are indivdually specified.
%---Carstensen remarks upon this in his book (page 61), saying that it is "not often used".
%---The Carstensen model does not include covariance terms for either VC matrices.
%---Some of Carstensens estimates are presented, but not extractable, from R code, so calculations have to be done by %---hand.
%---All of Roys stimates are  extractable from R code, so automatic compuation can be implemented
%---When there is negligible covariance between the two methods, Roys LoA and Carstensen's LoA are roughly the same.
%---When there is covariance between the two methods, Roy's LoA and Carstensen's LoA differ, Roys usually narrower.



%-----------------------------------------------------------------------------------%

\section{Carstensen's Limits of agreement}
\citet{bxc2008} presents a methodology to compute the limits of agreement based on LME models.
Importantly, Carstensen's underlying model differs from Roy's model in some key respects, and therefore a prior discussion of Carstensen's model is required.


Carstensen makes some interesting remarks in this regard.

\begin{quote}
	The only slightly non-standard (meaning "not often used") feature
	is the differing residual variances between methods.
\end{quote}

\subsection{Assumptions on Variability}

Aside from the fixed effects, another important difference is that
Carstensen's model requires that particular assumptions be
applied, specifically that the off-diagonal elements of the
between-item and within-item variability matrices are zero. By
extension the overall variability off-diagonal elements are also
zero. Also, implementation requires that the between-item
variances are estimated as the same value: $g^2_1 = g^2_2 = g^2$.
As a consequence, Carstensen's method does not allow for a formal
test of the between-item variability.

\[\left(\begin{array}{cc}
\omega^1_2  & 0 \\
0 & \omega^2_2 \\
\end{array}  \right)
=  \left(
\begin{array}{cc}
g^2  & 0 \\
0 & g^2 \\
\end{array} \right)+
\left(
\begin{array}{cc}
\sigma^2_1  & 0 \\
0 & \sigma^2_2 \\
\end{array}\right)
\]

In cases where the off-diagonal terms in the overall variability
matrix are close to zero, the limits of agreement due to
\citet{bxc2008} are very similar to the limits of agreement that
follow from the general model.

\newpage


\newpage

%-----------------------------------------------------------------------------------%

\section{Carstensen's Limits of agreement}
\citet{bxc2008} presents a methodology to compute the limits of agreement based on LME models. The method of computation is the
same as Roy's model, but with the covariance estimates set to zero.

In cases where there is negligible covariance between methods, the limits of agreement computed using Roy's model accord with those computed using Carstensen's model. In cases where some degree of
covariance is present between the two methods, the limits of agreement computed using models will differ. In the presented
example, it is shown that Roy's LoAs are lower than those of Carstensen, when covariance is present.

Importantly, estimates required to calculate the limits of agreement are not extractable, and therefore the calculation must
be done by hand.


%-----------------------------------------------------------------------------------%

\subsection{Computation of limits of agreement }

%---Carstensen's limits of agreement
%---The between item variances are not individually computed. An estimate for their sum is used.
%---The within item variances are indivdually specified.
%---Carstensen remarks upon this in his book (page 61), saying that it is "not often used".
%---The Carstensen model does not include covariance terms for either VC matrices.
%---Some of Carstensens estimates are presented, but not extractable, from R code, so calculations have to be done by %---hand.
%--Importantly, estimates required to calculate the limits of agreement are not extractable, and therefore the calculation must be done by hand.
%---All of Roys stimates are  extractable from R code, so automatic compuation can be implemented
%---When there is negligible covariance between the two methods, Roys LoA and Carstensen's LoA are roughly the same.
%---When there is covariance between the two methods, Roy's LoA and Carstensen's LoA differ, Roys usually narrower.

The computation thereof require that the variance of the difference of measurements. This variance is easily computable from the  variance estimates in the ${\mbox{Block - }\boldsymbol \Omega_{i}}$ matrix, i.e.
\[
% Check this
\operatorname{Var}(y_1 - y_2) = \sqrt{ \omega^2_1 + \omega^2_2 - 2\omega_{12}}.
\]

\citet{BXC2008} also presents a methodology to compute the limits of agreement based on LME models. The method of computation is similar Roy's model, but for absence of the covariance estimates. In cases where there is negligible covariance between methods, the limits of agreement computed using Roy's model accord with those computed using model described by (\ref{BXC-model}). In cases where some degree of covariance is present between the two methods, the limits of agreement computed using models will differ. In the presented example, it is shown that Roy's LOAs are lower than those of (\ref{BXC-model}), when covariance between methods is present.


\subsection{Note 2: Carstensen model in the single measurement case}
\citet{BXC2004} presents a model to describe the relationship between a value of measurement and its real value.
The non-replicate case is considered first, as it is the context of the Bland-Altman plots.
This model assumes that inter-method bias is the only difference between the two methods.


\begin{equation}
	y_{mi}  = \alpha_{m} + \mu_{i} + e_{mi} \qquad  e_{mi} \sim \mathcal{N}(0,\sigma^{2}_{m})
\end{equation}

The differences are expressed as $d_{i} = y_{1i} - y_{2i}$.

For the replicate case, an interaction term $c$ is added to the model, with an associated variance component.



\subsection{Note 3: Model terms}
It is important to note the following characteristics of this model.
\begin{itemize}
	\item Let the number of replicate measurements on each item $i$ for both methods be $n_i$, hence $2 \times n_i$ responses. However, it is assumed that there may be a different number of replicates made for different items. Let the maximum number of replicates be $p$. An item will have up to $2p$ measurements, i.e. $\max(n_{i}) = 2p$.
	
	% \item $\boldsymbol{y}_i$ is the $2n_i \times 1$ response vector for measurements on the $i-$th item.
	% \item $\boldsymbol{X}_i$ is the $2n_i \times  3$ model matrix for the fixed effects for observations on item $i$.
	% \item $\boldsymbol{\beta}$ is the $3 \times  1$ vector of fixed-effect coefficients, one for the true value for item $i$, and one effect each for both methods.
	
	\item Later on $\boldsymbol{X}_i$ will be reduced to a $2 \times 1$ matrix, to allow estimation of terms. This is due to a shortage of rank. The fixed effects vector can be modified accordingly.
	\item $\boldsymbol{Z}_i$ is the $2n_i \times  2$ model matrix for the random effects for measurement methods on item $i$.
	\item $\boldsymbol{b}_i$ is the $2 \times  1$ vector of random-effect coefficients on item $i$, one for each method.
	\item $\boldsymbol{\epsilon}$  is the $2n_i \times  1$ vector of residuals for measurements on item $i$.
	\item $\boldsymbol{G}$ is the $2 \times  2$ covariance matrix for the random effects.
	\item $\boldsymbol{R}_i$ is the $2n_i \times  2n_i$ covariance matrix for the residuals on item $i$.
	\item The expected value is given as $\mbox{E}(\boldsymbol{y}_i) = \boldsymbol{X}_i\boldsymbol{\beta}.$ \citep{hamlett}
	\item The variance of the response vector is given by $\mbox{Var}(\boldsymbol{y}_i)  = \boldsymbol{Z}_i \boldsymbol{G} \boldsymbol{Z}_i^{\prime} + \boldsymbol{R}_i$ \citep{hamlett}.
\end{itemize}
\newpage

%\chapter{Limits of Agreement}

%\section{Modelling Agreement with LME Models}

% Carstensen pages 22-23


Roys uses and LME model approach to provide a set of formal tests for method comparison studies.\\

Four candidates models are fitted to the data.\\

These models are similar to one another, but for the imposition of equality constraints.\\

These tests are the pairwise comparison of candidate models, one formulated without constraints, the other with a constraint.\\


Roy's model uses fixed effects $\beta_0 + \beta_1$ and $\beta_0 + \beta_1$ to specify the mean of all observationsby \\ methods 1 and 2 respectively.


Roy adheres to Random Effect ideas in ANOVA. Roy treats items as a sample from a population.\\

Allocation of fixed effects and random effects are very different in each model\\

Carstensen's interest lies in the difference between the population from which they were drawn.\\

Carstensen's model is a mixed effects ANOVA.\\

This model includes a method by item interaction term.\\

Carstensen presents two models. One for the case where the replicates, and a second for when they are linked.\\
Carstensen's model does not take into account either between-item or within-item covariance between methods.\\
In the presented example, it is shown that Roy's LoAs are lower than those of Carstensen.


\[\left(\begin{array}{cc}
\omega^1_2  & 0 \\
0 & \omega^2_2 \\
\end{array}  \right)
=  \left(
\begin{array}{cc}
\tau^2  & 0 \\
0 & \tau^2 \\
\end{array} \right)+
\left(
\begin{array}{cc}
\sigma^2_1  & 0 \\
0 & \sigma^2_2 \\
\end{array}\right)
\]


\newpage
