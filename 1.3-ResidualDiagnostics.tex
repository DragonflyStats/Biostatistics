\documentclass[Main.tex]{subfiles}
\begin{document}
%---------------------------------------------------------------------------%
Likelihood Ratio Tests
$L= - 2ln$ is approximately distributed as 2 under $H_0$ for large sample size and under the normality assumption.

The power of the likelihood ratio test may depends on specific sample size and the specific number of replications, and [Roy 2009] proposes simulation studies to examine this further.

%--------------------------------------------------%
\section{REML}

In statistics, the restricted (or residual, or reduced) maximum likelihood (REML) approach is a particular form of maximum likelihood estimation which does not base estimates on a maximum likelihood fit of all the information, but instead uses a likelihood function calculated from a transformed set of data, so that nuisance parameters have no effect.[1]

In the case of variance component estimation, the original data set is replaced by a set of contrasts calculated from the data, and the likelihood function is calculated from the probability distribution of these contrasts, according to the model for the complete data set. In particular, REML is used as a method for fitting linear mixed models. In contrast to the earlier maximum likelihood estimation, REML can produce unbiased estimates of variance and covariance parameters.[2]

%--------------------------------------------------%

% http://www.stats.ox.ac.uk/~ripley/LME4.pdf

REML estimates minimize

\[ (Y-X\beta)^{T}V^{-1}(\omega)(Y-X\beta) + \mbox{log}|V(\omega)|
+ \mbox{log}|X^{T}V^{-1}(\omega)X| \]

The REML estimator of $\beta$ is GLS.

%---------------------------------------------------%
REML fit criterion is the marginal likelihood, integrating $\beta$

%---------------------------------------------------%

\section{Best linear unbiased predictions}
In an LME it is not clear what fitted values and hence residuals are.
Our best prediction for subject i is not given by the mean relationship. We need to specify just what is
common with an example we have already seen.

BLUPs replace the random effects b by their conditional means $\hat{b}$ given the data, and then make
predictions using those values,
\[ \hat{Y} = X\hat{\beta}+ Z\hat{b}\]
%--------------------------------------------------%
\subsection{Estimation Methods}
Nested LME models, fitted by ML estimation, can be compared using the likelihood ratio test [Lehmann (1986)].
Models fitted using REML estimation can also be compared, but only if both were fitted using REML, and both have the same fixed effects specifications.

Likelihood ratio tests are generally used to test the significance of terms in the random effects structure.
%--------------------------------------------------%
\subsection{Information Criteria}
Additionally nested models may be compared by using the Akaike Information Criterion,(AIC) and the Bayesian Information Criterion (BIC).

When comparing the respective scores for nested models, the model with the smaller score is considered to be the preferable model.
ML / REML [Morrell 1998]
%--------------------------------------------------------------------------------%
The variance components in the LME model may be estimated by ML or REML.
Maximum Likelihood estimates do not take into account the estimation of fixed effects and so
are biased downwards.
REML estimates accounts for the presence of these nuisance parameters by maximising the linearly independent error contrasts to obtain more unbiased estimates.
Treatment of items as fixed effects

[Pinheiro Bates 2000] addresses the issue of treating items as fixed effects. Such a specification is useful only for the specific sample of items, rather than the population of items, where the interest would naturally lie.

[Pinheiro Bates 2000] advises the specification of random effects to correspond to items; treating the item effects as random deviations from the population mean.

Indeed [Roy 2009] follows this approach.
%----------------------------------------------------------------------------------%
\subsection{Grubb’s One Way Classification Model }
Carstensen develops a model that accords with a well-established method comparison methodology, that of Grubbs’ 1946 paper.




\newpage

	
	

	
		\newpage
		\subsection{Extension of techniques to LME Models} %1.2
		
		Model diagnostic techniques, well established for classical models, have since been adapted for use with linear mixed effects models. Diagnostic techniques for LME models are inevitably more difficult to implement, due to the increased complexity.
		
		Beckman, Nachtsheim and Cook (1987) \citet{Beckman} applied the \index{local influence}local influence method of Cook (1986) to the analysis of the linear mixed model.
		
		While the concept of influence analysis is straightforward, implementation in mixed models is more complex. Update formulae for fixed effects models are available only when the covariance parameters are assumed to be known.
		
		If the global measure suggests that the points in $U$ are influential, the nature of that influence should be determined. In particular, the points in $U$ can affect the following
		
		\begin{itemize}
			\item the estimates of fixed effects,
			\item the estimates of the precision of the fixed effects,
			\item the estimates of the covariance parameters,
			\item the estimates of the precision of the covariance parameters,
			\item fitted and predicted values.
		\end{itemize}
		
		
		
		
	
	
	%---------------------------------------------------------------------------%
	\newpage
	\subsection{Influence Statistics for LME models} %1.1.4
	Influence statistics can be coarsely grouped by the aspect of estimation that is their primary target:
	\begin{itemize}
		\item overall measures compare changes in objective functions: (restricted) likelihood distance (Cook and Weisberg 1982, Ch. 5.2)
		\item influence on parameter estimates: Cook's  (Cook 1977, 1979), MDFFITS (Belsley, Kuh, and Welsch 1980, p. 32)
		\item influence on precision of estimates: CovRatio and CovTrace
		\item influence on fitted and predicted values: PRESS residual, PRESS statistic (Allen 1974), DFFITS (Belsley, Kuh, and Welsch 1980, p. 15)
		\item outlier properties: internally and externally studentized residuals, leverage
	\end{itemize}
	%---------------------------------------------------------------------------%
	
	


\newpage
\subsection{Extension of techniques to LME Models} %1.2


Model diagnostic techniques, well established for classical models, have since been adapted for use with linear mixed effects models.Diagnostic techniques for LME models are inevitably more difficult to implement, due to the increased complexity.


Beckman, Nachtsheim and Cook (1987) \citet{Beckman} applied the \index{local influence}local influence method of Cook (1986) to the analysis of the linear mixed model.


While the concept of influence analysis is straightforward, implementation in mixed models is more complex. Update formulae for fixed effects models are available only when the covariance parameters are assumed to be known.


If the global measure suggests that the points in $U$ are influential, the nature of that influence should be determined. In particular, the points in $U$ can affect the following


\begin{itemize}
\item the estimates of fixed effects,
\item the estimates of the precision of the fixed effects,
\item the estimates of the covariance parameters,
\item the estimates of the precision of the covariance parameters,
\item fitted and predicted values.
\end{itemize}








\newpage
\subsection{Standardized and studentized residuals} %1.4
%--Studentized and Standardized Residuals

To alleviate the problem caused by inconstant variance, the residuals are scaled (i.e. divided) by their standard deviations. This results in a \index{standardized residual}`standardized residual'. Because true standard deviations are frequently unknown, one can instead divide a residual by the estimated standard deviation to obtain the \index{studentized residual}`studentized residual. 



%--------------------------------------------------%
\subsection{Residual Analysis for Linear Models, LME models and GLMs}

\textbf{Keywords:}

\begin{itemize}
	\item Residuals (\emph{Beginners}), 
	\item Testing the Assumption of Normality (\emph{Beginners})
	\item Diagnostic Plots with the \texttt{plot} function
	\item Cook's Distance
	\item DFFits and DFBeta
	\item Standardized and Studentized Residuals
	\item Influence Leverage and Outlierness
\end{itemize}
%=================================================================================================== %
\newpage

%http://www.artifex.org/~meiercl/R_statistics_guide.pdf
\subsection{Identifying outliers with a LME model object}

The process is slightly different than with standard LME model objects, since the \textbf{\emph{influence}}
function does not work on lme model objects. Given \textbf{\emph{mod.lme}}, we can use the plot function to
identify outliers.
%----------------------%
\subsection{Diagnostics for Random Effects}
Empirical best linear unbiased predictors EBLUPS provide the a useful way of diagnosing random effects.

EBLUPs are also known as ``shrinkage estimators" because they tend to be smaller than the estimated effects would be if they were computed by treating a random factor as if it was fixed (West et al )




\newpage


\subsection{Influence Diagnostics: Basic Idea and Statistics} %1.1.2
%http://support.sas.com/documentation/cdl/en/statug/63033/HTML/default/viewer.htm#statug_mixed_sect024.htm

The general idea of quantifying the influence of one or more observations relies on computing parameter estimates based on all data points, removing the cases in question from the data, refitting the model, and computing statistics based on the change between full-data and reduced-data estimation. 




\subsection{Case Deletion Diagnostics for Mixed Models}

\citet{Christiansen} notes the case deletion diagnostics techniques have not been applied to linear mixed effects models and seeks to develop methodologies in that respect.

\citet{Christiansen} develops these techniques in the context of REML

\subsection{Methods and Measures}
The key to making deletion diagnostics useable is the development of efficient computational formulas, allowing one to obtain the \index{case deletion diagnostics} case deletion diagnostics by making use of basic building blocks, computed only once for the full model.

\citet{Zewotir} lists several established methods of analyzing influence in LME models. These methods include \begin{itemize}
	\item Cook's distance for LME models,
	\item \index{likelihood distance} likelihood distance,
	\item the variance (information) ration,
	\item the \index{Cook-Weisberg statistic} Cook-Weisberg statistic,
	\item the \index{Andrews-Prebigon statistic} Andrews-Prebigon statistic.
\end{itemize}

%===================================================================================== %

\subsection{Cook's 1986 paper on Local Influence}%1.7.1
Cook 1986 introduced methods for local influence assessment. These methods provide a powerful tool for examining perturbations in the assumption of a model, particularly the effects of local perturbations of parameters of observations.

The local-influence approach to influence assessment is quitedifferent from the case deletion approach, comparisons are of
interest.




\end{document}
