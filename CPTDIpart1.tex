\section{Probability-based Measures of Agreement}
There are two measures of agreement based on the probability criteria. The first is the
$p_0$-th percentile of jDj, say Q($p_0$), where $p_0$ (> 0.5) is a specified large probability (usually
¸ 0:80). 
It was introduced by Lin (2000) who called it the total deviation index (TDI). Its
small value indicates a good agreement between (X; Y ). The TDI can be expressed as,

! EQUATION HERE
-distribution with a single degree of freedom
and non-centrality parameter $\Del$.

%-----------------------------------------------------------------------------------------%
%Coverarge Probability

The second measure, introduced by Lin et al. (2002), is the \textbf{coverage probability} (CP) of
the interval [¡±0; ±0], where a difference under §±0 is considered practically equivalent to
zero. There is no loss of generality in taking this interval to be symmetric around zero as it
can be achieved by a location shift. 

Letting,
dl = (¡±0 ¡ ¹)=¾; du = (±0 ¡ ¹)=¾; (2)


the CP can be expressed as
F(±0) = ©(du) ¡ ©(dl): (3)

A high value of F(±0) implies a good agreement between the methods.



