\documentclass[Chap3amain.tex]{subfiles}
\begin{document}

\begin{verbatim}
Fiducial approach for assessing agreement between two instruments
CM Wang and Hari K Iyer
1 Statistical Engineering Division, National Institute of Standards and Technology, Boulder,
CO 80305, USA
2 Department of Statistics, Colorado State University, Fort Collins, CO 80523, USA
%-----------------------------------------------------------------------------------------------%
\end{verbatim}
%DEMING REGRESSION
This paper presents an approach for making inferences about the intercept and slope of a linear
regression model when both variables are subject to measurement errors. The approach is
based on the principle of fiducial inference. A procedure is presented for computing
uncertainty regions for the intercept and slope that can be used to assess agreement between
two instruments. Computer codes for performing these calculations, written using open-source
software, are listed.
%-----------------------------------------------------------------------------------------------%

%EQUIVALENCE REGION
The equivalence region is specified by the user. It can be an ellipse, parallelogram,
rectangle, or a region of some other appropriate shape. The way we use the fiducial region in this method is as follows. If
the $1−\gamma$ fiducial region that we construct is completely inside the equivalence region then we have established agreement.

%MAXIMUM ALLOWABLE DIFFERENCE
maximum allowable difference of the two equivalent methods.

%-----------------------------------------------------------------------------------------------%
In this paper we have provided an approach for making inference on the intercept $\beta_0$ and slope $\beta_1$ of a linear regression
model with both X and Y subject to measurement errors. Specifically, we have provided procedures for constructing
uncertainty regions for ($\beta_0$, $\beta_1$ ) that can be used to assess agreement between two methods. The approach is based on
fiducial inference.


%-----------------------------------------------------------------------------------------------%
\end{document}
