\documentclass[MAIN.tex]{subfiles}

% Load any packages needed for this document
\begin{document}

	\chapter{Linear Mixed effects Models}
\section{Introduction to Mixed Models}

%\citet{BrownPrescott} defines random effects as realizations of
%samples from a normal distribution with mean equal to zero.

All models are characterized by the mean $\alpha$ and the error
terms. In addition to these terms, any model described so far will
have either random effects terms or fixed effects terms and
accordingly are referred to as random or fixed models. Models that
have both fixed effects terms and random effects terms are known
as 'mixed effects models'. Once the theory underlying fixed and
random effects models has been fully understood, the progression
to understanding mixed models is very simple.

Elaborating on the original mice litter example, the six litters
by each mouse were fed according to three different dietary
treatments \citep{Searle}. Therefore a fixed effect $\phi_{j}$ has
been added to the model, which is now formulated as follows;
\begin{equation}
y_{ij} = \mu + \delta_{i} + \phi_{j} + \gamma_{ij} +
\epsilon_{ijk}
\end{equation}
As before, an interaction effect $\gamma_{ij}$ must also be added
to the model. In cases where the interaction term describes the
combined effect of fixed and random components, it should be
treated as random effect. The variance of the above model is
composed of the $\sigma^{2}_{\delta}$, $\sigma^{2}_{\gamma}$ and
$\sigma^{2}_{\epsilon}$ .


It may be shown that the interaction factors make no contribution
to the outcome, i.e $\gamma_{ij}$ is consistently calculated as
zero. Considering the skin tumour example, a person's age would
bear no relation to their gender and hence there would be
plausible interaction between the two factors. Indeed , in keeping
with the `Law of Parsimony', factors should be specified such that
each would convey separate information. However, interaction terms
are extant when the model specifies repeated observations, as
there is necessarily a relationship between observations from the
same subject. Importantly, interaction effects, being random
effects, are attended by variance component terms and therefore
also contribute to the overall variance of the model.

\citet{Searle} gives a mixed effects model formulation for the
Grubbs artillery study. $y_{ij}$ is the muzzle velocity of the
$i$th shell, as measured by the $j$th chronometer.
\begin{equation}
y_{ij} = \mu + \alpha_{i} + \beta_{j}  + \epsilon_{ij}
\end{equation}
In this formulation $\alpha_{i}$ is the random effect of round
$i$, and the fixed effect component $\beta_{j}$ is the bias in
chronometer $j$. (Also, no interaction term is used).



\section{Linear Mixed effects Models}
	A linear mixed effects (LME) model is a statistical model containing both fixed effects and random effects (random effects are also known as variance components). LME models are a generalization of the classical linear model, which contain fixed effects only. When the levels of factors are considered to be sampled from a population,
	and each level is not of particular interest, they are considered random quantities with associated variances.
	The effects of the levels, as described, are known as random effects. Random effects are represented by unobservable
	normally distributed random variables. Conversely fixed effects are considered non-random and the
	levels of each factor are of specific interest.
	%LME models are useful models when considering repeated measurements or grouped observations.
	
	\citet{Fisher4} introduced variance components models for use in genetical studies. Whereas an estimate for variance must take an non-negative value, an individual variance component, i.e.\ a component of the overall variance, may be negative.
	
	The framework has developed since, including contributions from
	\citet{tippett}, who extend the use of variance components into linear models, and \citet{eisenhart}, who introduced the `mixed model' terminology and formally distinguished between mixed and random effects models. \citet{Henderson:1950} devised a framework for deriving estimates for both the fixed effects and the random effects, using a set of equations that would become known as `mixed model equations' or `Henderson's equations'.
	LME methodology is further enhanced by Henderson's later works \citep{Henderson53, Henderson59,Henderson63,Henderson73,Henderson84a}. The key features of Henderson's work provide the basis for the estimation techniques.
	
	\citet{HartleyRao} demonstrated that unique estimates of the variance components could be obtained using maximum likelihood methods. However these estimates are known to be biased `downwards' (i.e.\ underestimated) , because of the assumption that the fixed estimates are known, rather than being estimated from the data. \citet{PattersonThompson} produced an alternative set of estimates, known as the restricted maximum likelihood (REML) estimates, that do not require the fixed effects to be known. Thusly there is a distinction the REML estimates and the original estimates, now commonly referred to as ML estimates.
	

	%\subsection{Likelihood and estimation}
	
	% Likelihood is the hypothetical probability that an event that has already occurred would yield a specific outcome. Likelihood differs from probability in that probability refers to future occurrences, while likelihood refers to past known outcomes.
	
	% The likelihood function ($L(\theta)$)is a fundamental concept in statistical inference. It indicates how likely a particular population is to produce an observed sample. The set of values that maximize the likelihood function are considered to be optimal, and are used as the estimates of the parameters. For computational ease, it is common to use the logarithm of the likelihood function, known simply as the log-likelihood ($\ell(\theta)$).
	
	


\section{Linear mixed effects models}

% http://www.artifex.org/~meiercl/R_statistics_guide.pdf
These models are used when there are both fixed and random effects that need to be incorporated into a model.

Fixed effects usually correspond to experimental treatments for which one has data for the entire population of samples corresponding to that treatment.

Random effects,on the other hand, are assigned in the case where we have measurements on a group of samples, and those
samples are taken from some larger sample pool, and are presumed to be representative.

As such, linear mixed effects models treat the error for fixed effects differently than the error for random effects.





	\section{Limits of agreement in LME models}
	
	Limits of agreement are used extensively for assessing agreement, because they are intuitive and easy to use.
	Necessarily their prevalence in literature has meant that they are now the best known measurement for agreement, and therefore any newer methodology would benefit by making reference to them.
	
	\citet{BXC2008} uses LME models to determine the limits of agreement. Between-subject variation for method $m$ is given by $d^2_{m}$ and within-subject variation is given by $\lambda^2_{m}$.  \citet{BXC2008} remarks that for two methods $A$ and $B$, separate values of $d^2_{A}$ and $d^2_{B}$ cannot be estimated, only their average. Hence the assumption that $d_{x}= d_{y}= d$ is necessary. The between-subject variability $\boldsymbol{D}$ and within-subject variability $\boldsymbol{\Lambda}$ can be presented in matrix form,\[
	\boldsymbol{D} = \left(%
	\begin{array}{cc}
	d^2_{A}& 0 \\
	0 & d^2_{B} \\
	\end{array}%
	\right)=\left(%
	\begin{array}{cc}
	d^2& 0 \\
	0 & d^2\\
	\end{array}%
	\right),
	\hspace{1.5cm}
	\boldsymbol{\Lambda} = \left(%
	\begin{array}{cc}
	\lambda^2_{A}& 0 \\
	0 & \lambda^2_{B} \\
	\end{array}%
	\right).
	\]
	
	The variance for method $m$ is $d^2_{m}+\lambda^2_{m}$. Limits of agreement are determined using the standard deviation of the case-wise differences between the sets of measurements by two methods $A$ and $B$, given by
	\begin{equation}
	\mbox{var} (y_{A}-y_{B}) = 2d^2 + \lambda^2_{A}+ \lambda^2_{B}.
	\end{equation}
	Importantly the covariance terms in both variability matrices are zero, and no covariance component is present.
	
	\citet{BXC2008} presents a data set `fat', which is a comparison of measurements of subcutaneous fat
	by two observers at the Steno Diabetes Center, Copenhagen. Measurements are in millimeters
	(mm). Each person is measured three times by each observer. The observations are considered to be `true' replicates.
	
	A linear mixed effects model is formulated, and implementation through several software packages is demonstrated.
	All of the necessary terms are presented in the computer output. The limits of agreement are therefore,
	\begin{equation}
	0.0449  \pm 1.96 \times  \sqrt{2 \times 0.0596^2 + 0.0772^2 + 0.0724^2} = (-0.220,  0.309).
	\end{equation}
	
	\citet{roy} has demonstrated a methodology whereby $d^2_{A}$ and $d^2_{B}$ can be estimated separately. Also covariance terms are present in both $\boldsymbol{D}$ and $\boldsymbol{\Lambda}$. Using Roy's methodology, the variance of the differences is
	\begin{equation}
	\mbox{var} (y_{iA}-y_{iB})= d^2_{A} + \lambda^2_{B} + d^2_{A} + \lambda^2_{B} - 2(d_{AB} + \lambda_{AB})
	\end{equation}
	All of these terms are given or determinable in computer output.
	The limits of agreement can therefore be evaluated using
	\begin{equation}
	\bar{y_{A}}-\bar{y_{B}} \pm 1.96 \times \sqrt{ \sigma^2_{A} + \sigma^2_{B}  - 2(\sigma_{AB})}.
	\end{equation}
	
	For Carstensen's `fat' data, the limits of agreement computed using Roy's
	method are consistent with the estimates given by \citet{BXC2008}; $0.044884  \pm 1.96 \times  0.1373979 = (-0.224,  0.314).$
	
	\subsection{Linked replicates}
	
	\citet{BXC2008} proposes the addition of an random effects term to their model when the replicates are linked. This term is used to describe the `item by replicate' interaction, which is independent of the methods. This interaction is a source of variability independent of the methods. Therefore failure to account for it will result in variability being wrongly attributed to the methods.
	
	\citet{BXC2008} introduces a second data set; the oximetry study. This study done at the Royal Children’s Hospital in
	Melbourne to assess the agreement between co-oximetry and pulse oximetry in small babies.
	
	In most cases, measurements were taken by both method at three different times. In some cases there are either one or two pairs of measurements, hence the data is unbalanced. \citet{BXC2008} describes many of the children as being very sick, and with very low oxygen saturations levels. Therefore it must be assumed that a biological change can occur in interim periods, and measurements are not true replicates.
	
	\citet{BXC2008} demonstrate the necessity of accounting for linked replicated by comparing the limits of agreement from the `oximetry' data set using a model with the additional term, and one without. When the interaction is accounted for the limits of agreement are (-9.62,14.56). When the interaction is not accounted for, the limits of agreement are (-11.88,16.83). It is shown that the failure to include this additional term results in an over-estimation of the standard deviations of differences.
	
	Limits of agreement are determined using Roy's methodology, without adding any additional terms, are found to be consistent with the `interaction' model; $(-9.562, 14.504 )$. Roy's methodology assumes that replicates are linked. However, following Carstensen's example, an addition interaction term is added to the implementation of Roy's model to assess the effect, the limits of agreement estimates do not change. However there is a conspicuous difference in within-subject matrices of Roy's model and the modified model (denoted $1$ and $2$ respectively);
	\begin{equation}
	\hat{\boldsymbol{\Lambda}}_{1}= \left(\begin{array}{cc}
	16.61 &	11.67\\
	11.67 & 27.65 \end{array}\right) \qquad
	\boldsymbol{\hat{\Lambda}}_{2}= \left( \begin{array}{cc}
	7.55 & 2.60 \\
	2.60 & 18.59 \end{array} \right). 
	\end{equation}
	
	\noindent (The variance of the additional random effect in model $2$ is $3.01$.)
	
	\citet{akaike} introduces the Akaike information criterion ($AIC$), a model 
	selection tool based on the likelihood function. Given a data set, candidate models
	are ranked according to their AIC values, with the model having the lowest AIC being considered the best fit.Two candidate models can said to be equally good if there is a difference of less than $2$ in their AIC values.
	
	The Akaike information criterion (AIC) for both models are $AIC_{1} = 2304.226$ and $AIC_{2} = 2306.226$ , indicating little difference in models. The AIC values for the Carstensen `unlinked' and `linked' models are $1994.66$ and $1955.48$ respectively, indicating an improvement by adding the interaction term.
	
	The $\boldsymbol{\hat{\Lambda}}$ matrices are informative as to the difference between Carstensen's unlinked and linked models. For the oximetry data, the covariance terms (given above as 11.67 and 2.6 respectively ) are of similar magnitudes to the variance terms. Conversely for the `fat' data the covariance term ($-0.00032$) is negligible. When the interaction term is added to the model, the covariance term remains negligible. (For the `fat' data, the difference in AIC values is also approximately $2$).
	
	To conclude, Carstensen's models provided a rigorous way to determine limits of agreement, but don't provide for the computation of $\boldsymbol{\hat{D}}$ and $\boldsymbol{\hat{\Lambda}}$. Therefore the test's proposed by \citet{roy} can not be implemented. Conversely, accurate limits of agreement as determined by Carstensen's model may also be found using Roy's method. Addition of the interaction term erodes the capability of Roy's methodology to compare candidate models, and therefore shall not be adopted.
	
	Finally, to complement the blood pressure (i.e.`J vs S') method comparison from the previous section (i.e.`J vs S'), the limits of agreement are $15.62 \pm 1.96 \times 20.33 = (-24.22, 55.46)$.)
	\newpage







\section{Linear Mixed effects Models}
A linear mixed effects (LME) model is a statistical model containing both fixed effects and random effects (random effects are also known as variance components). LME models are a generalization of the classical linear model, which contain fixed effects only. When the levels of factors are considered to be sampled from a population,
and each level is not of particular interest, they are considered random quantities with associated variances.
The effects of the levels, as described, are known as random effects. Random effects are represented by unobservable
normally distributed random variables. Conversely fixed effects are considered non-random and the
levels of each factor are of specific interest.
%LME models are useful models when considering repeated measurements or grouped observations.

\citet{Fisher4} introduced variance components models for use in genetical studies. Whereas an estimate for variance must take an non-negative value, an individual variance component, i.e.\ a component of the overall variance, may be negative.

The methodology has developed since, including contributions from
\citet{tippett}, who extend the use of variance components into linear models, and \citet{eisenhart}, who introduced the `mixed model' terminology and formally distinguished between mixed and random effects models. \citet{Henderson:1950} devised a methodology for deriving estimates for both the fixed effects and the random effects, using a set of equations that would become known as `mixed model equations' or `Henderson's equations'.
LME methodology is further enhanced by Henderson's later works \citep{Henderson53, Henderson59,Henderson63,Henderson73,Henderson84a}. The key features of Henderson's work provide the basis for the estimation techniques.

\citet{HartleyRao} demonstrated that unique estimates of the variance components could be obtained using maximum likelihood methods. However these estimates are known to be biased `downwards' (i.e.\ underestimated) , because of the assumption that the fixed estimates are known, rather than being estimated from the data. \citet{PattersonThompson} produced an alternative set of estimates, known as the restricted maximum likelihood (REML) estimates, that do not require the fixed effects to be known. Thusly there is a distinction the REML estimates and the original estimates, now commonly referred to as ML estimates.

\citet{LW82} provides a form of notation for notation for LME models that has since become the standard form, or the basis for more complex formulations. Due to computation complexity, linear mixed effects models have not seen widespread use until many well known statistical software applications began facilitating them. SAS Institute added PROC MIXED to its software suite in 1992 \citep{singer}. \citet{PB} described how to compute LME models in the \texttt{S-plus} environment.

Using Laird-Ware form, the LME model is commonly described in matrix form,
\begin{equation}
y = X\beta + Zb + \epsilon
\label{LW}
\end{equation}

\noindent where $y$ is a vector of $N$ observable random variables, $\beta$ is a vector of $p$ fixed effects, $X$ and $Z$ are $N \times p$ and $N \times q$ known matrices, and $b$ and $\epsilon$  are vectors of $q$ and $N,$ respectively, random effects such that $\mathrm{E}(b)=0, \ \mathrm{E}(\epsilon)=0$
and
%	\[
%	\mathrm{var}
%	\begin{pmatrix}{
%		b \cr
%		\epsilon }  =
%	\begin{pmatrix}{
%		D & 0 \cr
%		0 & \Sigma }
%	\]
where $D$ and $\Sigma$ are positive definite matrices parameterized by an unknown variance component parameter vector $ \theta.$ The variance-covariance matrix for the vector of observations $y$ is given by $V = ZDZ^{\prime}+ \Sigma.$ This implies $y \sim(X\beta, V) = (X\beta,ZDZ^{\prime}+ \Sigma)$. It is worth noting that $V$ is an $n \times n$ matrix, as the dimensionality becomes relevant later on. The notation provided here is generic, and will be adapted to accord with complex formulations that will be encountered in due course.

%\subsection{Likelihood and estimation}

% Likelihood is the hypothetical probability that an event that has already occurred would yield a specific outcome. Likelihood differs from probability in that probability refers to future occurrences, while likelihood refers to past known outcomes.

% The likelihood function ($L(\theta)$)is a fundamental concept in statistical inference. It indicates how likely a particular population is to produce an observed sample. The set of values that maximize the likelihood function are considered to be optimal, and are used as the estimates of the parameters. For computational ease, it is common to use the logarithm of the likelihood function, known simply as the log-likelihood ($\ell(\theta)$).



\subsection{Formulation of the response vector}
Information of individual $i$ is recorded in a response vector $\boldsymbol{y}_{i}$. The response vector is constructed by stacking the response of the $2$ responses at the first instance, then the $2$ responses at the second instance, and so on. Therefore the response vector is a $2n_{i} \times 1$ column vector.
The covariance matrix of $\boldsymbol{y_{i}}$ is a $2n_{i} \times 2n_{i}$ positive definite matrix $\boldsymbol{\Omega}_{i}$.

Consider the case where three measurements are taken by both methods $A$ and $B$, $\boldsymbol{y}_{i}$ is a $6 \times 1$ random vector describing the $i$th subject.
\[
\boldsymbol{y}_{i} = (y_{i}^{A1},y_{i}^{B1},y_{i}^{A2},y_{i}^{B2},y_{i}^{A3},y_{i}^{B3}) \prime
\]

The response vector $\boldsymbol{y_{i}}$ can be formulated as an LME model according to Laird-Ware form.
\begin{eqnarray*}
	\boldsymbol{y_{i}} = \boldsymbol{X_{i}\beta}  + \boldsymbol{Z_{i}b_{i}} + \boldsymbol{\epsilon_{i}}\\
	\boldsymbol{b_{i}} \sim \mathcal{N}(\boldsymbol{0,D})\\
	\boldsymbol{\epsilon_{i}} \sim \mathcal{N}(\boldsymbol{0,R_{i}})
\end{eqnarray*}

Information on the fixed effects are contained in a three dimensional vector $\boldsymbol{\beta} = (\beta_{0},\beta_{1},\beta_{2})\prime$. For computational purposes $\beta_{2}$ is conventionally set to zero. Consequently $\boldsymbol{\beta}$ is the solutions of the means of the two methods, i.e. $E(\boldsymbol{y}_{i})  = \boldsymbol{X}_{i}\boldsymbol{\beta}$. The variance covariance matrix $\boldsymbol{D}$ is a general $2 \times 2$ matrix, while $\boldsymbol{R}_{i}$ is a $2n_{i} \times 2n_{i}$ matrix.

%------------------------------------------------------------------------------%
\subsection{Decomposition of the response covariance matrix}

The variance covariance structure can be re-expressed in the following form,
\[
\mbox{Cov}(\mbox{y}_{i}) = \boldsymbol{\Omega_{i}} = \boldsymbol{Z}_{i}\boldsymbol{D}\boldsymbol{Z}_{i}^\prime + \boldsymbol{R_{i}}.
\]

$\boldsymbol{R_{i}}$ can be shown to be the Kronecker product of a correlation matrix $\boldsymbol{V}$ and $\boldsymbol{\Lambda}$. The correlation matrix $\boldsymbol{V}$ of the repeated measures on a given response variable is assumed to be the same for all response variables. Both \citet{hamlett} and \citet{lam} use the identity matrix, with dimensions $n_{i} \times n_{i}$ as the formulation for $\boldsymbol{V}$. \citet{ARoy2009} remarks that, with repeated measures, the response for each subject is correlated for each variable, and that such correlation must be taken into account in order to produce a valid inference on correlation estimates.  \citet{ARoy20092006} proposes various correlation structures may be assumed for repeated measure correlations, such as the compound symmetry and autoregressive structures, as alternative to the identity matrix.

However, for the purposes of method comparison studies, the necessary estimates are currently only determinable when the identity matrix is specified, and the results in \citet{ARoy2009} indicate its use.

For the response vector described, \citet{hamlett} presents a detailed covariance matrix. A brief summary shall be presented here only. The overall variance matrix is a $6 \times 6$ matrix composed of two types of $2 \times 2$ blocks. Each block represents one separate time of measurement.

\[
\boldsymbol{\Omega}_{i} = \left(
\begin{array}{ccc}
\boldsymbol{\Sigma} & \boldsymbol{D} & \boldsymbol{D}\\
\boldsymbol{D} & \boldsymbol{\Sigma} & \boldsymbol{D}\\
\boldsymbol{D} & \boldsymbol{D} & \boldsymbol{\Sigma}\\
\end{array}\right)
\]

The diagonal blocks are $\Sigma$, as described previously. The $2 \times 2$ block diagonal matrix in $\boldsymbol{\Omega}$ gives $\boldsymbol{\Sigma}$. $\boldsymbol{\Sigma}$ is the sum of the between-subject variability $\boldsymbol{D}$ and the within subject variability $\boldsymbol{\Lambda}$.

$\boldsymbol{\Omega_{i}}$ can be expressed as
\[
\boldsymbol{\Omega_{i}} = \boldsymbol{Z}_{i}\boldsymbol{D}\boldsymbol{Z}_{i}^\prime + ({\boldsymbol{I_{n_{i}}} \otimes \boldsymbol{\Lambda}}).
\]
The notation $\mbox{dim}_{n_{i}}$ means an $n_{i} \times n_{i}$ diagonal block.




	\section{Repeated measurements in LME models}
	
	In many statistical analyzes, the need to determine parameter estimates where multiple measurements are available on each of a set of variables often arises. Further to \citet{lam}, \citet{hamlett} performs an analysis of the correlation of replicate measurements, for two variables of interest, using LME models.
	
	Let $y_{Aij}$ and $y_{Bij}$ be the $j$th repeated observations of the variables of interest $A$ and $B$ taken on the $i$th subject. The number of repeated measurements for each variable may differ for each individual.
	Both variables are measured on each time points. Let $n_{i}$ be the number of observations for each variable, hence $2\times n_{i}$ observations in total.
	
	It is assumed that the pair $y_{Aij}$ and $y_{Bij}$ follow a bivariate normal distribution.
	\begin{eqnarray*}
		\left(
		\begin{array}{c}
			y_{Aij} \\
			y_{Bij} \\
		\end{array}
		\right) \sim \mathcal{N}(
		\boldsymbol{\mu}, \boldsymbol{\Sigma})\mbox{   where } \boldsymbol{\mu} = \left(
		\begin{array}{c}
			\mu_{A} \\
			\mu_{B} \\
		\end{array}
		\right)
	\end{eqnarray*}
	
	The matrix $\Sigma$ represents the variance component matrix between response variables at a given time point $j$.
	
	\[
	\boldsymbol{\Sigma} = \left( \begin{array}{cc}
	\sigma^2_{A} & \sigma_{AB} \\
	\sigma_{AB} & \sigma^2_{B}\\
	\end{array}   \right)
	\]
	
	$\sigma^2_{A}$ is the variance of variable $A$, $\sigma^2_{B}$ is the variance of variable $B$ and $\sigma_{AB}$ is the covariance of the two variable. It is assumed that $\boldsymbol{\Sigma}$ does not depend on a particular time point, and is the same over all time points.
	
	%------------------------------------------------------------------------------%
	\subsection{Formulation of the Response Vector}
	Information of individual $i$ is recorded in a response vector $\boldsymbol{y}_{i}$. The response vector is constructed by stacking the response of the $2$ responses at the first instance, then the $2$ responses at the second instance, and so on. Therefore the response vector is a $2n_{i} \times 1$ column vector.
	The covariance matrix of $\boldsymbol{y_{i}}$ is a $2n_{i} \times 2n_{i}$ positive definite matrix $\boldsymbol{\Omega}_{i}$.
	
	Consider the case where three measurements are taken by both methods $A$ and $B$, $\boldsymbol{y}_{i}$ is a $6 \times 1$ random vector describing the $i$th subject.
	\[
	\boldsymbol{y}_{i} = (y_{i}^{A1},y_{i}^{B1},y_{i}^{A2},y_{i}^{B2},y_{i}^{A3},y_{i}^{B3}) \prime
	\]
	
	The response vector $\boldsymbol{y_{i}}$ can be formulated as an LME model according to Laird-Ware form.
	\begin{eqnarray*}
		\boldsymbol{y_{i}} = \boldsymbol{X_{i}\beta}  + \boldsymbol{Z_{i}b_{i}} + \boldsymbol{\epsilon_{i}}\\
		\boldsymbol{b_{i}} \sim \mathcal{N}(\boldsymbol{0,D})\\
		\boldsymbol{\epsilon_{i}} \sim \mathcal{N}(\boldsymbol{0,R_{i}})
	\end{eqnarray*}
	
	Information on the fixed effects are contained in a three dimensional vector $\boldsymbol{\beta} = (\beta_{0},\beta_{1},\beta_{2})\prime$. For computational purposes $\beta_{2}$ is conventionally set to zero. Consequently $\boldsymbol{\beta}$ is the solutions of the means of the two methods, i.e. $E(\boldsymbol{y}_{i})  = \boldsymbol{X}_{i}\boldsymbol{\beta}$. The variance covariance matrix $\boldsymbol{D}$ is a general $2 \times 2$ matrix, while $\boldsymbol{R}_{i}$ is a $2n_{i} \times 2n_{i}$ matrix.
	
	%------------------------------------------------------------------------------%
	\subsection{Decomposition of the response covariance matrix}
	
	The variance covariance structure can be re-expressed in the following form,
	\[
	\mbox{Cov}(\mbox{y}_{i}) = \boldsymbol{\Omega_{i}} = \boldsymbol{Z}_{i}\boldsymbol{D}\boldsymbol{Z}_{i}^\prime + \boldsymbol{R_{i}}.
	\]
	
	$\boldsymbol{R_{i}}$ can be shown to be the Kronecker product of a correlation matrix $\boldsymbol{V}$ and $\boldsymbol{\Lambda}$. The correlation matrix $\boldsymbol{V}$ of the repeated measures on a given response variable is assumed to be the same for all response variables. Both \citet{hamlett} and \citet{lam} use the identity matrix, with dimensions $n_{i} \times n_{i}$ as the formulation for $\boldsymbol{V}$. \citet{roy} remarks that, with repeated measures, the response for each subject is correlated for each variable, and that such correlation must be taken into account in order to produce a valid inference on correlation estimates.  \citet{roy2006} proposes various correlation structures may be assumed for repeated measure correlations, such as the compound symmetry and autoregressive structures, as alternative to the identity matrix.
	
	However, for the purposes of method comparison studies, the necessary estimates are currently only determinable when the identity matrix is specified, and the results in \citet{roy} indicate its use.
	
	For the response vector described, \citet{hamlett} presents a detailed covariance matrix. A brief summary shall be presented here only. The overall variance matrix is a $6 \times 6$ matrix composed of two types of $2 \times 2$ blocks. Each block represents one separate time of measurement.
	
	\[
	\boldsymbol{\Omega}_{i} = \left(
	\begin{array}{ccc}
	\boldsymbol{\Sigma} & \boldsymbol{D} & \boldsymbol{D}\\
	\boldsymbol{D} & \boldsymbol{\Sigma} & \boldsymbol{D}\\
	\boldsymbol{D} & \boldsymbol{D} & \boldsymbol{\Sigma}\\
	\end{array}\right)
	\]
	
	The diagonal blocks are $\Sigma$, as described previously. The $2 \times 2$ block diagonal matrix in $\boldsymbol{\Omega}$ gives $\boldsymbol{\Sigma}$. $\boldsymbol{\Sigma}$ is the sum of the between-subject variability $\boldsymbol{D}$ and the within subject variability $\boldsymbol{\Lambda}$.
	
	$\boldsymbol{\Omega_{i}}$ can be expressed as
	\[
	\boldsymbol{\Omega_{i}} = \boldsymbol{Z}_{i}\boldsymbol{D}\boldsymbol{Z}_{i}^\prime + ({\boldsymbol{I_{n_{i}}} \otimes \boldsymbol{\Lambda}}).
	\]
	The notation $\mbox{dim}_{n_{i}}$ means an $n_{i} \times n_{i}$ diagonal block.
	
	\subsection{Correlation terms}
	\citet{hamlett} demonstrated how the between-subject and within subject variabilities can be expressed in terms of
	correlation terms.
	
	\[
	\boldsymbol{D} = \left( \begin{array}{cc}
	\sigma^2_{A}\rho_{A} & \sigma_{A}\sigma_{b}\rho_{AB}\delta \\
	\sigma_{A}\sigma_{b}\rho_{AB}\delta & \sigma^2_{B}\rho_{B}\\
	
	\end{array}\right)
	\]
	
	\[
	\boldsymbol{\Lambda} = \left(
	\begin{array}{cc}
	\sigma^2_{A}(1-\rho_{A}) & \sigma_{AB}(1-\delta)  \\
	\sigma_{AB}(1-\delta) & \sigma^2_{B}(1-\rho_{B}) \\
	\end{array}\right).
	\]
	
	$\rho_{A}$ describe the correlations of measurements made by the method $A$ at different times. Similarly $\rho_{B}$ describe the correlation of measurements made by the method $B$ at different times. Correlations among repeated measures within the same method are known as intra-class correlation coefficients. $\rho_{AB}$ describes the correlation of measurements taken at the same same time by both methods. The coefficient $\delta$ is added for when the measurements are taken at different times, and is a constant of less than $1$ for linked replicates. This is based on the assumption that linked replicates measurements taken at the same time would have greater correlation than those taken at different times. For unlinked replicates $\delta$ is simply $1$. \citet{hamlett} provides a useful graphical depiction of the role of each correlation coefficients.
	
	\newpage
\section{Extended LME model}
% Pinheiro Bates Page 202
The extended single level LME model relaxes the independence assumption, allowing heteroscedastic and correlated within group errors.


\begin{equation}
\epsilon_{i} = \mathcal{N}(0, \sigma^2 \Lambda_{i})
\end{equation}

$\Lambda_{i}$ are positive definite matrices. $\sigma^2$ is factored out of the matrix for computational reasons.




\bibliographystyle{chicago}
\bibliography{DB-txfrbib}
\end{document}