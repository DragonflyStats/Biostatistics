
\documentclass[12pt, a4paper]{article}
\usepackage{epsfig}
\usepackage{subfigure}
%\usepackage{amscd}
\usepackage{amssymb}
\usepackage{amsbsy}
\usepackage{amsthm, amsmath}
%\usepackage[dvips]{graphicx}
\usepackage{natbib}
\bibliographystyle{chicago}
\usepackage{vmargin}
% left top textwidth textheight headheight
% headsep footheight footskip
\setmargins{3.0cm}{2.5cm}{15.5 cm}{22cm}{0.5cm}{0cm}{1cm}{1cm}
\renewcommand{\baselinestretch}{1.5}
\pagenumbering{arabic}
\theoremstyle{plain}
\newtheorem{theorem}{Theorem}[section]
\newtheorem{corollary}[theorem]{Corollary}
\newtheorem{ill}[theorem]{Example}
\newtheorem{lemma}[theorem]{Lemma}
\newtheorem{proposition}[theorem]{Proposition}
\newtheorem{conjecture}[theorem]{Conjecture}
\newtheorem{axiom}{Axiom}
\theoremstyle{definition}
\newtheorem{definition}{Definition}[section]
\newtheorem{notation}{Notation}
\theoremstyle{remark}
\newtheorem{remark}{Remark}[section]
\newtheorem{example}{Example}[section]
\renewcommand{\thenotation}{}
\renewcommand{\thetable}{\thesection.\arabic{table}}
\renewcommand{\thefigure}{\thesection.\arabic{figure}}
\title{Research notes: linear mixed effects models}
\author{ } \date{ }


\begin{document}
\author{Kevin O'Brien}
\title{Updating techniques for LME models}

\addcontentsline{toc}{section}{Bibliography}

%-----------------------------------------------------------------------------------------%

\newpage
\section{Introduction}

\citet{roy} uses an approach based on linear mixed effects (LME) models for the purpose of comparing the agreement between two methods of measurement, where replicate measurements on items (ofttimes individuals) by both methods are available. She provides three tests of hypothesis appropriate for evaluating the agreement between the two methods of measurement under this sampling scheme. These tests consider null hypotheses that assume: absence of inter-method bias; equality of between-subject variabilities of the two methods; equality of within-subject variabilities of the two methods. By inter-method bias we mean that a systematic difference exists between observations recorded by the two methods. Differences in between-subject variabilities of the two methods arise when one method is yielding average response levels for individuals than are more variable than the average response levels for the same sample of individuals taken by the other method.  Differences in within-subject variabilities of the two methods arise when one method is yielding responses for an individual than are more variable than the responses for this same individual taken by the other method. The two methods of measurement can be considered to agree, and subsequently can be used interchangeably, if all three null hypotheses are true.

\bigskip

Let $y_{mir} $ be the $r$th replicate measurement on the $i$th item by the $m$th method, where $m=1,2,$ $i=1,\ldots,N,$ and $r = 1,\ldots,n_i.$ When the design is balanced and there is no ambiguity we can set $n_i=n.$ The LME model underpinning Roy's approach can be written
\begin{equation}\label{Roy-model}
y_{mir} = \beta_{0} + \beta_{m} + b_{mi} + \epsilon_{mir}.
\end{equation}
Here $\beta_0$ and $\beta_m$ are fixed-effect terms representing, respectively, a model intercept and an overall effect for method $m.$
To determine an estimable solution, the $\beta$ terms can be gathered together into (fixed effect) intercept terms $\alpha_m=\beta_0+\beta_m.$ The $b_{1i}$ and $b_{2i}$ terms are correlated random effect parameters having $\mathrm{E}(b_{mi})=0$ with $\mathrm{Var}(b_{mi})=g^2_m$ and $\mathrm{Cov}(b_{mi}, b_{m^\prime i})=g_{12}.$ The random error term for each response is denoted $\epsilon_{mir}$ having $\mathrm{E}(\epsilon_{mir})=0$, $\mathrm{Var}(\epsilon_{mir})=\sigma^2_m$, $\mathrm{Cov}(b_{mir}, b_{m^\prime ir})=\sigma_{12}$, $\mathrm{Cov}(\epsilon_{mir}, \epsilon_{mir^\prime})= 0$ and $\mathrm{Cov}(\epsilon_{mir}, \epsilon_{m^\prime ir^\prime})= 0.$ Two methods of measurement are in complete agreement if the null hypotheses $\mathrm{H}_1\colon \alpha_1 = \alpha_2$ and $\mathrm{H}_2\colon \sigma^2_1 = \sigma^2_2 $ and $\mathrm{H}_3\colon g^2_1= g^2_2$ hold simultaneously. \citet{roy} proposes a Bonferroni correction to control the familywise error rate for tests of $\{\mathrm{H}_1, \mathrm{H}_2, \mathrm{H}_3\}$ and account for difficulties arising due to multiple testing. Roy also integrates $\mathrm{H}_2$ and $\mathrm{H}_3$ into a single testable hypothesis $\mathrm{H}_4\colon \omega^2_1=\omega^2_2,$ where $\omega^2_m = \sigma^2_m + g^2_m$ represent the overall variability of method $m.$  Disagreement in overall variability may be caused by different between-item variabilities, by different within-item variabilities, or by both.  If the exact cause of disagreement between the two methods is not of interest, then the overall variability test $H_4$ is an alternative to testing $H_2$ and $H_3$ separately.

\bigskip

\citet{BXC2008} proposes an approach, also based on the LME model, for the purpose of comparing two (or more) methods of measurement where replicate measurements are available on each item. Their approach is to adapt the popular limits-of-agreement (LOA) methodology, devised by \citet{BA86}, to consider the replicate measurements. They propose an appropriate estimate for the standard deviation of the case-wise differences for the purpose of determining the LOAs. As their interest lies in extending the Bland-Altman methodology, other formal tests are not considered.
 
\citet{BXC2008} presents how this model is developed from a standard two-way analysis of variance model, reformulated for the case of replicate measurements, specifying random effects terms as appropriate. Their model can be written:
\begin{equation}\label{BXC-model}
y_{mir}  = \alpha_{m} + \mu_{i} + a_{ir} + c_{mi} + \epsilon_{mir}.
\end{equation}
The $\alpha_{m}$ and $\mu_{i}$ terms represent, respectively, the fixed-effect for method $m$ and the fixed-effect for item $i.$ All random-effect terms are assumed to be independent, and incorporate an item-by-replicate interaction term $a_{ir} \sim \mathcal{N}(0,\varsigma^{2})$, a method-by-item interaction term $c_{mi} \sim \mathcal{N}(0,\tau^{2}_{m}),$ and model error terms $\varepsilon \sim \mathcal{N}(0,\varphi^{2}_{m}).$ 

There is a substantial difference in the number of fixed parameters used by each model. For the model in (\ref{Roy-model}) requires two fixed effect parameters, i.e. the means of the two methods, for any number of items $N$. In contrast, the model described by (\ref{BXC-model}) requires $N+2$ fixed effects for $N$ items. The inclusion of fixed effects to account for the `true value' of each item greatly increases the level of model complexity.

When only two methods are compared, \citet{bxc2008} notes that separate estimates of $\tau^2_m$ can not be obtained due to the model over-specification. To overcome this, the assumption of equality ,i.e. $\tau^2_1 = \tau^2_2$, is required.

%With regards to the specification of the variance terms, Carstensen  remarks that using their approach is common, %remarking that \emph{ The only slightly non-standard (meaning ``not often used") feature is the differing residual %variances between methods }\citep{bxc2010}.


\section{Roy's Hypotheses Tests}

In order to express Roy's LME model in matrix notation we gather all $2n_i$ observations specific to item $i$ into a single vector  $\boldsymbol{y}_{i} = (y_{1i1},y_{2i1},y_{1i2},\ldots,y_{mir},\ldots,y_{1in_{i}},y_{2in_{i}})^\prime.$ The LME model can be written
\[
\boldsymbol{y_{i}} = \boldsymbol{X_{i}\beta} + \boldsymbol{Z_{i}b_{i}} + \boldsymbol{\epsilon_{i}},
\]
where $\boldsymbol{\beta}=(\beta_0,\beta_1,\beta_2)^\prime$ is a vector of fixed effects, and $\boldsymbol{X}_i$ is a corresponding $2n_i\times 3$ design matrix for the fixed effects. The random effects are expressed in the vector $\boldsymbol{b}=(b_1,b_2)^\prime$, with $\boldsymbol{Z}_i$ the corresponding $2n_i\times 2$ design matrix. The vector $\boldsymbol{\epsilon}_i$ is a $2n_i\times 1$ vector of residual terms. Random effects and residuals are assumed to be independent of each other.

The random effects are assumed to be distributed as $\boldsymbol{b}_i \sim \mathcal{N}_2(0,\boldsymbol{G})$. The between-item variance covariance matrix $\boldsymbol{G}$ is constructed as follows:
\[ \boldsymbol{G} =\left(
            \begin{array}{cc}
              g^2_1  & g_{12} \\
              g_{12} & g^2_2 \\
            \end{array}
          \right) \]

The matrix of random errors $\boldsymbol{\epsilon}_i$ is distributed as $\mathcal{N}_2(0,\boldsymbol{R}_i)$.
\citet{hamlett} shows that the variance covariance matrix for the residuals(i.e. the within-item sources of variation between both methods) can be expressed as the Kroneckor product of an $n_i \times n_i$ identity matrix and the partial within-item variance covariance matrix $\boldsymbol{\Sigma}$, i.e. $\boldsymbol{R}_{i} = \boldsymbol{I}_{n_{i}} \otimes \boldsymbol{\Sigma}$.
\[
\boldsymbol{\Sigma} = \left( \begin{array}{cc}
  \sigma^2_{1} & \sigma_{12} \\
  \sigma_{12} & \sigma^2_{2} \\
\end{array}\right),
\]
where $\sigma^2_{1}$ and $\sigma^2_{2}$ are the within-subject variances of the respective methods, and $\sigma_{12}$ is the within-item covariance between the two methods. The within-item variance covariance matrix $\boldsymbol{\Sigma}$ is assumed to be the same for all replications.Computational analysis of linear mixed effects models allow for the explicit analysis of both $\boldsymbol{G}$ and $\boldsymbol{R_i}$.

For expository purposes consider the case where each item provides three replicate measurements by each method. In matrix form the model has the structure
\[
\boldsymbol{y}_{i} =
%---Design Matrix X ----%
\left(\begin{array}{ccc}
 1 & 1 & 0 \\ 1 & 0 & 1 \\ 1 & 1 & 0 \\
 1 & 0 & 1 \\ 1 & 1 & 0 \\ 1 & 0 & 1 \\
\end{array}\right)
%---FE Matrix----%
\left(\begin{array}{c}
 \beta_0 \\ \beta_1 \\ \beta_2 \\
\end{array}\right)
+
%---Design Matrix Z----%
\left(\begin{array}{cc}
1 & 0 \\0 & 1 \\1 & 0 \\0 & 1 \\0 & 1 \\\end{array}
\right)
%---RE Matrix----%
\left(\begin{array}{c}
b_{1i} \\   b_{2i} \\
\end{array}\right)
+
%------Errors Vector---%
\left( \begin{array}{c}
\epsilon_{1i1} \\\epsilon_{2i1} \\\epsilon_{1i2} \\ \epsilon_{2i2} \\\epsilon_{1i3} \\\epsilon_{2i3} \\
\end{array}\right).
\]
The between item variance covariance $\boldsymbol{G}$ is as before, while the within item variance covariance is given as
%------Specification of within item VC matrix R---%
\[
\boldsymbol{R}_i = \left(
\begin{array}{cccccc}
  \sigma^2_{1} & \sigma_{12} & 0 & 0 & 0 & 0 \\
  \sigma_{12} & \sigma^2_{2} & 0 & 0 & 0 & 0 \\
  0 & 0 & \sigma^2_{1} & \sigma_{12} & 0 & 0 \\
  0 & 0 & \sigma_{12} & \sigma^2_{2} & 0 & 0 \\
  0 & 0 & 0 & 0 & \sigma^2_{1} & \sigma_{12} \\
  0 & 0 & 0 & 0 & \sigma_{12} & \sigma^2_{2} \\
\end{array} \right)
\]

The overall variability between the two methods is the sum of between-item variability
$\boldsymbol{G}$ and partial within-item variability $\boldsymbol{\Sigma}$. \citet{roy} denotes the overall variability as ${\mbox{Block - }\boldsymbol \Omega_{i}}$. The overall variation for methods $1$ and $2$ are given by

%------Overall variability in terms of G and R ----%
\begin{equation}
\left(\begin{array}{cc}
              \omega^2_1  & \omega_{12} \\
              \omega_{12} & \omega^2_2 \\
       \end{array}  \right)
 =
\left(\begin{array}{cc}
              g^2_1  & g_{12} \\
              g_{12} & g^2_2 \\
\end{array} \right)
+
\left( \begin{array}{cc}
              \sigma^2_1  & \sigma_{12} \\
              \sigma_{12} & \sigma^2_2 \\
\end{array}\right)
\end{equation}


\subsection{Note 2: Carstensen model in the single measurement case}
\citet{BXC2004} presents a model to describe the relationship between a value of measurement and its real value.
The non-replicate case is considered first, as it is the context of the Bland-Altman plots.
This model assumes that inter-method bias is the only difference between the two methods.


\begin{equation}
y_{mi}  = \alpha_{m} + \mu_{i} + e_{mi} \qquad  e_{mi} \sim \mathcal{N}(0,\sigma^{2}_{m})
\end{equation}

The differences are expressed as $d_{i} = y_{1i} - y_{2i}$.

For the replicate case, an interaction term $c$ is added to the model, with an associated variance component.




\subsection{Note 3: Model terms}
It is important to note the following characteristics of this model.
\begin{itemize}
\item Let the number of replicate measurements on each item $i$ for both methods be $n_i$, hence $2 \times n_i$ responses. However, it is assumed that there may be a different number of replicates made for different items. Let the maximum number of replicates be $p$. An item will have up to $2p$ measurements, i.e. $\max(n_{i}) = 2p$.

% \item $\boldsymbol{y}_i$ is the $2n_i \times 1$ response vector for measurements on the $i-$th item.
% \item $\boldsymbol{X}_i$ is the $2n_i \times  3$ model matrix for the fixed effects for observations on item $i$.
% \item $\boldsymbol{\beta}$ is the $3 \times  1$ vector of fixed-effect coefficients, one for the true value for item $i$, and one effect each for both methods.

\item Later on $\boldsymbol{X}_i$ will be reduced to a $2 \times 1$ matrix, to allow estimation of terms. This is due to a shortage of rank. The fixed effects vector can be modified accordingly.
\item $\boldsymbol{Z}_i$ is the $2n_i \times  2$ model matrix for the random effects for measurement methods on item $i$.
\item $\boldsymbol{b}_i$ is the $2 \times  1$ vector of random-effect coefficients on item $i$, one for each method.
\item $\boldsymbol{\epsilon}$  is the $2n_i \times  1$ vector of residuals for measurements on item $i$.
\item $\boldsymbol{G}$ is the $2 \times  2$ covariance matrix for the random effects.
\item $\boldsymbol{R}_i$ is the $2n_i \times  2n_i$ covariance matrix for the residuals on item $i$.
\item The expected value is given as $\mbox{E}(\boldsymbol{y}_i) = \boldsymbol{X}_i\boldsymbol{\beta}.$ \citep{hamlett}
\item The variance of the response vector is given by $\mbox{Var}(\boldsymbol{y}_i)  = \boldsymbol{Z}_i \boldsymbol{G} \boldsymbol{Z}_i^{\prime} + \boldsymbol{R}_i$ \citep{hamlett}.
\end{itemize}
\newpage

%\chapter{Limits of Agreement}

%\section{Modelling Agreement with LME Models}

% Carstensen pages 22-23


Roys uses and LME model approach to provide a set of formal tests for method comparison studies.\\

Four candidates models are fitted to the data.\\

These models are similar to one another, but for the imposition of equality constraints.\\

These tests are the pairwise comparison of candidate models, one formulated without constraints, the other with a constraint.\\


Roy's model uses fixed effects $\beta_0 + \beta_1$ and $\beta_0 + \beta_1$ to specify the mean of all observationsby \\ methods 1 and 2 respectively.


Roy adheres to Random Effect ideas in ANOVA. Roy treats items as a sample from a population.\\

Allocation of fixed effects and random effects are very different in each model\\

Carstensen's interest lies in the difference between the population from which they were drawn.\\

Carstensen's model is a mixed effects ANOVA.\\

This model includes a method by item interaction term.\\

Carstensen presents two models. One for the case where the replicates, and a second for when they are linked.\\
Carstensen's model does not take into account either between-item or within-item covariance between methods.\\
In the presented example, it is shown that Roy's LoAs are lower than those of Carstensen.


\[\left(\begin{array}{cc}
                \omega^1_2  & 0 \\
              0 & \omega^2_2 \\
            \end{array}  \right)
            =  \left(
            \begin{array}{cc}
              \tau^2  & 0 \\
              0 & \tau^2 \\
            \end{array} \right)+
            \left(
            \begin{array}{cc}
              \sigma^2_1  & 0 \\
              0 & \sigma^2_2 \\
            \end{array}\right)
\]


\newpage
\bibliography{DB-txfrbib}
\end{document}
