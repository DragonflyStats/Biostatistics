
\noindent \textbf{Diagnostics for conformity of paired quantitative measurements}

\subsection*{Abstract}
\begin{itemize}
\item Matched pairs data arise in many contexts – in case-control clinical trials, for example, and from cross-over designs. They also arise in experiments to verify the equivalence of quantitative assays. This latter use (which is the main focus of this paper) raises difficulties not always seen in other matched pairs applications. 

\item Since the designs deliberately vary the analyte levels over a wide range, issues of variance dependent on mean, calibrations of differing slopes, and curvature all need to be added to the usual model assumptions such as normality. 

\item Violations in any of these assumptions invalidate the conventional matched pairs analysis. 

\item A graphical method, due to Bland and Altman, of looking at the relationship between the average and the difference of the members of the pairs is shown to correspond to a formal testable regression model. 

\item Using standard regression diagnostics, one may detect and diagnose departures from the model assumptions and remedy them – for example using variable transformations. Examples of different common scenarios and possible approaches to handling them are shown.
\end{itemize}
%====================================%

% 1. Problem Description
% 2. A Statistical Formulation
%    2.1 Regression Methods for Testing Equivalence
%    2.2 Matched Pair Analysis
% 3. Regression of Differences on Sums
%    3.1 Use of Formal regression Diagnostics
% 4. Examples
%    4.1 Example 1
%    4.2 Detection and Cure of Heteroscedascity
%    4.3 Example 2
%    4.4 Example 3
%    4.5 Example 4
%    4.6 Example 5
% 5. Precepts for Design
% 6. Conclusion


%====================================%
