Assessing equivalence of two assays using sensitivity and specificity.
Quiroz J1, Burdick RK.

% http://www.ncbi.nlm.nih.gov/pubmed/17479392
%----------------------------------------------------------------------------%

The equivalence of two assays is determined using the sensitivity and specificity relative to a gold standard.
The equivalence-testing criterion is based on a misclassification rate proposed by Burdick et al. (2005) and
the intersection-union test (IUT) method proposed by Berger (1982). 

Using a variance components model and IUT methods, we construct bounds for the sensitivity and specificity 
relative to the gold standard assay based on generalized confidence intervals. We conduct a simulation study 
to assess whether the bounds maintain the stated test size. 

We present a computational example to demonstrate the method described in the paper.

%----------------------------------------------------------------------------%
Assessing equivalence of two assays using sensitivity and specificity.
Quiroz J1, Burdick RK.

% http://www.ncbi.nlm.nih.gov/pubmed/17479392
%----------------------------------------------------------------------------%

The equivalence of two assays is determined using the sensitivity and specificity relative to a gold standard.
The equivalence-testing criterion is based on a misclassification rate proposed by Burdick et al. (2005) and
the intersection-union test (IUT) method proposed by Berger (1982). 

Using a variance components model and IUT methods, we construct bounds for the sensitivity and specificity 
relative to the gold standard assay based on generalized confidence intervals. We conduct a simulation study 
to assess whether the bounds maintain the stated test size. 

We present a computational example to demonstrate the method described in the paper.

%----------------------------------------------------------------------------%
