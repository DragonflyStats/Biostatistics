\documentclass[12pt, a4paper]{article}

\usepackage{amssymb}
\usepackage{amsbsy}
\usepackage{amsthm, amsmath}


\begin{document}
\author{Kevin O'Brien}
\title{Implementation of Model Diagnostics for LME models with \texttt{R}}

%====================================================================%

\section{Introduction}

Outliers and detection of influent observations is an important step in the analysis of a data set. There are several ways of evaluating the influence of perturbations in the data set and in the model given the parameter estimates. 

\subsection{Overview of R implementations}
Further to previous material, an appraisal of the current state of development (or lack thereof) for current implemenations for LME models, particularly for \texttt{nlme} and \texttt{lme4} fitted models.

Crucially, a review of internet resources indicates that almost all of the progress in this regard has been done for \texttt{lme4} fitted models, specifically the \textit{Influence.ME} \texttt{R} package. (Nieuwenhuis et al 2014)
Conversely there is very little for \texttt{nlme} models. One would immediately look at the current development workflow for both packages.
 
%======================%
% Douglas Bates

As an aside, Douglas Bates was arguably the most prominent \texttt{R} developer working in the LME area. 
However Bates has now prioritised the development of LME models in another computing environment , i.e Julia. 
% The current version of this is XXXX

%======================%
% nlme

With regards to \texttt{nlme}, the package is now maintained by the \texttt{R} core development team. The most recent major text is by Galecki \& Burzykowski, who have published \textit{ Linear Mixed Effects Models using \texttt{R}. }
Also, the accompanying \texttt{R} package, nlmeU package is under current development, with a version being released $0.70-3$.


%======================%
% lme4 and influence.ME

The \textbf{lme4} pacakge is used to fit linear and generalized linear mixed-effects models in the R environment.
The \textbf{lme4} package is also under active development, under the leadership of Ben Bolker (McMaster Uni., Canada).


%=====================%
\subsection*{Important Consideration for MCS}

The key issue is that \texttt{nlme} allows for the particular specification of Roy's Model, speciifically direct specification of the VC matrices for within subject and between subject residuals.
The \texttt{lme4} package does not allow for Roy's Model, for reasons that will identified shortly.
To advance the ideas that eminate from Roys' paper, one is required to use the \texttt{nlme} context. However, to take advantage of the infrastructure already provided for \texttt{lme4} models, one may change the research question away from that of Roy's paper. 
To this end, an exploration of what textbf{influence.ME} can accomplished is merited.

%==========================================================================%
\newpage
\section*{Influence Analysis}
The basic rationale behind measuring influential cases is that when iteratively single units are omitted
from the data, models based on these data should not produce substantially different estimates.
%===================================================================== %
\subsection*{Well Known Influence Measures}

\textit{``Regression Diagnostics: Identifying Influential Data
and Source of Collinearity (1980)"} by Belsley,Kuh,\& Welsch is a landmark text in the field of residual diagnostics, and
provides a foundation for much of the subsequent work.

\begin{description}
\item[Cook's Distance] Cook’s Distance is a measure indicating to what extent model parameters are influenced by (a set
of) influential data on which the model is based.
\item[DFBETAS] DFBETAS (standardized difference of the beta) is a measure that standardizes the absolute difference
in parameter estimates between a (mixed effects) regression model based on a full set of
data, and a model from which a (potentially influential) subset of data is removed. A value for
DFBETAS is calculated for each parameter in the model separately.
\end{description}


%================================================================== %
\newpage
\section*{Influence Measures with \textit{influence.ME}}influence.ME calculates measures of influence for mixed effects models estimated with the lme4 R package. The
basic rationale behind measuring influential cases is that when iteratively single units are omitted

from the data, models based on these data should not produce substantially different estimates. 

Calculating measures of influential data for an LME model requires the re-estimation
of this model for each set of potentially influential data separately. The \texttt{estex(} function does this,
and returns the altered estimates resulting from each re-estimation. 

The main function in the influence.ME package is the \texttt{influence()}.

\begin{quote} Based on a priorly estimated
mixed effects regression model (estimated using lme4), the \texttt{influence()} function iteratively modifies
the mixed effects model to neutralize the effect a grouped set of data has on the parameters, and
which returns returns the fixed parameters of these iteratively modified models. These are used to
compute measures of influential data. ()
\end{quote}

\subsection*{Using the influence.ME package}
Influence Analysis can onnly be carried out with LME models fitted using the functions in the \textbf{lme4} package. Such models are known as \texttt{mer} objects.
Hence the \texttt{estex()} function only works on LME
models of class \texttt{mer}.
The package developers advise that it is required that the \texttt{mer} model was estimated using a factor variable to indicate group levels.
When using something similar to \texttt{+ (1 | as.factor(variable))}, the function is not able of
identifying the correct grouping factors, and returns an error.

Executing this procedure can be computationally highly demanding, because \texttt{estex()} entails the re-estimation of the provided mixed effects model for each level of the specified grouping factor (after alteration of the data).
%============================================================================= %
\subsection*{Functionality of the influence.ME pacakge}
To standardize the assessment of how influential an observation (or group of observations)is, several measures
of influence are used by influence.ME.


\begin{itemize}
\item DFBETAS is a standardized measure of the absolute difference
between the estimate with a particular case included and the estimate without that particular
case. 
\item Cook’s distance provides an overall measurement of the change in all parameter
estimates, or a selection thereof.
\end{itemize}

The \texttt{estex()} command computes revised estimates can subsequently
be entered to the \texttt{cooks.distance} and \texttt{dfbetas} commands, to calculate Cook’s Distance
and the DFBETAS (standardized difference of the beta) measures.
%--------------------------------------------------------------%
\subsubsection*{The \texttt{pchange} command}

The \texttt{pchange} command computes the percentile change, as a measure of influential data. This unstandardized measure can
serve to help interpret the magnitude of the influence single or combined grouping levels exert on
mixed effects models. 

The percentage change in parameter estimates between an LME model based on a full set of data, and a model from which a (potentially influential)
subset of data is removed. A value of percentage change is calculated for each parameter in the
model separately, based on the information returned by the \texttt{estex()} function.

\subsection*{sigtest}

The \texttt{sigTest} function can test for changes in the level of statistical significance resulting from
the deletion of potentially influential observations


\subsubsection*{The \texttt{plot.estex} command}

%=====================================================================================================%

\newpage
%====================%
% Diagnostics with nlmeU

\section*{Leave-One-Out Diagnostics with \texttt{lmeU}}
Galecki et al provide a brief the matter of LME influence diagnostics in their book.

The command \texttt{lmeU} fits a model with a particular subject removed. The identifier of the subject to be removed is passed as the only argument

A plot ofthe per-observation diagnostics individual subject log-likelihood contributions can be rendered.

\subsubsection*{Likelihood Displacement}

\end{document}
