
\documentclass[12pt, a4paper]{article}
\usepackage{epsfig}
\usepackage{subfigure}
%\usepackage{amscd}
\usepackage{amssymb}
\usepackage{amsbsy}
\usepackage{amsthm, amsmath}
%\usepackage[dvips]{graphicx}
\usepackage{natbib}
\bibliographystyle{chicago}
\usepackage{vmargin}
% left top textwidth textheight headheight
% headsep footheight footskip
\setmargins{3.0cm}{2.5cm}{15.5 cm}{22cm}{0.5cm}{0cm}{1cm}{1cm}
\renewcommand{\baselinestretch}{1.5}
\pagenumbering{arabic}
\theoremstyle{plain}
\newtheorem{theorem}{Theorem}[section]
\newtheorem{corollary}[theorem]{Corollary}
\newtheorem{ill}[theorem]{Example}
\newtheorem{lemma}[theorem]{Lemma}
\newtheorem{proposition}[theorem]{Proposition}
\newtheorem{conjecture}[theorem]{Conjecture}
\newtheorem{axiom}{Axiom}
\theoremstyle{definition}
\newtheorem{definition}{Definition}[section]
\newtheorem{notation}{Notation}
\theoremstyle{remark}
\newtheorem{remark}{Remark}[section]
\newtheorem{example}{Example}[section]
\renewcommand{\thenotation}{}
\renewcommand{\thetable}{\thesection.\arabic{table}}
\renewcommand{\thefigure}{\thesection.\arabic{figure}}
\title{Research notes: linear mixed effects models}
\author{ } \date{ }


\begin{document}
\author{Kevin O'Brien}
\title{Updating techniques for LME models}

\addcontentsline{toc}{section}{Bibliography}

%----------------------------------------------------------------------------------------%
\newpage
%\chapter{Limits of Agreement}

%\section{Modelling Agreement with LME Models}

Roys uses and LME model approach to provide a set of fromal tests for method comparison studies.\\

Four candidates models are fitted to the data.\\

These models are similar to one another, but for the imposition of equality constraints.\\

These tests are the pairwise comparison of candidate models, one formulated without constraints, the other with a constraint.\\

Carstensen also uses a LME model for examining MCS with replicates.\\

Roy's model uses fixed effects $\beta_0 + \beta_1$ and $\beta_0 + \beta_1$ to specify the mean of all observationsby \\ methods 1 and 2 respectuively.

Carstensen allocates a fixed, but unknown, mean for each individual. [Grubbs(1948) model.]\\

His interest lies in calculating the LoA as opposed to Formalized testing.

Roy adheres to Random Effect ideas in ANOVA

Roy treats items as a sample from a population.\\

Allocation of fixed effects and random effects are very different in each model\\

Carstensen's interest lies in the difference between the population from which they were drawn.\\

Carstensen's model is a mixed effects ANOVA.\\

\[
Y_{mir}  =  \alpha_m + \mu_i + c_{mi} + e_{mir}, \qquad c_{mi} \sim \mathcal{\tau^2_m}, \qquad e_{mir} \sim \mathcal{\sigma^2_m},
\]

This model includes a method by item iteration term.\\

Carstensen presents two models. One for the case where the replicates, and a second for when they are linked.\\

Carstensen's model does not take into account either between-item or within-item covariance between methods.\\

Importantly, estimates required to calculate the limits of agreement are not extractable, and therefore the calculation must be done by hand.

In cases where there is negligible covariance between methods, the limits of agreement computed using Roy's model accord with those computed using Carstensen's model.
In cases where some degree of covariance is present between the two methods, the limits of agreement computed using models will differ.

In the presented example, it is shown that Roy's LoAs are lower than those of Carstensen.



Carstensen makes some interesting remarks in this regard.

\begin{quote}
The only slighlty non-standard (meaning "not often used") feature is the differing residual variances between methods.
\end{quote}






\newpage
\subsection{Remarks on the Multivariate Normal Distribution}

The multivariate normal distribution of a $k$-dimensional random vector $X = [X_1, X_2, \ldots, X_k]$
can be written in the following notation:
\[
    X\ \sim\ \mathcal{N}(\mu,\, \Sigma),
\]
or to make it explicitly known that $X$ is $k$-dimensional,
\[
    X\ \sim\ \mathcal{N}_k(\mu,\, \Sigma).
\]
with $k$-dimensional mean vector
\[ \mu = [ \operatorname{E}[X_1], \operatorname{E}[X_2], \ldots, \operatorname{E}[X_k]] \]
and $k \times k$ covariance matrix
\[ \Sigma = [\operatorname{Cov}[X_i, X_j]], \; i=1,2,\ldots,k; \; j=1,2,\ldots,k \]

\bigskip

\begin{enumerate}
\item Univariate Normal Distribution

\[
    X\ \sim\ \mathcal{N}(\mu,\, \sigma^2),
\]

\item Bivariate Normal Distribution

\begin{itemize}
\item[(a)] \[  X\ \sim\ \mathcal{N}_2(\mu,\, \Sigma), \vspace{1cm}\]
\item[(b)] \[    \mu = \begin{pmatrix} \mu_x \\ \mu_y \end{pmatrix}, \quad
    \Sigma = \begin{pmatrix} \sigma_x^2 & \rho \sigma_x \sigma_y \\
                             \rho \sigma_x \sigma_y  & \sigma_y^2 \end{pmatrix}.\]
\end{itemize}
\end{enumerate}

\bigskip

Diligence is required when considering the models. Carstensen specifies his models in terms of the univariate normal distribution. Roy's model is specified using the bivariate normal distribution.
This gives rises to a key difference between the two model, in that a bivariate model accounts for covariance between the variables of interest.

\bigskip




\newpage


\bigskip

Let $y_{mir} $ be the $r$th replicate measurement on the $i$th item by the $m$th method, where $m=1,2,$ $i=1,\ldots,N,$ and $r = 1,\ldots,n_i.$ When the design is balanced and there is no ambiguity we can set $n_i=n.$ The LME model can be written
\begin{equation}
y_{mir} = \beta_{0} + \beta_{m} + b_{mi} + \epsilon_{mir}.
\end{equation}
Here $\beta_0$ and $\beta_m$ are fixed-effect terms representing, respectively, a model intercept and an overall effect for method $m.$ The $b_{1i}$ and $b_{2i}$ terms represent random effect parameters corresponding to the two methods, having $\mathrm{E}(b_{mi})=0$ with $\mathrm{Var}(b_{mi})=g^2_m$ and $\mathrm{Cov}(b_{mi}, b_{m^\prime i})=g_{12}.$ The random error term for each response is denoted $\epsilon_{mir}$ having $\mathrm{E}(\epsilon_{mir})=0$, $\mathrm{Var}(\epsilon_{mir})=\sigma^2_m$, $\mathrm{Cov}(b_{mir}, b_{m^\prime ir})=\sigma_{12}$, $\mathrm{Cov}(\epsilon_{mir}, \epsilon_{mir^\prime})= 0$ and $\mathrm{Cov}(\epsilon_{mir}, \epsilon_{m^\prime ir^\prime})= 0.$
When two methods of measurement are in agreement, there is no significant differences between $\beta_1$ and $\beta_2,$ $g^2_1 $ and$ g^2_2$, and $\sigma^2_1 $ and$ \sigma^2_2$.
\bigskip

% Complete paragraph by specifying variances and covariances for epsilons.
% I thing that these are your sigmas?
% Also, state equality of the parameters in this model when each of the three hypotheses above are true.




$\boldsymbol{G}$ is the variance covariance matrix for the random effects $\boldsymbol{b}$.
i.e. between-item sources of variation. The between-item variance covariance matrix $\boldsymbol{G}$ is constructed as follows:

\[ \mbox{Var}  \left[
            \begin{array}{c}
              b_1   \\
              b_2  \\
            \end{array}
          \right] =  \boldsymbol{G} =\left(
            \begin{array}{cc}
              g^2_1  & g_{12} \\
              g_{12} & g^2_2 \\
            \end{array}
          \right) \]
It is important to note that no special assumptions about the structure of $\boldsymbol{G}$ are made. An example of such an assumption would be that $\boldsymbol{G}$ is the product of a scalar value and the identity matrix.

$\boldsymbol{R}_{i}$ is the variance covariance matrix for the residuals, i.e. the within-item sources of variation between both methods. Computational analysis of linear mixed effects models allow for the explicit analysis of both $\boldsymbol{G}$ and $\boldsymbol{R_i}$.

\citet{hamlett} shows that $\boldsymbol{R}_{i}$  can be expressed as $\boldsymbol{R}_{i} = \boldsymbol{I}_{n_{i}} \otimes \boldsymbol{\Sigma}$. The partial within-item variance?covariance matrix of two methods at any replicate is denoted $\boldsymbol{\Sigma}$, where $\sigma^2_{1}$ and $\sigma^2_{2}$ are the within-subject variances of the respective methods, and $\sigma_{12}$ is the within-item covariance between the two methods. It is assumed that the within-item variance?covariance matrix $\boldsymbol{\Sigma}$ is the same for all replications. Again it is important to note that no special assumptions are made about the structure of the matrix.



It is assumed that $\boldsymbol{b}_i \sim N(0,\boldsymbol{G})$,
$\boldsymbol{\epsilon}_i$ is a matrix of random errors distributed as $N(0,\boldsymbol{R}_i)$ and
that the random effects and residuals are independent of each other. Assumptions made on the structures of $\boldsymbol{G}$ and $\boldsymbol{R}_i$ will be discussed in due course.

\newpage





The overall variability between
the two methods is the sum of between-item variability
$\boldsymbol{G}$ and within-item variability
$\boldsymbol{\Sigma}$. \citet{roy} denotes the overall variability
as ${\mbox{Block - }\boldsymbol \Omega_{i}}$. The overall
variation for methods $1$ and $2$ are given by

\begin{center}
\[\left(\begin{array}{cc}
                \omega^2_1  & \omega_{12} \\
              \omega_{12} & \omega^2_2 \\
            \end{array}  \right)
            =  \left(
            \begin{array}{cc}
              g^2_1  & g_{12} \\
              g_{12} & g^2_2 \\
            \end{array} \right)+
            \left(
            \begin{array}{cc}
              \sigma^2_1  & \sigma_{12} \\
              \sigma_{12} & \sigma^2_2 \\
            \end{array}\right)
\]
\end{center}
The computation of the limits of agreement require that the variance of the difference of measurements. This variance is easily computable from the estimate of the ${\mbox{Block - }\boldsymbol \Omega_{i}}$ matrix. Lack of agreement can arise if there is a disagreement in overall variabilities. This may be due to due to the disagreement in either between-item
variabilities or within-item variabilities, or both. \citet{roy} allows for a formal test of each.




Of particular importance is terms of the model, a true value for item $i$ ($\mu_{i}$).  The fixed effect of Roy's model comprise of an intercept term and fixed effect terms for both methods, with no reference to the true value of any individual item. A distinction can be made between the two models: Roy's model is a standard LME model, whereas Carstensen's model is a more complex additive model.


\subsection{Assumptions on Variability}

Aside from the fixed effects, another important difference is that Carstensen's model requires that particular assumptions be applied, specifically that the off-diagonal elements of the between-item
and within-item variability matrices are zero. By extension the
overall variability off diagonal elements are also zero.

Also, implementation requires that the between-item variances are
estimated as the same value: $g^2_1 = g^2_2 = g^2$. Necessarily
Carstensen's method does not allow for a formal test of the
between-item variability.

\[\left(\begin{array}{cc}
                \omega^1_2  & 0 \\
              0 & \omega^2_2 \\
            \end{array}  \right)
            =  \left(
            \begin{array}{cc}
              g^2  & 0 \\
              0 & g^2 \\
            \end{array} \right)+
            \left(
            \begin{array}{cc}
              \sigma^2_1  & 0 \\
              0 & \sigma^2_2 \\
            \end{array}\right)
\]

In cases where the off-diagonal terms in the overall variability
matrix are close to zero, the limits of agreement due to
\citet{bxc2008} are very similar to the limits of agreement that
follow from the general model.

\newpage



\end{itemize}
\bibliography{DB-txfrbib}
\end{document}
