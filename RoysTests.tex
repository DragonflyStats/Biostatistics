\newpage
\subsection{Roy's hypothesis tests for variability}
% Three hypothesis tests follow from this equation.
Lack of agreement can arise if there is a disagreement in overall variabilities. This lack of agreement may be due to differing between-item variabilities, differing within-item variabilities, or both. The formulation presented above usefully facilitates a series of significance tests that assess if and where such differences arise. \citet{roy} allows for a formal test of each. These tests are comprised of a formal test for the equality of between-item variances,
\begin{eqnarray*}
\operatorname{H_0} : g^2_1 = g^2_2 \\
\operatorname{H_1} : g^2_1 \neq g^2_2
\end{eqnarray*}
a formal test for the equality of within-item variances,
\begin{eqnarray*}
\operatorname{H_0} : \sigma^2_1 = \sigma^2_2 \\
\operatorname{H_1} : \sigma^2_1 \neq \sigma^2_2
\end{eqnarray*}
and finally, a formal test for the equality of overall variances.
\begin{eqnarray*}
\operatorname{H_0} : \omega^2_1 = \omega^2_2 \\
\operatorname{H_1} : \omega^2_1 \neq \omega^2_2
\end{eqnarray*}

These tests are complemented by the ability to consider the inter-method bias and the overall correlation coefficient.
Two methods can be considered to be in agreement if criteria based upon these methodologies are met. Additionally Roy makes reference to the overall correlation coefficient of the two methods, which is determinable from variance estimates.
%------------------------------------------------------------------------%
\newpage
\subsection{Roy's hypothesis tests for variability}
% Three hypothesis tests follow from this equation.
Lack of agreement can arise if there is a disagreement in overall variabilities. This lack of agreement may be due to differing between-item variabilities, differing within-item variabilities, or both. The formulation presented above usefully facilitates a series of significance tests that assess if and where such differences arise. \citet{roy} allows for a formal test of each. These tests are comprised of a formal test for the equality of between-item variances,
\begin{eqnarray*}
	\operatorname{H_2} : g^2_1 = g^2_2 \\
	\operatorname{K_2} : g^2_1 \neq g^2_2
\end{eqnarray*}
and a formal test for the equality of within-item variances.
\begin{eqnarray*}
	\operatorname{H_3} : \sigma^2_1 = \sigma^2_2 \\
	\operatorname{K_3} : \sigma^2_1 \neq \sigma^2_2
\end{eqnarray*}
A formal test for the equality of overall variances is also presented.
\begin{eqnarray*}
	\operatorname{H_4} : \omega^2_1 = \omega^2_2 \\
	\operatorname{K_4} : \omega^2_1 \neq \omega^2_2
\end{eqnarray*}

These tests are complemented by the ability to consider the inter-method bias and the overall correlation coefficient.
Two methods can be considered to be in agreement if criteria based upon these methodologies are met. Additionally Roy makes reference to the overall correlation coefficient of the two methods, which is determinable from variance estimates.

%------------------------------------------------------------------------%
\newpage
\subsection{Roy's hypothesis tests for variability}
% Three hypothesis tests follow from this equation.
Lack of agreement can arise if there is a disagreement in overall variabilities. This may be due to due to the disagreement in either between-item variabilities or within-item variabilities, or both. The formulation presented above usefully facilitates a series of significance tests that advise as to how well the two methods agree. \citet{roy} allows for a formal test of each. These tests are comprised of a formal test for the equality of between-item variances,
\begin{eqnarray*}
	\operatorname{H_0} : g^2_1 = g^2_2 \\
	\operatorname{H_1} : g^2_1 \neq g^2_2
\end{eqnarray*}
a formal test for the equality of within-item variances,
\begin{eqnarray*}
	\operatorname{H_0} : \sigma^2_1 = \sigma^2_2 \\
	\operatorname{H_1} : \sigma^2_1 \neq \sigma^2_2
\end{eqnarray*}
and finally, a formal test for the equality of overall variances.
\begin{eqnarray*}
	\operatorname{H_0} : \omega^2_1 = \omega^2_2 \\
	\operatorname{H_1} : \omega^2_1 \neq \omega^2_2
\end{eqnarray*}

These tests are complemented by the ability to consider the inter-method bias and the overall correlation coefficient.
Two methods can be considered to be in agreement if criteria based upon these methodologies are met. Additionally Roy makes reference to the overall correlation coefficient of the two methods, which is determinable from variance estimates.

%-----------------------------------------------------------------------------------%
%------------------------------------------------------------------------%
\newpage
\subsection{Roy's hypothesis tests for variability}
% Three hypothesis tests follow from this equation.
Lack of agreement can arise if there is a disagreement in overall variabilities. This lack of agreement may be due to differing between-item variabilities, differing within-item variabilities, or both. The formulation presented above usefully facilitates a series of significance tests that assess if and where such differences arise. \citet{roy} allows for a formal test of each. These tests are comprised of a formal test for the equality of between-item variances,
\begin{eqnarray*}
	\operatorname{H_2} : g^2_1 = g^2_2 \\
	\operatorname{K_2} : g^2_1 \neq g^2_2
\end{eqnarray*}
a formal test for the equality of within-item variances,
\begin{eqnarray*}
	\operatorname{H_3} : \sigma^2_1 = \sigma^2_2 \\
	\operatorname{K_3} : \sigma^2_1 \neq \sigma^2_2
\end{eqnarray*}
and finally, a formal test for the equality of overall variances.
\begin{eqnarray*}
	\operatorname{H_4} : \omega^2_1 = \omega^2_2 \\
	\operatorname{K_4} : \omega^2_1 \neq \omega^2_2
\end{eqnarray*}

These tests are complemented by the ability to consider the inter-method bias and the overall correlation coefficient.
Two methods can be considered to be in agreement if criteria based upon these methodologies are met. Additionally Roy makes reference to the overall correlation coefficient of the two methods, which is determinable from variance estimates.

