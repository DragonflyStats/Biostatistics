
\documentclass[MAIN.tex]{subfiles}

% Load any packages needed for this document
\begin{document}

%\section{Technology Acceptance Model}
\section{Prevalence of the Bland-Altman plot}
%---------------------------------------------%

\citet*{BA86}, which further develops the Bland-Altman methodology,
was found to be the sixth most cited paper of all time by the
\citet{BAcite}. \cite{Dewitte} describes the rate at which
prevalence of the Bland-Altman plot has developed in scientific
literature. \citet{Dewitte} reviewed the use of Bland-Altman plots
by examining all articles in the journal `Clinical Chemistry'
between 1995 and 2001. This study concluded that use of the
Bland-Altman plot increased over the years, from 8\% in 1995 to
14\% in 1996, and 31-36\% in 2002.

The Bland-Altman Plot has since become expected, and
often obligatory, approach for presenting method comparison
studies in many scientific journals \citep{hollis}. Furthermore
\citet{BritHypSoc} recommend its use in papers pertaining to
method comparison studies for the journal of the British
Hypertension Society.

%% TAM
\section{The Technology Acceptance Model}
Davis (1989) proposes the TAM model, which suggests an hypothesis as to why users may adopt particular technologies, and not others. 
According to this theory, when users are presented with a new 
technology, two important factors will influence their decision about how and when they will adopt it.
\begin{description}
	\item[Perceived usefulness (PU)] - This was defined by Fred Davis as "the degree to which a person believes that using a particular system would enhance his or her job performance".
	\item[Perceived ease-of-use (PEOU)] - Davis defined this as "the degree to which a person believes that using a particular system would be free from effort" 
\end{description}

Davis's explanations of these term can be rephrased for application to statistical analysis. 
Perceived Use could refer to the degree to which an user would deem a particular statistical method would properly establish the results of an analaysis. In the case of method comparison studies, proper indication of agreement, or lack thereof.


Perceived ease-of-use requires only applying the context of a satistical problem. A very modest statistical skill set is the only prerequistive for constructing a Bland-Altman plot, and computing limits of agreement. The main building blocks 
are simple descriptive, statistics and a knowledge of the normal distribution. These are topics that feature in almost every undergraduate statistics courses.

%---------------------------------------------%


A very modest statistical skill set is the only prerequistive for constructing a Bland-Altman plot, and computing limits of agreement. The main building blocks 
are simple descriptive, statistics and a knowledge of the normal distribution. These are topics that feature in almost every undergraduate statistics courses.

In short, the user perceives the Bland-Altman methodology to be an easy-to-implement technique, that will properly address the question of agreement.

Conversely the Survival plot is a derivative of the Kaplan-Meier Curve, a non-parametric graphical technique that features in Survival Analysis. This subject area is a well known domain of statistics, but would be encountered 
on curriculums of specialist courses. The Mountain Plot is formally called the empirical folder cumulative distribution plot. 
Currently there is only one software implementation , medcalc.be toolkot (FIX)

Conversely the Survival plot is a derivative of the Kaplan-Meier Curve, a non-parametric graphical technique that features in Survival Analysis. This subject area is a well known domain of statistics, but would be encountered 
on curriculums of specialist courses. 

The Mountain Plot is formally called the empirical folder cumulative distribution plot. While not particularly hard to render, the procedure is not straight-forward for the casual user. Currently there is only one software implementation , \textbf{\textit{medcalc.be}} toolkit.





The ROC curve is a plot that is commonly used in the appraisal of a statistical analytics systems. Interpretation of the plot, the nearer the curve is to the
top left corner of the plot, the better the statistical method is at making predicting outcomes.
\addcontentsline{toc}{section}{Bibliography}

%--------------------------------------------------------------------------------------%

\bibliographystyle{chicago}
\bibliography{DB-txfrbib}
\end{document}
