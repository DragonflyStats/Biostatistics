\documentclass[MAIN.tex]{subfiles}
\begin{document}
	\section{Roy's Approach}
	
	For the purposes of comparing two methods of measurement, \citet{ARoy2009} presents a methodology utilizing linear mixed effects model. This methodology provides for the formal testing of inter-method bias, between-subject variability and within-subject variability of two methods. 
	
	\citet{ARoy2009} uses an approach based on linear mixed effects (LME) models for the purpose of comparing the agreement between two methods of measurement, where replicate measurements on items (often individuals) by both methods are available. Three tests of hypothesis appropriate are provided for evaluating the agreement between the two methods of measurement under this sampling scheme. 
	
	Importantly \citet{ARoy2009} further proposes a series of three tests on the variance components of an LME model, which allow decisions on the second and third of Roy's criteria. For these tests, four candidate LME models are constructed. 
	
	\citet{ARoy2009} proposes the use of LME models to perform a test on two methods of agreement to comparing the agreement between two methods of measurement, where replicate measurements on items (often individuals) by both methods are available, determining whether they can be used
	interchangeably. The methodology proposed by \citet{ARoy2009} is largely based on \citet{hamlett}, which in turn follows on from \citet{lam}.
	
	\citet{ARoy2009} proposes a novel method using the LME model with Kronecker product covariance structure in a doubly multivariate set-up to assess the agreement between a new method and an established method with unbalanced data and with unequal replications for different subjects. By doubly multivariate set up, Roy means that the information on each patient or item is multivariate at two levels, the number of methods and number of replicated measurements. Further to \citet{lam}, it is assumed that the replicates are linked over time. However it is easy to modify to the unlinked case.
	
	
	The well-known ``Limits of Agreement", as developed by \citet{BA86} are not referred to directly, but are easily computable using the framework proposed by \citet{ARoy2009}. Further discussion will be provided in due course.	
	
	
	
	
	
	%Two methods of measurement can be said to be in agreement if there is no significant difference between in three key respects. 
	%
	%Firstly, there is no inter-method bias between the two methods, i.e. there is no persistent tendency for one method to give higher values than the other.
	%
	%Secondly, both methods of measurement have the same  within-subject variability. In such a case the variance of the replicate measurements would consistent for both methods.
	%Lastly, the methods have equal between-subject variability.  Put simply, for the mean measurements for each case, the variances of the mean measurements from both methods are equal.
	
	Further to this, \citet{ARoy2009} demonstrates an suite of tests that can be used to determine how well two methods of measurement, in the presence of repeated measures, agree with each other.
	
	\begin{itemize}\itemsep0.5cm
		\item No Significant inter-method bias
		\item No difference in the between-subject variabilities of the two methods
		\item No difference in the within-subject variabilities of the two methods
	\end{itemize}
	
	The formulation presented above usefully facilitates a series of
	significance tests that advise as to how well the two methods
	agree. These tests are as follows:
	\begin{itemize}
		\item A formal test for the equality of between-item variances,
		\item A formal test for the equality of within-item variances,
		\item A formal test for the equality of overall variances.
	\end{itemize}
	These tests are complemented by the ability to consider the inter-method bias and the overall correlation coefficient. Two methods can be considered to be in agreement if criteria based upon these methodologies are met. Additionally Roy makes reference to the overall correlation coefficient of the two methods, which is determinable from variance estimates. Other important aspects of the method comparison study are consequent. The limits of agreement are computed using the results of the reference model.
	
	Differences in between-subject variabilities of the two methods arise when one method is yielding average response levels for individuals than are more variable than the average response levels for the same sample of individuals taken by the other method.  Differences in within-subject variabilities of the two methods arise when one method is yielding responses for an individual than are more variable than the responses for this same individual taken by the other method. The two methods of measurement can be considered to agree, and subsequently can be used interchangeably, if all three null hypotheses are true.	
	
	
	\subsection{LME Model Specification}
	
	Let $y_{mir} $ denote the $r$th replicate measurement on the $i$th item by the $m$th method, where $m=1,2,$ $i=1,\ldots,N,$ and $r = 1,\ldots,n_i.$ When the design is balanced and there is no ambiguity we can set $n_i=n.$ The LME model underpinning Roy's approach can be written
	\begin{equation}\label{Roy-model}
		y_{mir} = \beta_{0} + \beta_{m} + b_{mi} + \epsilon_{mir}.
	\end{equation}
	
	Here $\beta_0$ and $\beta_m$ are fixed-effect terms representing, respectively, a model intercept and an overall effect for method $m.$ 
	
	The $b_{1i}$ and $b_{2i}$ terms represent random effect parameters corresponding to the two methods, having $\mathrm{E}(b_{mi})=0$ with $\mathrm{Var}(b_{mi})=g^2_m$ and $\mathrm{Cov}(b_{mi}, b_{m^\prime i})=d_{12}.$ The random error term for each response is denoted $\epsilon_{mir}$ having $\mathrm{E}(\epsilon_{mir})=0$, $\mathrm{Var}(\epsilon_{mir})=\sigma^2_m$, $\mathrm{Cov}(b_{mir}, b_{m^\prime ir})=\sigma_{12}$, $\mathrm{Cov}(\epsilon_{mir}, \epsilon_{mir^\prime})= 0$ and $\mathrm{Cov}(\epsilon_{mir}, \epsilon_{m^\prime ir^\prime})= 0.$
	
	
	When two methods of measurement are in agreement, there is no significant differences between $\beta_1$ and $\beta_2,$ $g^2_1 $ and$ g^2_2$, and $\sigma^2_1 $ and$ \sigma^2_2$.
	Here $\beta_0$ and $\beta_m$ are fixed-effect terms representing, respectively, a model intercept and an overall effect for method $m.$ 
	
	The model can be reparameterized by gathering the $\beta$ terms together into (fixed effect) intercept terms $\alpha_m=\beta_0+\beta_m.$ The $b_{1i}$ and $b_{2i}$ terms are correlated random effect parameters having $\mathrm{E}(b_{mi})=0$ with $\mathrm{Var}(b_{mi})=g^2_m$ and $\mathrm{Cov}(b_{1i}, b_{2 i})=d_{12}.$ 
	
	The random error term for each response is denoted $\epsilon_{mir}$ having $\mathrm{E}(\epsilon_{mir})=0$, $\mathrm{Var}(\epsilon_{mir})=\sigma^2_m$, $\mathrm{Cov}(\epsilon_{1ir}, \epsilon_{2 ir})=\sigma_{12}$, $\mathrm{Cov}(\epsilon_{mir}, \epsilon_{mir^\prime})= 0$ and $\mathrm{Cov}(\epsilon_{1ir}, \epsilon_{2 ir^\prime})= 0.$ Two methods of measurement are in complete agreement if the null hypotheses $\mathrm{H}_1\colon \alpha_1 = \alpha_2$ and $\mathrm{H}_2\colon \sigma^2_1 = \sigma^2_2 $ and $\mathrm{H}_3\colon g^2_1= g^2_2$ hold simultaneously. \citet{ARoy2009} uses a Bonferroni correction to control the familywise error rate for tests of $\{\mathrm{H}_1, \mathrm{H}_2, \mathrm{H}_3\}$ and account for difficulties arising due to multiple testing. 
	\subsection{Variance Covariance Matrices }
	
	Under Roy's model, random effects are defined using a bivariate normal distribution. Consequently, the variance-covariance structures can be described using $2 \times 2$  matrices. A discussion of the various structures a variance-covariance matrix can be specified under is required before progressing. The following structures are relevant: the identity structure, the compound symmetric structure and the symmetric structure.
	
	The differences in the models are specifically in how the the $D$ and $\Lambda$ matrices are constructed, using either an unstructured form or a compound symmetry form. To illustrate these differences, consider a generic matrix $A$,
	
	\[
	\boldsymbol{A} = \left( \begin{array}{cc}
	a_{11} & a_{12}  \\
	a_{21} & a_{22}  \\
	\end{array}\right).
	\]
	
	A symmetric matrix allows the diagonal terms $a_{11}$ and $a_{22}$ to differ. The compound symmetry structure requires that both of these terms be equal, i.e $a_{11} = a_{22}$.
	
	
	The identity structure is simply an abstraction of the identity matrix. The compound symmetric structure and symmetric structure can be described with reference to the following matrix (here in the context of the overall covariance Block-$\boldsymbol{\Omega}_i$, but equally applicable to the component variabilities $\boldsymbol{D}$ and $\boldsymbol{\Sigma}$);
	
	\[\left( \begin{array}{cc}
	\omega^2_1  & \omega_{12} \\
	\omega_{12} & \omega^2_2 \\
	\end{array}\right) \]
	
	Symmetric structure requires the equality of all the diagonal terms, hence $\omega^2_1 = \omega^2_2$. Conversely compound symmetry make no such constraint on the diagonal elements. Under the identity structure, $\omega_{12} = 0$.
	A comparison of a model fitted using symmetric structure with that of a model fitted using the compound symmetric structure is equivalent to a test of the equality of variance.
	
	
	%In the presented example, it is shown that Roy's LOAs are lower than those of (\ref{BXC-model}), when covariance between methods is present.
	
	
	
	\subsubsection{Independence}
	
	As though analyzed using between subjects analysis.
	\[
	\left(
	\begin{array}{c c c}
	\psi^2 & 0 & 0   \\
	0 & \psi^2 & 0   \\
	0 & 0 & \psi^2   \\
	\end{array}%
	\right)
	\]
	
	
	
	\subsubsection{Compound Symmetry}
	
	Assumes that the variance-covariance structure has a single variance (represented by $\psi^2$)
	for all 3 of the time points and a single covariance (represented by $\psi_{ij}$) for each of the pairs of trials.
	
	\[
	\left(%
	\begin{array}{c c c}
	\psi^2 &  \psi_{12} & \psi_{13}   \\
	\psi_{21} & \psi^2 & \psi_{23}   \\
	\psi_{31} & \psi_{32} & \psi^2   \\
	\end{array}%
	\right)
	\]
	
	
	%\subsubsection{Unstructured}
	%
	%Assumes that each variance and covariance is unique.
	%Each trial has its own variance (e.g. s12 is the variance of trial 1)
	%and each pair of trials has its own covariance (e.g. s21 is the covariance of trial 1 and trial2).
	%This structure is illustrated by the half matrix below.
	%
	%
	%
	%\subsubsection{Autoregressive}
	%
	%Another common covariance structure which is frequently observed
	%in repeated measures data is an autoregressive structure,
	%which recognizes that observations which are more proximate
	%are more correlated than measures that are more distant.
	%
	%
	
	
	
	
	
	
	%-----------------------------------------------------------------------------------------------------%
	
	
	
	
	
	\subsection{Model Terms (Roy 2009)}
	\begin{itemize}
		\item Let $y_{mir}$ be the response of method $m$ on the $i$th subject
		at the $r-$th replicate.
		\item Let $\boldsymbol{y}_{ir}$ be the $2 \times 1$ vector of measurements
		corresponding to the $i-$th subject at the $r-$th replicate.
		\item Let $\boldsymbol{y}_{i}$ be the $R_i \times 1$ vector of
		measurements corresponding to the $i-$th subject, where $R_i$ is number of replicate measurements taken on item $i$.
		\item Let $\alpha_mi$ be the fixed effect parameter for method for subject $i$.
		\item Formally ARoy2009 uses a separate fixed effect parameter to describe the true value $\mu_i$, but later combines it with the other fixed effects when implementing the model.
		\item Let $u_{1i}$ and $u_{2i}$ be the random effects corresponding to methods for item $i$.
		
		\item $\boldsymbol{\epsilon}_{i}$ is a $n_{i}$-dimensional vector
		comprised of residual components. For the blood pressure data $n_{i} = 85$.
		
		\item $\boldsymbol{\beta}$ is the solutions of the means of the two methods. In the LME output, the bias ad corresponding
		t-value and p-values are presented. This is relevant to ARoy2009's first test.\end{itemize}
	
	\section{Using LME for method comparison}
	Due to the prevalence of modern statistical software, \citet{BXC2008} advocates the adoption of computer based approaches, such as LME models, to method comparison studies. \citet{BXC2008} remarks upon `by-hand' approaches advocated in \citet{BA99} discouragingly, describing them as tedious, unnecessary and `outdated'. Rather than using the `by hand' methods, estimates for required LME parameters can be read directly from program output. Furthermore, using computer approaches removes constraints associated with `by-hand' approaches, such as the need for the design to be perfectly balanced.
	
	\subsection{Roy's Approach}
	
	For the purposes of comparing two methods of measurement, \citet{roy} presents a framework that utilizes linear mixed effects model. This methodology provides for the formal testing of inter-method bias, between-subject variability and within-subject variability of two methods. The formulation contains a Kronecker product covariance structure in a doubly multivariate setup. By doubly multivariate set up, Roy means that the information on each patient or item is multivariate at two levels, the number of methods and number of replicated measurements. Further to \citet{lam}, it is assumed that the replicates are linked over time. However it is easy to modify to the unlinked case.
	
	\citet{roy} sets out three criteria for two methods to be considered in agreement. Firstly that there be no significant bias. Second that there is no difference in the between-subject variabilities, and lastly that there is no significant difference in the within-subject variabilities. Roy further proposes examination of the the overall variability by considering the second and third criteria be examined jointly. Should both the second and third criteria be fulfilled, then the overall variabilities of both methods would be equal.
	
	A formal test for inter-method bias can be implemented by examining the fixed effects of the model. This is common to well known classical linear model methodologies. The null hypotheses, that both methods have the same mean, which is tested against the alternative hypothesis, that both methods have different means.
	The inter-method bias and necessary $t-$value and $p-$value are presented in computer output. A decision on whether the first of Roy's criteria is fulfilled can be based on these values.
	
	Importantly \citet{roy} further proposes a series of three tests on the variance components of an LME model, which allow decisions on the second and third of Roy's criteria. For these tests, four candidate LME models are constructed. The differences in the models are specifically in how the the $D$ and $\Lambda$ matrices are constructed, using either an unstructured form or a compound symmetry form. To illustrate these differences, consider a generic matrix $A$,
	
	\[
	\boldsymbol{A} = \left( \begin{array}{cc}
	a_{11} & a_{12}  \\
	a_{21} & a_{22}  \\
	\end{array}\right).
	\]
	
	A symmetric matrix allows the diagonal terms $a_{11}$ and $a_{22}$ to differ. The compound symmetry structure requires that both of these terms be equal, i.e $a_{11} = a_{22}$.
	
	The first model acts as an alternative hypothesis to be compared against each of three other models, acting as null hypothesis models, successively. The models are compared using the likelihood ratio test. Likelihood ratio tests are a class of tests based on the comparison of the values of the likelihood functions of two candidate models. 
	
	
	
	
	
	
	\subsection{Correlation}
	In addition to the variability tests, Roy advises that it is preferable that a correlation of greater than $0.82$ exist for two methods to be considered interchangeable. However if two methods fulfil all the other conditions for agreement, failure to comply with this one can be overlooked. Indeed Roy demonstrates that placing undue importance to it can lead to incorrect conclusions. \citet{roy} remarks that current computer implementations only gives overall correlation coefficients, but not their variances. Consequently it is not possible to carry out inferences based on all overall correlation coefficients.
	
	%--------------------------------------------------%
	\subsection{Variability test 1}
	The first test determines whether or not both methods $A$ and $B$ have the same between-subject variability, further to the second of Roy's criteria.
	\begin{eqnarray*}
		H_{0}: \mbox{ }d_{A}  = d_{B} \\
		H_{A}: \mbox{ }d_{A}  \neq d_{B}
	\end{eqnarray*}
	This test is facilitated by constructing a model specifying a symmetric form for $D$ (i.e. the alternative model) and comparing it with a model that has compound symmetric form for $D$ (i.e. the null model). For this test $\boldsymbol{\hat{\Lambda}}$ has a symmetric form for both models, and will be the same for both.
	
	%---------------------------------------------%
	\subsection{Variability test 2}
	
	This test determines whether or not both methods $A$ and $B$ have the same within-subject variability, thus enabling a decision on the third of Roy's criteria.
	
	\begin{eqnarray*}
		H_{0}: \mbox{ }\lambda_{A}  = \lambda_{B} \\
		H_{A}: \mbox{ }\lambda_{A}  = \lambda_{B}
	\end{eqnarray*}
	
	This model is performed in the same manner as the first test, only reversing the roles of $\boldsymbol{\hat{D}}$ and $\boldsymbol{\hat{\Lambda}}$. The null model is constructed a symmetric form for $\boldsymbol{\hat{\Lambda}}$ while the alternative model uses a compound symmetry form. This time $\boldsymbol{\hat{D}}$ has a symmetric form for both models, and will be the same for both.
	
	As the within-subject variabilities are fundamental to the coefficient of repeatability, this variability test likelihood ratio test is equivalent to testing the equality of two coefficients of repeatability of two methods. In presenting the results of this test, \citet{roy} includes the coefficients of repeatability for both methods.
	
	%-----------------------------------------------%
	\subsection{Variability test 3}
	The last of the variability test examines whether or not methods $A$ and $B$ have the same overall variability. This enables the joint consideration of second and third criteria.
	\begin{eqnarray*}
		H_{0}: \mbox{ }\sigma_{A}  = \sigma_{B} \\
		H_{A}: \mbox{ }\sigma_{A}  = \sigma_{B}
	\end{eqnarray*}
	
	The null model is constructed a symmetric form for both $\boldsymbol{\hat{D}}$ and $\boldsymbol{\hat{\Lambda}}$ while the alternative model uses a compound symmetry form for both.
	
	
	
	
	\subsection{Model Terms (Roy 2009)}
	It is important to note the following characteristics of this model.
	
	Let the number of replicate measurements on each item $i$ for both methods be $n_i$, hence $2 \times n_i$ responses. However, it is assumed that there may be a different number of replicates made for different items. Let the maximum number of replicates be $p$. An item will have up to $2p$ measurements, i.e. $\max(n_{i}) = 2p$.
	
	% \item $\boldsymbol{y}_i$ is the $2n_i \times 1$ response vector for measurements on the $i-$th item.
	% \item $\boldsymbol{X}_i$ is the $2n_i \times  3$ model matrix for the fixed effects for observations on item $i$.
	% \item $\boldsymbol{\beta}$ is the $3 \times  1$ vector of fixed-effect coefficients, one for the true value for item $i$, and one effect each for both methods.
	
	Later on $\boldsymbol{X}_i$ will be reduced to a $2 \times 1$ matrix, to allow estimation of terms. This is due to a shortage of rank. The fixed effects vector can be modified accordingly.
	
	$\boldsymbol{Z}_i$ is the $2n_i \times  2$ model matrix for the random effects for measurement methods on item $i$.\\
	\bigskip
	
	$\boldsymbol{b}_i$ is the $2 \times  1$ vector of random-effect coefficients on item $i$, one for each method.
	
	$\boldsymbol{\epsilon}$  is the $2n_i \times  1$ vector of residuals for measurements on item $i$.\\
	\bigskip
	
	$\boldsymbol{G}$ is the $2 \times  2$ covariance matrix for the random effects.
	
	$\boldsymbol{R}_i$ is the $2n_i \times  2n_i$ covariance matrix for the residuals on item $i$.
	
	The expected value is given as $\mbox{E}(\boldsymbol{y}_i) = \boldsymbol{X}_i\boldsymbol{\beta}.$ \citep{hamlett}\\
	\bigskip
	
	The variance of the response vector is given by $\mbox{Var}(\boldsymbol{y}_i)  = \boldsymbol{Z}_i \boldsymbol{G} \boldsymbol{Z}_i^{\prime} + \boldsymbol{R}_i$ \citep{hamlett}.
	
	
	
	$\boldsymbol{b}_{i}$ is a $m-$dimensional vector comprised of
	the random effects.
	\begin{equation}
		\boldsymbol{b}_{i} = \left( \begin{array}{c}
			b_{1i} \\
			b_{21}  \\
		\end{array}\right)
	\end{equation}
	
	
	$\boldsymbol{V}$ represents the correlation matrix of the replicated measurements on a given method.
	$\boldsymbol{\Sigma}$ is the within-subject VC matrix.\\
	\bigskip
	
	
	$\boldsymbol{V}$ and $\boldsymbol{\Sigma}$ are positive
	definite matrices. The dimensions of $\boldsymbol{V}$ and
	$\boldsymbol{\Sigma}$ are $3 \times 3 ( = p \times p )$ and $ 2 \times
	2 (= k \times k)$.\\
	\bigskip
	
	
	It is assumed that $\boldsymbol{V}$ is the same for both methods and $\boldsymbol{\Sigma}$ is
	the same for all replications.\\
	\bigskip
	
	$\boldsymbol{V} \bigotimes \boldsymbol{\Sigma}$ creates a $ 6 \times 6 ( = kp \times
	kp)$ matrix.
	$\boldsymbol{R}_{i}$ is a sub-matrix of this.\\
	\bigskip
	
\section{Extension of Roy's methodology}
Roy's methodology is constructed to compare two methods in the presence of replicate measurements. Necessarily it is worth examining whether this methodology can be adapted for different circumstances.

An implementation of Roy's methodology, whereby three or more methods are used, is not feasible due to computational restrictions. Specifically there is a failure to reach convergence before the iteration limit is reached. This may be due to the presence of additional variables, causing the problem of non-identifiability. In the case of two variables, it is required to estimate two variance terms and four correlation terms, six in all. For the case of three variabilities, three variance terms must be estimated as well as nine correlation terms, twelve in all. In general for $n$ methods has $2 \times T_{n}$ variance terms, where $T_n$ is the triangular number for $n$, i.e. the addition analogue of the factorial. Hence the computational complexity quite increases substantially for every increase in $n$.

Should an implementation be feasible, further difficulty arises when interpreting the results. The fundamental question is whether two methods have close agreement so as to be interchangeable. When three methods are present in the model, the null hypothesis is that all three methods have the same variability relevant to the respective tests. The outcome of the analysis will either be that all three are interchangeable or that all three are not interchangeable.

The tests would not be informative as to whether any two of those three were interchangeable, or equivalently if one method in particular disagreed with the other two. Indeed it is easier to perform three pair-wise comparisons separately and then to combine the results.

Roy's methodology is not suitable for the case of single measurements because it follows from the decomposition for the covariance matrix of the response vector $y_{i}$, as presented in \citet{hamlett}. The decomposition depends on the estimation of correlation terms, which would be absent in the single measurement case. Indeed there can be no within-subject variability if there are no repeated terms for it to describe. There would only be the covariance matrix of the measurements by both methods, which doesn't require the use of LME models. To conclude, simpler existing methodologies, such as Deming regression, would be the correct approach where there only one measurements by each method.

\section{Conclusion}
\citet{BXC2008} and \citet{roy} highlight the need for method comparison methodologies suitable for use in the presence of replicate measurements. \citet{roy} presents a comprehensive methodology for assessing the agreement of two methods, for replicate measurements. This methodology has the added benefit of overcoming the problems of unbalanced data and unequal numbers of replicates. Implementation of the methodology, and interpretation of the results, is relatively easy for practitioners who have only basic statistical training. Furthermore, it can be shown that widely used existing methodologies, such as the limits of agreement, can be incorporated into Roy's methodology.


\addcontentsline{toc}{section}{Bibliography}

%--------------------------------------------------------------------------------------%

\bibliographystyle{chicago}
\bibliography{DB-txfrbib}
\end{document}



